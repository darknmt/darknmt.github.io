\iffalse
\begin{info}
The PDF version of this page can be downloaded by replacing \texttt{html} in the its address by
\texttt{pdf}. 
For example \texttt{/html/sheaf-cohomology.html} should become \texttt{/pdf/sheaf-cohomology.pdf}.
\end{info}
\fi

\iffalse
This post is my reading note for \cite{sacks_existence_1981}. I want to present the authors'
ideas as clear as possible and I may probably skip a few (important) details and computations.
\fi

\section{Brief view of Sacks and Uhlenbeck's strategy.}
\label{sec:org4fcb43b}

Let \(M\) and \(N\) be compact Riemannian manifolds (without boundary), \(M\) is a
surface and \(N\) is isometrically embedded in \(\mathbb{R}^k\). It was showed by
Eells and Sampson \cite{eells_harmonic_1964} that if \(N\) is \href{harmonic-map-existence.org}{negatively curved} than any
map from \(M\) to \(N\) is homotopic to a harmonic map. The idea of Sacks and
Uhlenbeck in \cite{sacks_existence_1981} consists of (1) approximating the energy functional
\(E\) by a family \(E_\alpha\) satisfying Palais-Smale condition, whose
\emph{nontrivial} critical values can be more easily proved to exist and (2) trying to prove
that the critical maps \(s_\alpha\) of \(E_\alpha\) converge in \(C^1\)-topology.

We will first review the general machinery of Morse-Palais-Smale theory and prove the existence
of \(s_\alpha\). The convergence of \(s_\alpha\) in the case of surface is due to the
facts that energy functional \(E\) is a conformal invariant of \(M\), in particular \(E\) is
invariant by homotheties (i.e.  \(E\) remains unchanged when we zoom in and out), which
allows us to justify the \(C^1\)-convergence (under conditions of \(N\)) except at
finitely many points using a local estimate and a suitable
covering of \(M\). 

Sacks and Uhlenbeck used an extension result for harmonic map, in an elegant argument
to prove that if the above sequence \(\{s_\alpha\}\) fails to converge at a point,
for a certain surface \(M\), then one has a nontrivial harmonic map from \(\mathbb{S}^2\) to \(N\). Therefore if such sequence \(\{ s_\alpha \}\) from \(\mathbb{S}^2\) to \(N\) exists, for example when \(\pi_k (N)\) is nontrivial for a certain \(k\geq 2\) then, 
whether \(s_\alpha\) converges or not, there exists a nontrivial harmonic map from \(\mathbb{S}^2\) to \(N\). 

Finally, the theory of branched immersion of surfaces by Gulliver-Osserman-Royden
\cite{gulliver_theory_1973} can be applied to show that the harmonic map obtained this way
is a conformal, branched, minimal immersion of \(\mathbb{S}^2\) to \(N\).

\section{General machinery by Morse-Palais-Smale.}
\label{sec:org4d8655c}

\subsection{Perturbed functionals \(E_\alpha\).}
\label{sec:orgb908691}

Let \(s: M \longrightarrow N \hookrightarrow \mathbb{R}^k\) be a map from a compact
surface \(M\) to a compact Riemannian manifold \(N\) isometrically embedded into \(\mathbb{R}^k\). Recall that the energy functional of \(s\) is given by \(E(s):=
\frac{1}{2}\int_M |ds|^2 dV_M = \frac{1}{2}\int_M \langle s^* g_N, g_M \rangle dV_M\). The perturbed energy functionals are
\[
 E_\alpha(s) := \int_M\left(1 + |ds|^2\right)^\alpha dV,\quad \alpha \geq 1
\]
We will suppose, by rescaling the metric \(g_M\) of \(M\) that the volume of \(M\)
is 1, so when \(\alpha=1\), \(E_1 =  1+ 2E(s)\) is just the previously defined energy. Using \((a+b)^\alpha \geq a^\alpha + b^\alpha\) and Jensen's inequality, one has \(E_\alpha(s) \geq 1 + (2E(s))^\alpha\) for all \(\alpha \geq 1\). Also, since we only
interest in the case \(\alpha\) close to \(1\), let us also suppose that \(\alpha\) from now on is smaller than 2.

By Sobolev embedding, one has \(W^{1,2\alpha}(M, \mathbb{R}^k) \subset C^0(M,
\mathbb{R}^k)\) compactly for all \(\alpha >1\). It then makes sense to talk about \(W^{1,2\alpha}(M,N) \subset C^0(M,N)\) which consist of elements of \(W^{1,2\alpha}(M,
\mathbb{R}^k)\subset C^0(M, \mathbb{R}^k)\) whose image lies in \(N\).

\begin{theorem}[Palais]
\label{thm:Palais-1}
The spaces \(C^\infty (M,N)\subset W^{1,2\alpha}(M,N)\subset C^0(M,N)\), where \(\alpha>1\), are
of the same homotopy type and the inclusions are homotopy equivalences. In particular,
their connected components are naturally in bijection.
\end{theorem}

We will also need the following version of Morse theory for function spaces, also
developed by R. Palais.

\begin{theorem}[Morse theory for Banach manifolds]
\label{thm:Palais-2}
\begin{enumerate}
\item If \(F\) is a \(C^2\) functional on a complete \(C^2\) Finsler manifold \(L\)
and \(F\) satisfies Palais-Smale condition then
\begin{enumerate}
\item The functional \(F\) admits minimum on each connected component of \(L\).
\item If \(F\) has no critical value in \([a,b]\) then the sublevel \(\{F\leq b\}\)
retracts by deformation to the sublevel \(\{F\leq a\}\).
\end{enumerate}
\item The pair \((L,F)=(W^{1,2\alpha}(M,N), E_\alpha)\) with \(\alpha > 1\) satisfies the
condition of the first part.
\end{enumerate}
\end{theorem}

By consequence, one has

\begin{corollary}[Component-wise minimum of \( E_\alpha \)]
\label{cor:Palais-1-2}
The minimum of \(E_\alpha\) in each connected component \(C\) of \(W^{1,2\alpha}(M,N)\), \(\alpha >1\) is taken by some \(s_\alpha \in C^\infty(M,N)\) and there exists \(B>0\)
depending on the component \(C\) such that 
\[
 \min_{C}E_\alpha \leq (1+B^2)^\alpha
\]
\end{corollary}

\begin{proof}
By Theorem \ref{thm:Palais-2}, \(E_\alpha\) admits minimum at \(s_\alpha\) on each
component \(C\) of \(W^{1,2\alpha}(M,N)\). By writing down the Euler-Lagrange equation
of \(E_\alpha\) and apply regularity estimates, one can prove that \(s_\alpha\) is
actually smooth. By Theorem \ref{thm:Palais-1}, the preimage of \(C\) by inclusion \(C^\infty(M,N)\subset W^{1,2\alpha}(M,N)\) is a connected component \(C'\) of \(C^\infty(M,N)\) over which \(s_\alpha\)
is the minimum of \(E_\alpha\). Take \(B = \sup_M |du|\) for an arbitrary element \(u\in C'\) and the conclusion follows.
\end{proof}

\begin{remark}
Corollary \ref{cor:Palais-1-2} is trivialised when \(W^{1,2\alpha}(M,N)\) is connected
(for one \(\alpha\) or equivalently for all \(\alpha\)). In
this case, \(s_\alpha\) is a constant map and \(B=0\).
\end{remark}

To establish a nontrivial analog of Corollary \ref{cor:Palais-1-2} in the case where the
spaces of maps from \(M\) to \(N\) are connected, we will have to look at the
submanifold \(N_0\cong N\) formed by constant maps.

\subsection{Tubular neighborhood of the submanifold of trivial maps.}
\label{sec:orgde54354}

Fix \(y\in N\), considered as a constant maps in \(N_0\). We will summarise a few facts about the tangent space of \(W^{1,2\alpha}(M,N)\) at \(y\).

\begin{remark}
\label{rem:tangent-palais}
\begin{enumerate}
\item The tangent \(T_y W^{1, 2\alpha}(M,N)\) can be identified with \(W^{1,2\alpha}(M, T_y
   N)\). The subspace \(T_y N_0\) contains constant maps from \(M\) to \(T_y N\).
\item The fiber \(\mathcal{N}_y\) over \(y\) of the normal bundle \(\mathcal{N}\) of \(N_0\) can be identified with
\[
    \mathcal{N}_y = \left\{v\in W^{1,\alpha}(M, T_y N): \int_M v dV = 0\right\}
   \]
\end{enumerate}
\end{remark}

The exponential map on \(TW^{1,2\alpha}(M,N)\) can be defined as follows:
\begin{align*}
  e:\ TW^{1,2\alpha}(M,N) & \longrightarrow W^{1,2\alpha}(M,N)\\
  (s,v) 		  &\longmapsto \left(x\mapsto \exp_{s(x)}v(x) \right)
\end{align*}
where \(s\in W^{1,2\alpha}(M,N)\) and \(v\in T_s W^{1,2\alpha}(M,N)\) is a \(W^{1,2\alpha}\) vector field along \(s(x)\). With the representation of normal bundle
\(\mathcal{N}\) as Remark \ref{rem:tangent-palais}, the restriction of \(e\) on \(\mathcal{N}\) is given by
\begin{align*}
\restr{e}{\mathcal{N}}:\ \mathcal{N} &\longrightarrow  W^{1,2\alpha}(M,N)\\
			 (y,v) &\longmapsto	       \left(x\mapsto \exp_y (v(x)) \right)
\end{align*}
where \(y\in N_0 \cong N\) and \(v \in W^{1,2\alpha}(M, T_yN)\).

\begin{lemma}
\label{lem:local-isom-e}
The restriction \(\restr{e}{\mathcal{N}}\) of \(e\) on \(\mathcal{N}\) is a local
diffeomorphism mapping a neighborhood of the zero-section of \(\mathcal{N}\) onto a
neighborhood of \(N_0\) in \(W^{1,2\alpha}(M,N)\).
\end{lemma}
\begin{proof}
It can be calculated that 
\[
  de_{(y,0)}(a,v) = \left(x\mapsto a + v(x)\right) \in T_yW^{1,2\alpha}(M,N) =
W^{1,2\alpha}(M, T_yN)
\]
for \(a\in T_y N\) and \(v\in \mathcal{N}_y \subset W^{1,2\alpha}(M, T_y N)\). It
is invertible since \(a\) is tangential to \(N_0\) and \(v\in \mathcal{N}_y\) is
in the normal component. The Inverse function theorem applies.
\end{proof}

\subsection{Critical values of \(E_\alpha\).}
\label{sec:org7be5a37}

The exponential map previously defined on the normal bundle of \(N_0\) in  \(W^{1,
2\alpha}(M,N)\) allows us to retract by deformation a small neighborhood of \(N_0\) to
\(N_0\). We will prove that if the energy \(E_\alpha(s)\) is sufficiently close to \(1=E_\alpha(N_0)\) then \(s\) is sufficiently \(W^{1,2\alpha}\)-close to \(N_0\) and hence can be retracted to \(N_0\), in other words, \(E_\alpha^{-1}[1, 1+\delta]\) retracts by deformation to \(N_0 = E_\alpha^{-1}(1)\).

\begin{proposition}[]
\label{prop:crit-val-1}
Given \(\alpha>1\), there exists \(\delta >0\) depending on \(\alpha\) such that \(E_\alpha^{-1}[1, 1+\delta]\) retracts by deformation to \(E_\alpha^{-1}(1) = N_0\).
\end{proposition}

\begin{proof}
Let \(s\in E_\alpha^{-1}[1, 1+\delta]\), using \((a+b)^\alpha \geq a^\alpha + b^\alpha\), one has
\[
 1 + \delta > \int_M (1+|ds|^2)^\alpha dV > 1 + \int_M |ds|^{2\alpha}dV
\]
therefore \(\|ds\|_{L^{2\alpha}}\leq \delta ^{1/{2\alpha}}\). By Poincaré-Wirtinger
inequality, \(\|s-\int_M s\|_{W^{1,2\alpha}}\leq C \delta^{1/4}\) where \(C\) is the
Poincaré-Wirtinger constant.

By Sobolev embedding, \(\max_M |s-\int_M s| \leq C_\alpha \|s -\int_M s\|_{W^{1,2\alpha}}\)
where the Sobolev constant \(C_\alpha\) can no longer be chosen uniformly in \(\alpha
\to 1\). Fix an \(x_0\in M\), one has
\[
 d_{W^{1,2\alpha}}(s, N_0) \leq \|s - s(x_0)\|_{W^{1,2\alpha}} \leq \left\|s-\int_M
s\right\|_{W^{1,2\alpha}} + \left|\int_M s - s(x_0)\right| \leq C_\alpha \delta^{1/4}
\]

Now choose \(\delta \ll 1\) depending on \(\alpha\) such that \(s\) is in the
neighborhood of \(N_0\) given by Lemma \ref{lem:local-isom-e}, \(s\) can be written as
\[
 s(x) = e(y,v(x)) = \exp_y v(x)
\]
where \(y\in N_0\) and \(v\in W^{1,2\alpha}(M, T_yN)\) depend continuously on \(s\in
W^{1,2\alpha}(M,N)\). We can define the deformation retraction by
\begin{align*}
  \sigma:\ E^{-1}_\alpha [1,1+\delta]\times [0,1] & \longrightarrow E^{-1}_\alpha[1,1+\delta]\\
  	   		 (s,t)		    	  &\longmapsto \left( x\mapsto \exp_y tv(x)\right)
\end{align*}
It is clear that \(\sigma\) is continuous and \(\sigma_0\) is a retraction. The only
thing to check is that the image of \(\sigma\) remains in \(E_\alpha^{-1}[1,1+\delta]\) at all time. This can be checked by showing that \(\frac{d}{dt}E_\alpha(\sigma_t) \geq
0\), hence \(E_\alpha(\sigma_t) \leq E_\alpha(\sigma_1) \leq 1+\delta\) for all \(0\leq
t\leq 1\).
\end{proof}

We will now prove the existence of nontrivial critical value of \(E_\alpha\) in an
interval \((1, B)\) for a certain \(B>1\) sufficiently big independently of \(\alpha
> 1\).

Fix \(z_0\in M\) and consider the map 
\begin{align*}
  p: \ C^0(M,N) &\longrightarrow N\\
       s	&\longmapsto	 f(z_0)
\end{align*}
then \(p\) is a fiber bundle and therefore is a \emph{Serre fibration}. In fact fix \(q_0\in N\) then for all \(q\in N\) near \(q_0\), there is a vector field \(v_q\) supported
in a small ball centered at \(q_0\) such that the flow of \(v_q\) from time 0 to 1
turns \(q_0\) to \(q\), i.e. \({\Phi_{v_q}}_0^1(q_0)=q\), and that \(v_q\) varies continuously
in \(q\). Then any fiber \(p^{-1}(q)\) can be identified with \(p^{-1}(q_0)\) using
the flow of \(v_q\). We will denote by \(\Omega(M,N)\) the topological fiber of \(p\).


We will use a few facts from algebraic topology, briefly summarised here.

\begin{fact}
\label{fact:alg-top-Omega-M-N}
\begin{enumerate}
\item (Long exact sequence of homotopy) Let \(p: E \longrightarrow B\) be a fiber bundle
of fiber \(F = p^{-1}(b_0) \ni f_0\), then one has the following long exact sequence
\[
    \xymatrix{
    \dots \ar@{->}[r]^{\partial} & \pi_n(F) \ar@{->}[r]^{\iota_*} & \pi_n(E) \ar@{->}[r]^{p_*} & \pi_n(B) \ar@{->}[r]^{\partial} & \pi_{n-1}(F) \ar@{->}[r] & \dots \ar@{->}[r] & \pi_0(E) \ar@{->}[r] & 0
    }
   \]
 where \(\iota: F \longrightarrow E\) is the inclusion.
\item If \(p\) admits a global section \(s\), then one has a retraction \(s_*\) of \(p_*\):
\[
    \xymatrix{
    \pi_n(E) \ar@{->}[r]^{p_*} & \pi_n(B) \ar@/^/@{->}[l]^{s_*}
    }
   \]
hence \(p_*\) is surjective and \(\partial\) factors through \(0\), which gives
us the short exact sequence
\[
    \xymatrix{
    0 \ar@{->}[r] & \pi_n(F) \ar@{->}[r]^{\iota_*} & \pi_n(E) \ar@{->}[r]^{p_*} & \pi_n(B) \ar@/^/@{->}[l]^{s_*} \ar@{->}[r] & 0
    }
   \]
where \(p_*\) admits a retraction \(s_*\), so the short exact sequence splits and
we have
\[
   \pi_n(E) \cong \pi_n(F) \oplus \pi_n(B).
   \]
\end{enumerate}
\end{fact}

Now apply this result to the fiber bundle \(p: C^0(M,N)
\longrightarrow N\) of fiber \(\Omega(M,N)\), which has \(N_0\) as a global section,
one obtains
\[
 \pi_n(C^0(M,N)) \cong \pi_n(N) \oplus \pi_n(\Omega(M,N)).
\]


\begin{theorem}[Nontrivial critical value of \( E_\alpha \)]
\label{thm:3-nontrivial-crit}
If \(C^0(M,N)\) is not connected, or if \(\Omega(M,N)\) is not contractible, then
there exists \(B>0\) such that for all \(\alpha >1\), \(E_\alpha\) has critical
values in the interval \((1, (1+B^2)^\alpha)\). 

In particular, if \(M=\mathbb{S}^2\) and if the universal covering \(\tilde N\) of \(N\) is not contractible then \(E_\alpha\) has critical values in \((1, (1+B^2)^\alpha)\).
\end{theorem}

\begin{proof}
If \(C^0(M,N)\) is not connected, one only needs to apply Corollary \ref{cor:Palais-1-2} to a
connected component of \(W^{1,2\alpha}(M,N)\) not containing \(N_0\). We now suppose
that \(C^0(M,N)\) is connected and \(\Omega(M,N)\) is not contractible.


In this case, there exists \(n>0\) such that \(\pi_n(\Omega(M,N))\) is nontrivial and
contains a nonzero element \(\gamma:\ \mathbb{S}^n \longrightarrow  \Omega(M,N)\)
which is not homotopic to any \(\tilde\gamma:\ \mathbb{S}^n \longrightarrow N_0\) in \(\pi_n(C^0(M,N))\).


Choose \(B:= \max_{\theta\in \mathbb{S}^n,x\in M} |d\gamma(\theta)(x)|\) then by
definition
\[
 E_\alpha(\gamma(\theta)) \leq (1+B^2)^\alpha\quad\forall \theta\in \mathbb{S}^n,\alpha>1.
\]
If \(E_\alpha\) has no critical value in \([1+\frac{\delta_\alpha}{2}, (1+B^2)^\alpha]\) where \(\delta_\alpha\) is given by Proposition \ref{prop:crit-val-1}, then by Theorem
\ref{thm:Palais-2}, \(E_\alpha^{-1}[1, (1+B^2)^\alpha]\) retracts by deformation to \(E_\alpha^{-1}[1, 1+\delta_\alpha]\) which retracts by deformation to \(E_\alpha^{-1}(1)=N_0\). But this means that \(\gamma\) is homotopic to a certain \(\tilde\gamma\in \pi_n(N)\), which is a contradiction.


As an application, if \(M= \mathbb{S}^2\) and the universal covering \(\tilde N\) is
not contractible then the long exact sequence of homotopy for the bundle \(\tilde N
\longrightarrow N\) with fiber of dimension \(0\), gives
\[
 \pi_n(\tilde N) = \pi_n(N),\quad \forall n\geq 2.
\]
Since \(\tilde N\) is simply-connected and not contractible, there exists \(n\geq 2\)
such that \(0\ne\pi_n(\tilde N) = \pi_n(N) = \pi_{n-2}(\Omega(\mathbb{S}^2,N))\), where
the last equality follows from definition of homotopy group. The
general argument applies.
\end{proof}


\section{Local results: Estimates and extension.}
\label{sec:org5c54772}

We will say that the map \(s: M \longrightarrow N\) is a critical point of \(E_\alpha\) on a small disc \(D(R)\subset M\) if \(s\) satisfies the Euler-Lagrange equation of
\(E_\alpha\) (as functional on \(W^{1,2\alpha}(M,N)\)) on \(D(R)\).

\begin{remark}
\label{rem:rescal-Euclide}
Rescaling \((D(R), g_M)\), where \(R\ll 1\) and \(g_M\) is
\(\epsilon\)-close to the Euclidean metric, to the unit disc \(D\) one obtains a
metric \(\tilde g_M\) that is still \(\epsilon\)-close to Euclidean metric. The
curvature of \(\tilde g_M\) is \(R^2\) times smaller than that of \(g_M\).
\end{remark}

If \(s:\ D(R) \longrightarrow N\) is a critical map
of \(E_\alpha\) on \(D(R)\), then the composition \(\tilde s\) of \(s\) and the rescaling operator \(D
\longrightarrow D(R)\) satisfies the Euler-Lagrange equation of \(\tilde E_\alpha =
R^{2(1-\alpha)}\int_D (R^2 + |d\tilde s|^2)^\alpha d\tilde V\) where \(d\tilde V\) is
the volume form of the rescaled metric \(\tilde g_M\). We will abusively
use the same notation for \(\tilde s\) and \(s\) and regard \(s\) as a map on the unit disc \(D\). 

\begin{lemma}[Sacks-Uhlenback's Main estimate]
\label{lem:main-est}
For all \(p\in (1,+\infty)\), there exists \(\epsilon>0\) and \(\alpha_0 >1\)
depending on \(p\) such that if
\begin{itemize}
\item \(s:\ (D,\tilde g) \longrightarrow N\) is a critical map of \(E_\alpha\) on \(D(R)\)
\item \(E(s) < \epsilon\), \(1 < \alpha <\alpha_0\)
\end{itemize}
then
\[
 \|ds\|_{W^{1,p}(D')} < C(p,D') \|ds\|_{L^2(D)},\quad \text{for all disc }  D'\Subset D 
\]
\end{lemma}

\begin{remark}
In fact \(\alpha_0, \epsilon\) and \(C(p,D')\) depend on the rescaled metric \(\tilde
g\) on \(D\), but if \(R \ll 1\) and \(\tilde g\) is very close to Euclidean metric,
then one can choose these parameters independently of \(\tilde g\). 
\end{remark}

A consequence of (the proof of) Lemma \ref{lem:main-est} is the following global result:

\begin{theorem}[Critical maps of low energy are trivial]
\label{thm:4-trivial-energy}
There exists \(\epsilon' >0\) and \(\alpha_0 > 1\)  such that if
\begin{itemize}
\item \(s: M \longrightarrow N\) is critical map of \(E_\alpha\)
\item \(E(s)<\epsilon'\), 1 < \(\alpha\) <\(\alpha_{\text{0}}\)
\end{itemize}
then \(s\in N_0\) and \(E(s) = 0\).
\end{theorem}

We proved in the last section that, under certain algebraic topological condition on \(N\), \(E_\alpha\) admits critical value \(v_\alpha \in (1, (1+B^2)^\alpha)\). We now
can conclude that, by Theorem \ref{thm:4-trivial-energy}, the critical values \(v_\alpha\)
are bounded away from \(1\), i.e. \(\inf_{\alpha} v_\alpha > 1\).


We will also need the following extension theorem:

\begin{theorem}[Extension of harmonic maps]
\label{thm:extension-sacks-uhlenbeck}
If \(s:\ D\setminus \{0\} \longrightarrow N\) is a harmonic map with finite energy \(E(s) < \infty\), then \(s\) extends to a smooth harmonic map \(\tilde s:\ D \longrightarrow N\).
\end{theorem}



\section{Convergence of critical maps of \(E_{\alpha}\).}
\label{sec:orgfd03f49}

We proved in Theorem \ref{thm:3-nontrivial-crit} that if \(C^0(M,N)\) is not connected or
if \(\Omega(M,N)\) is not contractible, then there exists a family \(\{s_\alpha\}\) of
critical maps  of \(E_\alpha\) with bounded, nontrivial energy \(E_\alpha(s_\alpha) < B\). Since 
\begin{itemize}
\item \(\int_M |ds_\alpha|^2 \leq \left(E_\alpha(s_\alpha)-1\right)^{1/\alpha}\) is bounded
uniformly on \(\alpha\)
\item \(\|s_\alpha\|_{L^\infty}\) is bounded by compactness of \(N\).
\end{itemize}
the \(W^{1,2}(M, \mathbb{R}^k)\)-norms of \(\{s_\alpha\}\) are bounded. By
reflexivity of Sobolev spaces, there exists a subsequence \(\{s_\beta\}\) weakly
converging to \(s\) in \(W^{1,2}(M,\mathbb{R}^k)\) with
\[
\|s\|_{W^{1,2}}\leq \liminf_{\beta\to 1} \|s_\beta\|_{W^{1,2}}
\]
We do not know at this moment if the convergence is \(C^0\), or if \(s\) is
continuous, or even if the image of \(s\) remains in \(N\). The following key lemma
answer these questions on a small disc of \(M\) in the case the energy of \(s_\alpha\)
is small.

\begin{lemma}[Key]
\label{lem:key-sacks-uhlenbeck}
There exists an \(\epsilon>0\), in fact given by the Main estimate Lemma \ref{lem:main-est} with
\(p=4\), such that if
\begin{itemize}
\item \(s_\alpha:\ D(R) \longrightarrow N\subset \mathbb{R}^k\) are critical maps of \(E_\alpha\) in \(W^{1, 2\alpha}(D(R),N)\),
\item \(E(s_\alpha) < \epsilon\) and \(s_\alpha\) converges weakly to \(s\) in \(W^{1,2}(D(R),\mathbb{R}^k)\),
\end{itemize}
then 
\begin{itemize}
\item the restriction of \(s\) on \(\overline{D(R/2)}\) is smooth harmonic map with image in \(N\),
\item \(s_\alpha \to s\) in \(C^1(\overline{D(R/2)},N))\).
\end{itemize}
\end{lemma}

\begin{remark}
\label{rem:topo-C1}
There are two different ways to define convergence of a sequence \(s_n\) to \(s\) in
\(C^1(\Omega)\) on an open set \(\Omega\):
\begin{enumerate}
\item The sequence \(s_\alpha\) and \(s\) extend to \(C^1(\bar\Omega)\) and have finite norm \(\max_\Omega |s| +
   \max_\Omega|ds|\) and \(\max_\Omega |s_\alpha| + \max_\Omega|ds_\alpha|\) and
\[
     \max_\Omega |s_\alpha - s| + \max_\Omega |ds-ds_\alpha| \to 0.
   \]
In this case, we will say that \(s_\alpha\) converges to \(s\) in \(C^1(\bar\Omega)\).
\item \(C^1(\Omega)\) is topologised by a family of seminorms
\(\Gamma_K:\ s\longmapsto \max_{K} |s| + \max_{K}|ds|\)
for \(K\Subset\Omega\). This makes \(C^1(\Omega)\) a Fréchet topological vector space.
If the sequence \(s_\alpha\) converges to \(s\) under this topology then we will
say that \(s_\alpha\) converges uniformly to \(s\) on compacts of \(\Omega\).
\end{enumerate}
\end{remark}

\begin{proof}
We consider \(s_\alpha\) and \(s\) as maps from the unit disc \(D\) to \(\mathbb{R}^k\), then by Main estimate Lemma \ref{lem:main-est} for \(p=4\), since \(E(s_\alpha) < \epsilon\), one has:
\[
\|ds_\alpha\|_{W^{1,4}(D(1/2), \mathbb{R}^k)} \leq C(4, D(1/2)) \| ds_\alpha\|_{L^2(D)} =
C(4, D(1/2)) E(s_\alpha)^{1/2}
\]
So \(\{s_\alpha\}\) is bounded in \(W^{1,4}(D(1/2), \mathbb{R}^k)\) which is embedded
compactly into \(C^1(\overline{D(1/2)}, \mathbb{R}^k)\).

We now can prove that \(s_\alpha\) converges strongly to \(s\) in \(C^1(\overline{D(1/2)},\mathbb{R}^k)\): If there was a subsequence \(\{s_\beta\}\)
whose restriction to \(\overline{D(1/2)}\) remains \(C^1\)-away from \(s\), then by compactness of \(W^{1,4}(D(1/2),
\mathbb{R}^k) \hookrightarrow C^1(\overline{D(1/2)}, \mathbb{R}^k)\), we can suppose that \(\{s_\beta\}\) converges in \(C^1\) to a certain \(\bar s\ne s\) on \(\overline{D(1/2)}\). But as a
subsequence of \(\{s_\alpha\}\), \(\{s_\beta\}\) converges weakly to \(s\) on \(D\), hence on \(\overline{D(1/2)}\), we than obtain a contradiction using the uniqueness of
weak limit.

By considering the Euler-Lagrange equation and letting \(\alpha\to 0\), one concludes
that \(s\) is a harmonic map from \(D(1/2)\) to \(N\).
\end{proof}

The global convergence of \(\{s_\alpha\}\) can be established by a well-chosen covering
of \(M\) by small balls or radius \(R\).

\begin{proposition}
\label{prop:2-sacks-uhlenbeck}
Let \(s_\alpha:M \longrightarrow N\subset \mathbb{R}^k\) be critical maps of \(E_\alpha\) on \(M\) such that \(s_\alpha\) converges weakly to \(s\) in \(W^{1,2}(M,
\mathbb{R}^k)\) and \(E(s_\alpha) < B\). Then there exists \(l=l(B,N)\) such that given any \(m>0\),
one can find a sequence \(\{x_{m,1},\dots, x_{m,l}\}\subset M\) and a subsequence \(\{s_{\alpha(m)}\}\) of \(\{s_\alpha\}\) such that
\[
 s_{\alpha(m)} \longrightarrow  s\ \text{in } C^1\left(M\setminus\bigcup_{i=1}^{l} D(x_{m,i}, 2^{-m+1}),N\right)
\]
\end{proposition}
\begin{proof}
We cover \(M\) by finitely many balls \(D(y_i, 2^{-m})\) such that each point is covered
at most \(h\) times by the bigger balls \(D(y_i,2^{-m+1})\). By Lemma
\ref{lem:uni-loc-finite-cover}, \(h\) can be chosen independently of \(m\) as \(2^{-m} \to 0\).


Since \(\sum_i \int_{D(y_i, 2^{-m+1})} |ds_\alpha|^2 < B h\), choosing \(l = \lceil
\frac{Bh}{2\epsilon} \rceil\), we see that there are at most \(l\) balls \(D(y_{\alpha,i},
2^{-m+1})\) with centers depending on \(\alpha\), on which the energy \(E(s_\alpha)\) is less than
\(\epsilon\). Passing to a subsequence \(\{s_{\alpha(m)}\}\) of \(\{s_\alpha\}\),
we can suppose that \(\{y_{\alpha(m),i} \}\) converges to \(x_{m,i}\) as \(\{\alpha(m)\} \to 1\). But since the points \(\{y_i\}\) are of finite number and
separated, \(y_{\alpha(m),i}\equiv x_{m,i}\) eventually and we can suppose that the bad balls
\(D(y_{\alpha(m),i})\) where energy of \(s_{\alpha(m)}\) surpasses \(\epsilon\) are
the same for every \(s_{\alpha(m)}\).

Now apply Lemma \ref{lem:key-sacks-uhlenbeck} to the sequence \(\{s_{\alpha(m)}\}\) on all
the other \(2^{-m+1}\)-balls, one sees that \(\{s_{\alpha(m)}\}\) converges in \(C^1\) to \(s\) on all \(\overline{D(y_i, 2^{-m})}\) except those centered at \(x_{m,i}\). The conclusion follows.
\end{proof}

Using a diagonal argument, we can find a subsequence \(\{s_\beta\}\) of \(\{s_\alpha\}\) that converges to \(s\) uniformly on compacts of \(M\setminus \{x_1,\dots, x_l\}\).

\begin{theorem}[Convergence of \( \{s_\alpha\} \)]
\label{thm:5-convergence-crit-map}
Let \(s_\alpha:M \longrightarrow N\subset \mathbb{R}^k\) be critical maps of \(E_\alpha\) on \(M\) such that \(s_\alpha\) converges weakly to \(s\) in \(W^{1,2}(M,
\mathbb{R}^k)\) and \(E(s_\alpha) < B\). Then there exist at most \(l\) points \(x_1,\dots, x_l\) in \(M\), where \(l\) is given by Proposition
\ref{prop:2-sacks-uhlenbeck}, and a subsequence \(\{s_\beta\}\) of \(\{s_\alpha\}\) such that
\[
s_\beta \longrightarrow  s \ \text{in } C^1(M\setminus\{x_1,\dots, x_l\}, \mathbb{R}^k) \
\text{uniformly on compacts}.
\]
\end{theorem}
\begin{proof}
By passing to a subsequence \(\{m_k\}\) of \(\{ m\}\), we can suppose that
\(\{x_{m,i}\}\) converges to \(x_i\) in \(M\). Choose the diagonal subsequence \(\{s_\beta\}\) from \(\{s_{\alpha(m)}\}\) that consists of \(s_{\alpha(m)(a_m)}\)
where \(a_m\) is sufficiently big such that \(\alpha(m)(a_m)\) is increasing and
\(\|s_{\alpha(m)(b)} - s_{\alpha(m)(c)} \|_{C^1(M\setminus \cup_i D(x_{m,i},2^{-m+1})} <
\frac{1}{m}\) for all \(b,c\geq a_m\). Then the sequence \(\{s_\beta\}\) converges
uniformly on compacts of \(M\setminus\{x_1,\dots,x_l\}\) because \(\{\bigcup_i
D(x_{m,i}, 2^{-m+1})\}_m\) is an exhaustive family of compacts of \(M\setminus\{x_1,\dots,x_l\}\).
\end{proof}

\begin{remark}
With the same notation as Theorem \ref{thm:5-convergence-crit-map},
\begin{enumerate}
\item The image \(s(M\setminus\{x_1,\dots,x_l\})\) lies in \(N\). Also, using the
Euler-Lagrange equation, one sees that \(s\) is a (smooth) harmonic map from \(M\setminus\{x_1,\dots, x_l\}\) to \(N\).
\item Since \(E(s) \leq \|s\|_{W^{1,2}}^2 \leq \liminf_{\alpha\to 1} \|s_\alpha\|^2 <+\infty\), \(\restr{s}{M\setminus\{x_1,\dots, x_l\}}\) extends to a harmonic
map \(\tilde s:\ M \longrightarrow N\). We can therefore suppose that the limit \(s\) of
Theorem \ref{thm:5-convergence-crit-map} is smooth harmonic map on \(M\) and of image
in \(N\).
\end{enumerate}
\end{remark}


\section{Nontrivial harmonic maps from \(\mathbb{S}^2\).}
\label{sec:orgffd7b44}
We will now prove the existence of nontrivial harmonic maps from \(\mathbb{S}^2\) to a
compact Riemannian manifold \(N\) satisfying the conditions of Theorem
\ref{thm:3-nontrivial-crit}. 

The following theorem does not suppose any condition on \(N\).

\begin{theorem}
\label{thm:6-final-sacks-uhlenbeck}
Let \(M\) be a compact surface and \(s_\alpha\) be critical maps of \(E_\alpha\). Suppose that
\begin{itemize}
\item \(s_\alpha\) converges in \(C^1\) to \(s\) uniformly on compacts of \(M\setminus\{x_1,\dots,x_l\}\) but not on \(M\setminus\{x_2,\dots,x_l\}\).
\item \(E(s_\alpha) < B\)
\end{itemize}
Then there exists a nontrivial harmonic map \(s_*: \mathbb{S}^2 \longrightarrow N\).
\end{theorem}

Before proving the theorem, let us state its corollary.

\begin{corollary}[Nontrivial harmonic map from \(\mathbb{S}^2\)]
If the universal covering \(\tilde N\) of \(N\) is not contractible then there exists
a nontrivial harmonic map \(s: \mathbb{S}^2 \longrightarrow N\).
\end{corollary}

\begin{proof}
By Theorem \ref{thm:3-nontrivial-crit} and Theorem \ref{thm:4-trivial-energy}, there exist
critical maps \(s_\alpha: \mathbb{S}^2 \longrightarrow N\) of \(E_\alpha\) corresponding to critical values \(E_\alpha(s_\alpha)\) in \((1+\delta,B)\). We claim that \(\{s_\alpha\}\) cannot converge
in \(C^1(M)\) to a trivial harmonic map \(s\in N_0\). In fact, if it did,
\[
 1+\delta  \leq \lim_{\alpha\to 1} \int_M (1+|ds_\alpha|^2)^\alpha dV = \int_M (1+|ds|^2)dV
= 1
\]
which is contradictory.

Therefore, we only have two possibilities: 
\begin{itemize}
\item \(\{s_\alpha\}\) does not converge in \(C^1(M)\) to \(s\), then by Theorem
\ref{thm:6-final-sacks-uhlenbeck}, there exists a nontrivial harmonic map \(s_*:
  \mathbb{S}^2 \longrightarrow N\).
\item If \(\{s_\alpha\}\) converges in \(C^1(M)\) to a certain \(\tilde s\), then as
argued above, \(\tilde s\) is nontrivial.
\end{itemize}
In both cases, nontrivial harmonic map from \(\mathbb{S}^2\) to \(N\) exists.
\end{proof}

Let us now prove Theorem \ref{thm:6-final-sacks-uhlenbeck}.

\begin{proof}[Proof of Theorem \ref{thm:6-final-sacks-uhlenbeck}]
If there is no \(C^1\) convergence near \(x_1\), we claim that:

\begin{assertion}
\label{assert:3-star}
For all \(C>0\) and \(\delta >0\), there exists \(\alpha >1\) arbitrarily close to 1 such that
\[
 \max_{\overline{D}(x_1,2\delta)}|ds_\alpha|  > C.
\]
Moreover, we can suppose that \(\max_{\overline{D}(x_1,2\delta)}|ds_\alpha| = \max_{D(x_1,\delta)}|ds_\alpha|\).
\end{assertion}


Suppose that was not the case, then there exist \(C,\delta >0\) such that \(\max_{D(x_1,2\delta)} |ds_\alpha| \leq C\) for all \(\alpha>1\) sufficiently close
to 1. Choose a radius \(R \ll \delta\) such that
\[
 \int_{D(x_1,R)} |ds_\alpha|^2 \leq \pi R^2 C^2 <\epsilon
\]
It suffices to apply Key lemma \ref{lem:key-sacks-uhlenbeck} to see that \(s_\alpha \to s\) in \(C^1(D(x_1,R/2))\), hence \(s_\alpha\) converges to \(s\) in \(C^1(M\setminus\{x_2,\dots,x_l\})\) uniformly on compacts. Moreover, since \(\{ds_\alpha\}\) converges uniformly to \(ds\) on \(\overline{D}(x_1,2\delta)\setminus
D(x_1,\delta)\), we can suppose, with \(\alpha\) sufficiently close to \(1\), that the
maximum is actually attained in \(D(x_1,\delta)\).


Therefore, we can choose a sequence \(\{C_n\}\) increasing to \(+\infty\) and \(\{\delta_n\}\)
decreasing to \(0\), such that \(C_n\delta_n\) diverges to \(+\infty\) and there exists a sequence \(\{\alpha_n\}\) decreasing to \(1\)
such that
\[
 \left| ds_{\alpha_n}(y_n)\right|:=\max_{D(x_1,\delta_n)}|ds_{\alpha_n}|=\max_{D(x_1,2\delta_n)}|ds_{\alpha_n}| = C_n
\]

We define 
\begin{align*}
  \tilde s_{\alpha_n}:\ D(\delta_n C_n) & \longrightarrow N \\
  	 		x	   	&\longmapsto 	  s_{\alpha_n}(y_n + C^{-1}_nx)
\end{align*}
then \(|d\tilde s_{\alpha_n}(0)|= \max_{D(C_n\delta_n)}|d\tilde s_{\alpha_n}| = 1\).

Fix any large \(R <+\infty\), since \(C_n\delta_n\to +\infty\), \(\tilde s_{\alpha_n}\)  is eventually defined on \(D(R)\) and is a critical point of \(E_{\alpha_n}\) with
respect to a metric \(\tilde g_n\) on \(D(R)\) converging to the Euclidean metric. The
energy \(E(\restr{\tilde s_{\alpha_n}}{D(C_n\delta_n)},\tilde g_n) = E(\restr{\tilde
s_{\alpha_n}}{D(y_n,\delta_n)},g_M)\leq B\).

We claim that Proposition \ref{prop:2-sacks-uhlenbeck} and Theorem \ref{thm:5-convergence-crit-map} remain
correct when \(M=D(R)\) and \(s_\alpha\) are critical maps of \(E_\alpha\) with
respect to metrics \(\tilde g_\alpha\) converging to the Euclidean metric. To be
precise:


\begin{assertion}
\label{prop:3-star}
Let \(\tilde s_\alpha:\ (D(R),\tilde g_\alpha) \longrightarrow N\subset \mathbb{R}^k\)
be critical maps of \(E_\alpha\) such that
\begin{itemize}
\item \(s_\alpha\) converges weakly to \(s_*\) in \(W^{1,2}(D(R), \text{Euclid})\),
\item \(E(s_\alpha) < B\)
\end{itemize}
then there exists at most \(l\) points \(\{x_1,\dots,x_l\}\) in \(\overline{D}(R)\)
and a subsequence \(\{s_\beta\}\) such that \(s_\beta\) converges to \(s_*\) in \(C^1(\overline{D}({R/2})\setminus
\{x_1,\dots, x_l\}, \mathbb{R}^k)\) uniformly on compacts, and \(s_*\) is harmonic in
\(D(R/2)\).
\end{assertion}


The two ingredients of the proof of Proposition \ref{prop:2-sacks-uhlenbeck} and
Theorem \ref{thm:5-convergence-crit-map} to be investigated are the covering and the estimate from Lemma
\ref{lem:main-est}. For the estimates, we already remarked that the parameters \(\alpha_0,\epsilon,C(p,D')\) of Lemma \ref{lem:main-est} can be chosen independent of the
metric \(\tilde g_\alpha\) if they are close to Euclidean. For the covering, the investigation is not on the constant
\(h\), which can be chosen to be
\(3^{\dim M}\), but on how small the radius of the covering balls must be, but Lemma
\ref{lem:uni-loc-finite-cover} states that their size is dictated by the Ricci curvature and
sectional curvature of \(\tilde g_\alpha\), which are also uniformly bounded.


Using Assertion \ref{prop:3-star}, passing to a subsequence of \(\{\tilde s_{\alpha_n}\}\)
if necessary, we can suppose that \(\tilde s_{\alpha_n} \to s_*\) in \(C^1(D(R),
\mathbb{R}^k)\). Note that there is no singular point where \(\{\tilde s_{\alpha_n}\}\) fails to converge because \(|d\tilde s_{\alpha_n}|\) is bounded uniformly on \(D(R)\) (hence cannot explode as in Assertion \ref{assert:3-star}). We can also choose, by a
diagonal argument, a
subsequence of \(\{\tilde s_{\alpha_n}\}\) that converges to \(s_*\) in \(C^1(\mathbb{R}^2)\) uniformly on compacts.

It is clear that \(s_*: \mathbb{R}^2 \longrightarrow N\) is harmonic and nontrivial
because
\[
 \left| ds_*(0)\right|_{\text{Euclid}} = \lim_{\alpha_{n}\to 1}|d\tilde s_{\alpha_n}(0)|_{\tilde
g_{\alpha_n}} = 1.
\]
Also,
\[
 \int_{D(R)}|ds_*|^2 dE = \lim_{\alpha_n \to 1}\int_{D(R)} |d\tilde
s_{\alpha_n}|^2dV_{\tilde g_\alpha} \leq \limsup_{\alpha\to 1}
2E(\restr{s_{\alpha}}{D(x_1,2\delta_n)}) < 2B
\]
which means the energy of \(s_*\) on \(\mathbb{R}^2\) is bounded above by \(2B\).

Now since \((\mathbb{R}^2, \text{Euclid})\) is conformal to \(\mathbb{S}^2\setminus\{p\}\), \(s_*\) can be seen as a harmonic map on \(\mathbb{S}^2\setminus\{p\}\) with the
same (finite) energy. By Extension theorem \ref{thm:extension-sacks-uhlenbeck}, \(s_*\)
extends to a nontrivial harmonic map from \(\mathbb{S}^2\) to \(N\).
\end{proof}


\begin{remark}
\label{rem:final-sacks-uhlenbeck}
\begin{enumerate}
\item We can have a better estimate of \(E(s_*)\). For any \(R>0\), one has
\[
    E(\restr{s_*}{D(R)}) + E(\restr{s}{M\setminus \cup_{i=1}^l D(x_i,\delta_n)}) \leq
    \limsup_{\alpha_n\to 1} \left[E(\restr{s_{\alpha_n}}{D(x_1,\delta_n)}) + E(\restr{s_{\alpha_n}}{M\setminus\cup_{i=1}^l D(x_i,\delta_n)})\right]
   \]
 Let \(\delta \to 0\) then \(R\to +\infty\), one has
\[
   E(s_*) + E(s) \leq \limsup_{\alpha\to 1} E(s_\alpha).
   \]
\item The proof of Theorem \ref{thm:6-final-sacks-uhlenbeck} also gives a constraint on the
image of \(s_*\): since \(s_*(D(R))\subset
   \overline{\cup_{1<\beta<\alpha}s_\beta(D(x_1,2\delta))}\) for all \(\alpha\)
arbitrarily close to \(1\) and \(\delta\) arbitrarily small, one has
\[
   s_*(\mathbb{S}^2)\subset \bigcap_{\delta \to 0}\bigcap_{\alpha \to 1}
   \overline{\bigcup_{1<\beta<\alpha}s_\beta(D(x_1, \delta))}
   \]
\end{enumerate}
\end{remark}



\section{Minimal immersions of \(\mathbb{S}^2\).}
\label{sec:org1da06c8}

We use the following result:

\begin{theorem}[\cite{chern_volume_1975}, \cite{gulliver_theory_1973}, \cite{eells_harmonic_1964}]
If \(s: \mathbb{S}^2 \longrightarrow N\) is a nontrivial harmonic map and \(\dim N\geq
3\), then \(s\) is a \(C^\infty\) conformal, branched, minimal immersion.
\end{theorem}

The "minimal" part follows from \cite{eells_harmonic_1964}, the "branched" part
follows from \cite{gulliver_theory_1973} and the "conformal" part follows from
\cite{chern_volume_1975} and the fact that there is no nontrivial holomorphic quadratic
differential on \(\mathbb{S}^2\). Theorem \ref{thm:6-final-sacks-uhlenbeck} gives:

\begin{theorem}
If the universal covering \(\tilde N\) of \(N\) is not contractible then there exists
a \(C^\infty\) conformal, branched, minimal immersion \(s: \mathbb{S}^2 \longrightarrow N\).
\end{theorem}


\iffalse
\bibliographystyle{alpha}
\bibliography{../res/Stage2018}
\fi
