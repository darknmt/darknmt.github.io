Recall that in the Euclidean space \(\mathbb{R}^n\), one obtains a representation of the
solution \(u\) of equation \(\Delta u = f\) by
\begin{itemize}
\item first solving for an explicit radial solution of \(\Delta G = \delta_0\). In
particular, \(G = [(n-2)\omega_{n-1}]^{-1}r^{2-n}\) if \(n>2\) and \(G=
   -(2\pi)^{-1}\log(r)\) if \(n=2\)
\item then tensoring \(G\) by \(f\), one has the solution \(u=G* f\) of \(\Delta u = f\)
\end{itemize}

To generalise this argument for Riemannian manifolds, there are a few points that have to be
modified:
\begin{enumerate}
\item Since it does not make sense to add/substract points of a manifold, one will need to
find different fundamental solutions for different points, so instead of fundamental
solution, we will find the Green's function \(G=G(p,q)  (p,q\in M)\). The convolution will be replace by the following operation on functions \(X, Y\) defined on \((M\times M)\setminus \Delta_M\) where \(\Delta_M\) denotes the diagonal:
\[
    (X * Y) (p,q)= \int_M X(p,r) Y(r,q) dV(r)
   \].
\item The distance function \(q \mapsto d(p,q)\) is only smooth near \(p\), outside of
the cut-locus, the best one can says is that the function is Lipschitz. Since cut-loci
are almost impossible to calculate or visualise (the cut-locus of an ellipsoid is still
a conjecture, according to \cite{berger_panoramic_2003}), one will cut-off the Euclidean
solution, try to solve the equation near \(p\) and later add a correcting term. This
inspires the definition of parametrix.
\item Another reason that we have to approximate the exact solution by parametrix, that also
explain the iteration in Theorem \ref{thm:existence-green}, is that the expression of
Laplacian, even in the geodesic polar coordinate and even near the origin, involves the
metric, hence the Euclidean fundamental solution is not yet a solution even near the
origin.
\end{enumerate}

\begin{remark}
\label{rem:green-riem-surf}
To give a simplified analogy of what we will be doing, let us prove the existence of
"Green's function" on Riemann surfaces (with boundary, so that we do not have to deal with
the volume). The "Laplace equation" is
\begin{equation}
\label{eq:lap-complex}
-2i \partial \bar \partial g = \delta_0
\end{equation}
where the LHS is a 2-form and the RHS is a generalised 2-form in the sense of
current. Contrary to the previous point 3, one knows the exact local solution of
\eqref{eq:lap-complex}, namely \(z\mapsto -(2\pi)^{-1}\log(|z|)\). Therefore, the argument
will be simplified as:
\begin{itemize}
\item Given a holomorphic chart of a point \(0\in M\), pose \(h(z):= -(2\pi)^{-1}\log(|z|)\chi(|z|)\) where \(\chi\) is a cut-off function that
is \(1\) on a neighborhood of \(0\)
\item The 2-form \(\alpha=-2i \partial \bar \partial h\) is well-defined everywhere except \(0\), and vanishes on a neighborhood of \(0\). Denote by \(\alpha^{\rm naiv}\) its
extension to \(M\).
\item Recall the fact that every smooth 2-form on a compact, connected, Riemann surface with
boundary can be writen as \(\alpha^{\rm naiv} = -2i \partial \bar \partial \varphi\),
pose \(g = h-\varphi\).
\end{itemize}

For Riemann surface without boundary, the equation is
\(-2i \partial \bar \partial g = \delta_0 -2i\int_M \partial \bar \partial g\) and the
fact to evoke is that any smooth 2-form \(\alpha\) with \(\int_M \alpha = 0\) is of
form \(\alpha = -2i \partial \bar \partial \varphi\) 
\end{remark}

We will suppose that \(M^n\) is a Riemannian manifold with injectivity radius \(\delta_0
>0\), and of bounded curvature. Compact manifolds, for example, fall in this category.

\section{Parametrix and the Green's formula}
\label{sec:org49c865e}

\begin{definition}
A \textbf{Green's function} \(G(p,q)\) of a compact Riemannian manifold is a function defined on
\((M\times M)\setminus \Delta_M\) such that
\begin{enumerate}
\item \(\Delta_q^{\rm dist}G(p,q) = \delta_p(q)\) if \(M\) has boundary.
\item \(\Delta_q^{\rm dist}G(p,q) = \delta_p(q) - V^{-1}\)
\end{enumerate}
where \(\Delta_q^{\rm dist}\) concerns the distribution derivatives and \(V\) is the
volume of \(M\).

Let \(p,q\in M\) be distinct points, the \textbf{parametrix} \(H\) is defined by
\[
 H(p,q) = \begin{cases}
 [(n-2)\omega_{n-1}]^{-1} r^{2-n}\chi(r),  & \text{if $n>2$} \\
 -(2\pi)^{-1} \chi(r)\log r, & \text{if $n=2$}
\end{cases}
\]
where \(r=d(p,q)\), \(\chi: \mathbb{R}_{\geq 0} \longrightarrow \mathbb{R}\) is smooth, \(\chi=1\)
in a neighborhood of \(0\) and \(\chi(t)=0\) if \(t >\delta_0\).
\end{definition}


Recall that in the geodesic polar coordinates, i.e. the polar coordinates on the tangent \(T_pM\) at \(p\in  M\), identified with a neighborhood of \(p\in M\), the metric \(g\) is given by
\[
g: ds^2 = dr^2 + r^2 g_{\theta_i\theta_j}(r,\theta) d\theta^i d\theta^j
\]
and one denotes \(| g_\theta |:= \det (g_{\theta_i \theta_j})\), therefore
\(|g|=\det(g_{ij}) = r^{2(n-1)}|g_\theta|\)

\begin{lemma}
\label{lem:lap-radial}
If a function \(\varphi \in C^2\) defined locally around \(p\in M\) and \(\varphi\)
is radial, i.e. \(\varphi = f(r)\) in a small geodesic ball \(B(p, \delta)\) then
\[
-\Delta \varphi = f'' + \frac{n-1}{r} f' + f' \partial_r \log \sqrt{|g_\theta|}
\]
\end{lemma}
\begin{proof}
One has
\begin{equation*}
\begin{split}
\Delta \varphi &= - \tr\left(\nabla_i(g^{kj} \partial_j \varphi\ e_k)\right)_{i,k} = -\partial_i (g^{ij} \partial_j\varphi) - g^{kj} \partial_j \varphi \Gamma^i_{ik}\\
       	       &= -|g|^{-1/2} \partial_i (g^{ij} |g|^{1/2} \partial_j \varphi)
\end{split}   
\end{equation*}
since \(\Gamma^{i}_{ik} = \partial_k \log \sqrt{|g|} = \frac{\partial_i |g|}{2|g|}\). One concludes by substituting \(|g| = r^{2n-2}|g_\theta|\) and noticing that \(g^{r\theta_i} = g^{\theta_i\theta_j} = 0\ (i\ne j)\).
\end{proof}

\begin{remark}
\label{rem:lap-rad}
\begin{enumerate}
\item The Laplacian of the metric \(g\), viewed in polar geodesic coordinates centered at \(p\), i.e. in the
tangent space \(T_pM\) is not the Euclidean Laplacian of \(T_pM\), however the
difference if \(O(r)\) since \(\partial_r \log\sqrt{|g_\theta|} \leq Ar\) where the bound
\(A\) is given by Ricci curvature, see the \href{sobolev-riemannian.org}{Volume comparison theorem}.
\item Applied the formula for \(q \mapsto H(p,q)\), one has
\begin{equation}
\label{eq:lap-Hpq}
 \Delta_q^{\rm naiv} H(p,q) = [(n-2)\omega_{n-1}]^{-1}r^{1-n}\left( (n-3)\chi' - r \chi'' + ((n-2)\chi - r\chi') \partial_r \log\sqrt{|g_\theta|} \right)
 \end{equation}
therefore \(\Delta_q^{\rm naiv} H(p,q) \leq B r^{2-n}\) where \(B\) does not depend on \(p\).
\item Unlike the case of Remark \ref{rem:green-riem-surf} where we know the exact fundamental
solution and the form \(\alpha^{\rm naiv}\) has no singularity, there is no reason
for that this holds true for \(\Delta_q^{\rm naiv} H(p,q)\). However, we proved that
the order of singularity at \(q=p\) can be controlled.
\end{enumerate}
\end{remark}


\begin{proposition}[Green's formula]
\label{prop:green-formula}
For any function \(\psi\in C^2(M)\), one has
\begin{equation}
\label{eq:green-formula}
 \psi(p) = \int_M H(p,q)\Delta\psi(q) dV(q) - \int_M \Delta_q^{\rm naiv} H(p,q) \psi(q) dV(q)
\end{equation}
where \(\Delta_q^{\rm naiv} H(p,q)\) denotes the pointwise derivative of \(H(p,q)\), \uline{not}
the distribution derivative.
\end{proposition}

\begin{remark}
\label{rem:transposition}
\begin{enumerate}
\item In other words, the theorem says that \(\Delta_q^{\rm dist} H(p,q) = \Delta_q^{\rm naiv}H(p,q) + \delta_p(q)\) where \(\Delta_q^{\rm dist}\) is the distribution
derivative. In particular, if there is no concern about regularity of the distance
function \(d(p,q)\) (as in the Euclidean case), allowing us to take the cut-off
function \(\chi=1\) in the definition of parametrix, then
\(\Delta_q^{\rm naiv}H(p,q) = 0\) and \(\Delta_q^{\rm dist}H(p,q) = \delta_p(q)\)
which is not a surprise since \(H(p,q)\) is also the Green's function.
\item Taking \(\psi=1\), one has
\[
    \int_M \Delta_q^{\rm naiv}H(p,q) = -1
   \]
\item Multiplying \eqref{eq:green-formula} by \(\phi(p)\) and integrate over \(M\), one has
\[
   \int_M \phi(q) \psi(q) dV(q) = \int_M \left(\int_M H(p,q) \phi(p) dV(p)\right)\Delta\psi(q)
   dV(q) - \int_M\left(\int_M \Delta_q^{\rm naiv} H(p,q)\phi(p)\right) \psi(q) dV(q)
   \]
hence in distribution sense
\begin{equation}
\label{eq:transposition-parametrix}
\phi(q) = \Delta_q \int_M H(p,q)\phi(p) dV(p) - \int_M \Delta_q^{\rm naiv} H(p,q)\phi(p)
\end{equation}
The equation \eqref{eq:transposition-parametrix} is called the \uline{transposition} of
equation \eqref{eq:green-formula} and what we have just done is a rigourous proof of the
following heuristic justification of \eqref{eq:transposition-parametrix}: "Take the
derivative \(\Delta_q\) inside the integral, then use \(\int_M \delta_p(q)\phi(p)
   dV(p) \text{ "=" } \phi(q)\)".
\end{enumerate}
\end{remark}



\begin{proof}
The intuition is clear:
\begin{itemize}
\item since one only modifies the fundamental solution at points \(q\) far from \(p\), one
only needs to recompense by \(\Delta_q^{\rm naiv}H(p,q)\)
\item there may be trouble near \(p\) caused by the difference between the Euclidean Laplacian
and the metric Laplacian, however as explained by Remark \ref{rem:lap-rad}, this difference is
\(O(r)\) as \(r\to 0\).
\end{itemize}
For a rigorous proof, one calculates \(\int_M H(p,q)\Delta \psi(q) dV(q)\) by
decomposing \(M\) to \(B(p,\epsilon)\) and \(M\setminus B(p,\epsilon)\) with \(0<\epsilon<\delta_0\) tending to \(0\) eventually, then
\begin{equation*}
\begin{split}
\int_{M\setminus B(p,\epsilon)} H(p,q)\Delta\psi(q)dV(q) &= \int_{M\setminus B(p,\epsilon)} \left(\Delta_q^{\rm naiv}H(p,q)\psi(q) + d(\psi\wedge *dH - H\wedge *d\psi)\right) dV(q)\\
		 &= \int_{M\setminus B(p,\epsilon)} \Delta_q^{\rm naiv}H(p,q)\psi(q) dV(q) +\int_{\partial B(p,\epsilon)} (\psi\wedge *dH - H\wedge *d\psi) dV(q)
\end{split}   
\end{equation*}
by Stokes' theorem, where \(*\) denotes the Hodge star. Therefore
\[ \int_M H(p,q)\Delta\psi(q) dV(q) = \int_M\Delta_q^{\rm naiv} H(p,q) \psi(q) dV(q) + I_1 +
I_2\]
where \(I_1 = \lim_{\epsilon\to 0} \int_{\partial B(p,\epsilon)}(\psi\wedge *dH
-H\wedge * d\psi)\) and \(I_2 = \lim_{\epsilon\to 0} \int_{B(p,\epsilon)}
H(p,q)\Delta\psi(q) dV(q)\).

Now \(I_2 = \psi(p)\) since \((\frac{\sin (b\epsilon)}{b\epsilon})^{n-1} \leq dV/dE \leq
(\frac{\sin (\alpha \epsilon)}{\alpha\epsilon})^{n-1}\) in \(B(p,\epsilon)\) by \href{sobolev-riemannian.org}{Volume comparison theorem}
where \(b^2\) is an upper bound of sectional curvature and \((n-1)\alpha^2\) is a lower
bound of Ricci curvature (\(\alpha\in \mathbb{C}\)), and since \(\Delta\psi(q)
-\Delta_E\psi(q) = O(\varepsilon)\) in \(B(p,\epsilon)\) where \(\Delta_E\) is the
Euclidean Laplacian.

For \(I_1\), with \(\epsilon\) small enough such that \(\chi=1\), one has
\(|H\wedge * d\psi| \leq \const \epsilon^{2-n} (* d\psi)\). By straightforward
computation:
\begin{equation*}
\begin{split}
dH &= -\omega_{n-1}^{-1} r^{1-n}dr,\quad dV = r^{n-1}\sqrt{|g_\theta|} dr\wedge d\theta^1\wedge\dots\wedge d\theta^{n-1}\\
*dH &= -\omega_{n-1}r^{2n-2}\sqrt{|g_\theta|} d\theta^1\wedge\dots\wedge d\theta^{n-1}
\end{split}   
\end{equation*}
hence \(\int_{\partial B(p,\epsilon)} H\wedge * d\psi = O(\epsilon)\) and \(\int_{\partial B(p,\epsilon)} \psi\wedge *dH = O(\epsilon^{2n-2})\). Therefore \(I_1=0\) and the conclusion follows.
\end{proof}

\section{Existence of Green's function on compact Riemannian manifolds}
\label{sec:orgf5e83c1}

Our goal is to prove the following theorem
\begin{theorem}[Existence of Green's function]
\label{thm:existence-green}
Let \(M^n\) be a compact Riemannian manifold without boundary, there exists a Green's
function \(G(p,q)\) of the Laplacian such that
\begin{enumerate}
\item \emph{Green's function}. For all \(\varphi\in C^2(M)\), 
\begin{equation}
\label{eq:green-int}
   \varphi(p) = V^{-1}\int_M\varphi(q)dV(q) + \int_M G(p,q)\Delta\varphi(q) dV(q)   
\end{equation}
\item \emph{Smooth}. \(G\in C^\infty((M\times M)\setminus\Delta_M)\).
\item \emph{Radial estimates}. There exists a constant \(k\) such that
\begin{equation}
\label{eq:G-bound}
 \left| G(p,q)\right| \leq \begin{cases}
 k(1 +|\log r|)	 ,  & \text{if $n=2$} \\
 kr^{2-n}	 , & \text{if $n>2$}
 \end{cases}   
\end{equation}
for \(r=d(p,q)\). Moreover, one has the derivative estimates:
\begin{equation}
\label{eq:derivative-bound} 
\left|\nabla_q G(p,q) \right| \leq kr^{1-n},\qquad \left|\nabla^2_q G(p,q) \right| \leq kr^{-n},
\end{equation}
\item \(G\) is bounded below. Since \(G\) is defined upto a constant, one can choose the constant so that \(G>0\).
\item \emph{Constant integral}. The integral \(\int_M G(p,q) dV(p)\) is constant in \(q\). Since \(G\) is defined upto a constant, one can choose the constant so that \(\int_M G(p,q) dV(p)=0\).
\item \emph{Symmetric}. \(G(p,q) = G(q,p)\) for \(p\ne q\) in \(M\).
\end{enumerate}
\end{theorem}

For a better notation, let us replace \(\Delta_p U(p,q)\) by \(\Delta_2 U(p,q)\). Recall that we already know how to solve the equation \(\Delta u = f\) for \(f\in
L^2(M)\), this means we can solve \(\Delta_2 U(p,q) = f_p(q)\) for double-integrable functions
\(f_p\), or briefly we can \emph{solve} \(L^2\) functions. Now, define
\[
 (X* Y)(p,q):= \int_M X(p,r) Y(r,q) dV(r)
\]
if the integration is possible and if it commutes with derivation, one has
\[
 \Delta_2(F_1 * H) = F_1 *\Delta_2^{\rm dist}H = F_1 + F_1 *\Delta_2^{\rm naiv} H
\]
So if one can \emph{solve} \(F_1*\Delta^{\rm naiv}H\), then one can \emph{solve} \(F_1\),
i.e. if \(\Delta_2 E_2 = F_1 * \Delta_2^{\rm naiv}H\) then take \(E_1:= F_1*H- E_2\),
one has \(\Delta_2 E_1 = F_1\).

Now in order to prove that \(\delta_\Delta - V^{-1}\) can be solved, it remains to check that
\begin{equation}
\label{eq:convo-k}
 \delta_\Delta * (\Delta_2^{\rm naiv} H)^{*k} \in L^2(M)\quad \text{for \( k \gg 1 \)}.
\end{equation}
This is the content of the following lemma.

\begin{lemma}
\label{lem:reg-conv}
Let \(\Omega\subset \mathbb{R}^n\) be a bounded open set,
\(X,Y:\ (\Omega\times\Omega)\setminus\Delta_\Omega \longrightarrow \mathbb{R}\) be
continuous functions such that
\[ 
\left| X(p,q) \right| \leq \const d(p,q)^{\alpha-n},\qquad \left| Y(p,q) \right|\leq
\const d(p,q)^{\beta-n}, \qquad  \alpha,\beta \in (0,n) 
\]
then
\[
 Z(p,q):=\int_\Omega X(p,r)Y(r,q) dV(r)
\]
is continuous in \((\Omega\times\Omega)\setminus\Delta_\Omega\) and
\[
\left| Z(p,q) \right| \leq \begin{cases}
\const d(p,q)^{\alpha +\beta -n}   ,  & \text{if \(\alpha +\beta < n\)} \\
\const (1 + |\log d(p,q) |)	   , & \text{if \(\alpha +\beta = n\)} \\
\const	   , & \text{if \(\alpha +\beta > n\)}
\end{cases}
\]
In the case \(\alpha +\beta >n\), \(Z\) admits a continuous extension to \(\Omega\times\Omega\).
The result also holds for compact Riemannian manifolds.
\end{lemma}

\begin{proof}
It suffices to consider \(p,q\) closed to each other. Let \(d(p,q) = 2\rho\). Decompose \(\Omega = (\Omega\cap B(p, \rho))\cup (\omega \setminus B(q, 3\rho)) \cup
\Omega\cap (B(q, 3\rho)\setminus B(p,\rho))\), then
\begin{equation*}
\begin{split}
\left| \int_{\Omega\cap B(p,\rho)}X(p,r)Y(r,q) dV(r) \right| &\leq C\rho^{\alpha + \beta -n}\\
\left| \int_{\Omega\cap B(q,3\rho)\setminus B(p, \rho)} X(p,r)Y(r,q) dV(r) \right| &\leq C\rho^{\alpha + \beta -n}\\
\left| \int_{\Omega\setminus B(q,3\rho)}X(p,r)Y(r,q) dV(r) \right| &\leq C\int_\rho^D \frac{dr}{r^{n-\alpha -\beta - 1}}\\
\end{split}   
\end{equation*}
where \(D\) is the diameter of \(\Omega\). For compact Riemannian manifold, take \(\rho \ll \delta_0\), the injectivity radius and use Comparison theorem, one has the same estimates.
\end{proof}

Back to the proof of Theorem \ref{thm:existence-green}, one can see that it suffices to
choose \(k > \frac{n}{2}\) in \eqref{eq:convo-k}. The rigorous proof is given below.

\begin{proof}[Proof of Theorem ref:thm:existence-green]
Carefully do the algebraic part of the above argument, one poses
\[
 G(p,q) = H(p,q) + \sum_{i=1}^{k-1} (-\Delta_2^{\rm naiv}H)^{*i} * H + F_k(p,q)
\]
where \(F_k(p,q)\) satisfies
\[
 \Delta_2 F_k(p,q) = (-\Delta_2^{\rm naiv}H)^{*k} - V^{-1}
\]
This is possible if one chooses \(k>n/2\) since by repeated application of Lemma
\ref{lem:reg-conv}, \((-\Delta_2^{\rm naiv}H)^{*k}\) is continuous. By regularity result of
equation \(\Delta u = f\), the function \(q \mapsto F_k(p,q)\) is in
\(C^2(M\setminus \{ p\})\). Each function \(F_k(p,\cdot)\) is uniquely defined up to a
constant, choose the constant such that \(\int_M G(p,q) dV(q)=0\), then the function \(p \mapsto \int_M F_k(p,q) dV(q)\) is continuous. The condition 1) of the Theorem can be
verified without difficulty.
Moreover, since \(\Delta_2 G(p,q)=0\) if \(q\ne p\), the function \(q \mapsto G(p,q)\) is \(C^\infty\).

We will prove such \(G(p,q)\) satisfies the statements 2-6, starting from a weaker form 2-)
of 2), that is we will prove that \(p\mapsto G(p,q)\) is continuous, then using this, we
will prove 3-6, and eventually come back to prove 2 completely.

For 2-) we will use the following fact:

\textbf{Fact.} If \(\Delta u = f\) and \(f\in C^0(M)\) (hence \(u\in C^2(M)\) and
\(\int_M u = 0\), then one has \(\sup |u| \leq C \sup |f|\) where \(C>0\) is a constant.

Denote \(\Gamma_i:= (-\Delta_2^{\rm naiv} H)^{*i}\) and apply the result for
\(u= F(p,\cdot) - V^{-1}\int_M F(p,q) dV(q)\) and \(f=\Gamma_k(p,\cdot)\), one has
\[
 \sup \left|F(p,\cdot) - F(r,\cdot) - V^{-1}\int_M \left(F(p,\cdot)-F(r,\cdot)\right)\right| \leq C \sup_q |\Gamma_k(p,q)-\Gamma_k(r,q)|
\]
Then the continuity of \(p\mapsto F(p,\cdot)\) under \(C^0\) topology is given by
\begin{itemize}
\item \(p\mapsto \int_M F(p,\cdot)\) is continuous by the previous choice of constant.
\item The uniform continuity of \(\Gamma_k\) on \(M\times M\), which is the result of its
continuity and the compactness of \(M\times M\).
\end{itemize}
Hence \(p\mapsto G(p,q)\) is continuous on \(M\setminus \{ q\}\) for all \(q\in M\).

For 3), fix \(p\in M\) and let \(r=d(p,q)\) be small, then \(H(p,q) = O(r^{2-n}\),
\((\Gamma_i*H)(p,q) = O(r^{2i+2-n})\) by Lemma \ref{lem:reg-conv} and \(F(p,q)=O(1)\) if
\(n>2\). Hence \(G(p,q)=O(r^{2-n})\), where here the constant in \(O(r^{2-n})\), if
checked carefully, does not depend on \(p\). The case \(n=2\) can be treated
similarly.
For the derivative estimates, note that \(\nabla_q G(p,q) = \nabla_q H(p,q) +
\sum_{i=1}^{k-1}(\Gamma_i * \nabla_2 H)(p,q) + \nabla_q F(p,q)\) and \(\nabla_q^2 G(p,q) = \nabla_q^2 H(p,q) +
\sum_{i=1}^{k-1}(\Gamma_i * \nabla_2^2 H)(p,q) + \nabla_q^2 F(p,q)\)
where the commutative of derivation and integration can be justified by Lebesgue's
Dominated convergence. In both case, the dominant terms as \(q\to p\) are \(\nabla_q
H(p,q)\) and \(\nabla^2_q H(p,q)\) respectively, which is \(O(r^{1-n})\) and \(O(r^{-n})\) where the constants in big-O do not depend on \(p\).

For 4), note that \(H(p,q)\) is the dominant term of \(G(p,q)\) as \(q\to p\) and \(H(p,q)>0\), one see that \(G(p,q) >0\) in a neighborhood of \(\Delta_M\). By the
compactness of \(M\) and the continuity of \(G\) outside of \(\Delta_M\), one sees
that \(G\) is bounded below.

To prove 5), take to transposition of \eqref{eq:green-int}, i.e. multiply by \(\psi(p)\) and
integrate, as in Remark \ref{rem:transposition}, one obtains
\begin{equation}
\label{eq:transp-green-int}
\Delta_q\int_M G(p,q)\psi(p) dV(p) = \psi(q) -V^{-1}\int_M \psi(p) dV(p)
\end{equation}
Substitute \(\psi=1\), one sees that \(q\mapsto \int_M G(p,q) dV(p)\) is harmonic on
\(M\), hence is constant by compactness of \(M\).

We will now prove 6). It follows from \eqref{eq:green-int} that
\begin{equation}
\label{eq:3}
 \Delta_q\int_M G(p,q)\psi(q) dV(q) = \Delta_q \psi(q)
\end{equation}
Also, from the transposition \eqref{eq:transp-green-int}, replace \(\psi\) by \(\Delta\psi\), one has
\[
 \Delta_q\int_M G(p,q)\Delta\psi(p) dV(p) = \Delta_q \psi(q)
\]
Swap \(p\) and \(q\) and subtract to \eqref{eq:3}, one has
\[
 \Delta_p\int_M\left(G(p,q) - G(q,p) \right)\Delta\psi(q) dV(q) = 0
\]
Hence \(\int_M\left(G(p,q) - G(q,p) \right)\Delta\psi(q) = C \const\). Integrate by \(p\in M\) and use the fact that we chose \(\int_M G(q,p) dV(p) = 0\), one has \(C=0\),
meaning that \(\Delta_q \left(G(p,q) - G(q,p) \right) = C(p)\), being independent of \(q\). By swapping \(p,q\), one has \(C(p) = -C(q)\) for all \(p\ne q\). Since \(M\) contains more than 3 points, these constants are 0. Hence \(G(p,q) = G(q,p)\).

Now coming back to 2), since \(G(p,q) = G(q,p)\), we see that \(p\mapsto G(p,q)\) is
\(C^\infty\) for all \(q\in M\). It remains to prove that \(p\mapsto \nabla_q^h
G(p,q)\) is continuous on \(M\setminus \{q\}\), then Schwarz's lemma applies. For that,
one may try the following argument:
\[
 \Delta_p \nabla_q^hG(p,q) = \nabla_q^h\Delta_pG(p,q) = 0,\quad p\in M\setminus[q\}
\]
hence \(p\mapsto \nabla_q^h G(p,q)\) is \(C^\infty\). It is however difficult to
justify the commutativity of derivations, which is equivalent to
\begin{equation}
\label{eq:4}
\int_M \nabla_q^h G(p,q)\Delta \varphi(p) dV(p) = \nabla_q^h\int_M G(p,q) \Delta\varphi(p) dV(p),
\end{equation}
that is the ability to derive in the integral sign. A justification for this can be done
in the case \(h\leq 2\) using estimates of 3). 

A simpler way is to note that it suffices to prove the continuity of \(p\mapsto
\nabla_q^hG(p,q)\)  for \(p\) in a small open set \(V\) with \(\bar V\) not containing \(q\). Then
claim that \(\Delta_p \nabla_q^h G(p,q) = \nabla_q\Delta_p G(p,q)=0\) as distributions
on \(V\), which is equivalent to \eqref{eq:4} for all test functions \(\varphi\) with \(\supp\varphi\in V\). Then Dominated convergence applies since
\(|\nabla_q^{h+1}G(p,q)|\leq Cd(q,\bar V)^{1-n-h}\) hence is bounded.
\end{proof}
