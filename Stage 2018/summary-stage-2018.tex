
The goal of this part is to give a summary of what will be developed in the next chapters. In brief, we are interested in
maps \(f: M \longrightarrow M'\) between Riemannian manifolds (that to simplify, are
supposed to be compact) that are critical points of the energy functional
\[
 E(f) = \frac{1}{2}\int_M |\nabla f|^2 dV
\]
that is, by taking first order variation of \(E\), those whose \textbf{tension field} \(\tau(f)\) vanish. 

We wish to prove that any smooth map \(f_0: M \longrightarrow M'\) can be deform to a
harmonic map using the gradient descent equation, that is to show the equation
\begin{equation}
\label{eq:intro:1}
\begin{cases}
\frac{df_t}{dt} = \tau(f_t)\\
\restr{f}{t=0} = f_0
\end{cases}
\end{equation}
We prove, in the rest of the memoire, that if \(M'\) is negatively curved then this PDE admits a globally defined smooth
solution \(f_t\) and that \(f_{\infty}:=\lim_{t\to \infty} f_t\) in \(C^\infty\) is
a harmonic map.

The resolution of \eqref{eq:intro:1} can be organised in 3 steps:
\begin{enumerate}
\item Find the global equation. We will find a global frame of \(M'\) and express
\(f\) in this frame, so that in stead of solving for a map, we will have to solve for functions.
\item Study linear PDEs on manifolds. The equation, expressed in local coordinates is a nonlinear heat equation, i.e. other
than a heat operator, it has a nonlinear differential operator of strictly lower
degree. Short-time existence and regularity for \eqref{eq:intro:1} follows from \emph{standard} results of
parabolic equation.
\item Prove long-time existence. This follows from several energy estimate.
\end{enumerate}

Local form of \eqref{eq:intro:1} can be found using calculus on Riemannian-connected vector
bundle. The relevant vector bundle here is \(f^* TM'\) over \(M\) in case of a single
map \(f: M \longrightarrow M'\), or \(F^* TM'\) over \(M\times[\alpha,\omega]\) for
a deformation \(F_t:\ M\times [\alpha,\omega] \longrightarrow M'\).

\section{Global equation.}
\label{sec:orgbbc11fd}

We will explain here how the step 1 is done. We will embed \(M'\) in a Euclidean space
\(\mathbb{R}^N\), not necessarily isometric because we will not use the Euclidean metric
on \(R^N\) anyway. We will equip a tubular neighborhood \(T\) of \(M'\), which is
diffeomorphic to \(M'\times D\) where \(D\) is an open disc of dimension equal the
codimension of \(M'\), with the product of the metric of \(M'\) and the Euclidean
metric of \(D\). The global equation of \eqref{eq:intro:1}, as we will prove, is the flow
along the tension field of \(T\), i.e. 
\begin{equation}
\label{eq:intro:2}
\frac{df_t}{dt} = \tau_T(f_t). 
\end{equation}

What we want in a global equation is that locally
it has to be the same as \(\frac{df_t}{dt} = \tau_{M'}(f_t)\) and globally the image of \(f_t\) has to remain in \(M'\). So in fact, if there is a global equation of \eqref{eq:intro:1}, then it has to be \(\frac{df_t}{dt} = \tau_T(f_t)\) because of the following fact:

\textbf{Fact.} If the inclusion \(M' \longrightarrow T\) is totally geodesic and \(f: M
\longrightarrow M'\) be a smooth map. Then the tension field of \(f\) in \(M'\) is
actually the tension field of \(f\) as a map to \(T\).

It is, however just a necessary condition. To complete the argument, one needs to justify
that the \(\tau\)-flow in \(T\) always remain in \(M'\). The following idea is due
to Hamilton \cite{hamilton_harmonic_1975}. The advantage, in comparison with Eells and
Sampson \cite{eells_harmonic_1964} is its clarity in idea. The disadvantage, is that one needs to
establish uniqueness of solution first. Let \(\iota\) be the reflection in \(T\)
around \(M'\) then if \(f_t\) is a solution of \eqref{eq:intro:2} then \(\iota f_t\)
is also a solution with the same initial condition, since \(\iota\) preserves \(M'\). Then by uniqueness of solution, \(f_t = \iota f_t\) for all relevant time, meaning
that \(f_t\) remains in \(M'\) since \(\iota\) only preserves \(M'\).

\section{Linear PDEs on manifolds.}
\label{sec:orga7f5a8f}

I passed more than half of my stage learning how to solve linear equations on manifold. I
started with \cite{aubin_nonlinear_1998} and \cite{jost_riemannian_2008} as reference, where Sobolev spaces are defined by
density and Laplace equation, heat equation are solved using parametrix. This approach has advantage of
being quick, intuitive and clear in ideas. The disadvantage is that, while parametrix
works perfectly for smooth functions, we have to remain fuzzy on function spaces while
solving parabolic equation. I later discovered \cite{hamilton_harmonic_1975} where a
thorough treatment of Sobolev spaces on manifold was an important part of the paper. While
the conceptual points behind are clear, the limit of this approach is that it only works
for compact manifolds, unlike \cite{aubin_nonlinear_1998} and \cite{jost_riemannian_2008},
whose manifolds are supposed to be complete, having strictly positive injectivity radius
and bounded curvatures. This disadvantage cannot be remedied because for non-compact
manifolds, Sobolev spaces, set-theoretically and as Banach spaces, depend on the metric. I
choose to present here the second approach.

\subsection{Sobolev spaces}
\label{sec:orge6386bd}
Using a partition of unity \(\{\psi_i\}\) subordinated to a finite atlas of a compact manifold \(M\), the Sobolev
space \(W^{k,p}(M)\) is defined as the preimage of \(\bigoplus_i W^{k,p}(\mathbb{R}^n)\) of:

\begin{align}
\label{eq:intro:2.5}
  \iota:\ \mathcal{S}^*(M) &\longrightarrow \bigoplus_i \mathcal{S}^*(\mathbb{R}^n) \\
  	  f &\longmapsto \oplus_i \psi_i f \nonumber
\end{align}

meaning that we a natural inclusion
\begin{equation}
\label{eq:intro:3}
\iota: \mathcal{S}^*(M)\supset W^{k,p}(M) \hookrightarrow  \bigoplus_i W^{k,p}(\mathbb{R}^n)
\end{equation}

The definition of \(W^{k,p}(M)\) as a subspace of a direct sum renders the task of
generalising operators on \(W^{k,p}(M)\) indirect. 

For example, to define a differential
operator \(A\) of order \(r\) on \(W^{k,p}(M)\), we have to define it
component-wise, that is \(Af:= \oplus_i A(\psi_i f_i)\) then verify that the RHS is
actually in the image of \(\iota:\ W^{k-r,p}(M) \hookrightarrow \bigoplus_i
W^{k-r,p}(\mathbb{R}^n)\). This is straightforward for differential operators be cause we
can differentiate elements of \(\mathcal{S}^*(M)\): \(\iota(Af) = \oplus_i A(\psi_i
f_i)\).

A less straightforward example is the definition of trace operator of elements in \(W^{k,p}(M)\) where \(M\) is a compact manifold with boundary. The Sobolev space \(W^{k,p}(M)\) when \(\partial M\ne \emptyset\) is defined in the same spirit: so that we
have the inclusion
\begin{align}
\label{eq:intro:4}
  \iota:\  W^{k,p}(M) &\longrightarrow \bigoplus_i W^{k,p}(R_i) \\
  	  f &\longmapsto \oplus_i \psi_i f \nonumber
\end{align}
where \(R_i\) is \(\mathbb{R}^n\) if the \(i^{\rm th}\) chart does not intersect the
boundary, or the upper half plan \(\mathbb{R}^{n-1}\times \mathbb{R}_{\geq 0}\) if the
intersection is not trivial. Then similarly, if \(k\) is sufficiently large, we can
define a component-wise trace operator \(W^{k,p}(\mathbb{R}^{n-1}\times \mathbb{R}_{\geq
0}) \longrightarrow W^{l,p}(\mathbb{R}^{n-1})\). On \(\partial M\), the
restriction of the atlas of \(M\) is still a finite atlas, subordinated by the partition
of unity \(\{\psi_i\}\), one therefore has the following diagram, where the vertical
arrow on the right was defined.
\[
\xymatrix{
W^{k,p}(M) \ar@{^{(}->}[r] \ar@{-->}[d] & \bigoplus_i W^{k,p}(R_i) \ar@{->}[d]^{\rm Tr} \\
W^{l,p}(\partial M) \ar@{->}[r] & \bigoplus_i W^{l,p}(\partial R_i)
}
\]
The tricky part to define the dashed vertical arrow, in comparison with the case of
differential operators, is that we cannot define trace operator on \(\mathcal{S}^*(M)\).

The situation is resolved because the maps \(\iota\) in \eqref{eq:intro:2.5}, \eqref{eq:intro:3} and
\eqref{eq:intro:4} admit a projection \(\pi\) (i.e. right-inverse) given by multiplication with a
family of cut-off functions \(\tilde\psi_i\) that are still supported in the chart,
but are identically 1 on the supports of \(\psi_i\). So to check whether an element is
in \(\im \iota\), one only has to check if it is fixed by \(\pi\circ\iota\), which is
simple.

The existence of \(\pi\) also shows that \(\im\iota\) is closed, hence \(W^{k,p}\)
is a reflexive Banach space, and that we can extend interpolation theory for \(W^{k,p}(M)\).

\subsection{Elliptic and parabolic equations}
\label{sec:org780a3df}
We will encode classical results of linear equations in an exact diagram such as
\begin{equation}
\label{eq:intro:5}
\xymatrix{
E \ar@{->}[r]^{l} \ar@{->}[d]^{m} & F \ar@{->}[d]^{p} \\
G \ar@{->}[r]^{q} & H
}
\end{equation}
where exactness means that
\[
 \xymatrix{
0 \ar@{->}[r] & E \ar@{->}[r]^{l\oplus m} & F\oplus G \ar@{->}[r]^{p\ominus q} & H \ar@{->}[r] & 0
}
\]
is a short exact sequence.

As a simplified example, for elliptic operator \(A\) with constant coefficient, one
has the following exact diagram
\[
\xymatrix{
W^{n,p}(\mathbb{R}^n) \ar@{->}[r]^{A} \ar@{^{(}->}[d]^{i} & W^{n-r,p}(\mathbb{R}^n) \ar@{^{(}->}[d]^{i} \\
W^{k,p}(\mathbb{R}^n) \ar@{->}[r]^{A} & W^{k-r,p}(\mathbb{R}^n)
}
\]
The closedness of \(\im A\oplus \iota = \ker A\ominus \iota\) implies, through Open
mapping theorem, Gårding's inequality. The equality \(\im A\oplus \iota = \ker A\ominus
\iota\) itself is the regularity result for elliptic equation. The subjectivity of \(A\ominus\iota\) is the existence of approximate solution.

The exactness of the previous diagram comes from the fact that it splits, meaning that we
can find compatible maps \(G\) and \(\psi(D)\) such that
\[
\xymatrix{
W^{k,p}(\mathbb{R}^n) \ar@/_/@{->}[r]_{A} \ar@{->}[d]^{i} & W^{k-r,p}(\mathbb{R}^n) \ar@{->}[d]_{i} \ar@/_/@{->}[l]_{G} \\
W^{l,p}(\mathbb{R}^n) \ar@/^/@{->}[r]^{A} \ar@/^/@{->}[u]^{\psi(D)} & W^{l-r,p}(\mathbb{R}^n) \ar@/^/@{->}[l]^{G} \ar@/_/@{->}[u]_{\psi(D)}
}
\]
is a split diagram, where \(\psi(D)\) is certain cut-off function on the frequency
space. The splitness of the diagram in local (in \(\mathbb{R}^n\)) instead of just
exactness reflects the fact that we have an algebraic formula of the solution/ of the
Green kernel in \(\mathbb{R}^n\).

The idea to go from local to global, naturally since the equation is linear, is to use a
partition of unity and to remark that the commutator of a differential operator and the
multiplication by a cut-off function is a differential operator of strictly lower
order. This however is not the only ingredient. In the same spirit (but not the same technical reason) as the parametrix
approach where we have to iterate to find the Green kernel, in the "diagram" approach,
we lose algebraic control of the solution in an argument of the following type: a diagram
sufficiently closed to an exact diagram \eqref{eq:intro:5} in \(L(E,F)\times L(F,H)\times
L(E,G)\times L(G,H)\) is still exact.

One also has a similar diagram for parabolic equation. The only difference is that one has
\emph{causality} in the parabolic case, meaning that the operator \(A\) is now an
isomorphism. This is because when
the initial condition is a vanishing condition \(\restr{f}{t=\alpha}=0\) and when the
boundary conditions on \(\partial M \times [\alpha,\omega]\) is independent of time, we
can make translation in time of the solution, and still have a solution. In the general
case, we use Fredholm's Index theory.

\section{Energy estimates.}
\label{sec:orgded6fed}
The ingredients to bound higher order derivative of the solution include: (1) estimates of
physical quantities, i.e. the total potential and kinetic energy, (2) Maximum principle
and  Gårding's inequality, (3) estimates of nonlinear differential operators using Besov
spaces, and (4) \(L^1\)-comparison theorem for linear heat equation. Let us explain the
last item. If \(f\) is a smooth solution of a linear heat equation
\[
 \frac{df}{dt} = -\Delta f + Cf \quad \text{on } [\alpha,\omega]
\]
then by an argument similar to the proof of Maximum principle, one can estimate the \(L^\infty\)-norm of \(\restr{f}{t=\omega}\) in term of \(\|\restr{f}{\alpha}\|_{L^\infty}\):
\[
 \|\restr{f}{\omega}\|_{L^\infty} \leq e^{B(\omega-\alpha)}
\|\restr{f}{\alpha}\|_{L^\infty}.
\]
We will need in certain moment to estimate \(\|\restr{f}{\omega}\|_{L^1}\) in term of \(\|\restr{f}{\alpha} \|_{L^1}\). Since \(L^1\) is the dual space of \(L^\infty\), one
can estimate \(\|\restr{f}{\omega}\|_{L^1}\) by finding an upper bound of \(\int_{t=\omega} fh\) in term of \(\|h\|_{L^\infty}\) where \(h\) is a smooth function
on \(M\times\{\omega\}\).

This can be done by considering the backwards heat equation propagating from time \(\omega\) to \(\alpha\): 
\begin{equation*}
 \begin{cases}
\frac{d g}{dt} = \Delta g - C g,  & \text{on $ N\times{[\alpha,\omega]}$} \\
\restr{g}{\omega} = h, 
\end{cases}
\end{equation*}
This equation was chosen so that \(\frac{d}{dt}\int_M fg = \int_M f \frac{d }{dt}g + g
\frac{d }{dt}f =0\), therefore \(\int_{M\times\{\alpha\}}fg = \int_{M\times\{\omega\}} fg\). Apply \(L^\infty\)-estimate to \(g\) and one obtains an \(L^1\)-estimate for \(f\).