\iffalse
\begin{info}
The PDF version of this page can be downloaded by replacing \texttt{html} in the its address by
\texttt{pdf}. 
For example \texttt{/html/sheaf-cohomology.html} should become \texttt{/pdf/sheaf-cohomology.pdf}.
\end{info}
\fi

\iffalse
Here is the \href{../Stage 2018/main.pdf}{memoire}.
\fi

\section{Summary}
\label{sec:org7287a1a}

The goal of this part is to give a summary of what will be developed in the next chapters. In brief, we are interested in
maps \(f: M \longrightarrow M'\) between Riemannian manifolds (that to simplify, are
supposed to be compact) that are critical points of the energy functional
\[
 E(f) = \frac{1}{2}\int_M |\nabla f|^2 dV.
\]
By taking first order variation of \(E\), these are maps whose \textbf{tension field} \(\tau(f)\) vanishes. 


\subsection{Deformation using nonlinear heat equation.}
\label{sec:org9617dcd}

The approach of \cite{eells_harmonic_1964} is to prove that, if the target space is
negatively curved, then any smooth map \(f_0: M \longrightarrow M'\) can be deformed to a
harmonic map using the gradient descent equation:
\begin{equation}
\label{eq:intro:1}
\begin{cases}
\frac{df_t}{dt} = \tau(f_t)\\
\restr{f}{t=0} = f_0
\end{cases}
\end{equation}
We will prove that if \(M'\) is negatively curved then this PDE admits a globally defined smooth
solution \(f_t\) and that \(f_{\infty}:=\lim_{t\to \infty} f_t\) in \(C^\infty\) is
a harmonic map.

The resolution of \eqref{eq:intro:1} can be organised in 3 steps:
\begin{enumerate}
\item Find the global equation. We will find a global frame of \(M'\) and express
\(f\) in this frame, so that instead of solving for a map, we will have to solve for functions.
\item Study linear PDEs on manifolds. The equation, expressed in local coordinates, is a nonlinear heat equation, i.e. other
than a heat operator, it has a quadratic term. Short-time existence and regularity for \eqref{eq:intro:1} follows from \emph{standard} results of
parabolic equation.
\item Prove long-time existence. In order to use continuity method, we will have to prove
that \(W^{k,p}\)-norms of the solution \(f_t\) do not explode. This will be
established first in the case \(W^{2,2}\) using physical quantities, namely the
potential energy \(E\) and the kinetic energy \(K\). The general case is proved
from the \(W^{2,2}\) estimate using Gårding's inequality and Comparison theorem for
parabolic equation.
\end{enumerate}

The hypothesis of negative curvature is only used to establish the energy
estimates. During deformation, the rate of potential energy can be calculated as: 
\[
 \frac{d e(f_t)}{dt}= -\Delta e(f_t) - |\beta(f_t)|^2 - \left\langle \Ric(M) \nabla_v
f_t,\nabla_v f_t \rangle + \langle \Riem(M') (\nabla_v f_t,\nabla_w f_t)\nabla_v
f_t,\nabla_w f_t \right\rangle
\]
and the kinetic energy as:
\[
 \frac{d k(f_t)}{dt}= -\Delta k(f_t) - \left|\nabla \frac{\partial f_t}{\partial t}\right|^2 +
\left\langle \Riem(M') (\nabla_v f_t,\frac{\partial f_t}{\partial t})\nabla_v
f_t,\frac{\partial f_t}{\partial t} \right\rangle
\]
Therefore if all sectional curvatures of \(M'\) are negative, these rates can be
controlled and the energies are guaranteed not to explode.

\subsection{Existence using Morse-Palais-Smale theory.}
\label{sec:orgc4e17a9}
We also give a less detailed review of the work by Sacks and Uhlenbeck
\cite{sacks_existence_1981}. This approach uses an approximating family \(E_\alpha\) of the energy functional \(E\) whose critical functions in \(W^{1,2\alpha}\) can be easily proved
to exist using Morse-Palais-Smale theory. One then tries to prove that the critical sequence
\(C^1\)-converges to a nontrivial limit. 

As a concrete result, the authors proved,
using an extension theorem for harmonic maps on surface and a suitable covering of \(M\) by small discs on which the energy \(E\) is sufficiently small, that if
the fundamental group \(\pi_k(M')\) is nontrivial for a certain \(k\geq 2\), or
equivalently, if the universal covering \(\tilde M'\) of \(M'\) is not contractible,
then there exists a nontrivial harmonic map from \(\mathbb{S}^2\) to \(M'\).


\iffalse
\bibliographystyle{alpha}
\bibliography{../res/Stage2018}
\fi
