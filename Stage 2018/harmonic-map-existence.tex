\iffalse
\begin{info}
The PDF version of this page can be downloaded by replacing \texttt{html} in the its address by
\texttt{pdf}. 
For example \texttt{/html/sheaf-cohomology.html} should become \texttt{/pdf/sheaf-cohomology.pdf}.
\end{info}
\fi

Let \(M\) be a compact Riemannian manifold. We want to solve the following nonlinear
heat equation where \(F: M \longrightarrow M'\subset
B\subset V = \mathbb{R}^N\):
\[
 \frac{d F_t}{d t} = -\Delta F_t + \Gamma(F_t) (\nabla F_t)^2
\]
We have proved that the solution \href{polynomial-besov.org}{exists in short-time and is smooth whenever it exists}. We will now establish long-time existence using continuity
method: we will show that if the solution exists on \([\alpha,\omega_n]\) where
\(\omega_n\) is an increasing sequence to \(\omega\), then the solution exists on \([\alpha,\omega]\). We then apply short-time existence to gain a small open interval where
solution still exists. We then conclude that the solution exists globally on \([\alpha,+\infty)\) since this interval is connected.

The crucial step to prove that the solution can be extended on \([\alpha,\omega]\) is to
uniformly bound all of its derivatives in time of evolution \([\alpha,\omega)\). These
estimates will also be useful to justify the convergence of \(F_t\) in \(C^\infty(M)\) to a smooth function \(F_\infty\) which will eventually be a harmonic map
from \(M\) to \(M'\).

Recall that we \href{harmonic-maps.org}{proved} in Corollary \ref{cor:bound-2-2}, under the hypothesis of negative curvature, the boundedness of \(\|F_t\|_{W^{2,2}(M)}\) by a constant \(C\) depending only on curvatures of \(M, M'\) and
the initial total energies. Since \(\frac{dF_t}{dt }\) relates to spatial derivatives of
\(F\) by the nonlinear heat equation, it is easy to see that \(\|F_t\|_{W^{2,2}(M\times[\tau,\tau +\delta])}\) is bounded by a constant independent of
\(\tau\). We will denote \(W^{k,p}(M\times [\beta,\gamma])\) by \(W^{k,p}([\beta,\gamma])\).

\begin{theorem}[\(W^{2,2}\)-boundedness]
\label{thm:bound-2-2}
Suppose \(\Riem(M') \leq 0\). There exists a constant \(C\) depending only on \(\delta\), the metrics and initial
total energies such that
\[
 \|F\|_{W^{2,2}(\tau,\tau+\delta)}\leq C\quad\text{for all } \alpha \leq \tau <\omega-\delta.
\]
\end{theorem}
\begin{proof}
Since 
\[ \|F\|^2_{W^{2,2}([\tau,\tau+\delta])} \leq \int_\tau^{\tau+\delta}
\|F_t\|^2_{W^{2,2}(M)} dt + 2\int_\tau^{\tau+\delta} \left\| \Delta F_t
\right\|^2_{L^2}dt + 2\int_\tau^{\tau+\delta} \left\| \Gamma(F_t)(\nabla F_t)^2
\right\|^2_{L^2}dt
\]
The first term and the second term are bounded by \(C^2\delta\), the third one, since \(\Gamma(F_t)\) is bounded, by \(C^2\delta\) where \(C\) is a constant only depending on
the metrics and initial total energies.
\end{proof}

The estimates of higher derivatives of \(F\) will be established in the same strategy as
the bootstrap: first in \(W^{2,p}\) for all \(p\) then in \(W^{k,p}\) for all \(k,p\), then in \(C^\infty\).

\section{Estimate of higher derivatives.}
\label{sec:org50e3df5}

\begin{lemma}[\( W^{2,p} \)-boundedness]
\label{lem:bound-2-p}
Suppose \(\Riem(M') \leq 0\). For all \(p \in (1,+\infty)\), there exists a constant \(C>0\) depending only on \(\delta\), \(p\), the metrics and initial energies such that for all \(\alpha +\delta \leq \tau \leq \omega-\delta\):
\[
\|F\|_{W^{2,p}([\tau,\tau+\delta])}\leq C
\]
\end{lemma}

\begin{proof}
Applying Gårding Inequality to the parabolic equation \(AF = \Gamma(F) (\nabla F)^2\)
where \(A:= \frac{\partial}{\partial t} + \Delta\) is the heat operator, one has
\[
 \|F\|_{W^{2,p}([\tau,\tau+\delta])} \leq C\left( \|\Gamma(F)(\nabla F)^2\|_{L^p([\tau
 -\frac{\delta}{3},\tau +\delta ])} + \|F\|_{W^{2,2}([\tau-\frac{\delta}{3},\tau +\delta])}  \right)
\]
The second term of RHS is already bounded by applying Theorem \ref{thm:bound-2-2} to \(\frac{4\delta}{3}\). For the first term:
\[
\left\| \Gamma(F) (\nabla F)^2\right\|_{L^p([\tau-\frac{\delta}{3},\tau+\delta])} \leq
C(M') \| |\nabla F|^2 \|_{L^p([\tau-\frac{\delta}{3}, \tau+\delta])} = C(M') \| e(F)
\|_{L^p([\tau - \frac{\delta}{3},\tau+\delta])}.
\]


Recall that by Theorem \ref{thm:den-pot}, the \href{harmonic-maps.org}{potential density} satisfies \(\frac{de}{dt} +\Delta e -Ce \leq 0\) for
certain constant \(C\) depending only on the metric of \(M\). By \href{elliptic-parabolic.org}{Maximum principle}
(Theorem \ref{thm:max-princ-d}), one has \(e \leq \psi_\tau\) where \(\psi_\tau\) is
the solution of

\begin{cases}
\frac{d}{dt}\psi_\tau + \Delta\psi_\tau -C\psi_\tau = 0  \\
\restr{\psi_\tau}{\tau - \frac{\delta}{2}} = \restr{e}{\tau - \frac{\delta}{2}}
\end{cases}

We apply Gårding Inequality again for \(\psi_\tau\) and obtain
\begin{equation}
\label{eq:lem:bound-2-p:2}
 \| e(F) \|_{L^p([\tau - \frac{\delta}{3},\tau+\delta])} \leq \|\psi_\tau\|_{L^p([\tau-
\frac{\delta}{3},\tau+\delta])} \leq C \|\psi_\tau\|_{L^1([\tau-
\frac{\delta}{2},\tau+\delta])}.
\end{equation}


Now apply \(L^1\)-\href{elliptic-parabolic.org}{Comparison Theorem} \ref{thm:1-comparison-d} to \(\psi_\tau\), one has
\begin{equation}
\label{eq:lem:bound-2-p:3}
\|\psi_\tau\|_{L^1([\tau-
\frac{\delta}{2},\tau+\delta])} \leq \int_0^{3\delta/2} \|\restr{\psi_\tau}{\tau-\frac{\delta}{2}}\|_{L^1}e^{Bt}dt \leq \int_0^{3\delta/2}e^{Bt}dt.  \|\restr{e}{\tau-\frac{\delta}{2}}\|_{L^1}\leq C.
\end{equation}
The lemma follows from \eqref{eq:lem:bound-2-p:2} and \eqref{eq:lem:bound-2-p:3}.
\end{proof}

We can now estimate higher order derivatives.

\begin{theorem}[\( W^{k,p} \)-boundedness]
\label{thm:bound-k-p}
Suppose \(\Riem(M') \leq 0\). For all \(p\in (1,+\infty)\) and \(k<+\infty\), there exists \(C\) depending only on
\(k\), \(p\), the metrics and initial energies such that
\[
\|F\|_{W^{k,p}([\tau, \tau+\delta])} \leq C
\]
for all \(\alpha +\delta \leq\tau\leq\omega-\delta\).
\end{theorem}
\begin{proof}
Applying Gårding Inequality to the equation \(\frac{dF}{dt} + \Delta F_t =
\Gamma(F)(\nabla F)^2\) then Regularity estimate for the quadratic term (Theorem \ref{thm:reg-quad}), one has for \(\epsilon \ll \delta\):
\begin{align*}
 \|F\|_{W^{k,p}([\tau,\tau+\delta])} &\leq C_\epsilon \left(
\|F\|_{W^{2,p}([\tau-\epsilon,\tau+\delta])} + \|\Gamma(F)(\nabla F)^2\|_{W^{k-2,p}([\tau-\epsilon,\tau+\delta])}\right)  \\
					   &\leq C_\epsilon\left(1 + C\left(1+\|F\|_{W^{s,q}([\tau-\epsilon,\tau+\delta])}\right)^{q/p}\right)
\end{align*}
as long as \(k-1 < s\) and \(\frac{1}{p} > \frac{k}{s}.\frac{1}{q}\). Therefore if \(\|F\|_{W^{s,q}([\tau,\tau+\delta])}\leq C(\delta,s,q)\) for all \(\beta\leq
\tau\leq\omega-\delta\) and \(q\in (1,+\infty)\), we just proved that
\[
 \|F\|_{W^{k,p}([\tau,\tau+\delta])} \leq C({\epsilon}, k,p)
\]
for all 
\begin{cases}
\beta + \epsilon \leq \tau \leq \omega-\delta \\
k < s+1, p \in(1,+\infty)
\end{cases}
since \(\|F\|_{W^{s,q}([\tau-\epsilon,\tau+\delta])} \leq 2C(\delta,s,q)\).

One can then conclude by induction on \(k\), with step \(\frac{1}{2}\), starting with
\(k=2\) and \(\epsilon=\frac{\delta}{2}\) divided by 2 after each induction step.
\end{proof}


\section{Global existence for nonlinear heat equation.}
\label{sec:orga0359a0}

\begin{theorem}[Global existence]
\label{thm:global-heat-existence}
Suppose \(\Riem(M') \leq 0\). The solution of nonlinear heat equation
\begin{equation}
\label{eq:thm:global-heat}
 \frac{dF}{dt} = -\Delta F +\Gamma(F) (\nabla F)^2
\end{equation}
with smooth initial condition exists globally for all time \(t >\alpha\).
\end{theorem}

\begin{proof}
Let \(F_n\) be a sequence of solution of \eqref{eq:thm:global-heat} on \([\alpha,\omega_n]\) with \(\omega_n\) increasing to \(\omega\) then they coincide by
uniqueness of the solution. As discussed in the beginning of this part, it is
sufficient to prove that the solution extends to \([\alpha,\omega]\). Let \(F\) be the
solution on \([\alpha,\omega)\) such that \(\restr{F}{[\alpha,\omega_n]} = F_n\), then
by Theorem \ref{thm:bound-k-p}, for all \(\tau \in [\alpha,\omega-\delta)\):
\[
 \| D^u_t D^v_x F \|_{L^\infty(M\times[\tau,\tau+\delta])} \leq C_{\rm Sobolev} \| D^u_t D^v_x F
\|_{W^{k,p}(M\times[\tau,\tau+\delta])} \leq C_{\rm Sobolev} . C(k,p,\delta)
\]
where, if we choose \(k\) sufficiently large, \(C_{\rm Sobolev}\) is the constant of Sobolev imbedding \(W^{k,p}(M\times[0,\delta]) \hookrightarrow C(M\times[0,\delta])\) and \(C(k,p,\delta)\)
is the constant provided by Theorem \ref{thm:bound-k-p}. 

So all partial derivatives of \(F\) are uniformly bounded on \([\alpha,\omega)\). This proves that \(F\) extends
to a solution on \([\alpha,\omega]\). In fact \(\restr{F}{\tau}:=\restr{F}{M\times\{\tau\}}\) converges
in \(C^\infty(M)\) as \(\tau\to \omega\), since 
\[ \|D^\alpha \restr{F}{\tau} -
D^\alpha\restr{F}{\tau'}\|_{L^\infty} \leq \max_{\|\beta\| =
\|\alpha\|+1}\|D^{\beta}F\|_{L^\infty}|\tau-\tau'|. 
\]
\end{proof}

We have just proved the first part of the following theorem.

\begin{theorem}[Eells-Sampson]
\label{thm:final}
\begin{enumerate}
\item Let \(M, M'\) be compact Riemannian manifolds with \(\Riem(M') \leq 0\). Then for
every smooth map \(f_0:\ M \longrightarrow M'\subset B\subset \mathbb{R}^N\), the
nonlinear heat equation
\begin{equation*}
\begin{cases}
\frac{df_t}{dt} = \tau(f_t),  & \text{for all $t\geq 0$} \\
\restr{f}{t=0} = f_0,
\end{cases}      
\end{equation*}
admits a globally defined smooth solution \(f_t\). Moreover, all derivatives \(D^\alpha
   f_t\) remain bounded as \(t\to +\infty\).
\item For a suitable sequence \(\{t_n\}\) increasing to \(+\infty\) the sequence \(\{f_{t_n}\}\) converges in \(C^\infty(M)\) to a function \(f_\infty\) with \(\tau(f_\infty)=0\). Therefore any map \(f_0:\ M \longrightarrow M'\) is homotopic to a harmonic map.
\end{enumerate}
\end{theorem}
\begin{proof}
For any sequence \(\{t_n\}\), one can extract from \(\{f_{t_n}\}\), since their derivatives are
uniformly bounded, a subsequence \(\{f_{t_{n_i}}\}\) converging in \(C^k(M,
\mathbb{R}^N)\). By a diagonal argument, one can extract from any sequence \(\{f_{t_n}\}\) a subsequence converging in \(C^\infty(M, \mathbb{R}^N)\) to \(f_\infty\). Abusively denote this subsequence by \(\{f_{t_n}\}\), by Theorem \ref{thm:den-kin}
\[
 \lim_{n\to\infty} K(f_{t_n}) = \lim_{n\to\infty} \int_M |\tau(f_{t_n})|^2 = 0
\]
Therefore \(\tau(f_{t_n}) \to 0\) in \(L^2(M)^{\oplus N}\). But also \(\tau(f_{t_n})
\to \tau(f_\infty)\) in \(C^\infty(M, \mathbb{R}^N)\), one has \(\tau(f_\infty)=0\).
\end{proof}
