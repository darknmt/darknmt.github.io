\section{Motivation}
\label{sec:orgc3441c8}

We will define a more general notion of Sobolev spaces on compact manifold than those in
\cite{aubin_nonlinear_1998} and \cite{jost_riemannian_2008}, where Sobolev spaces on a
(Riemannian) manifold \(W^{k,p}(M)\) of dimension \(n\) are defined for
\(k\in \mathbb{Z}_{\geq 0}\) and for \emph{uniform weight}, meaning that a function \(f\in W^{k,p}(M)\) is supposed to be weakly differentiable up to order \(k\) in every variables \(x_1,\dots, x_n\) in each smooth coordinates. The space \(W^{k,p}(M)\) in this case can
be defined by density with respect to a norm involving derivatives \(\frac{\partial f}{\partial x^\alpha}\).

Meanwhile, the suitable function spaces to
solve parabolic equations are those whose regularity in time is half of that in space,
i.e. we will solve parabolic equations on the Sobolev spaces \(W^{k,p}(M\times T)\) of functions
\(k\) times regular in \(M\) and \(k/2\) times regular in \(T\). We cannot always,
(for example when \(k\) is odd) find a simple norm involving derivatives of \(f\) 
to define \(W^{k,p}\) by density. This generalisation will be done using Stein's
multipliers.

Another generalisation will be made is to allow the manifold to have boundary. Even when
we only want to solve parabolic equation on manifold \(M\) without boundary, the
underlying space is \(M\times [0, T]\) which has boundary. Moreover, we will have to
discuss the notion of trace in order to use the initial condition at \(t=0\). 

In this part, all manifolds will be compact, with no given metric. This is not really a
generalisation since on compact manifolds, Sobolev spaces \(W^{k,p}(M)\), as defined in
\cite{aubin_nonlinear_1998} and \cite{jost_riemannian_2008} set theoretically do not depend on
the metric and (the equivalent class of) their norms also independent of the metric.

We will mainly follow the discussion in \cite{hamilton_harmonic_1975}, where the author also
works on manifold with \emph{corner}, i.e. irregular boundary. The corners, modeled by \(\mathbb{R}^{n-k}\times \mathbb{R}_{\geq 0}^k\), appear naturally,
for example at the boundary \(\partial M\) in \(t=0\). The extra effort to cover the
case of corners is not much (see \cite[page 50]{hamilton_harmonic_1975}) and essentially algebraic.

\section{Preparatory material}
\label{sec:orgd4a87c1}
We will recall here basic elements of Fourier transform on the space of tempered
distributions and then we will have a quick review of interpolation theory.

\subsection{Stein's multiplier}
\label{sec:org7695576}
Let \(X = \mathbb{R}^n\) be the Euclidean space, coordinated by \(x_1,\dots, x_n\) and
\(\mathcal{E}= \mathbb{R}^n\), coordinated by \(\xi_1,\dots,\xi_n\) be the frequency
domain of \(X\). Recall that Fourier transform is an isomorphism in the following three levels
\begin{enumerate}
\item The Schwartz space of rapidly decreasing smooth functions \(\mathcal{S}(X)\) whose elements
are smooth and decrease more rapidly then any rational function. The Schwartz space are
topologized by the family of semi-norms \(|f|_{\alpha,\beta}= \sup_X |x^{\alpha} D^b_x f(x)|\).
\item The space \(L^2(X)\) of doubly-integrable functions.
\item The space of tempered distributions, i.e. the dual space \(\mathcal{S}^*(X)\) of \(\mathcal{S}(X)\) under the weak-* topology given by \(\mathcal{S}(X)\).
\end{enumerate}

To simplify the notation, we use \(D^\alpha_x = \left(\frac{1}{i}\frac{\partial}{\partial
x_1}\right)^{\alpha_1} \dots  \left(\frac{1}{i}\frac{\partial}{\partial
x_n}\right)^{\alpha_n}\) and \(P(D) = \sum_\alpha c_\alpha D^\alpha\) for any polynomial
\(P\). 

Recall that for any \(u\in \mathcal{S}(X)\)
and for any polynomial \(P\), one has  \(\widehat{P(D)u} = P(\xi)\hat u(\xi)\). This
can be extended to non-polynomial function of \(M(D)\) of \(D\) by
\[
 \widehat{M(D)u} := M(\xi)\hat u(\xi)
\]
where \(M\) is a slowly growing function, i.e. \(D^\alpha M(\xi)\) grows slower than
certain polynomial as \(|\xi|\to \infty\).

The following theorem give a criteria of the function \(M\) such that \(M(D):
\mathcal{S}(X) \longrightarrow \mathcal{S}(X)\) extend to \(L^p(X) \longrightarrow
L^p(X)\).

\begin{theorem}[Stein]
\label{thm:stein-crit}
If for any \emph{primitive} index \(\alpha=(\alpha_1,\dots,\alpha_n)\), i.e. each \(\alpha_i\) being \(0\) or \(1\) (there are exactly \(2^n\) primitive indices), one has
\[
 \left| \xi^\alpha D^\alpha M(\xi)\right| \leq C_\alpha
\]
then \(M(D)\) extend to a bounded linear operator on \(L^p(X)\).
\end{theorem}

\begin{definition}
\begin{enumerate}
\item A slowly growing function \(W\) on \(\mathcal{E}\) with \(W(\xi)>0\) is called a
\textbf{weight} if for all primitive index \(\alpha\), one has
\[
   \left|\xi^\alpha D^\alpha W(\xi)\right|\leq C_\alpha W(\xi).
   \]
\item The \textbf{Sobolev space} \(W^{k,p}(X,W)\) with respect to weight \(W\), \(k\in
   \mathbb{R}, 1<p<\infty\) is the vector space
\[
    W^{k,p}(X,W) = \left\{ u\in \mathcal{S}^*(X):\ W(D)^k u \in L^p(X)\right\}
   \]
normed by \(\|u\|_{W^{k,p}} = \| W(D)^k u\|_{L^p}\).
\end{enumerate}
\end{definition}

\begin{exampl}[Weight given by \( \Sigma=(\sigma_1,\dots, \sigma_n) \)]
Note by \(\sigma:= \lcm(\sigma_1,\dots,\sigma_n)\) then \(W_\Sigma(\xi) = \left( 1 +
\xi_1^{2\sigma_1} + \dots + \xi_n^{2\sigma_n} \right)^{1/2\sigma}\) is a weight. We will
only use weights of this type in our discussion. The index \(\Sigma =
(\sigma_1,\dots,\sigma_n)\) is chosen according to the differential operator in the
elliptic/parabolic equation. In particular, for Laplace equation, one chooses \(\Sigma=(1,\dots, 1)\) and for heat equation \(\Sigma = (1,2,\dots, 2)\) where \(1\)
is in the time component.
\end{exampl}

\begin{remark}
\label{rem:weight}
\begin{enumerate}
\item If \(W_1,W_2\) are weights then \(W_1 + s W_2, W_1 W_2, W_1^s (s>0)\) are also weights.
\item The operator \(W(D):\ W^{k+r},p(X,W) \longrightarrow W^{k,p}(X)\) is bounded.
\item Given another weight \(V(\xi) \leq C W(\xi)\), by Stein's criteria (Theorem
\ref{thm:stein-crit}) one has a bounded embedding \(W^{k,p}(X,W) \hookrightarrow  W^{k,p}(X,V)\).
\end{enumerate}
\end{remark}

The Sobolev space \(W^{k,p}(X,W_\Sigma\) has a simple definition by density when \(\sigma \mid
k\). Given an index \(\alpha = (\alpha_1,\dots,\alpha_n)\), note by
\(\|\alpha\| := \sum_{i=1}^n \alpha_i \frac{\sigma}{\sigma_i}.\)

\begin{theorem}[Equivalent norm when \( \sigma \mid k \)]
\label{thm:equiv-norm-Sobolev}
If \(k>0\) and \(\sigma \mid k\) and \(1<p<\infty\), then given \(u\in
\mathcal{S}^*(X)\), one has
\begin{enumerate}
\item \(u\in W^{k,p}(X)\) if and only if \(D^\alpha u\in L^p(X)\) for all \(\|\alpha\|\leq k\) and the norm \(\sum_{\| \alpha \leq k\|} \|D^\alpha u\|_{L^p}\) is equivalent to \(\|u\|_{W^{k,p}}\).
\item \(u\in W^{-k,p}\) if and only if there exists \(g_\alpha\in L^p\) such that \(u =
   \sum_{\|\alpha\|\leq k} D^\alpha g_\alpha\) and \(\|u\|_{W^{-k,p}}\) is equivalent
to
\[
    \inf \left\{ \sum_{\|\alpha\| \leq k} \|g_\alpha\|_{L^p}:\ u = \sum_{\|\alpha\|\leq k}
   D^\alpha g_\alpha \right\}
   \]
\end{enumerate}
\end{theorem}

\begin{exampl}
\begin{enumerate}
\item When \(\sigma_1=\dots=\sigma_n=1\), one has the familiar Sobolev spaces.
\item For (the weight of) heat equation, \(W^{2,p}\) can be defined by density using the
norm
\[
    \|u(t,x)\| = \left\|\frac{\partial u}{\partial t}\right\|_{L^p} +
   \left\|Du \right\|_{L^p} + \left\|D u\right\|_{L^p}
   \]
where \(L^p\) stands for \(L^p(X\times [0,T])\).
\end{enumerate}
\end{exampl}

\subsection{Holomorphic interpolation of Banach spaces}
\label{sec:orgf24d7d0}

The Interpolation theory is based on the following Three-lines theorem whose proof follows
from the classic \href{https://en.wikipedia.org/wiki/Hadamard\_three-lines\_theorem}{Hadamard's three-lines theorem} (the case \(A=\mathbb{C}\)) and the way we
define complex Banach spaces and holomorphic maps taking value there.

\begin{theorem}[Three-lines]
\label{thm:3-line}
Let \(A\) be a complex Banach space and \(h:\ S=\left \{0\leq \re z\leq 1\right\}\subset
\mathbb{C} \longrightarrow  A\) be a holomorphic map, i.e. continuous and
holomorphic in the interior such that \(h\) is bounded at infinity, i.e. \(h(x+iy)
\to 0\) as \(y \to \infty\). Let \(M(x):=\sup_y \|h(x+iy)\|\) then one has
\[
 M(x)\leq M(1)^x M(0)^{1-x}
\]
\end{theorem}

Let \(A_0, A_1\) be complex Banach spaces such that 
\begin{enumerate}
\item \(A_0, A_1\) can be continuously embedded into a Hausdorff topological complex vector
space \(E\) such that the complex structures are compatible with each others,
i.e. the linear embeddings \(A_i \hookrightarrow E\) preserve complex structures.
\item The intersection \(A_0\cap A_1\) in \(E\) is dense in \((A_i,\|\|_{A_i})\) for \(i=0,1\).
\end{enumerate}
such \((A_0,A_1)\) is called an \textbf{interpolatable} pair.

The norms of \(A_0\cap A_1\) and \(A_0 + A_1\) are defined such that the these spaces
are Banach and the diagram
\begin{equation}
\label{fig:interpol-pair}
\xymatrix{
0 \ar@{->}[r] & A_0\cap A_1 \ar@{->}[r] & A_0\oplus A_1 \ar@{->}[r] & A_0+A_1 \ar@{->}[r] & 0
}
\end{equation}
commutes and the arrows are
continuous. By Open mapping theorem, this means that the norm on \(A_0\cap A_1\) is
equivalent to \(\|x\|_{A_0\cap A_1} = \|x\|_{A_0} + \|x\|_{A_1}\) and the norm on \(A_0+A_1\) is equivalent to \(\|x\|_{A_0+ A_1} = \inf_{x=x_0 + x_1, x_i\in A_i} \left\{\|x_0\|_{A_0} + \|x\|_{A_1}\right\}\).

\begin{remark}
\label{rem:interp-pair}
A pair \((A_0, A_1)\) of Banach spaces may give different interpolatable pairs depending
how they are embedded into a common space \(E\). It is not difficult to see that the data of \emph{interpolatable pair} is
uniquely determined by 2 complex Banach spaces \(U,V\) (which are eventually \(A\cap B\) and \(A+B\)) and the diagram
\begin{equation}
\label{fig:unique-interpol-pair}
\xymatrix{
 &  & 0 \ar@/_/@{->}[d] &  &  \\
 &  & A_0 \ar@{->}[rd] \ar@/_/@{->}[d] \ar@/_/@{->}[u] &  &  \\
0 \ar@{->}[r] & U \ar@{->}[r] \ar@{->}[ru] \ar@{->}[rd] & A_0\oplus A_1 \ar@{->}[r] \ar@/_/@{->}[d] \ar@/_/@{->}[u] & V \ar@{->}[r] & 0 \\
 &  & A_1 \ar@{->}[ru] \ar@/_/@{->}[d] \ar@/_/@{->}[u] &  &  \\
 &  & 0 \ar@/_/@{->}[u] &  & 
}
\end{equation}
in which 
\begin{enumerate}
\item All arrows are continuous and compatible with complex structures. The horizontal sequence
is exact, the vertical sequence is exact and canonical.
\item The diagonal arrows from \(U\) to \(A_0, A_1\) are injective and of dense image in
\(A_0, A_1\).
\item The maps composed by the diagonal arrows \(U \to A_i \to V\) are injective for \(i=0,1\). Since the two maps are additive inverse, it suffices to have injectivity for
one of them.
\end{enumerate}

In the language that we will use to solve \href{./elliptic-parabolic.org}{linear equation}, these properties of diagram
\eqref{fig:unique-interpol-pair} are equivalent to the square
\[
 \xymatrix{
U \ar@{->}[r] \ar@{->}[d] & A_0 \ar@{->}[d] \\
A_1 \ar@{->}[r] & V
}
\]
being \emph{exact}.
\end{remark}

The following construction will give a family of complex subspace \(A_\theta\) of \(A_0+A_1\) containing \(A_0\cap A_1\) for \(0\leq \theta\leq 1\) that interpolates \(A_0\) and \(A_1\) that satisfies the following properties, called interpolation
inequalities

\begin{theorem}[Interpolation inequality for elements in the intersection]
\label{thm:interp-ineq-ele}
Let \(a\in A_0\cap A_1\) then \(a\in A_\theta\) and
\[
 \|a\|_{A_\theta}\leq 2 \|a\|_{A_1}^\theta \|a\|_{A_0}^{1-\theta}
\]
\end{theorem}


\begin{theorem}[Interpolation inequality for operators]
\label{thm:interp-ineq-op}
Given interpolatable pairs \((A_0,A_1)\) and \(B_0, B_1)\), and \(T\) a
bounded linear operator \(T: A_0 \longrightarrow B_0\) and
\(T: A_1 \longrightarrow B_1\) such that \(T\) is well-defined on \(A_0\cap A_1\). Then \(T\) extends linearly and continuously to \(T:\ A_0+A_1 \longrightarrow B_0+
B_1\), that is

\begin{equation}
\xymatrix{
0 \ar@{->}[r] & A_0\cap A_1 \ar@{->}[r] \ar@{->}[d]^{T} & A_0\oplus A_1 \ar@{->}[r] \ar@{->}[d]^{T\oplus T} & A_0+A_1 \ar@{->}[r] \ar@{->}[d]^{T} & 0 \\
0 \ar@{->}[r] & B_0\cap B_1 \ar@{->}[r] & B_0\oplus B_1 \ar@{->}[r] & B_0+B_1 \ar@{->}[r] & 0
}
\end{equation}

Also, \(T\) defines a bounded operator \(T:\ A_\theta \longrightarrow  B_\theta\) and
\[
 \|T\|_{L(A_\theta,B_\theta)}\leq 2 \|E\|_{L(A_1,B_1)}^\theta \|E\|_{L(A_0,B_0)}^{1-\theta}
\]
\end{theorem}

To define \(A_\theta\), let 
\[
 \mathcal{H}(A_0,A_1):= \left\{ h: S \longrightarrow A_0+ A_1 :\ h \text{ is holomorphic, } \lim_{|y|\to\infty}h(z) = 0, h(iy)\in A_0,\ h(1+iy)\in A_1 \}
\]
where, as above, \(S\) denotes the strip \(0\leq \re z\leq 1\). Then \(\mathcal{H}(A_0, A_1)\) is a Banach space with the norm
\[
 \|h\|_{\mathcal{H}(A_0, A_1)}:= \sup_y \|h(iy)\|_{A_0} + \sup_y \| h(1+iy)\|_{A_1}
\]

The space \(A_\theta\) is defined set-theoretically as the space of all value in \(A_0+
A_1\) that a function \(h\in \mathcal{H}(A_0, A_1)\) can take at \(\theta\in [0,1]\in S\). Therefore, set-theoretically \(A_\theta\) coincides with \(A_0\) and \(A_1\)
when \(\theta=0\) and \(\theta=1\). To define the norm on \(A_\theta\), let
\[
 \mathcal{K}_\theta(A_0, A_1):= \left\{h\in \mathcal{H}(A_0, A_1):\ h(\theta)=0 \right\}
\]
then \(\mathcal{K}_\theta(A_0,A_1)\) is a closed complex subspace of the Banach space \(\mathcal{H}(A_0, A_1)\). Then \(A_\theta:= \mathcal{H}(A_0, A_1)/
\mathcal{K}_\theta(A_0, A_1)\) has the natural quotient norm inherited from \(\mathcal{H}(A_0, A_1)\) and is still a Banach space.

It is not difficult to see that the norm on \(A_\theta\) coincides with the norm \(\|\cdot\|_{A_0} , \|\cdot\|_{A_1}\) when \(\theta=0\) or \(\theta=1\)

Theorem \ref{thm:interp-ineq-ele} follows from the this lemma when one takes \(h\) to be a
constant, and is in \(A_0\cap A_1\).

\begin{lemma}
\label{lem:interp-ineq}
If \(h\in \mathcal{H}(A_0, A_1)\) then \(\| h(\theta)\|_{A_\theta}\leq 2 M_1^{\theta}  M_0^{1-\theta}\)
where
\[
 M_0:= \sup_y \|h(iy)\|_{A_0} ,\quad M_1:=\sup_y \| h(1+iy)\|_{A_1}
\]
\end{lemma}
\begin{proof}
The \(A_\theta\)-norm of \(h(\theta)\) only depends on the value of \(h\) at \(\theta\), one can therefore replace \(h\) by a function of form \(h_{c,\epsilon}(z) = \exp
(c(z-\theta) + \epsilon z^2) h(z)\), then let \(\epsilon\to 0\) and choose the optimal
\(c\), which is \(e^c=M_0/M_1\).
\end{proof}

Theorem \ref{thm:interp-ineq-op} follows from Theorem \ref{thm:interp-ineq-ele} and the very
definition of quotient norm.

\begin{remark}
The optimal constant, as given by the proofs, is \(\theta^{-\theta}(1-\theta)^{\theta-1} < 2\)
\end{remark}

The interest of holomorphic interpolation theory comes from the fact that interpolation of
Sobolev spaces are still Sobolev spaces, which, together with Theorem
\ref{thm:interp-ineq-op} and Theorem \ref{thm:interp-ineq-ele}, gives a class of useful
inequalities generally called interpolation inequalities.

\begin{theorem}[Interpolation of Sobolev spaces]
\label{thm:interp-sobolev}
Let \(p,q\in (1,+\infty)\) and \(k,l\in \mathbb{R}\) and \(X=\mathbb{R}^n\). Take 
\[
 A_0:= W^{k,p}(X),\quad A_1:=W^{l,q}(X)
\]
then \(A_\theta=W^{s,r}(X)\) where
\[
 \theta l + (1-\theta)k = s,\qquad \theta \frac{1}{q} + (1-\theta) \frac{1}{p} = \frac{1}{r}
\]
\end{theorem}

The holomorphic interpolation behaves predictably with direct sum and compact operators

\begin{theorem}
\label{thm:dir-sum-interp}
Let \((A_0,A_1), (B_0, B_1)\) be interpolatable pairs and denotes by \((A\oplus
B)_\theta\) be the interpolation of \(A_0 \oplus B_0\) and \(A_1 \oplus B_1\) then
one has \((A\oplus B)_\theta \cong A_\theta \oplus B_\theta\) by a canonical isomorphism.
\end{theorem}
\begin{proof}
The set-theoretical bijection is easy to see: note that there is a natural inclusion
\((A\oplus B)_\theta \hookrightarrow A_\theta\oplus B_\theta\), which is also a
bijection because  \(\mathcal{H}(A_0\oplus B_0, A_1 \oplus B_1) = \mathcal{H}(A_0, A_1)
\oplus \mathcal{H}(B_0,B_1)\).

The most difficult part is to know what we mean by \emph{isomorphism}. In fact the two norms
(the interpolation norm and the direct-sum norm) do not coincide, but they are
equivalent. One can prove, with basic sup-inf analysis that 
\[
 \frac{1}{2}\|\cdot \|_{A_\theta\oplus B_\theta} \leq \|\cdot \|_{(A\oplus B)_\theta} \leq
\|\cdot\|_{A_\theta \oplus B_\theta}
\]
\end{proof}

Theorem \ref{thm:dir-sum-interp} can be generalised to the following result.

\begin{theorem}[*]
\label{thm:interp-closed-emb}
Let \((X_0, X_1)\) and \((Y_0, Y_1)\) be interpolatable pairs. Suppose that there are
inclusion
\(X_0 \hookrightarrow  Y_0\) and \(X_1 \hookrightarrow  Y_1\)
with closed images in \(Y_0\) and \(Y_1\) respectively and the inclusions agree on \(X_0 \cap X_1\) as mappings from \(X_0\cap X_1\) to \(Y_0 + Y_1\). Moreover, suppose
that the image of \(X_0+X_1\) in \(Y_0+Y_1\) is closed. Then there is a
natural inclusion \(X_\theta \hookrightarrow Y_\theta\) with closed image in \(Y_\theta\)
\end{theorem}
\begin{remark}
\begin{enumerate}
\item The condition \(X_0+X_1 \hookrightarrow Y_0+Y_1\) being of closed image is redundant
if \(X_1 \hookrightarrow X_0\) and \(Y_1 \hookrightarrow Y_0\), as in the case of
interpolation of \emph{certain} Sobolev spaces on manifolds. In general, one can also check that this condition holds for the
maps \(\iota_{k,p}\) and \(\iota_{l,q}\) in Definition \ref{def:sobolev-space} of
Sobolev spaces using the fact that they admit left-inverse given by \(\{\tilde \psi_i\}\). See Remark \ref{rem:hamilton-typo}.
\item If one has two exact sequences
\begin{equation}
\label{eq:ses-interp}
0 \longrightarrow X_i \longrightarrow Y_i \longrightarrow Z_i \longrightarrow 0,\quad i=0,1
\end{equation}
whose arrows commute with ones from the intersection and ambient spaces of
interpolatable pairs \((X_0,X_1), (Y_0, Y_1), (Z_0, Z_1)\) then, since the images of
\(X_i \longrightarrow Y_i\) being kernel of \(Y_i \longrightarrow Z_i\) are
closed, one has the inclusion for interpolation spaces, also of closed image:
\[
    0 \longrightarrow X_\theta \longrightarrow Y_\theta,\quad 0\leq\theta\leq 1.
   \]
\item In particular, if the sequences in \eqref{eq:ses-interp} split, meaning that one can find
a retraction \(0 \longrightarrow Z_i \longrightarrow  Y_i\), then by applying the
theorem for the retractions, one sees that the interpolation sequence extend to \(Z_\theta\), i.e.
\[
    0 \longrightarrow X_\theta \longrightarrow Y_\theta \longrightarrow Z_\theta \longrightarrow 0
   \]
and also splits, meaning \(Y_\theta \cong X_\theta \oplus Z_\theta\). Applying this
results to the split-exact sequences \[ 0 \longrightarrow A_i \longrightarrow A_i\oplus
   B_i \longrightarrow B_i \longrightarrow 0\] one then obtains Theorem \ref{thm:dir-sum-interp}.
\end{enumerate}
\end{remark}

\begin{proof}
The inclusion \(X_\theta \hookrightarrow Y_\theta\) is natural and due to the fact that
\(\mathcal{H}(X_0, X_1) \subset \mathcal{H}(Y_0, Y_1)\). The equivalence of the
interpolation norm \(X_\theta\) and the norm inherited from \(Y_\theta\) on \(X_\theta\) requires more than a simple sup-inf analysis as in the proof
of Theorem \ref{thm:dir-sum-interp} since \(\mathcal{H}(X_0, X_1)\) is strictly included
in \(\mathcal{H}(Y_0, Y_1)\). What we can say is that the interpolation norm \(X_\theta\) dominates the interpolation norm of \(Y_\theta\), since it involves the infimum on
the smaller set. In other words, it means that the inclusion \(X_\theta \hookrightarrow
Y_\theta\) is continuous. It remains to check that the image of \(X_\theta \hookrightarrow Y_\theta\) is closed.

Since
\[
 \xymatrix{
X_\theta \ar@{->}[r] & Y_\theta \\
\mathcal{H}(X_0, X_1)/\mathcal{K}_\theta(X_0, X_1) \ar@{=}[u] & \mathcal{H}(Y_0,Y_1)/\mathcal{K}_\theta(Y_0, Y_1) \ar@{=}[u] \\
\mathcal{H}(X_0, X_1) \ar@{->}[r] \ar@{->>}[u] & \mathcal{H}(Y_0, Y_1) \ar@{->>}[u]
}
\]
it suffices to show that the image \(\mathcal{H}(X_0,X_1) \hookrightarrow
\mathcal{H}(Y_0,Y_1)\) is closed, meaning if \(\mathcal{H}(X_0,X_1)\ni h_n \to h\) in \(\mathcal{H}(Y_0,Y_1)\), then \(h\) must take value in \(X_0 + X_1\). This is easy to
verify on \(\partial S\): By the
equivalence of the norm on \(X_i\) and the restricted norm from \(Y_i\), \(i=0,1\),
one sees that \(h(iy)\in X_0\) and \(h(1+iy)\in X_1\).

Since \(X_0+X_1\) is
closed in \(Y_0+Y_1\), any holomorphic map \(\mathcal{H}(Y_0,Y_1)\ni f: S
\longrightarrow Y_0+Y_1\) passes holomorphically to the quotient \(S \longrightarrow (Y_0
+Y_1) /(X_0+X_1)\). The fact that \(h\) takes value in \(X_0 + X_1\) follows from
Maximum modulus principle for holomorphic functions.
\end{proof}


\begin{theorem}[Interpolation of compact embedding]
\label{thm:compact-interp}
If \(A_1 \hookrightarrow A_0\) is a compact embedding, then \(A_1 \cong A_\theta\cap A_1
\hookrightarrow A_\theta\) is a compact embedding where the first \(\cong\) denotes the
same space with equivalent norms.
\end{theorem}
\begin{proof}
It follows from Theorem \ref{thm:interp-ineq-ele}:
\[
 \|x_m - x_n\|_{A_\theta}\leq 2 \|x_m-x_n\|^{1-\theta}_{A_0} \|x_m-x_n\|_{A_1}^\theta
\]
Hence if \(\{x_n\}\) is a bounded sequence in \(A_1\), it converges in \(A_0\) and
therefore \(A_\theta\).
\end{proof}

The previous Theorem \ref{thm:interp-ineq-ele}, together with Theorem \ref{thm:interp-sobolev}
also gives a proof of Kondrachov's Theorem, that is the embedding \(W^{k,p}(X)
\hookrightarrow  W^{l,p}(X)\) is compact if \(k>h>\geq 0\). This follows from the
following 2 remarks
\begin{enumerate}
\item The case \(l=0\) and \(k\gg 1\) follows from the embedding \(W^{k,p}
   \hookrightarrow C^1\) and Ascoli's theorem. Hence by Theorem \ref{thm:interp-ineq-ele},
one has the compactness embedding if \(k \gg 1\) and \(l<k\).
\item For the case of small \(k\), note that 
\[ 
   W^{k+r,p}(X) \twoheadrightarrow W^{k,p}(X):\ v \longmapsto W(D)^r u 
   \]
is surjective and any \(u\in W^{k,p}(X)\) can be lifted to an element \(\tilde u\in
   W^{k+r,p}(X)\) of the same norm. In fact, if \(W(\xi)^k \hat u \in L^p\) then choose
\(\tilde u\) such that \(\widehat{\tilde u} = W(\xi)^{-r} \hat u\). Kondrachov's
theorem follows from the diagram:
\[
    \xymatrix{
    W^{k+r,p}(X) \ar@{->>}[r] \ar@{^{(}->}[d]_{\rm compact} & W^{k,p}(X) \ar@{^{(}->}[d] \\
    W^{h+r,p}(X) \ar@{->>}[r] & W^{h,p}(X)
    }
   \]
\end{enumerate}

\begin{remark}
The advantage of this proof is that it is valid for weighted Sobolev spaces over manifolds.
\end{remark}

\section{Sobolev spaces on compact manifold without boundary}
\label{sec:orgb45ccd2}
Let \(M\) be a compact manifold without boundary. We fix a finite atlas of \(M\) by
chart \(\varphi_i:\ M\supset U_i \longrightarrow V_i \subset \mathbb{R}^n\) such that
the transitions \(\varphi_{ij}=\varphi_i \circ \varphi_j^{-1}: V_j \longrightarrow V_i\)
are of strictly positive and bounded derivatives, i.e. \(C(\alpha)^{-1}\leq D^\alpha
\varphi_{ij}\leq C(\alpha)\) for all indices \(\alpha\). We will called such atlas a
\emph{good atlas}. 
One can always obtain such atlas by shrinking a bit each chart of a given atlas of \(M\). Let \(\psi_i\) be a
partition of unity subordinated to \(\{U_i\}\)


\begin{definition}
\label{def:sobolev-space}
\begin{enumerate}
\item The \textbf{Sobolev spaces} \(W^{k,p}(M)\) is defined as
\[
   W^{k,p}(M):=\left\{ f\in \mathcal{S}(M)^*:\ (\psi_i f)\circ\varphi_i^{-1}\in W^{k,p}(\mathbb{R}^n)\right\}
   \]
with the norm
\[
    \|f\|_{W^{k,p}}= \sum_i \|(\psi_i f)\circ \varphi_i^{-1}\|_{W^{k,p}(\mathbb{R}^n)}
   \]
\item Weighted Sobolev spaces can be defined when \(M\) has a foliation structure, i.e. \(M\) is locally modeled by
\(0\subsetneq F_1\subsetneq \dots \subset F_k \subsetneq \mathbb{R}^n\) where
\(F_i\) are vector subspace of \(\mathbb{R}^n\) of dimension
\(0<n_1 <\dots < n_k< n\) respectively and \(F_k\) are preserved by the transition
maps \(\varphi_{ij}\), for example when \(M\) is a product of manifolds of lower
dimension. Then the above definition extends to weighted Sobolev spaces with weight
\(\sigma_1= \dots = \sigma_{n_1}\), \(\sigma_{n_1+1} =\dots = \sigma_{n_2},\dots\)
\(\sigma_{n_k+1}=\dots = \sigma_{n}}\).
\end{enumerate}
\end{definition}

\begin{remark}
\begin{enumerate}
\item One can define \(\mathcal{S}(M)^*\) as the dual space of
\(\mathcal{S}(M) = C^\infty(M)\) under Schwartz topology with respect to any metric,
because by compactness any two metrics on \(M\) are comparable. The distributions 
\(\psi_i f\) are tempered because they are compactly supported.
\item One can identify \(C^\infty(M)\) with a subspace of \(\mathcal{S}^*(M)\) that is
contained in any Sobolev space \(W^{k,p}(M)\) by fixing a Riemannian metric \(g\)
on \(M\). The map \(C^\infty(M) \hookrightarrow \mathcal{S}^*(M)\) may depend on
\(g\), but its image does not. Similarly, one can also identify an element of \(W^{k,p}(\mathbb{R}^n)\) supported in \(V_i\) with an element in \(W^{k,p}(M)\).
\item If one uses another good atlas \(U'_i\) or a different partition of unity, one
obtains the same set \(W^{k,p}(M)\) and an equivalent norm. To see this, let us call
two good atlas \emph{compatible} if their union is also a good atlas, then the statement
holds for two compatible atlas by comparing their union. Moreover, for any two
arbitrary good atlas \(\{U_i\},\{U'_j\}\), one can find a good atlas compatible with
both of them by shrinking their union.
\end{enumerate}
\end{remark}

By definition, one has an inclusion \(\iota:\ W^{k,p}(M) \hookrightarrow \bigoplus_i
W^{k,p}(\mathbb{R}^n)\). Also \(\iota\) is of closed image because one can find a
projection \(\pi:\ \bigoplus_i W^{k,p}(\mathbb{R}^n) \longrightarrow W^{k,p}(M)\) with \(\pi\circ \iota = {\rm Id}\). In fact, let \(\tilde \psi_i\) be functions supported in \(U_i\) that equal \(1\) in the support of \(\psi_i\), then 
\[ 
\pi:\ g\mapsto \sum \tilde \psi_i. (g\circ \varphi_i) 
\]
works. The continuity of \(\pi\) follows from straight-forward calculations.

The closedness of image of \(\iota\) is equivalent to the fact that \(W^{k,p}(M)\) is complete.

\begin{remark}
\label{rem:hamilton-typo}
Although \(\iota\) preserves the norm of \(W^{k,p}(M)\) and has a right-inverse, it is far
from being an isomorphism (it is not surjective). Each summand of an element in the
image of \(\iota\) tends to 0 on the boundary of \(V_i\) (take \(k \gg 1\) then everyone
is continuous by Sobolev embedding, there is no subtlety in what we mean by "tends to
0"). \cite[page 54]{hamilton_harmonic_1975} seems to claim that \(\iota\) is an
isomorphism and apply Theorem \ref{thm:dir-sum-interp}
repeatedly to deduce Theorem \ref{thm:interp-sobolev} for Sobolev spaces on manifold,
then the Sobolev embedding \(W^{k,p} \hookrightarrow C^{l}(M)\) and Kondrachov's
theorem.

The above results are true and the correction is not difficult (use Theorem \ref{thm:interp-closed-emb}).
\end{remark}

From the remark, one has
\begin{theorem}[Interpolation of Sobolev spaces on manifold]
\label{thm:interp-sobolev-M}
Theorem \ref{thm:interp-sobolev} holds for Sobolev spaces \(W^{k,p}(M)\) on compact
manifold \(M\).
\end{theorem}


\section{Sobolev spaces on compact manifold with boundary}
\label{sec:orgd932199}

In this part, we will define the Sobolev spaces \(W^{k,p}(M/ \mathcal{A})\) where \(k\in
\mathbb{R}, p\in(1, \infty)\) and \(M\)
is a manifold with boundary and \(\mathcal{A}\) is union of connected components of \(\partial M\) the boundary of \(M\). These spaces contain \(W^{k,p}(M)\) "functions" who vanish
on \(\mathcal{A}\). The motivation is that we will later
take \(M = M'\times [0,T]\) where \(M'\) is a manifold without boundary where we want
to solve heat equation, and the natural \(\mathcal{A}\) would be \(M \times
\{0\}\). We also want that the new definition coincides with the case of no boundary
when \(\mathcal{A} = \emptyset\)


Suppose that we already define the Sobolev spaces on \(X\times Y^+\) where
\(X = \mathbb{R}^n\) and \(Y^+ = \mathbb{R}_{\geq 0}\), that is the space \(W^{k,p}(X\times Y^+) = W^{k,p}(X\times Y^+/\emptyset)\) and \(W^{k,p}(X\times Y^+, X\times \{0\})\). Then then we define the space
\(W^{k,p}(M/ \mathcal{A})\) in analog of Definition \ref{def:sobolev-space} as follows

\begin{definition}
\label{def:sobolev-space-b}
\begin{enumerate}
\item The \textbf{Sobolev spaces} \(W^{k,p}(M/ \mathcal{A})\) where \(A\) is a connected
component of \(\partial M\) is defined as
\[
   W^{k,p}(M/ \mathcal{A}):=\left\{ f\in \mathcal{S}(M)^*:\ (\psi_i f)\circ
   \varphi_i^{-1}\in W^{k,p}(R_i/\mathcal{A}_i)\right\}
   \]
where \(\mathcal{A}_i = \varphi_i(U_i \cap \mathcal{A})\) and \(R_i\) is the
Euclidean space containing \(\varphi(U_i)\), that is either \(\mathbb{R}^{n+1}\) when \(\mathcal{A}_i=\emptyset\) or \(\mathbb{R}^{n}\times \mathbb{R}_{\geq 0}\) when \(\mathcal{A}_i\subset
   \mathbb{R}^n\times \{0\}\). The norm is given by
\[
    \|f\|_{W^{k,p}}= \sum_i \|(\psi_i f)\circ \varphi_i^{-1}\|_{W^{k,p}(R_i/\mathcal{A}_i)}
   \]
\item As before, weighted Sobolev spaces can be defined when \(M\) has a foliation
structure compatible with its boundary.
\end{enumerate}
\end{definition}

The fact that different good atlas and different partition of unity defines the same space
\(W^{k,p}(M/\mathcal{A})\) (as a subset of \(\mathcal{S}^*(M)\)) with equivalents norm
comes from the following lemma, which is just a formulation of arguments in the case of no
boundary. For the proof, one reduces the lemma, by interpolation
inequality, to the case \(k\) is a multiple of \(\sigma\) and use the criteria in
Theorem \ref{thm:equiv-norm-Sobolev} and the boundedness of derivative
of the transition map.

\begin{lemma}
\label{lem:equiv-norm-sobolev}
Let \((U,\mathcal{A}_U)\) and \((V,\mathcal{A}_V)\) be subsets of \((X\times Y^+,
X\times \{0\})\) and \(\varphi_{VU}:\ (U, \mathcal{A}_U) \longrightarrow (V, \mathcal{A}_V)\)
being a diffeomorphism between \(U\) and \(V\) mapping \(\mathcal{A}_U\subset \partial U\) to \(\mathcal{A}_V \subset \partial V\)
bijectively and of bounded derivatives. Let \(0\leq \psi \leq 1\) be a smooth function compactly supported in \(V\). Then the linear mapping \(T: \mathcal{S}^*(X\times Y^+/X\times \{0\}) \longrightarrow
\mathcal{S}^*(X\times Y^+/X\times\{0\}):\ f \longrightarrow \psi. (f\circ
\varphi^{-1}_{VU})\) extends to a bounded operator from \(W^{k,p}(U/\mathcal{A}_U)
\longrightarrow W^{k,p}(V, \mathcal{A}_V)\).
\end{lemma}

We will sketch
rapidly the (well known) ideas to define Sobolev spaces on half-plan and the trace
operator in the next sections.

\subsection{Sobolev spaces on half-plan}
\label{sec:org043596f}
In this section, the Sobolev spaces on \(X\times Y\) or \(X\times Y^+\) are defined
with weight \((\sigma_1,\dots,\sigma_n,\rho)\) and \(\sigma:= \lcm (\sigma_1,\dots,\sigma_n,\rho)\).
\subsubsection{Smooth extensions}
\label{sec:orge91e420}
Let \(\mathcal{S}(X\times Y^+)\) denote the space of
smooth, rapidly decreasing functions (and all of their derivatives) on \(X\times Y^+\)
and  \(\mathcal{S}(X\times Y^+/0)\) denotes the subspace of functions who vanish, together with
all their derivatives, at \(X\times \{0\}\). Similar definition for \(\mathcal{S}(X\times Y^-)\) and
\(\mathcal{S}(X\times Y^-/0)\). The following exact sequence is obvious and the arrows are continuous under Schwartz topology.
\begin{equation}
\label{eq:ses-S}
 \xymatrix{
0 \ar@{->}[r] & \mathcal{S}(X\times Y^-/0) \ar@{->}[r]^{Z_-} & \mathcal{S}(X\times Y) \ar@{->}[r]^{C_+} & \mathcal{S}(X\times Y^+) \ar@{->}[r] & 0
}
\end{equation}
where \(Z_-\) be the extension by \(0\) and \(C_+\) be the cut-off operator. 

It is however not obvious that the sequence in \eqref{eq:ses-S} splits. Algebraically this
is equivalent to the fact that \(C_+\) admits a retraction, that we will note by \(E_+\) since it is
in fact an extension to the negative half-plan, which is continuous
under Schwartz topology. The construction of \(E_+\) is as follows
\begin{align*}
  E_+: \mathcal{S}(X\times Y^+) &\longrightarrow \mathcal{S}(X\times Y) \\
  	f 		     	  &\longmapsto	  \left( (x,y) \longmapsto \begin{cases}
								       f(x,y),  & \text{if $y\geq 0$} \\
								       \int_0^\infty \varphi(\lambda) f(x,-\lambda y) d\lambda, & \text{if $y<0$}
								       \end{cases} \right)
\end{align*}
where the difficult part is the choice of \(\varphi\), which is resolved by
the following lemma.

\begin{lemma}
\label{lem:construction-varphi}
There exists a smooth function \(\varphi:\ \mathbb{R}_{\geq 0} \longrightarrow
\mathbb{R}\) such that
\(\int_0^{+\infty} x^n |\varphi(x)|dx <\infty \quad\forall n\in \mathbb{Z}\)
and 
\[
  \int_0^{+\infty} x^n \varphi(x)dx = (-1)^n \quad\forall n\in \mathbb{Z}\setminus\{0\}
\]
Moreover, \(\varphi(\frac{1}{x}) = -x \varphi(x)\) for all \(x>0\).
\end{lemma}

In fact, the function
\[
 \varphi(x) = \frac{e^4}{\pi}.\frac{e^{-(x^{1/4} + x^{-1/4})} \sin(x^{1/4} - x^{-1/4})}{1+x}
\]
works. The continuity of operator \(E_+\) comes from these properties of \(\varphi\)
and basic justification of Lebesgue's Dominated convergence. The projection \(R_-\) of
\(Z_-\) in the sequence \eqref{eq:ses-S} is constructed algebraically:
\begin{align*}
  R_-\ \mathcal{S}(X\times Y) &\longrightarrow \mathcal{S}(X\times Y^-/0)\\
       f &\longmapsto f- E_+C_+ f
\end{align*}
which is also continuous in Schwartz topology. To resume, one has the split exact sequence
\begin{equation}
\label{eq:ses-S-split}
\xymatrix{
0 \ar@{->}[r] & \mathcal{S}(X\times Y^-/0) \ar@/^/@{->}[r]^{Z_-} & \mathcal{S}(X\times Y) \ar@/^/@{->}[r]^{C_+} \ar@/^/@{->}[l]^{R_-} & \mathcal{S}(X\times Y^+) \ar@{->}[r] \ar@/^/@{->}[l]^{E_+} & 0
}
\end{equation}
and a similar sequence for \(\mathcal{S}(X\times Y^+/0\) and \(\mathcal{S}(X\times Y^-)\) with
operators \(Z_+, C_-, E_-\) and \(R_+\).


Also, note that
\begin{equation}
\label{eq:Eplus-Rplus}
\langle E_+ f, g \rangle = \langle f, R_+ g \rangle
\end{equation}
where the first pairing is on \(\mathcal{S}(X\times Y) \times \mathcal{S}(X\times Y)\)
and the second is on \(\mathcal{S}(X\times Y^+)\times \mathcal{S}(X\times Y^+/0)\).

\begin{remark}
\label{rem:coupling-S}
\begin{enumerate}
\item The two pairings satisfy \(\langle D^\alpha u, v \rangle = (-1)^{|\alpha|} \langle u,
   D^\alpha v \rangle\).
\item The second pairing gives two natural identifications
\[
    \mathcal{S}(X\times Y^+/0) \hookrightarrow \mathcal{S}^*(X\times Y^+),\quad
   \mathcal{S}(X\times Y^+) \hookrightarrow \mathcal{S}^*(X\times Y^+/0)
   \]
while the first pairing gives \(\mathcal{S}(X\times Y) \hookrightarrow
   \mathcal{S}^*(X\times Y)\).
\item \eqref{eq:Eplus-Rplus} shows that \(E_+\) and \(R_+\) are adjoint, strictly speaking
\(E_+\) is the restriction of \(R_+^*\), that is
\[
    \xymatrix{
    \mathcal{S}(X\times Y^+) \ar@{^{(}->}[d] \ar@{->}[r]^{E_+} & \mathcal{S}(X\times Y) \ar@{^{(}->}[d] \\
    \mathcal{S}^*(X\times Y^+/0) \ar@{->}[r]^{R_+^*} & \mathcal{S}^*(X\times Y)
    }
   \]
\end{enumerate}
\end{remark}

Similarly. since \(\langle C_- f, g \rangle = \langle f, Z_- g \rangle\), one has
\[
     \xymatrix{
    \mathcal{S}(X\times Y^-/0) \ar@{^{(}->}[d] \ar@{->}[r]^{Z_-} & \mathcal{S}(X\times Y) \ar@{^{(}->}[d] \\
    \mathcal{S}^*(X\times Y^-) \ar@{->}[r]^{C_-^*} & \mathcal{S}^*(X\times Y)
    }
\]

To resume, one can extend the sequence in \eqref{eq:ses-S} to the following diagram
\begin{equation}
\label{eq:ses-S-Sstar}
\xymatrix{
0 \ar@{->}[r] & \mathcal{S}(X\times Y^-/0) \ar@{^{(}->}[d] \ar@/^/@{->}[r]^{Z_-} & \mathcal{S}(X\times Y) \ar@{^{(}->}[d] \ar@/^/@{->}[r]^{C_+} \ar@/^/@{->}[l]^{R_-} & \mathcal{S}(X\times Y^+) \ar@{^{(}->}[d] \ar@{->}[r] \ar@/^/@{->}[l]^{E_+} & 0 \\
0 \ar@{->}[r] & \mathcal{S}^*(X\times Y^-) \ar@/^/@{->}[r]^{C_-^*} & \mathcal{S}^*(X\times Y) \ar@/^/@{->}[r]^{Z_+^*} \ar@/^/@{->}[l]^{E_-^*} & \mathcal{S}^*(X\times Y^+/0) \ar@{->}[r] \ar@/^/@{->}[l]^{R_+^*} & 0
}
\end{equation}

We will define Sobolev spaces \(W^{k,p}(X\times Y^-/0)\) and \(W^{k,p}(X\times Y^+)\)
so that they form an intermediate row in diagram. Since the center cell \(\mathcal{S}(X\times Y)\subset W^{k,p}(X\times Y) \subset \mathcal{S}^*(X\times Y)\) is
already defined, there is only one natural way to do this.

\begin{definition}
\begin{enumerate}
\item The \textbf{Sobolev space upper on half-plan} is \[ W^{k,p}(X\times Y^+) := \left\{ f\in
   \mathcal{S}^*(X\times Y^+/0):\ \exists g\in W^{k,p}(X\times Y), f= Z_+^* g\right\}\] with norm
\(\|f\|_{W^{k,p}(X\times Y^+)} = \inf_g \|g\|_{W^{k,p}(X\times Y)}\).
\item The \textbf{Sobolev space on lower half-plan with vanishing trace} \[ W^{k,p}(X\times Y^-/0):=
   \left\{f\in \mathcal{S}^*(X\times Y^-):\ C_-^* f\in W^{k,p}(X\times Y)\right\}\] with
the induced norm \(\|f\|_{W^{k,p}(X\times Y^-/0)}:= \|C^*_- f\|_{W^{k,p}(X\times Y)}\).
\end{enumerate}
\end{definition}

\begin{remark}
\begin{enumerate}
\item In other words, \(W^{k,p}(X\times Y^-/0) = {C^*_-}^{-1}(W^{k,p}(X\times Y))\) and
\(W^{k,p}(X\times Y^+) = Z^*_+ (W^{k,p}(X\times Y))\) and they are given by the induced
norm and the quotient norm of \(W^{k,p}(X\times Y)\) respectively. The operator
\(C_-^*\) and \(Z_+^*\) are by definition bounded under Sobolev norm.
\item The topology of \(W^{k,p}(X\times Y)\) being finer than the induced of weak-* topology from \(\mathcal{S}^*(X\times Y)\), the restricted operator \(\restr{Z_+^*}{W^{k,p}}: W^{k,p}(X\times Y)
   \longrightarrow \mathcal{S}^*(X\times Y^+/0)\) is continuous, hence \(\ker
   \restr{Z_+^*}{W^{k,p}}\subset W^{k,p}(X\times Y)\) is a closed subspace of the Banach space \(W^{k,p}(X\times Y)\). But this is also the image by \(C_-^*\) of
\(W^{k,p}(X\times Y^-/0)\). Therefore \(W^{k,p}(X\times Y^-/0)\) and \(W^{k,p}(X\times
   Y^+)\) are Banach spaces.
\item Idem for \(W^{k,p}(X\times Y^+/0)\) and \(W^{k,p}(X\times
   Y^-)\).
\end{enumerate}
\end{remark}


\begin{theorem}
\label{thm:sobolev-def-ses}
\begin{enumerate}
\item For all \(k\in \mathbb{R}\) and \(p\in (1,\infty)\), the three lines of the following diagram are split-exact and the arrows of the second
lines are bounded operators under Sobolev norms.
\begin{equation}
\label{eq:ses-S-W-Sstar}
\xymatrix{
0 \ar@{->}[r] & \mathcal{S}(X\times Y^-/0) \ar@/^/@{->}[r]^{Z_-} \ar@{^{(}->}[d] & \mathcal{S}(X\times Y) \ar@/^/@{->}[r]^{C_+} \ar@/^/@{->}[l]^{R_-} \ar@{^{(}->}[d] & \mathcal{S}(X\times Y^+) \ar@{->}[r] \ar@/^/@{->}[l]^{E_+} \ar@{^{(}->}[d] & 0 \\
0 \ar@{->}[r] & W^{k,p}(X\times Y^-/0) \ar@{^{(}->}[d] \ar@/^/@{->}[r]^{C_-^*} & W^{k,p}(X\times Y) \ar@{^{(}->}[d] \ar@/^/@{->}[r]^{Z_+^*} \ar@/^/@{->}[l]^{E_-^*} & W^{k,p}(X\times Y^+) \ar@{->}[r] \ar@/^/@{->}[l]^{R_+^*} \ar@{^{(}->}[d] & 0 \\
0 \ar@{->}[r] & \mathcal{S}^*(X\times Y^-) \ar@/^/@{->}[r]^{C_-^*} & \mathcal{S}^*(X\times Y) \ar@/^/@{->}[r]^{Z_+^*} \ar@/^/@{->}[l]^{E_-^*} & \mathcal{S}^*(X\times Y^+/0) \ar@{->}[r] \ar@/^/@{->}[l]^{R_+^*} & 0
}
\end{equation}
\item The subspaces \(\mathcal{S}(X\times Y^-/0)\) and \(\mathcal{S}(X\times Y^+)\) are dense
in \(W^{k,p}(X\times Y^0/0)\) and \(W^{k,p}(X\times Y^+)\) respectively.
\item Interpolation theorem \ref{thm:interp-sobolev} holds for \(W^{k,p}(X\times Y^-/0)\) and
\(W^{k,p}(X\times Y^+)\).
\end{enumerate}
\end{theorem}
\begin{proof}
The commutativity of the diagram is purely algebraic. The continuity of \(C^*_-\) and \(Z^*_+\) in the \(W^{k,p}\)-row follows from the definition of norms in this row. The
only non-trivial part is the continuity of \(E^*_-\) and \(R^*_+\) in the \(W^{k,p}\)-row, and it suffices to prove that \(C^*_- E^*_-\) and \(R^*_+ Z^*_+\) are bounded
as automorphism of \(W^{k,p}(X\times Y)\). This follows from direct computation of these
norm in the case \(\sigma \mid k\in \mathbb{R}\) and interpolation inequality (Theorem
\ref{thm:interp-ineq-op}) for intermediate \(k\).

Once the continuity of \(E^*_-\) and \(R_+^*\) is established, the density of \(\mathcal{S}(X\times Y^-/0)\) follows straight-forwardly and we see that \(W^{k,p}(X\times Y^-/0)\) and \(W^{l,p}(X\times Y^-/0)\) are interpolatable (the two
spaces share a dense subspace). Theorem \ref{thm:interp-closed-emb} applies and shows that
Theorem \ref{thm:interp-sobolev} holds for \(W^{k,p}(X\times Y^-/0)\). 

Idem for the side of \(\mathcal{S}(X\times Y^+)\subset W^{k,p}(X\times Y^+)\).
\end{proof}

\begin{remark}
By dualising the diagram \eqref{eq:ses-S-W-Sstar} and using the fact that the dual space of
\(W^{k,p}(X\times Y)\) is \(W^{-k, p'}(X\times Y)\), one can prove that the dual space
of \(W^{k,p}(X\times Y^+)\) is \(W^{-k, p'}(X\times Y^+/0)\). 
\end{remark}

\subsubsection{Functoriality of \(D_x\) and equivalent definitions}
\label{sec:org163f9fc}
The following discussion appeared as 4 lemmas in \cite[page
38-42]{hamilton_harmonic_1975} in the proof of Vanishing trace theorem \ref{thm:vanishing-trace}. I
think these ideas can be presented without much computation.

Note that the weight \(W(\xi,\eta) = \left(1 +\xi_1^{2\sigma_1} +\dots +
\xi_n^{2\sigma_n} +\eta^{2\rho} \right)^{1/2\sigma}\) is comparable to \(W(\xi) +
W(\eta)\) where
\[
 W(\xi) = \left(1 +\xi_1^{2\sigma_1} +\dots +
\xi_n^{2\sigma_n} \right)^{1/2\sigma},\quad W(\eta) = \left(1 +\eta^{2\rho} \right)^{1/2\sigma}
\]
and also \(W^k(\xi,\eta)\) is comparable to \(W(\xi)^k + W(\eta)^k\). Hence \(W(D_x)^l:\ W^{k,p}(X\times Y) \longrightarrow W^{k-l,p}(X\times Y)\) is a bounded
operator.

The vertical arrows in the following diagram are the vertical arrows of
\eqref{eq:ses-S-W-Sstar}. The dashed horizontal arrow indicates that it is established only
in the center cells \(W^{k,p}(X\times Y) \longrightarrow W^{k-l,p}(X\times Y)\).
\begin{equation}
\label{eq:S-W-S-3d}
\xymatrix{
 & \mathcal{S}-{\rm row} \ar@{->}[ld] \ar@{->}[rd] &  \\
W^{k,p}-{\rm row} \ar@{->}[rd] \ar@{-->}[rr]^{W(D_x)^l} &  & W^{k-l,p}-{\rm row} \ar@{->}[ld] \\
 & \mathcal{S}^*-{\rm row} & 
}
\end{equation}

We will see that the dashed arrow can be extended to a full arrow, that is 3 arrows
between the \(W^{k,p}\)-row and \(W^{k-l,p}\)-row that are compatible with the diagram
\eqref{eq:ses-S-W-Sstar}.

One can construct \(W(D_x)^l\) arrows from \(W^{k,p}(X\times Y^-/0)
\longrightarrow W^{k-l,p}(X\times Y^-/0)\) and \(W^{k,p}(X\times Y^+)
\longrightarrow W^{k-l,p}(X\times Y^+)\) as adjoint of \(W(D_x)^l\) on
\(\mathcal{S}(X\times Y^+)\) and \(\mathcal{S}(X\times Y^+/0)\). They are by definition continuous on the
weak-* topology. It is easy to see that if we can prove that these two \(W(D_x)^l\) arrows commute with \(C^*_-, E^*_-\) and \(Z^*_+, R^*_+\) on \(W^{k,p}\)-row and \(W^{k-l,p}\)-row, then
by the continuity of the \(W(D_x)^l\) arrow from \(W^{k,p}(X\times Y) \longrightarrow
W^{k-l,p}(X\times Y)\), these \(W(D_x)^l\) arrows are bounded in \(W^{k,p}\) norm.

The two new \(W(D_x)^l\) arrows commute with all "\(\longrightarrow\)" arrows in the \(W^{k,p}\)-row of \eqref{eq:ses-S-W-Sstar}, i.e. \(C^*_-\) and \(Z^*_+\), since for
smooth functions, \(D_x\) commutes with \(Z_-\) (extension by 0) and \(C_+\)
(cut-off).

The fact that \(W(D_x)^l\) commutes with the "\(\longleftarrow\)" arrows,
i.e. \(E^*_-\) and \(R^*_+\) is due to:

\(\xymatrix{
\mathcal{S}(X\times Y^+) \ar@{->}[d]^{W(D_x)^l} \ar@{->}[r]^{E_+} & \mathcal{S}(X\times Y) \ar@{->}[d]^{W(D_x)^l} \\
\mathcal{S}(X\times Y^+) \ar@{->}[r]^{E_+} & \mathcal{S}(X\times Y)
}\)
and
\(\xymatrix{
\mathcal{S}(X\times Y) \ar@{->}[d]^{W(D_x)^l} \ar@{->}[r]^{R_-} & \mathcal{S}(X\times Y^-/0) \ar@{->}[d]^{W(D_x)^l} \\
\mathcal{S}(X\times Y) \ar@{->}[r]^{R_-} & \mathcal{S}(X\times Y^-/0)
}\)

\begin{remark}
\label{rem:funct-Dy}
There is no functoriality of \(D_y\) since for \(y<0\)
\[
 D_y^l E_+ f(x,y) = \int_0^\infty (-\lambda)^l \varphi(\lambda)D_y^l f(x,-\lambda
y)d\lambda \ne E_+ D_y^l f(x,y)
\]
meaning that the \(D_y\) does not commute with \(E_+\).

However \(D_y^l E_+ f\in L^p(X\times Y)\) if and only if \(E_+ D^l_y f\in L^p(X\times
Y)\) if and only if \(D^l_y f  \in L^p(X\times Y^+)\). Moreover the 3 \(L^p\) norms
are equivalent.
\end{remark}


The density of \(\mathcal{S}(X\times Y^-/0)\) and \(\mathcal{S}(X\times Y^+)\) in the
corresponding \(W^{k,p}\) shows that the new \(W^{k,p}\) spaces can also be defined by
density using the \(W^{k,p}\)-norm of the extension (\(Z_-\) and \(E_+\) respectively) from half-plan to the whole plan. By the continuity of \(R_+^*\) in the second row of \eqref{eq:ses-S-W-Sstar} when \(k=0\), one sees that
the \(L^p\)-norms of the extensions by \(Z_-\) and \(E_+\) are equivalent to the \(L^p\) norm on the half-plan. Therefore, one has the following analog of Theorem \ref{thm:equiv-norm-Sobolev}.

\begin{theorem}
\label{thm:descrpt-sobolev}
Given \(k>0\) and \(\sigma \mid k\),
\begin{enumerate}
\item If \(f\in \mathcal{S}^*(X\times Y^+/0)\) then 
\begin{enumerate}
\item \(f\in W^{k,p}(X\times Y^+)\) if and only if \(D^\alpha_x D^\beta_y f\in
      L^p(X\times Y^+)\) for \(\|(\alpha,\beta)\|\leq k\).
\item \(f\in W^{-k,p}(X\times Y^+)\) if and only if there exists \(g_{\alpha\beta}\in
      L^p(X\times Y^+)\) such that \(f = \sum_{\|(\alpha,\beta)\|\leq k} D^\alpha_x
      D^\beta_y g_{\alpha\beta}\).
\end{enumerate}
\item If \(f\in \mathcal{S}^*(X\times Y^+)\) then
\begin{enumerate}
\item \(f\in W^{k,p}(X\times Y^+/0)\) if and only if \(D^\alpha_x D^\beta_y f\in
      L^p(X\times Y^+)\) for \(\|(\alpha,\beta)\|\leq k\).
\item \(f\in W^{-k,p}(X\times Y^+/0)\) if and only if there exists \(g_{\alpha\beta}\in
      L^p(X\times Y^+)\) such that \(f = \sum_{\|(\alpha,\beta)\|\leq k} D^\alpha_x
      D^\beta_y g_{\alpha\beta}\).
\end{enumerate}
\end{enumerate}
\end{theorem}



\subsection{Trace theorems}
\label{sec:orgfbd3028}
To make the notation more intuitive, we abusively denote the horizontal arrows in the \(W^{k,p}\)-row and the \(\mathcal{S}^*\)-row by their corresponding arrows in the \(\mathcal{S}\)-row (i.e. their restriction on the space of smooth functions), that is we will use \(Z_-, C_+, R_-, E_+\) instead of \(C_-^*,
Z_+^*, E_-^*, R_+^*\).

The goal of this section is to define the restriction of a function \(f\in
W^{k,p}(X\times Y^+)\) on \(X\times \{0\}\). The pointwise restriction of \(f\) does
not make sense because \(f\) is only defined up to a negligible set (i.e. of Lebesgue
measure 0). The strategy is to take a sequence \(f_n\in \mathcal{S}(X\times Y^+)\) that is
\(W^{k,p}\)-converging to \(f\) and to see if \(\{\restr{f_n}{X\times \{0\}}\}\)
converges in \(L^p(X\times \{0\})\). If it does one calls the limit \emph{trace} of \(f\) on
\(X\times \{0\}\). Theorem \ref{thm:vanishing-trace}, Example \ref{ex:vanishing-trace} and
Theorem \ref{thm:def-trace} show that one should expect 
\begin{itemize}
\item high regularity of \(f\), i.e. \(k\) large enough, so that the limit exists,
\item a drop of regularity of the restriction.
\end{itemize}

From diagram \eqref{eq:ses-S-W-Sstar} and its opposite version (with all \(+\) and \(-\)
signs interchanged), there is a natural inclusion \(\iota:\ W^{k,p}(X\times
Y^+/0)\) to \(W^{k,p}(X\times Y^+)\), by first extending by zero, then cutting-off
\[
 \xymatrix{
W^{k,p}(X\times Y^+/0) \ar@{^{(}->}[rr]^{\iota} \ar@{->}[rd]^{Z_+} &  & W^{k,p}(X\times Y^+) \\
 & W^{k,p}(X\times Y) \ar@{->}[ru]^{C_+} & 
}
\]

\begin{theorem}[Vanishing trace]
\label{thm:vanishing-trace}
If \(p\in (1,+\infty)\) and \(-1+ \frac{1}{p} < \rho \frac{k}{\sigma} < \frac{1}{p}\)
then \(\iota\) is an isomorphic
\end{theorem}



\begin{proof}
Define
\begin{align*}
  M_+(\lambda):\ \mathcal{S}(X\times Y^+) &\longrightarrow \mathcal{S}(X\times Y^+) \\
  		 f(x,y) &\longmapsto f(x, \lambda y)
\end{align*}
Since \(\langle M_+(\lambda)f, g \rangle = \langle f , N_+(\lambda) g \rangle\) for all
\(f\in \mathcal{S}(X\times Y^+), g\in \mathcal{S}(X\times Y^+/0)\) and \(\lambda >0\) where \(N_+(\lambda)g(x,y):=
\lambda^{-1}g(x, \lambda^{-1}y)\), one sees that \(M_+(\lambda)\) extends to \(\mathcal{S}^*(X\times Y^+/0) \longrightarrow \mathcal{S}^*(X\times Y^+/0)\) and that one
extension of it is \(N^*_+(\lambda)\) the adjoint of \(N_+(\lambda)\):
\[
 \xymatrix{
\mathcal{S}(X\times Y^+) \ar@{^{(}->}[d] \ar@{->}[r]^{M_+(\lambda)} & \mathcal{S}(X\times Y^+) \ar@{^{(}->}[d] \\
\mathcal{S}^*(X\times Y^+/0) \ar@{->}[r]^{N_+^*(\lambda)} & \mathcal{S}^*(X\times Y^+/0)
}
\]
We abusively call \(N_+^*(\lambda)\) by \(M_+(\lambda)\). We will let \(\lambda \to
+\infty\), the operator \(M_+(\lambda)\) intuitively "shrinks" to the boundary \(X\times \{0\}\).

\begin{lemma}
\label{lem:Nplus}
For \(k\geq 0, \lambda \geq 1\), \(M_+(\lambda): W^{k,p}(X\times Y^+) \longrightarrow
W^{k,p}(X\times Y^+)\) is bounded and
\[
\| M_+(\lambda) f\|_{W^{k,p}(X\times Y^+)} \leq C \lambda^{\frac{\rho
k}{\sigma}-\frac{1}{p} } \|f\|_{W^{k,p}(X\times Y^+)}
\]
where \(C\) does not depend on \(\lambda\).
\end{lemma}

The proof of the Lemma \ref{lem:Nplus} is straightforward: it suffices to prove the
boundedness in the case \(\sigma \mid k\) an use interpolation inequality
\ref{thm:interp-ineq-op}, also one can suppose that \(f\in \mathcal{S}(X\times Y^+)\). Note that \((\frac{\partial}{\partial y})^l M_+(\lambda) =\lambda^l
M_+(\lambda)(\frac{\partial}{\partial y})^l\) while
\(\frac{\partial}{\partial x}\) commutes with \(M_+(\lambda)\), hence in general \(|D_{(x,y)}^\alpha M_+(\lambda) f| \leq
\lambda^{k\rho/\sigma}|D^\alpha_{(x,y)} f|\) for all \(\|\alpha\|\leq k, \lambda \geq 1\). The \(-\frac{1}{p}\) in the exponent of \(\lambda\) is due to: \(\|M_+(\lambda)g\|_{L^p} =\lambda^{-1/p} \| g\|_{L^p}\).

Back to Theorem \ref{thm:vanishing-trace}, let \(f\in \mathcal{S}(X\times Y^+)\) and
define \(\tilde M(\lambda) f\) to be \(f\) on \(X\times Y^+\) and
\(M_-(\lambda) C_-E_+ f\) on \(X\times Y^-\), then
\(\tilde M(\lambda)f\in W^{\sigma/\rho,p}(X\times Y)\). Note that
\(D_y \tilde M(\lambda) f\) is not continuous at \(X\times \{0\}\) but is still in
\(L^p(X\times Y)\) because \(f\) and \(M_-(\lambda) C_-E_+ f\) agrees on
\(X\times \{0\}\). Suppose we can prove that as \(\lambda \to +\infty\) the sequence
\(\tilde M(\lambda) f\) converges to \(\tilde M f\) in \(W^{k,p}(X\times Y)\) then
\(C_- \tilde M f = \lim_{\lambda\to +\infty}M_-(\lambda)C_-E_+f=0\). One obtains, by
exactness of the second row of diagram \eqref{eq:ses-S-W-Sstar}, existence of a \(g\in
W^{k,p}(X\times Y^+/0)\) such that \(\tilde M f = Z_+ g\). Moreover, since \(C_+
\tilde M(\lambda) f = f\) for all \(\lambda > 0\), one has \(C_+ \tilde M f = f\), hence \(\iota g = C_+Z_+ g = C_+
\tilde M f = f\).

It remains to prove the existence of such \(\tilde M f\). By Lemma \ref{lem:Nplus} and the
fact that all \(\tilde M(\lambda)f\) are the same on \(X\times Y^+\), one has
\[
\| \tilde M(\lambda)f - \tilde M(2\lambda) f \|_{W^{k,p}(X\times Y)} \leq 2C \lambda^{\frac{\rho
k}{\sigma}-\frac{1}{p} } \|f\|_{W^{k,p}(X\times Y^+)}
\]
Therefore if \(\frac{\rho k}{\sigma}< \frac{1}{p}\), the sequence \(\tilde M (2^n) f\)
converge in \(W^{k,p}(X\times Y)\) to \(\tilde M f\).
\end{proof}

\begin{remark}
\label{ex:vanishing-trace}
If \(\rho=\sigma_i=1\) then \(\sigma=1\), take \(k=0\) then the Theorem
\ref{thm:vanishing-trace} claims that \(\mathcal{S}(X\times Y^+/0)\) is dense in \(L^P(X\times Y^+) \supset \mathcal{S}(X\times Y^+)\), or equivalently any smooth function
\(f\in \mathcal{S}(X\times Y^+)\) not necessarily vanishes on \(X\times \{0\}\) can be
\(L^p\)-approximated by smooth functions with all derivative vanishes on \(X\times
\{0\}\). This means that one cannot define any notion of trace on \(X\times \{0\}\)
that varies continuously under the \(L^p\) norm.
\end{remark}

In case of high regularity \(\frac{\rho k}{\sigma} > \frac{1}{p}\), one can define a
meaningful notion of trace.

\begin{theorem}[Well-defined trace]
\label{thm:def-trace}
If  \(\frac{\rho k}{\sigma} > \frac{1}{p}\) then the restriction map
\begin{align*}
  B: \mathcal{S}(X\times Y^+) &\longrightarrow \mathcal{S}(X)\\
     		f(x,y)	      &\longmapsto     f(x,0)	
\end{align*}
extends to a bounded operator, abusively noted by \(B:\ W^{k,p}(X\times Y^+) \longrightarrow L^p(X)\).
\end{theorem}

\begin{definition}
We call \(\partial W^{k,p}(X\times Y^+):= W^{k,p}(X\times Y^+)/\ker B\) the \textbf{space of
boundary value} of function in \(W^{k,p}(X\times Y^+)\).
\end{definition}

Theorem \ref{thm:def-trace} can be strengthen by remarking that if \(\sigma:=
\lcm(\sigma_1,\dots,\sigma_n,\rho) = \lcm (\sigma_1,\dots,\sigma_n)\) and if \(W(\xi)\)
denotes the weight \((1+\xi_1^{2\sigma_1}+\dots+\xi_n^{2\sigma_n})^{1/2\sigma}\) then \(B\) and \(W(D_x)\) commute, i.e.
\[
 \xymatrix{
W^{k,p}(X\times Y^+) \ar@{->}[d]_{W(D_x)^l} \ar@{->}[r]^{B} & L^p(X)\subset \mathcal{S}^*(X) \ar@{->}[d]^{W(D_x)^l} \\
W^{k-l,p}(X\times Y^+) \ar@{->}[r]^{B} & L^p(X)
}
\]
as long as \(\frac{\rho(k-l)}{\sigma} > \frac{1}{p}\). Therefore, one has

\begin{theorem}[Regularity of trace]
\label{thm:trace-reg}
If \(0\leq l < k- \frac{\sigma}{\rho p}\) then the trace operator \(B\) in Theorem
\ref{thm:def-trace} actually of image in \(W^{l,p}(X)\) and the operator \[ B:\
W^{k,p}(X\times Y^+) \longrightarrow W^{l,p}(X) \] is bounded.
\end{theorem}

\begin{proof}[Proof of Theorem \ref{thm:def-trace}]
It suffices to prove that \(\|Bf\|_{L^p(X)} \leq C\|f\|_{W^{k,p}(X\times Y^+)}\) for all
\(f\in \mathcal{S}(X\times Y^+)\) and \(1 \geq \frac{\rho k}{\sigma} > \frac{1}{p}\)
(for higher \(k\), embed in the \(W^{k,p}\) smaller \(k\)). Define
\begin{align*}
  T_v:\ \mathcal{S}(X\times Y^+) &\longrightarrow \mathcal{S}(X\times Y^+)\\
  	f		    	 &\longmapsto  \left((x,y) \longmapsto \frac{1}{v}\int_0^v f(x,y+w)dw\right)
\end{align*}
for \(v>0\). One can check that \(T_v\) extends to a bounded operator \(T_v:\
W^{k,p}(X\times Y^+) \longrightarrow W^{k,p}(X\times Y^+)\) for all \(k\geq 0\) and
that
\begin{equation*}
\begin{cases}
\|D_y T_v f\|_{L^p(X\times Y^+)} \leq Cv^{-1} \|f\|_{L^p(X\times Y^+)},  \\
\|D_y T_v f\|_{L^p(X\times Y^+)} \leq C\|f\|_{W^{\sigma/\rho,p}(X\times Y^+)}
\end{cases}
\end{equation*}
hence by Interpolation inequality Theorem \ref{thm:interp-ineq-op}, one obtains for all \(0\leq
k\leq \sigma/\rho\): \(\|D_y T_v f\|_{L^p(X\times Y^+)} \leq C v^{\rho k/\sigma -
1}\|f\|_{W^{k,p}(X\times Y^+)}\) hence
\begin{equation}
\label{eq:def-trace-1}
\|D_y (T_{v/2}- T_v f)\|_{L^p(X\times Y^+)} \leq C v^{\rho k/\sigma - 1}\|f\|_{W^{k,p}(X\times Y^+)}
\end{equation}

Similarly, one can prove that for all \(0\leq
k\leq \sigma/\rho\): \(\|({\rm Id} -  T_v) f\|_{L^p(X\times Y^+)} \leq C v^{\rho
k/\sigma}\|f\|_{W^{k,p}(X\times Y^+)}\) therefore
\begin{equation}
\label{eq:def-trace-2}
\|(T_{v/2} -  T_v) f\|_{L^p(X\times Y^+)} \leq C v^{\rho k/\sigma}\|f\|_{W^{k,p}(X\times Y^+)}
\end{equation}

Moreover, using Hölder inequality and Fundamental theorem of calculus, one has: if \(g\in
\mathcal{S}(X\times Y^+)\) then
\begin{equation}
\label{eq:def-trace-3}
\|B g\|_{L^p(X)}\leq C\|g\|^{1/p'}_{L^p(X\times Y^+)}\|D_y g\|^{1/p}_{L^p(X\times Y^+)}
\end{equation}

Substitute \(g\) by \((T_{v/2}- T_v)f\) in \eqref{eq:def-trace-3} then use apply
\eqref{eq:def-trace-1} and \eqref{eq:def-trace-2}, one has
\[
 \|B(T_{v/2}-T_v)\|_{L^p(X)}\leq C v^{\frac{\rho k}{\sigma} -
\frac{1}{p}}\|f\|_{W^{k,p}(X\times Y^+)}
\]
Therefore if \(\frac{1}{p} < \frac{\rho k}{\sigma}\leq 1\), the sequence \(B T_{2^-n}f\) converges in \(L^p(X)\) and the limit is of \(L^p\)-norm less than \(C\|f\|_{W^{k,p}(X\times Y^+)}\). Since \(f\) is continuous, the limit is \(\restr{f}{X\times \{0\}}\). The theorem follows.
\end{proof}

\begin{remark}
The fact that the condition on \(l\) in Theorem \ref{thm:trace-reg} is an open condition
explains why we topologize the space of boundary value  \(\partial W^{k,p}(X\times Y^+)\) by the quotient \(W^{k,p}\)-norm instead of any \(W^{l,p}\)-norm. Also, we have completeness for free.
\end{remark}


In the proof of Theorem \ref{thm:vanishing-trace}, we glue a function \(f_+\in
\mathcal{S}(X\times Y^+)\) with \(f_-\in \mathcal{S}(X\times Y^-)\) of the same value
on \(X\times \{0\}\) and the result is a function in \(W^{\sigma/\rho,p}(X\times Y)\). This can be generalised as follow

\begin{theorem}[Patching theorem]
\label{thm:patching}
If \(p\in (1,+\infty)\) and \(\frac{1}{p} < \rho \frac{k}{\sigma} < 1+\frac{1}{p}\),
then given \(f_+\in W^{k,p}(X\times Y^+)\) and \(f_-\in W^{k,p}(X\times Y^-)\) such
that \(B f_+ = B f_-\) in \(L^p(X)\), one defines \(f\in L^p(X\times Y)\) such that \(f=
f_+\) on \(X\times Y^+\) and \(f= f_-\) on \(X\times Y^-\). Then actually \(f\in
W^{k,p}(X\times Y)\).
\end{theorem}

\subsection{Trace operator on manifold}
\label{sec:orgbb83fb6}

The following paragraph does not appear in \cite{hamilton_harmonic_1975} because of Remark \ref{rem:hamilton-typo}.

To resume, we have defined Sobolev spaces on manifold with boundary as the space of
currents whose cut-off restrictions on each chart are in \(W^{k,p}\). Also we have defined
trace operator of Sobolev spaces on half-plan in a vision to extend the notion to
manifold.

Let \(f\in W^{k,p}(M/\mathcal{A})\) and \(\mathcal{B}\) be a
connected component of \(\partial M\). With the same notation as Definition
\ref{def:sobolev-space-b}, \(f\) gives the data of \(f_i = (\psi_i f)\circ
   \varphi_i^{-1}\in W^{k,p}(R_i/\mathcal{A}_i)\) the cut-off restriction of \(f\) on
each chart using a partition of unity \(\{\psi_i\}_i\) subordinated to a good atlas \((U_i)_i\) of \(M\), where \(R_i\) is an Euclidean space of the same
dimension as \(M\) (\(\mathcal{A}_i = \emptyset\)), or a half-plan (\(\mathcal{A}_i\subset \partial R_i\)). Note that \(U_i\cap \mathcal{B}\) is a good atlas
of \(\mathcal{B}\) and \(\psi_i\) is still a partition of unity subordinated to this
atlas, therefore take \(g_i\in W^{l,p}(\partial R_i)\) to be
trace of \(f_i\) on the image of \(\mathcal{B}\) of each chart. It remains to check
that the data \((g_i)\) corresponds to a unique element \(g\in W^{l,p}(\mathcal{B})\). Recall that we have the following diagram:
\[
\xymatrix{
0 \ar@{->}[r] & W^{l,p}(\mathcal{B}) \ar@/^/@{->}[r]^{\iota} & \bigoplus_i W^{l,p}(\partial R_i) \ar@/^/@{->>}[l]^{\pi}
}
\]
where \(\iota\) admits a projection \(\pi\) given by the cut-off functions \(\tilde
\psi_i\) that we choose to be the same ones used for \(M\). Hence to see that \((g_i)_i\) is in the image of \(\iota\), it suffices to
check that \(\iota\circ\pi((g_i)_i) = (g_i)_i\) which should be straightforward, since
\(\sum_i \tilde \psi_i \psi_i = 1\).

Now that we defined a trace operator \(B:\ W^{k,p}(M) \longrightarrow L^p( \partial M)\)
that factor through \(W^{k,p}(M) \longrightarrow W^{l,p}(\partial M)\) for all \(0\leq l < k -
\frac{\sigma}{\rho p}\), we can define the space of boundary value of function in
\(W^{k,p}(M)\) by 
\[
\partial W^{k,p}(M):= W^{k,p}(M) / \ker B
\]
which has a finer topology than its image in any \(W^{l,p}(\partial M)\) for \(0\leq
l< k- \frac{\sigma}{\rho p}\).
