\section{Commutative diagram and linear PDE. Example: Semi-elliptic equation on \(\mathbb{R}^n\)}
\label{sec:org2267616}
Fix a weight \((\sigma_1,\dots,\sigma_n)\) on \(X = \mathbb{R}^n\) and recall that
for an index \(\alpha\), we note \(\|\alpha\| := \sum_i \frac{\sigma}{\sigma_i}\alpha_i\).

We will consider in this section a partial differential operator \(A\) that is
\uline{heterogeneous}, of \uline{constant coefficient} and of weight \(r\), i.e.
\[
 A(D) = \sum_{\|\alpha\|=r} a_\alpha D^\alpha,\quad D^\alpha = \left(\frac{1}{i}
\frac{\partial}{\partial x}\right)^\alpha
\]


The \textbf{symbol} of \(A\) is \(A(\xi):= \sum_{\|\alpha\| = r} a_\alpha\xi^\alpha\) and \(A\) is
called \textbf{semi-elliptic} if \(A(\xi) \ne 0\) for all \(\xi\in \mathbb{R}^n\setminus 0\)

\begin{remark}
If \(A\) is semi-elliptic then \(\sigma \mid r\). In fact choose all \(\xi_j = 0\)
except \(\xi_i\ne 0\), one sees that there must be a non-zero coefficient \(a_{(0,\dots, \frac{r\sigma_i}{\sigma},\dots, 0)}\), i.e. \(\frac{r\sigma_i}{\sigma}\in
\mathbb{Z}\) for all \(i=\overline{1,n}\). Hence \(\sigma \mid r\) (\(\sigma= \lcm
(\sigma_i)\) being a combination of \(\sigma_i\), look at the same combination of \(\frac{r\sigma_i}{\sigma}\)).
\end{remark}

It is clear that the operator \(A: W^{n,p}(X) \longrightarrow W^{n-r,p}(X)\) is bounded
for all \(n\in \mathbb{R}\) and the following diagram commutes for every real numbers \(k < n\).
\begin{equation}
\label{diag:elliptic}
\xymatrix{
W^{n,p}(X) \ar@{->}[r]^{A(D)} \ar@{^{(}->}[d]^{i} & W^{n-r,p}(X) \ar@{^{(}->}[d]^{i} \\
W^{k,p}(X) \ar@{->}[r]^{A(D)} & W^{k-r,p}(X)
}
\end{equation}

\begin{definition}
Let \(E,F,G,H\) be Banach spaces and \(l,m,p,q\) are bounded operator such that the
following diagram \eqref{diag:D} commutes
\begin{equation}
\label{diag:D}
\tag{diag:D}
\xymatrix{
E \ar@{->}[r]^{l} \ar@{->}[d]^{m} & F \ar@{->}[d]^{p} \\
G \ar@{->}[r]^{q} & H
}
\end{equation}
Then \eqref{diag:D} is said to be an \textbf{exact square} if the following associated sequence is exact
\[
 \xymatrix{
0 \ar@{->}[r] & E \ar@{->}[r]^{l\oplus m} & F\oplus G \ar@{->}[r]^{p\ominus q} & H \ar@{->}[r] & 0
}
\]
\end{definition}

\begin{exampl}
If \((A,B)\) is an interpolatable pair of Banach spaces then
\[
 \xymatrix{
A\cap B \ar@{->}[d] \ar@{->}[r] & A \ar@{->}[d] \\
B \ar@{->}[r] & A+B
}
\]
is exact, where arrows are natural inclusions.
\end{exampl}

The notion of exact square allows us to reformulate classical results of elliptic
equation as

\begin{theorem}[Elliptic equation with constant coefficients]
\label{thm:elliptic-const}
The square \eqref{diag:elliptic} is exact for all \(k < n\) in \(\mathbb{R}\). This encodes the
following 3 results:
\begin{enumerate}
\item \(\xymatrix{
   W^{n,p}(X) \ar@{^{(}->}[r]^<<<<<{A\oplus i} & W^{n-r,p}(X)\oplus W^{k,p}(X)
   }\) is of closed image, i.e. there exists \(C>0\) such that
\[
    \|f\|_{W^{n,p}(X)}\leq C \left( \|A f\|_{W^{n-r,p}(X)} +\|f\|_{W^{k,p}(X)}  \right)
   \]
 which is Gårding's inequality.
\item \(\ker A\ominus i = \im A\oplus i\), i.e. if \(f\in W^{k,p}(X)\) and \(Af\in
   W^{n-r}(X)\) then actually \(f\in W^{n,p}(X)\), which is regularity theorem.
\item \(\im A\ominus i = W^{k-r,p}(X)\), i.e. for all \(g\in W^{k-r,p}(X)\), there exists
\(f\in W^{k,p}(X)\) such that \(Af - g\in W^{n-r,p}(X)\), which is the existence of
approximate solution (the idea behind \href{./green-function.org}{\emph{parametrix}}).
\end{enumerate}
\end{theorem}

A way to prove that a square is exact is to show that it splits

\begin{definition}
The square \eqref{diag:D} is called \textbf{split} if there exists \(l',m',p',q'\) such that
\begin{equation}
\label{diag:split}
\xymatrix{
E \ar@{->}[r]_{l} \ar@{->}[d]^{m} & F \ar@{->}[d]_{p} \ar@/_/@{->}[l]_{l'} \\
G \ar@{->}[r]^{q} \ar@/^/@{->}[u]^{m'} & H \ar@/_/@{->}[u]_{p'} \ar@/^/@{->}[l]^{q'}
}
\end{equation}
commutes in 4 ways: 
\[
 \xymatrix{
{} \ar@{-}[r] \ar@{-}[d] & {} \ar@{->}[d] \\
{} \ar@{->}[r] & {}
}\quad
\xymatrix{
{} \ar@{->}[r] & {} \\
{} \ar@{-}[u] \ar@{-}[r] & {} \ar@{->}[u]
}\quad
\xymatrix{
{} & {} \ar@{->}[l] \\
{} \ar@{->}[u] \ar@{-}[r] & {} \ar@{-}[u]
}\quad
\xymatrix{
{} \ar@{->}[d] & {} \ar@{-}[l] \\
{} & {} \ar@{-}[u] \ar@{->}[l]
}
\]
and splits in 4 ways
\[
 \xymatrix{
{} \ar@/_/@{->}[r] \ar@/^/@{->}[d] & {} \ar@/_/@{->}[l] \\
{} \ar@/^/@{->}[u] & 
}\quad
\xymatrix{
{} \ar@/_/@{->}[r] & {} \ar@/_/@{->}[l] \ar@/^/@{->}[d] \\
 & {} \ar@/^/@{->}[u]
}\quad
\xymatrix{
 & {} \ar@/^/@{->}[d] \\
{} \ar@/^/@{->}[r] & {} \ar@/^/@{->}[u] \ar@/^/@{->}[l]
}\quad
\xymatrix{
{} \ar@/^/@{->}[d] &  \\
{} \ar@/^/@{->}[u] \ar@/_/@{->}[r] & {} \ar@/_/@{->}[l]
},
\]
i.e. the sum of two circle in each diagram is the identities.
\end{definition}

\begin{theorem}
\label{thm:split-exact}
\begin{enumerate}
\item A split square is exact. In fact, if a square splits, then the associated short sequence splits.
\item If \(E,F,G,H\) are Hilbert spaces then any exact square splits.
\end{enumerate}
\end{theorem}

\begin{proof}[Proof of Theorem \ref{thm:elliptic-const} ]
Since \(A\) is semi-elliptic, there exists \(\epsilon>0\) such that \(|A(\xi)|\geq
\epsilon \|\xi\|^r\) for \(\|\xi\| := (\xi_1^{2\sigma_1}+\dots +
\xi_n^{2\sigma_n})^{1/2\sigma}\). Let \(\psi(\xi)\) be a radial function in \(\xi\)
that is identically \(1\) for \(\|\xi\|\leq 1\) and \(0\) for \(\|\xi\|\geq 2\)
and define
\[
G(\xi):=\begin{cases}
\frac{1-\psi(\xi)}{A(\xi)},  & \text{if $\|\xi\|\geq 1$} \\
0, & \text{if $\|\xi\| \leq 1$}
\end{cases}
\]
Then by Stein's multiplier theorem, 
\begin{align*}
   G(D):\ W^{k-r,p}(X) &\longrightarrow W^{k,p}(X) \quad\forall k\in \mathbb{R},\\ 
\psi(D):\ W^{l,p}(X) &\longrightarrow W^{k,p}(X) \quad \forall k,l\in \mathbb{R}
\end{align*}
are bounded operators. We say that \(G(D)\) is an \emph{approximate inverse} of \(A(D)\)
because \(G(D)A(D)=A(D)G(D) = 1-\psi(D)\). It is easy to check that \eqref{diag:elliptic}
splits: 
\[
\xymatrix{
W^{k,p}(X) \ar@/_/@{->}[r]_{A(D)} \ar@{->}[d]^{i} & W^{k-r,p}(X) \ar@{->}[d]_{i} \ar@/_/@{->}[l]_{G(D)} \\
W^{l,p}(X) \ar@/^/@{->}[r]^{A(D)} \ar@/^/@{->}[u]^{\psi(D)} & W^{l-r,p}(X) \ar@/^/@{->}[l]^{G(D)} \ar@/_/@{->}[u]_{\psi(D)}
}
\]
\end{proof}

The following abstract result shows that solutions of homogeneous equation \(Af=0\) are
smooth (also proved in the second point of Theorem \ref{thm:elliptic-const}) and the
solution space is of finite dimension.

\begin{theorem}
\label{thm:compact-op-exact}
Suppose that the square 
\[
 \xymatrix{
E \ar@{->}[r]^{l} \ar@{->}[d]^{m} & F \ar@{->}[d]^{p} \\
G \ar@{->}[r]^{q} & H
}
\]
is exact and \(m,p\) are compact operators. Then \(l\) and \(q\) have closed image,
and their kernels and cokernels are isomorphic through \(m\) and \(p\), and are of finite dimensional.
\end{theorem}

\begin{proof}
By basic diagram chasing, one can see that the restriction of \(m\) is an isomorphism \(\ker l \longrightarrow \ker q\). But \(m\) is compact, \(\ker l \cong \ker q\) are
locally compact, hence of finite dimension.

It is easy to check (with sequential limit) that \(\im l\) is closed in \(F\), since \(\im (l\oplus m) =\ker
p\ominus q\) is closed and \(m\) is compact. So \(\coker l\) is a Banach space.

Let \(p'': \coker l = F/l(E) \longrightarrow H/\overline{ q(G)}\) be the map induced by
\(p\) to the quotients, note that we have to take the closure of \(q(G)\) to ensure
that the quotient is Banach. Then \(p''\) is obviously continuous and compact. Also \(p''\) is surjective because \(\xymatrix{F\oplus G \ar@{->>}[r]^{p\ominus q} & H}\).

We will prove that \(p''\) is injective. If \(f\in F\setminus l(E)\) then by
\href{https://en.wikipedia.org/wiki/Hahn–Banach\_theorem}{Hahn-Banach theorem}, there exists a linear functional \(\lambda\in F^*\) such that \(\lambda(f)=1\) and \(\lambda (l(E))=0\). One has
\[
 \xymatrix{
0 \ar@{->}[r] & H^* \ar@{->}[r]^{(p\ominus q)^*} & F^*\oplus G^* \ar@{->>}[r]^{(l\oplus m)^*} & E^* \ar@{->}[r] & 0
}
\]
and that \((l\oplus m)^*(\lambda \oplus 0) = 0\), hence there exists \(\lambda'\in H^*\) such that \(\lambda \oplus 0 = (p\ominus q)^* \lambda'\), i.e. \(\lambda'\circ q = 0\) and \(\lambda'\circ p =\lambda\), which means \(\lambda'\) vanishes on \(q(G)\), hence \(\overline
q(G)\), and that \(\lambda'(p(f)) = \lambda (f)=1\). Hence \(p(f)\not\in
\overline{q(G)}\) and \(p''\) is injective.

The injectivity of \(p''\) has 2 consequences. First, it means that \(\coker l \cong
H/\overline{q(G)}\) by a compact operator, hence the two are locally compact and of
finite dimension.

Second, it proves that \(q(G)\) is closed in \(H\). In fact, given \(h\in
\overline{q(G)}\), by surjectivity of \(p\ominus q\), one has \(h= pf + qg\) for \(f\in F\) and \(g\in G\), this means \(p''(\bar f) = \bar 0 \in H/\overline{q(G)}\), hence \(\bar f =
\bar 0 \in \coker l\), i.e. \(f = l(e)\) for some \(e\in E\). Therefore
\[
 h = p\circ l (e) + q(g)=  q(m(e) +g)\in q(G)
\]
and \(p(G)=\overline{p(G)}\) is closed in \(H\).
\end{proof}

\begin{remark}
The proof of Theorem \ref{thm:compact-op-exact} is much simpler for split squares. We
presented the version for exact squares because we will use it later. The advantage of
using exact squares instead of split square is, as we will see, that among commutative squares, exact
squares form a relatively open set, allowing us to "pertube" an exact square and extend
the theory to cover the case \(A\) of variable coefficients.
\end{remark}


\section{Elliptic equation on half-plan \(X\times Y^+\). Boundary conditions. \label{ell-half-plan}}
\label{sec:org0deb1e7}
We will quickly review in this part the ideas to solve elliptic equations with constant
coefficients on half-plan. This does not require any more abstract (i.e. with diagram)
results. The main tasks will be using suitable cut-off function on the frequent space (1)
to define the approximate inverse of an elliptic operator on half-plan that is adapted to
the boundary structure and (2) to approximately inverse the boundary operators.

We will solve elliptic equation on \(X\times Y^+\) where the variables are \(x_1,\dots,
x_n\) and \(y\), under weight \(\Sigma = (\sigma_1,\dots, \sigma_n,\rho)\). Recall
that \(A(D) = \sum_{\|(\alpha,\beta)\|=r}a_{\alpha \beta}D^\alpha_x D^\beta_y\) with
symbol \(A(\xi,\eta)=\sum_{\|(\alpha,\beta)\|=r}\xi^\alpha\eta^\beta\).

If \(A(D)\) is semi-elliptic then for all \(\xi \ne 0\) the polynomial \(\eta \mapsto
A(\xi,\eta)\) has no real zeros, hence can be factorized to
\[
 A(\xi,\eta)=A^+(\xi,\eta)A^-(\xi,\eta)
\]
where \(A^+(\xi,\eta)\) (resp. \(A^-(\xi,\eta)\)) only has zeros \(\eta\) with \(\im\eta >0\) (resp. \(\im \eta <0\)).

\begin{remark}
\begin{enumerate}
\item By semi-ellipticity, the monomial \(a_{\alpha\beta}\xi^\alpha \eta^\beta\) with biggest \(\beta\) has
index \(\alpha = 0\). Hence we can suppose that the leading coefficients, as
polynomials in \(\eta\) of \(A, A^+, A^-\) are \(1\).
\item As polynomial in \(\eta\), \(A^+(\xi,\eta)=\sum_{\beta=0}^m a^+_\beta(\xi)\eta^\beta\) where \(m=r\rho/\sigma\) and \(a^+_\beta(\xi)\) are \(\Sigma\)-heterogeneous
of weight \((m-\beta)\rho\), i.e. 
\[
    a^+_\beta(t^{\sigma/\sigma_1}\xi_1,\dots,t^{\sigma/\sigma_n}\xi_n) = t^{(m-\beta)\rho}a^+_\beta(\xi)
   \]
 Also, the coefficients \(a^+_\beta\) are smooth in \(\xi\).
\end{enumerate}
\end{remark}

We will solve the elliptic equation under some \emph{suitable} boundary conditions. Let \(B^j\), \(1\leq j\leq m\) be \(m\) \(\Sigma\)-heterogeneous \emph{boundary operators} of
weights \(r_j\), i.e.
\[
B^j(D) = \sum_{\|(\alpha,\beta)\| = r_j} b^j_{\alpha\beta}D^\alpha_x D^\beta_y
\]
of symbol 
\[ 
B^j(\xi,\eta) = \sum_{\|(\alpha,\beta)\| = r_j} b^j_{\alpha\beta}\xi^\alpha
\eta^\beta = \sum_{\|(\alpha,\beta)\| = r_j} b^j_{\beta}(\xi)\eta^\beta  
\]
where \(b_\beta^j\) are heterogeneous in \(\xi\) (actually polynomials) and of weight \(r_j-\beta\).


As our \href{./interpolation-sobolev.org}{discussion on trace operator}, if \(k > r_j + \frac{\sigma}{\rho p}\) then \(B^j\) extends to a bounded operator 
\[
 \xymatrix{
W^{k,p}(X\times Y^+) \ar@{->}[rd] \ar@{->}[rr]^{B_j} &  & \partial W^{k-r_j,p}(X) \ar@{^{(}->}[r] & W^{l,p}(X) \\
 & W^{k-r_j,p}(X\times Y^+) \ar@{->}[ru] &  & 
}
\]
for all \(0\leq l < n - r_j - \frac{\sigma}{\rho p}\).


\begin{definition}
We will say that the operators \(Bf = (B^1 f, \dots, B^m f)\) satisfy the \textbf{complementary
boundary condition (CBC)} if the 
\[
 \det \left(c^j_\beta(\xi)\right)_{j,\beta}\ne 0\quad\forall \xi\in \mathbb{R}^n\setminus\{0\}
\]
where \(c^j_\beta(\xi)\) are the cofficients of the remanders \(C^j(\xi,\eta)\) when one divides \(B^j(\xi,\eta)\) by \(A^+(\xi,\eta)\) as polynomials in \(\eta\), i.e.
\[
B^j(\xi,\eta) \equiv C^j(\xi,\eta) = \sum_{\beta =
0}^{m-1}c^j_\beta(\xi)\eta^\beta\quad\mod A^+(\xi,\eta)
\]
\end{definition}

\paragraph{Approximate inverse of boundary operator \(B\).}
\label{sec:org52b14de}
The CBC condition allows us to approximately inverse boundary operator \(B\).

\begin{theorem}[Approximate inverse of \( B \)]
\label{thm:approx-inv-B}
Let \(B:\ \mathcal{S}(X\times Y^+) \longrightarrow \mathcal{S}(X)^{\oplus m}\) be a
boundary operator that satisfies CBC condition, then there exists an operator
\begin{align*}
  H: \mathcal{S}(X)^{\oplus m} &\longrightarrow \mathcal{S}(X\times Y^+)\\
     (h_1,\dots,h_m) &\longmapsto H_1 h_1 + \dots + H_m h_m
\end{align*}
such that
\begin{enumerate}
\item \(({\rm Id} - BH)h = \psi(D_x)h\) for all \(h\in \mathcal{S}(X)^{\oplus m}\).
\item \(({\rm Id} - HB)f = \psi(D_x)f\) for all \(f\in \ker A(D): \mathcal{S}(X\times Y^+)
   \longrightarrow \mathcal{S}(X\times Y^+)\).
\end{enumerate}
where \(\psi(\xi)\) is the radial smooth cut-off function in \(\xi\) that equals \(1\) when
\(\|\xi\|\leq 1\) and \(0\) when \(\|\xi\|\geq 2\).

Moreover, if \(k>r_j + \frac{\sigma}{\rho p}\) then the operators \(H_j: \mathcal{S}(X)
\longrightarrow \mathcal{S}(X\times Y^+)\) extends to a bounded operator
\[
 H_j:\ \partial W^{n-r_j,p}(X) \longrightarrow W^{k,p}(X\times Y^+)
\]
\end{theorem}

\begin{proof}[Sketch of proof]
We define \(H_j:\ \mathcal{S}(X) \longrightarrow \mathcal{S}(X\times Y^+)\) by its
action on the frequent space of \(X\), in particular, set
\[
\tilde{H_j h}(\xi,\eta):= H_j(\xi,y) \tilde h(\xi)
\]
where \(\tilde f\) is the partial (in
\(x\)) Fourier transform of \(f\) and \(H_j(\xi,y)\) is given by
\[
 H_j(\xi,y):= (1-\psi(\xi))\int_\Gamma \sum_{\alpha= 0}^{m-1}e^\alpha_j(\xi)
 \frac{A_\alpha^+(\xi,\eta)}{A^+(\xi,\eta)}e^{i\eta y}d\eta
\]
where \(\Gamma\subset \mathbb{C}\) is a curve enclosing all zeros of \(A(\xi,\eta)\) with \(\im
\eta >0\), \((e^\alpha_j(\xi))_{\alpha,j}\) is the inverse matrix of \((c^j_\beta(\xi))_{j,\beta}\) and \(A_\alpha^+(\xi,\eta):= \sum_{\beta=0}^{m-\alpha-1} a^+_{\alpha+\beta+1}(\xi)\eta^\beta\).
\end{proof}

\paragraph{Some auxilary functions.}
\label{sec:org25e9d81}
We cannot use the operator \(G\) as in the case of whole plan as an inverse of \(A\)
on the half-plan \(X\times Y^+\), since we only have access to the frequent space of \(X\). However we can modify the cut-off function to create an approximate inverse of \(A\) on the half-plan.

Let \(\varphi: \mathbb{R}_{\geq 0}\longrightarrow \mathbb{R}\) be the function that we used in the
definition of \(E_+\), i.e.
\[
\varphi(x) := \frac{e^4}{\pi}.\frac{e^{-(x^{1/4} + x^{-1/4})} \sin(x^{1/4} -
x^{-1/4})}{1+x} , \quad x\geq 0
\]
with the properties \(\int_0^{\infty}x^n \varphi(x)dx = (-1)^n\) for all \(n\in
\mathbb{Z}\setminus\{0\}\) and \(\int_0^\infty \varphi(x)dx=0\).
Extending \(\varphi\) by 0 for \(x<0\), one still has a smooth function. Define
\(\chi(y):=-\varphi(-y-1),\)
then \(\chi \in \mathcal{S}(Y)\), with support in \((-\infty,-1]\) and
\[
\int_{\mathbb{R}} y^n\chi(y) dy = \begin{cases}
0,  & \text{if $n > 0$} \\
1, & \text{if $n=0$}
\end{cases}
\]
In the frequent space of \(Y\), this means \(\hat\chi(0)=1\) and \(D^k_{\eta}\hat\chi(0)=0\), i.e. \(1-\hat\chi(\eta)\) has a zero of infinite order at \(\eta=0\).

Also, since \(\chi = 0\) when \(y>-1\), the convolution 
\[ 
f\longmapsto \hat\chi(D_y) f = \chi * f 
\]
maps \(\mathcal{S}(Y^-/0)\) to itself, hence induces a mapping from \(\mathcal{S}(Y^+)\) to itself, since
\[
 \xymatrix{
0 \ar@{->}[r] & \mathcal{S}(Y^-/0) \ar@{->}[r] & \mathcal{S}(Y) \ar@{->}[r] & \mathcal{S}(Y^+) \ar@{->}[r] & 0
}
\]
(given any \(f\in \mathcal{S}(Y^+)\), any extension \(\tilde f\) of \(f\)
to \(\mathcal{S}(Y)\) has the same restriction of \(\hat\chi(D_y)\tilde f\) on \(Y^+\)).

Let \(w(\xi,\eta):=\psi(\xi)\hat\chi(\eta)\) then \(w\) defines an operator
\[
 w(D): \mathcal{S}(X\times Y^+) \longrightarrow \mathcal{S}(X\times Y^+)
\]
In fact, for all \(k,l\in \mathbb{R}\), there exists \(C>0\) such that
\[
 \|w(D)f\|_{W^{k,p}(X\times Y^+)} \leq C \|f\|_{W^{l,p}(X\times Y^+)}.
\]

\iffalse

\begin{remark}
\label{rem:hamilton-sign-issue}
I changed the definition of \(\chi\) in \cite{hamilton_harmonic_1975},
where originally \(\chi(y):=\varphi(y-1)\) and should maps \(\mathcal{S}(Y^+/0)\) to
itself, not \(\mathcal{S}(Y^-/0)\) to itself as we wish.
\end{remark}
\fi


\paragraph{Approximate inverse of elliptic operator \(A\) on half-plan.}
\label{sec:org9757bc8}
The auxilary function \(w\) will play the role of \(\psi\) in the whole plan case.

\begin{theorem}[Approximate inverse of \( A \) on \( X\times Y^+ \)]
\label{thm:approx-inv-A}
There exists an operator \(G: \mathcal{S}(X\times Y^+) \longrightarrow
\mathcal{S}(X\times Y^+)\) such that:
\begin{enumerate}
\item \(({\rm Id} - AG) = w(D)\)
\item For all \(k,l\in \mathbb{R}\), there exists \(C>0\) such that for all \(f\in
   \mathcal{S}(X\times Y^+)\):
\[
    \|({\rm Id} - GA)\psi(D_x)\|_{W^{k,p}(X\times Y^+)}\leq C \|f\|_{W^{l,p}(X\times Y^+)}
   \]
\end{enumerate}
Also \(G\) extends to a bounded operator \(G:\ W^{k-r,p}(X\times Y^+) \longrightarrow
W^{k,p}(X\times Y^+)\) for all \(k\in \mathbb{R}\).
\end{theorem}

\begin{proof}[Sketch of proof]
In fact \(G\) is defined as follows:
\[
 G_0(\xi,\eta):= \frac{1-w(\xi,\eta)}{A(\xi,\eta)}
\]
which is smooth at \((0,0)\), where \(1- w\) has a zero of infinite order. Then \(G_0(D): \mathcal{S}(X\times Y) \longrightarrow \mathcal{S}(X\times Y)\) extends to \(W^{k-r,p}(X\times Y) \longrightarrow W^{k,p}(X\times Y)\). Finally, take \(G=C_+ G E_+\), which maps \(\mathcal{S}(X\times Y^+) \longrightarrow \mathcal{S}(X\times Y^+)\) by
first extending a function to the whole plan, applying \(G_0\) and finally cutting-off.
\end{proof}


\paragraph{Approximate inverse of the combined operator.}
\label{sec:orgac1470a}
Let \(\mathcal{C}\) be the combined operator:

\begin{align*}
   \mathcal{C}: \mathcal{S}(X\times Y^+) &\longrightarrow \mathcal{S}(X\times
Y^+)\oplus \mathcal{S}(X)^{\oplus m}\\
f &\longmapsto (Af, Bf)
\end{align*} 
and define the operator \(\mathcal{J}\) as

\begin{align*}
   \mathcal{J}: \mathcal{S}(X\times
Y^+)\oplus \mathcal{S}(X)^{\oplus m} & \longrightarrow  \mathcal{S}(X\times Y^+) \\
(g,h) &\longmapsto Gg+ H(h-BGg)
\end{align*}
then one can prove with straightforward computation that \(\mathcal{J}\) is an
approximate inverse of \(\mathcal{C}\).

\begin{theorem}[Approximate inverse of \mathcal{C}]
\label{thm:approx-inv-C}
For smooth functions \(f\in \mathcal{S}(X\times Y^+)\) and \((g,h)\in
\mathcal{S}(X\times Y^+) \oplus \mathcal{S}(X)^{\oplus m}\), one has
\begin{enumerate}
\item \(({\rm Id} - \mathcal{C}\mathcal{J})(g,h) = (w(D)g, \psi(D_x)(h-BGg))=:\lambda(g,h)\)
\item \(({\rm Id} - \mathcal{J}\mathcal{C})f = \psi(D_x) \left({\rm Id} -GA)f + ({\rm Id} -HB -\psi(D_x)\right)w(D)f=:\mu(f)\)
\end{enumerate}
\end{theorem}

Since \(G,H\) extend to Sobolev spaces, one also has
\[
\mathcal{J}:\ W^{k-r,p}(X\times Y^+)\bigoplus_{j=1}^m \partial W^{k-r_j,p}(X)
\longrightarrow W^{k,p}(X\times Y^+)
\]
whenever \(k > \frac{\sigma}{\rho p} + \max_j r_j\).

\begin{theorem}
\label{thm:exact-half}
In analogue of Theorem \ref{thm:elliptic-const}, one has the following exact-split square
\begin{equation}
\label{eq:elliptic-half}
\xymatrix{
W^{k,p}(X\times Y^+) \ar@{->}[r]_<<<<<{\mathcal{C}} \ar@{->}[d]^{\iota} & W^{k-r,p}(X\times Y^+)\bigoplus_{j=1}^m \partial W^{k-r_j,p}(X) \ar@{->}[d]_{\iota} \ar@/_/@{->}[l]_>>>{\mathcal{J}} \\
W^{l,p}(X\times Y^+) \ar@{->}[r]^<<<<<{\mathcal{C}} \ar@/^/@{->}[u]^{\mu} & W^{l-r,p}(X\times Y^+)\bigoplus_{j=1}^m \partial W^{l-r_j,p}(X) \ar@/^/@{->}[l]^>>>{\mathcal{J}} \ar@/_/@{->}[u]_{\lambda}
}
\end{equation}
for all \(k>l> \frac{\sigma}{\rho p} + \max_j r_j\).
\end{theorem}


\section{From local to global.}
\label{sec:org7498191}
\subsection{Pertubation of exact squares and consequences.}
\label{sec:orgfa2fb3e}
We will extend the result of Theorem \ref{thm:elliptic-const} (exactness of heterogeneous elliptic
operator with constant coefficient on Euclidean plan) in 2 levels: (1) for general
elliptic operators (non-heterogeneous and with variable coefficients) and (2) for such
operators on compact manifold (with boundary if needed). These 2 generalizations will be
done using the same technique: "cube by cube" approximating an exact square.

We topologize the space of commutative squares \(\xymatrix{
E \ar@{->}[r]^{l} \ar@{->}[d]_{m} & F \ar@{->}[d]^{p} \\
G \ar@{->}[r]^{q} & H
}\) as a closed subspace \(SQ(E,F,G,H)\) of \(L(E,F)\times L(F,H)\times L(E,G)\times L(G,H)\) defined
by the equation \(q\circ m = p\circ l\).

\begin{theorem}
\label{thm:exact-open}
In \(SQ(E,F,G,H)\), the exact squares form an open set.
\end{theorem}

Instead of giving a proof (see \cite[page 75-77]{hamilton_harmonic_1975}), let us explain why Theorem \ref{thm:exact-open} is true. The
commutativity already tells us that the composition of any two consecutive arrows in
\[
 \xymatrix{
0 \ar@{->}[r] & E \ar@{->}[r]^{l\oplus m} & F\oplus G \ar@{->}[r]^{p\ominus q} & H \ar@{->}[r] & 0
}
\]
is 0, and exactness is an extra condition of type "maximal rank", which is an open
condition (For matrices, this means the derterminant does not vanish. The analogous
phenomenon for Banach spaces is that a linear map sufficiently close to an invertible one
is also invertible).


We will distinguish the following 2 types of cubes that we will use to cover a
manifold. We will call the following set an \emph{interior cube}
\[ 
B_\epsilon := \left\{ (x_1,\dots,
x_n): |x_1|\leq\epsilon^{\sigma/\sigma_1},\dots |x_n|
\leq\epsilon^{\sigma/\sigma_n}\right\} 
\]
and the following an \emph{boundary cube}
\[ 
B^+_\epsilon := \left\{ (x_1,\dots,
x_n,y): |x_1|\leq\epsilon^{\sigma/\sigma_1},\dots |x_n|
\leq\epsilon^{\sigma/\sigma_n}, 0\leq y\leq \epsilon^{\sigma/\rho}\right\} 
\]

For the second type, we note by \(\partial_0\) the part \(y=0\) of the boundary of \(B^+_\epsilon\), and by \(\partial_e\) the remaining part.

We will say that the \(A:= \sum_{\|\alpha\|\leq r}a_{\alpha}(x) D^\alpha\) is
\textbf{semi-elliptic at \(0\)} if \(A_0:=  \sum_{\|\alpha\|= r}a_{\alpha}(0) D^\alpha\) is a
semi-elliptic operator.

\begin{proposition}[Approximate operator on interior cube]
\label{prop:interior-cube}
Suppose that \(A:=  \sum_{\|\alpha\|\leq r}a_{\alpha}(x) D^\alpha\) is defined in \(B_{\epsilon_0}\) and \(A\) is semi-elliptic at \(x = 0\). Fix \(-\infty < l <k <+\infty\). Then there exists an \(\epsilon>0\) sufficiently small and an operator \(A^\# =  \sum_{\|\alpha\|\leq
r}a^\#_{\alpha}(x) D^\alpha\) with smooth coefficients defined on \(X\) such that
\[
 A^\# = \begin{cases}
A       ,  & \text{inside $B_\epsilon$} \\
A_0       , & \text{outside $B_{2\epsilon}$}
       \end{cases}
\]
and the "\(k,l\)" square corresponding to \(A^\#\), i.e.
\[
 \xymatrix{
W^{k,p}(X) \ar@{->}[r]^{A^\#} \ar@{^{(}->}[d]^{\iota} & W^{k-r,p}(X) \ar@{^{(}->}[d]^{\iota} \\
W^{l,p}(X) \ar@{->}[r]^{A^\#} & W^{l-r,p}(X)
}
\]
is exact.
\end{proposition}

An analoguous result holds for boundary problem. The setup for boundary problem on
half-plan \(X\times Y^+\) is as follows.
\begin{align*}
  A &:=  \sum_{\|(\alpha,\beta)\|\leq r}a_{\alpha,\beta}(x,y) D^\alpha_x D^\beta_y \\
  B^j &:=  \sum_{\|(\alpha,\beta)\|\leq r_j}b^j_{\alpha,\beta}(x,y) D^\alpha_x D^\beta_y,\quad j=\overline{1,m}
\end{align*}
are operators with smooth coefficients on \(B^+_{\epsilon_0}\) and
\begin{align*}
  A_0 &:=  \sum_{\|(\alpha,\beta)\|= r}a_{\alpha,\beta}(0,0) D^\alpha_x D^\beta_y \\
  B^j_0 &:=  \sum_{\|(\alpha,\beta)\|= r_j}b^j_{\alpha,\beta}(0,0) D^\alpha_x D^\beta_y,\quad j=\overline{1,m}
\end{align*}
If \(A\) is semi-elliptic at \(0\) then we say that \(\{B^j\}\) satisfy the CBC
condition at \(0\) uf \(\{B^j_0\}\) are CBC with respect to \(A_0\). Note that this
is an "open condition", i.e. if the condition is satisfied at \((0,0)\) then it is also
satisfied in a neighborhood of \((0,0)\) in \(X\times \{0\}\). The analoguous result
for boundary problem can then be stated.

\begin{proposition}[Approximate opearator on boundary cube]
\label{prop:boundary-cube}
Under the previous setup and with \(\frac{\sigma}{\rho p} + \max_j r_j < l < k < +\infty\), for \(\epsilon > 0\) sufficiently small, there exists operators \(C^\#=(A^\#, B^\#)\)
with smooth coefficient in \(X\times Y^+\) agreeing with \((A,B)\) in \(B^+_\epsilon\) and with \((A_0,B_0)\) outside of \(B^+_{2\epsilon}\) such that the square
\[
 \xymatrix{
W^{k,p}(X\times Y^+) \ar@{->}[r]^<<<<<{\mathcal{C}^\#} \ar@{^{(}->}[d]_{\iota} & W^{k-r,p}(X\times Y^+)\bigoplus_{j=1}^m \partial W^{k-r_j,p}(X) \ar@{^{(}->}[d]^{\iota} \\
W^{l,p}(X\times Y^+) \ar@{->}[r]_<<<<<{\mathcal{C}^\#} & W^{l-r,p}(X\times Y^+)\bigoplus_{j=1}^m \partial W^{l-r_j,p}(X) 
}
\]
is exact.
\end{proposition}

We will prove Proposition \ref{prop:interior-cube} here to demonstrate how Theorem
\ref{thm:exact-open} is employed. Another reason is that the corresponding proof in
\cite{hamilton_harmonic_1975} is not very readable due to a notation/printing issue.

\begin{proof}[Proof of Proposition \ref{prop:interior-cube}]
We will use the change of coordinates
\(\tilde x_i = \lambda^{-\sigma/\sigma_i}x_i,\)
which gives a diffeomorphism \(h_\lambda\) from \(B_{\epsilon_0}\) to \(B_{\epsilon_0/\lambda}\), in which the derivative operators are 
\[ 
\left(\frac{\partial}{\partial \tilde{x_i}}\right)^\alpha_i =
\lambda^{\alpha_i\sigma/\sigma_i}\left(\frac{\partial}{\partial x_i}\right)^\alpha_i,\quad
\tilde D^\alpha  = \lambda^{\|\alpha\|}D^\alpha 
\]

The operator \(A\), viewed in \(h_\lambda\), i.e. the operator \(f\mapsto
A(f\circ h_\lambda)\), is \(\sum_{\|\alpha\|\leq r}
a_\alpha(\lambda^{\sigma/\sigma_i}\tilde{x_i}) \lambda^{-\|\alpha\|} \tilde D^\alpha\). We pose
\begin{align*}
  \tilde A_\lambda &:= \sum_{\|\alpha\|\leq r}\lambda^{r-\|\alpha\|} a_{\alpha}(\lambda^{\sigma/\sigma_i}\tilde{x_i}) \tilde D^\alpha \\  
  \tilde A_0 &:= \sum_{\|\alpha\|= r} a_{\alpha}(0) \tilde D^\alpha\\
  \tilde A_\lambda^* &:= \varphi(\tilde x) \tilde A_\lambda + (1-\varphi(\tilde x)) \tilde A_0
\end{align*}
where \(\varphi\) is radial in \(\tilde x\), equals \(1\) for \(\|\tilde x\|\leq 1\) and \(0\) for \(\|\tilde x\| \geq 2\).

The coefficient before \(\tilde D^\alpha\) of \(\tilde A_\lambda^*\) is \(\lambda^{r-\|\alpha\|} \left[\varphi(\tilde x) a_\alpha(\lambda^{\sigma/\sigma_i} \tilde{x_i}) +
(1-\varphi(\tilde x))a_\alpha(0)\delta_{\|\alpha\|=r}\right]\) is the same as that of \(\tilde A_0\) for \(\tilde x\) outside of \(B_2\) and \(C^0\)-converges to that of \(\tilde
A_0\) inside \(B_1\). Hence for \(\lambda\) sufficiently small the corresponding "\(k,l\)" diagram of \(\tilde A^*_\lambda\) is exact, hence so is the diagram of \(\lambda^{-r}\tilde
A_\lambda^*\). Choose \(\epsilon = \lambda\) and \(A^\#\) to be \(\lambda^{-r}\tilde
A_\lambda^*\) viewed in \(X\) through \(h\lambda\).
\end{proof}

\begin{remark}
To avoid making infinite intersection of open sets, we have to fix \(k\) and \(l\) first in Proposition \ref{prop:interior-cube}
and Proposition \ref{prop:boundary-cube}. The approximate operators \(A^\#, B^\#\) and the
size \(\epsilon\) of the cube therefore depend on \(k,l\), but this dependence will
not be a trouble when we pass from local to global situation.
\end{remark}


The exactness of semi-elliptic operator with variable coefficients on manifold will be
establish analytically, meaning through the 3 statements similar to those of Theorem \ref{thm:elliptic-const}.
Proposition \ref{prop:interior-cube} and \ref{prop:boundary-cube} can be applied to prove the
the local version of these statements.

\begin{lemma}
\label{lem:ell-loc-cube}
With the same \(\epsilon\) and \(k,l\) as Proposition \ref{prop:interior-cube} and the
extra condition that \(l\geq k-1\), one has for all \(0< \delta <\epsilon\)
\begin{enumerate}
\item \(\|f\|_{W^{k,p}(B_\delta)}\leq C \left( \|Af\|_{W^{k-r,p}(B_\epsilon)} +
   \|f\|_{W^{l,p}(B_\epsilon)} \right)\) for all \(f\in W^{k,p}(B_\epsilon)\).
\item If \(f\in W^{l,p}(B_\epsilon)\) and \(Af\in W^{k-r,p}(B_\epsilon)\) then \(f\in W^{k,p}(B_\delta)\).
\item If \(g\in W^{l-r,p}(B_\delta/ \partial)\) then there exists \(f\in
   W^{l,p}(B_\epsilon, \partial)\) such that
\[
    g-Af \in W^{k-r,p}(B_\epsilon, \partial B_\epsilon).
   \]
\end{enumerate}
\end{lemma}


\begin{proof}
Let \(\psi\) be a cut-off function that equals \(1\) on \(B_\delta\) and \(0\)
outside of \(B_\epsilon\) and \(A^\#\) be the differential operator on \(X\) with
exact "\(k,l\)" diagram given by Proposition \ref{prop:interior-cube} which equals \(A\)
on \(B_\epsilon\). The idea of the remaining computation is to use the exactness of \(A^\#\)
on \(\psi f\) and the reason for which the local-global passage is not trivial is that
the operator \(A^\#\) and the multiplication by \(\psi\) do not commute. The
commutator \([A^\#,\psi]\), however, is of weight at least \(1\) less than \(A\) and
with the choice \(l\geq k-1\) the norm \(\|[\psi, A^\#]f\|_{W^{k-r,p}(X)}\) is
dominated by \(\|f\|_{W^{l,p}}\).

\begin{enumerate}
\item If \(f\in W^{k,p}(B_\epsilon)\) then \(\psi f\in W^{k,p}(B_\epsilon, \partial)\)
and 
\begin{align*}
\|f\|_{W^{k,p}(B_\delta)} &\leq \| \psi f\|_{W^{k,p}(X)} \leq C \left( \|A^\#\psi f\|_{W^{k-r,p}(X)} + \|\psi f\|_{W^{l,p}(X)}  \right)\\
   			  &\leq C \left( \|\psi A^\# f\|_{W^{k-r,p}(X)} + \| [\psi, A^\#] f \|_{W^{k-r,p}(X)} + \|\psi f \|_{W^{l,p}(X)}  \right)\\
			  &\leq C' \left( \| Af \|_{W^{k-r,p}(B_\epsilon)} + \| f \|_{W^{l,p}(B_\epsilon)} \right)
\end{align*}
\item Given \(f\in W^{l,p}(B_\epsilon)\) and \(Af\in W^{k-r,p}(B_\epsilon)\), one has \(\psi f\in W^{l,p}(X)\). Also, \([A^\#, \psi] f \in W^{l-r+1,p}(X)\subset W^{k-r,p}(X)\) and \(\psi A^\# f = \psi A f\in W^{k-r,p}(X)\), therefore \(A^\#(\psi f)\in
   W^{k-r,p}(X)\). By exactness of \(A^\#\), one has \(\psi f\in W^{k,r}(X)\), so \(f\in W^{k,r}(B_\delta)\).
\item If \(g\in W^{l-r,p}(B_\delta/ \partial) \subset W^{l-r,p}(X)\), by exactness
of \(A^\#\) we can find
\(\tilde f \in W^{l,p}(X)\) such that \(g-A^\# \tilde f \in W^{k-r,p}(X)\). Choose \(f =
   \psi \tilde f \in W^{l,p}(B_\epsilon/ \partial)\) then 
\[
    g- Af = g - A^\#(\psi \tilde f) = \psi(g- A^\# \tilde f) + [\psi, A^\#]\tilde f \in W^{k-r,p}(B_\epsilon)
   \]
since \(\psi(g- A^\# \tilde f)\in W^{k-r,p}(B_\epsilon)\) and \([\psi, A^\#]\tilde f
   \in W^{l-r+1,p}(B_\epsilon) \subset W^{k-r,p}(B_\epsilon)\).
\end{enumerate}
\end{proof}

\begin{lemma}
\label{lem:ell-loc-bndry}
With \((A,B)\) and \(\epsilon, k,l\) as in Proposition \ref{prop:boundary-cube} with the
extra condition \(l\geq k-1\), then for all
\(\delta < \epsilon\), one has
\begin{enumerate}
\item \(\|f\|_{W^{k,p}(B^+_\delta)}\leq C \left( \|Af\|_{W^{k-r,p}(B^+_\epsilon)} +
   \sum_{j=1}^m \|B^j f\|_{\partial W^{k-r_j,p}(\partial_0 B^+_\epsilon)}
   +\|f\|_{W^{l,p}(B^+_\epsilon)} \right)\) for all \(f\in W^{k,p}(B^+_\epsilon)\).
\item If \(f\in W^{l,p}(B^+_\epsilon)\) and \(Af\in W^{k-r,p}(B^+_\epsilon)\) and \(B^j
   f\in \partial W^{k-r_j,p}(\partial_0 B^+_\epsilon)\) then actually \(f\in W^{k,p}(B^+_\delta)\).
\item If \(g\in W^{l-r,p}(B^+_\delta/ \partial_e )\) and \(h_j\in \partial
   W^{l-r_j,p}(\partial_0 B^+_\delta/ \partial)\) then there exists \(f\in
   W^{l,p}(B^+_\epsilon, \partial_e)\) with
\[
    g-Af \in W^{k-r,p}(B^+_\epsilon, \partial_e),\quad h_j-B^j f \in \partial
   W^{k-r_j,p}(\partial_0 B^+_\epsilon/ \partial).
   \]
\end{enumerate}
\end{lemma}


The generalisation of Theorem \ref{thm:elliptic-const} on manifold with variable
coefficients is now straightforward. The only nontrivial issue is the definition of
semi-elliptic operator \(A\) on manifold. This requires a Riemannian metric \(g\) and
ellipticity is naturally defined at every point, viewed in a chart, as we did
\href{./interpolation-sobolev.org}{before}. But this only defines the action of \(A\) on \(C^\infty(M)\) (or \(C^r(M)\)
if regularity is important), but not on \(W^{k,p}(M/ \mathcal{A})\) where \(\mathcal{A}\subset \partial M\) is a connected component.

The action of a differential operator \(A\) can be defined to be component-wise on \(W^{k,p}(M/ \mathcal{A}) \hookrightarrow \bigoplus_i
W^{k,p}(\mathcal{R}_i/\mathcal{A}_i)\) where \(\mathcal{R}_i\) is an Euclidean plan or a half-plan and \(\mathcal{A}_i\) the corresponding boundary part,
i.e.
\[
 \xymatrix{
W^{k,p}(M/\mathcal{A}) \ar@{->}[r]^{\iota} \ar@{-->}[d]_{{A}} & \bigoplus_i W^{k,p}(\mathcal{R}_i/\mathcal{A}_i) \ar@{->}[d]^{{A}} \\
W^{l,p}(M/\mathcal{A}) \ar@{->}[r]^{\iota} & \bigoplus_iW^{l,p}(\mathcal{R}_i, \mathcal{A}_i)
}
\]
It remains to check that the component-wise operation of \({A}\) maps an element
in the image
on \(W^{k,p}(M/\mathcal{A})\) to an element in the image of \(W^{l,p}(M/\mathcal{A})\). This can be done using the projection as we did when defining trace operator on
manifold, but the situation is much simpler here since we can differentiate directly an element
in \(\mathcal{S}^*(M)\).



\begin{theorem}[Elliptic equation on manifold]
\label{thm:elliptic-general}
Let \(M\) be a compact manifold possibly with boundary (and a compatible foliation if
the weight is not uniform). Let \(A\) be a general semi-elliptic operator of weight \(r\), of variable coefficients and \(\{B^j\}_j\) be a set boundary operators of weight \(r_j\) satisfying
CBC with respect to \(A\). Then for all \(\frac{\sigma}{\rho p}+\max_j r_j < l<
k<+\infty\), the square
\[
 \xymatrix{
W^{k,p}(M) \ar@{->}[r]^<<<<<{\mathcal{C}} \ar@{^{(}->}[d]_{\iota} &
W^{k-r,p}(M)\bigoplus_{j=1}^m \partial W^{k-r_j,p}(\partial M) \ar@{^{(}->}[d]^{\iota} \\
W^{l,p}(M) \ar@{->}[r]_<<<<<{\mathcal{C}} & W^{l-r,p}(M)\bigoplus_{j=1}^m \partial
W^{l-r_j,p}(\partial M) 
}
\]
is exact where \(\mathcal{C}=(A,B^j)\).
\end{theorem}

\begin{proof}
We can suppose \(l\geq k-1\), the general case follows using

\begin{lemma}
\label{lem:compose-exact}
If the two following squares are exact
\[
 \xymatrix{
E \ar@{->}[r]^{l} \ar@{->}[d]_{m} & F \ar@{->}[d]^{p} \\
G \ar@{->}[r]_{q} & H
},\quad
\xymatrix{
G \ar@{->}[r]^{q} \ar@{->}[d]_{r} & H \ar@{->}[d]^{s} \\
K \ar@{->}[r]_{t} & L
}
\]
then
\[
 \xymatrix{
E \ar@{->}[r]^{l} \ar@{->}[d]_{rm} & F \ar@{->}[d]^{sp} \\
K \ar@{->}[r]_{t} & L
}
\]
is exact.
\end{lemma}

Now covering \(M\) by finitely many charts of type \(B_\delta\subset B_\epsilon\) and
\(B^+_\delta\subset B^+_\epsilon\) such that the interior of \(B_{\delta}\) and of \(B^+_{\delta}\)
cover \(M\). Also, choose a partition of unity \(\sum \psi = 1\) subordinated to \(B_\delta\) and \(B^+_{\delta}\). The exactness will be established if we can prove the
analogue of the 2 last statements of Theorem \ref{thm:elliptic-const}

For the regularity statement: If \(f\in W^{l,p}(M)\), \(Af\in W^{k-r,p}(M)\) and \(B^j f\in W^{k-r_j,p}(\partial M)\) then the same holds for \(\psi f\) in \(B_\epsilon\) and \(B_\epsilon^+\) since 
\[
 [A,\psi]f\in W^{l-r+1,p}\subset W^{k-r,p},\quad [B^j,\psi] f \in
\partial W^{l-r_j+1,p}\subset \partial W^{k-r_j,p}
\]
Therefore \(\psi f \in W^{k,p}(B_\delta)\) or \(W^{k,p}(B^+_\delta)\) hence \(f\in W^{k,p}(M)\).

For the approximation: If \(g\in W^{l-r,p}(M)\) and \(h_j\in \partial
W^{l-r_j,p}(\partial M)\) then \(\psi g\in W^{l-r,p}(B_\delta/ \partial)\) or
\(W^{l-r,p}(B_\delta^+/ \partial_e)\) and \(\psi h_j\in \partial W^{l-r_j,p}(\partial_0
B^+_\delta / \partial)\). Then by Lemma \ref{lem:ell-loc-bndry}, we can find \(\tilde f\in
W^{l,p}(B_\epsilon/\partial)\) with \(\psi
g - A \tilde f \in W^{k-r,p}(B_\epsilon/ \partial)\) or in a boundary cube  \(\tilde f\in
W^{l,p}(B^+_\epsilon/ \partial_e)\) with \(\psi
g - A \tilde f \in W^{k-r,p}(B^+_\epsilon/ \partial_e)\) with \(\psi h_j - B^j \tilde
f\in \partial W^{k-r_j,p}(\partial_0 B^+_\epsilon/ \partial)\).
Then \(f:=\sum \tilde f\) makes sense and satisfies
 \begin{cases}
g - Af = \sum (\psi g - A\tilde f)  & \text{is in $W^{k-r,p}(M)$} \\
h - B^j f = \sum (\psi h_j - B^j \tilde f)  & \text{is in $ \partial W^{k-r_j,p}(\partial M)$}
 \end{cases}
\end{proof}


\subsection{Consequences of Theorem \ref{thm:compact-op-exact}.}
\label{sec:orge2f277f}
Under the same setup as Theorem \ref{thm:elliptic-general}, one has

\begin{theorem}[Regularity of kernel and cokernel]
\label{thm:ref-kernel}
The map \(\mathcal{C} = (A,B): W^{k,p}(M) \longrightarrow W^{k-r,p}(M) \bigoplus_{j=1}^m
\partial W^{k-r_j,p}( \partial M)\) has closed range, finite dimensional kernel and
cokernel and the kernel and cokernel are independent of \(k\) in the sense of Theorem
\ref{thm:compact-op-exact}. In particular, \(\ker C \subset C^\infty(M)\)
\end{theorem}

The analoguous regularity for cokernel is less straightforward. We resume here the result.

\begin{theorem}[Regularity of cokernel]
\label{thm:reg-coker}
If \(r>\max r_j\) then the image of \(\mathcal{C}\) can be represented by finitely
many linear relations: \((g,h)\in \im \mathcal{C}\) if and only if it satisfies finitely
many equations of type:
\[
 \langle g,\gamma \rangle_M + \sum_{j=1}^m \langle h_j, \eta_j \rangle_{\partial M} = 0
\]
with \(\gamma\in C^\infty(M)\) and \(\eta_j\in C^\infty (\partial M)\).

If \(\max r_j -r = k \geq 0\) then for all \(g\in W^{k-r,p}(M)\), the normal
derivatives \(\frac{\partial g}{\partial \nu^i}\) are well defined if \(\frac{\sigma_i}{\rho}\leq k\). The cokernel is then given by the relations
\[
 \langle g, \gamma \rangle_M + \sum_{\sigma_i/\rho \leq k} \langle \frac{\partial
g}{\partial \nu^i},\chi_i \rangle_{\partial M} + \sum_{j=1}^m \langle h_j, \eta_j
\rangle_{\partial M}=0
\]
with \(\gamma\in C^\infty(M)\), \(\chi_i\in C^\infty (\partial M)\), \(\eta_j\in
C^\infty (\partial M)\). 
\end{theorem}


\section{Parabolic equation on manifold.}
\label{sec:org414d75c}

\subsection{Parabolicity and local results.}
\label{sec:org4c87ce2}

\begin{definition}
The constant coefficient differential operator \(A(D_x,D_t) =
\sum_{\|(\alpha,\beta)\|\leq r}a_{\alpha\beta}D^\alpha_x D^\beta_t\) is called
\textbf{parabolic} if its symbol \(A(\xi,\theta):= \sum_{\|(\alpha,\beta)\| = r}
a_{\alpha\beta}\xi^\alpha\theta^\beta\) has no zero when \(\xi\in \mathbb{R}\) and \(\im \theta \leq 0\) except \(\xi=\theta=0\).
\end{definition}

\begin{exampl}
Take \(A= \partial_t - \partial_x^2 - \partial_y^2 - \partial_z^2 = iD_t + D_x^2 +
D_y^2 + D_z^2\), the symbol is \(i\theta + \sum \xi_i^2\) has no zero \(\xi\in
\mathbb{R}^3,\im\theta \leq 0\) except \(0\). Generally, the operator \(\partial_t + A(D_{x^i})\) is parabolic if \(A\) is an elliptic operator with the symbol
\(A(\xi)\geq 0\) for all \(\xi\in \mathbb{R}\) with equality only at \(\xi=0\).
\end{exampl}

\begin{remark}
\begin{enumerate}
\item If \(\sigma = \lcm (\sigma_1,\dots,\sigma_n)\) is the \(\lcm\) of weights of variable
\(x_i\) and \(\tau\) is the weight of \(t\), then parabolicity implies \(2\tau \mid
   \sigma\). Therefore if the weights of \(x_i\) are uniform, one can suppose that \(\tau
   = 1\).
\item Parabolicity implies ellipticity.
\end{enumerate}
\end{remark}

Similarly to the elliptic case, we attempt to define an approximate inverse \(G\) of \(A\), of the form
\[
 G(\xi,\theta) = (1-\psi(\xi,\theta))/A(\xi,\theta)
\]
such that \(G(D_x,D_t): W^{k-r,p}(X\times T^+/0) \longrightarrow  W^{k,p}(X\times T^+/0)\) and \(\psi(D_x,D_t): W^{k,p}(X\times T^+/0) \longrightarrow W^{k,p}(X\times T^+/0)\)
for all \(k,l\in \mathbb{R}\).

The sufficient condition for this is that \(\psi(\xi,\theta) = \psi(\xi)\hat\chi(\theta)\) where \(\psi\) is compactly support and \(\hat\chi\in \mathcal{S}(T)\) with \(\hat\chi - 1\) having a zero of infinite order at \(\theta = 0\), and \(\hat\chi\)
extends to a holomorphic function in \(\im \theta \leq 0\). The function \(\hat\chi\)
used in section \ref{ell-half-plan} suffices. We then have the following exact square
\[
\xymatrix{
W^{k,p}(X\times T^+/0) \ar@{->}[r]_{A} \ar@{^{(}->}[d]^{\iota} & W^{k-r,p}(X\times T^+/0) \ar@{^{(}->}[d]_{\iota} \ar@/_/@{->}[l]_{G} \\
W^{l,p}(X\times T^+/0) \ar@{->}[r]^{A} \ar@/^/@{->}[u]^{\psi} & W^{l-r,p}(X\times T^+/0) \ar@/^/@{->}[l]^{G} \ar@/_/@{->}[u]_{\psi}
}
\]

The theory in section \ref{ell-half-plan} also allows us to treat spatial boundary
condition, that is, to replace the Euclidean plan \(X\) by the half-plan \(X\times Y^+\). The analog of \textbf{CBC condition} for boundary operators
\[
 B^j(D_x,D_y,D_t) = \sum_{\|(\alpha,\beta,\gamma)\|\leq r_j} b_{\alpha\beta\gamma}^j
D_x^\alpha D_y^\beta D_t^\gamma
\]
is that the symbols
\[
B^j(\xi,\eta,\theta) = \sum_{\|(\alpha,\beta,\gamma)\|= r_j} b_{\alpha\beta\gamma}^j
\xi^\alpha \eta^\beta \theta^\gamma
\]
are linearly independent modulo \(A^+(\xi,\eta,\theta)\) as polynomial in \(\eta\) for
all \(\xi\in \mathbb{R}^n\) \textbf{and} for all \(\im\theta \leq 0\) except when \(\xi=\theta= 0\). In that case we have the exactness of
\[
\xymatrix{
W^{k,p}(X\times Y^+\times T^+/0) \ar@{->}[r]^<<<<<{(A,B^j)} \ar@{^{(}->}[d]^{\iota} & W^{k-r,p}(X\times Y^+\times T^+/0)\bigoplus_{j=1}^m \partial W^{k-r_j,p}(X\times T^+/0) \ar@{^{(}->}[d]_{\iota} \\
W^{l,p}(X\times Y^+\times T^+/0) \ar@{->}[r]^<<<<<{(A,B^j)} & W^{l-r,p}(X\times Y^+\times T^+/0)\bigoplus_{j=1}^m\partial W^{l-r_j,p}(X\times T^+/0)
}
\]


\subsection{Global results and causality. \label{sec:global-result-para}}
\label{sec:org33fecf2}

We will use the following setup. Let \(M\) be a compact manifold (possibly with boundary), of the form \(N\times
[\alpha,\omega]\ni (x,t)\). The global product gives a foliation that allows us to set the
spatial weight to be uniformly \(\sigma\) and the temporal weight to be \(\tau\). The
boundary of \(M\) has 3 parts: \(\partial_\alpha M := N\times \alpha\), \(\partial_\omega M := N\times\omega\) and \(\partial_S M := \partial N\times [\alpha,\omega]\).

Let \(A\) be a parabolic operator, meaning that \(A\) is parabolic at every point and
\(B^j, j=\overline{1,m}\) be a set of boundary operator satisfying CBC condition at
every point on \(\partial_S M\). We take into account the initial condition by only
considering the space \(W^{k,p}(M/\partial_\alpha)\) of function vanishing before time \(t=\alpha\). As before the operator
\[
 \mathcal{C}:=(A,B^j):\  W^{k,p}(M/ \partial_\alpha) \longrightarrow W^{k-r,p}(M/
\partial_\alpha)\bigoplus_{j=1}^m \partial W^{k-r_j,p}(\partial_S M/ \partial_\alpha)
\]
has closed range, finite dimensional kernel and cokernel which are independent of \(k >
\frac{1}{p}+\max r_j\).

The same method allows us to conclude that \(\ker \mathcal{C}\subset C^\infty(M)\) and
the cokernel is given by finitely many linear relations of type
\[
\langle g,\gamma \rangle_M + \sum_j \langle h_j,\eta_j \rangle_{\partial M} + \sum_i \langle
\frac{\partial}{\partial \nu^i} g ,\chi_i \rangle_{\partial_S M}
\]
where \(\gamma\in C^\infty (M/ \partial_\omega)\), \(\chi_i\in C^\infty(\partial_S M/
\partial_\omega)\) and \(\eta_j\in C^\infty(\partial_S M/ \partial_\omega)\).

The difference with elliptic equation is that the kernel and cokernel of \(\mathcal{C}\)
are not only of finite dimension, but are zero.

\begin{theorem}[Causality]
\label{thm:para-causality}
With the previous setup, the operator \(\mathcal{C}=(A, B^j)\) defines an isomorphism
\[
\xymatrix{
W^{k,p}(M/\partial_\alpha) \ar@{->}[r]^<<<<<{\mathcal{C}} & W^{k-r,p}(M/\partial_\alpha)\bigoplus_{j=1}^m \partial W^{k-r_j,p}(\partial_S M/\partial_\alpha)
}
\]
for all \(k > \frac{1}{p} + \max_j r_j\), and therefore an isomorphism
\[
 \xymatrix{
C^\infty(M/\partial_\alpha) \ar@{->}[r]^<<<<<{\mathcal{C}} & C^\infty(M/\partial_\alpha)\bigoplus_{j=1}^mC^\infty(\partial_S M/\partial_\alpha)
}
\]
\end{theorem}

\begin{proof}
Let \(\beta \leq \gamma\) be real numbers in \([\alpha,\omega]\) and let \(\ker
(\beta,\gamma)\) and \(\coker (\beta,\gamma)\) be the kernel and cokernel of operator
\(\mathcal{C}\) on \(N\times [\beta,\gamma]\) with vanishing initial condition at \(\beta\). Since \(\dim \ker (\beta,\gamma)\) and \(\dim\coker (\beta,\gamma)\) are
integer-valued, using the fact that \(\dim \ker (\beta,\omega)\) is decreasing in \(\beta\) and \(\dim\coker (\alpha,\gamma)\) is increasing in \(\gamma\), one can
easily check that it suffices to show that the two functions are
continuous in \((\beta,\gamma)\) to prove that they are identically \(0\).

The following statements can be verified mechanically:
\begin{enumerate}
\item \emph{Monotonicity:} \(\dim\ker(\beta,\gamma)\) is decreasing in \(\beta\), \(\dim\coker
   (\beta,\gamma)\) is increasing in \(\gamma\).
\item \emph{One-sided continuity:} \(\dim \ker (\beta,\gamma)\) is left-continuous in \(\beta\),
\(\dim\coker (\beta,\gamma)\) is right-continuous in \(\gamma\).
\item \emph{One-sided semi-continuity:} \(\dim\ker (\beta,\gamma)\) is left upper semi-continuous in
\(\gamma\), i.e.
\[
   \lim_{\gamma_1 \to \gamma_2^-} \inf \dim \ker (\beta,\gamma_1) \geq \dim\ker (\beta,\gamma_2)
   \]
This is due to the left-continuity in first variable of \(\dim\ker\) and the exact
sequence
\[
    0 \longrightarrow \ker (\gamma_1,\gamma_2) \longrightarrow \ker(\beta,\gamma_2) \longrightarrow \ker(\beta,\gamma_1)
   \]
 where the last arrow is the restriction. Similar statement for \(\coker\):
\[
    \lim_{\beta_2 \to \beta_1^+} \dim \inf \coker (\beta_2,\gamma) \geq \dim\coker (\beta_1,\gamma)
   \]
\end{enumerate}

This 3 statements suffice to finish the proof in the case where boundary conditions \(B^j\) on \(\partial_S M\) are of constant coefficients since \(\ker, \coker\)
only depend on the difference \(\gamma -\beta\), up to a translation in time of the solutions.

In case \(B^j\) are of variable coefficients, the idea of making translation in time can
be formulated using Index theory for Fredholm maps:

We recall that Fredholm maps between Banach spaces \(E, F\) are those in \(L(E,F)\) with
closed image and finite dimensional kernel and cokernel. It is a classical result that
\begin{enumerate}
\item The set \(\mathcal{F}\) of Fredholm maps are open in \(L(E,F)\).
\item The index \(i(l):= \dim \ker l - \dim \coker l\) is continuous in \(\mathcal{F}\).
\end{enumerate}

The difference \(\dim \ker (\beta,\gamma) -\dim\coker (\beta,\gamma)\) can be
regarded as the index of a continuous family \(\mathcal{C}_{(\beta,\gamma)}\) of
operators on the same space \(N\times [0,1]\) using the diffeomorphism
\[
 N\times [0,1] \xrightarrow{\sim} N\times [\beta,\gamma].
\]
Hence \(\dim\ker (\beta,\gamma) - \dim \coker (\beta,\gamma)\) is constant. It follows
that \(\dim \ker (\beta,\gamma)\) is both increasing and one-sided semi-continuous in \(\gamma\) hence is right-continuous in \(\gamma\), hence \(\dim \coker (\beta,\gamma)\) is continuous in \(\gamma\). Other continuities follows similarly.  
\end{proof}

\begin{remark}
\label{rem:init-cond-para}
To take into account the initial condition \(\restr{f}{\alpha} = f_\alpha\) smooth, one looks
for solution of the form \(f = f_b + f_\#\) where \(f_b\) satisfies the initial
condition and \(f_\#\in W^{k,p}(N\times[\alpha,\omega]/\alpha)\) satisfying a parabolic
equation \((A f_\#, B^j f) = (g,h)\) where \(g,h\) and the coefficients of \(A\) and
\(B^j\) depend smoothly on \(f_b\), and therefore still \(C^\infty\) in \((x,t)\).
\end{remark}



\subsection{Regularisation effect and Gårding inequality.}
\label{sec:org13c850a}

With the same technique used for elliptic equation, one can also prove regularity result
for parabolic equation. There are 2 different points, in comparison with the elliptic
case:
\begin{enumerate}
\item There is a regularisation effect of parabolic equation: An arbitrarily weak estimate in
the past gives an arbitrarily strong estimate in the future. We will see that this is
in fact a consequence of the causality of parabolic equation (Theorem \ref{thm:para-causality}) and Kondrachov's theorem.
\item The temporal boundary condition is thicken: We will look at the norm on \(N\times [\alpha,\pi]\) rather
than the restriction to \(\partial_\alpha M\).
\end{enumerate}

\begin{theorem}[Regularity and Garding inequality]
\label{thm:reg-parabolic}
Under the same setup and notation as Section \ref{sec:global-result-para}, let \(p\in
(1,+\infty)\) and \(k>l> \frac{1}{p}+\max r_j\). We denote by \(W^{k,p}([\beta,\gamma])\) the Sobolev space \(W^{k,p}(N\times[\beta,\gamma])\). Suppose that
\[
 f\in W^{l,p}([\alpha,\omega]),\quad Af\in W^{k-r,p}( [\alpha,\omega]),\quad
B^j f \in \partial W^{k-r_j,p}( [\alpha,\omega])
\]
then \(f\in W^{k,p}([\pi,\omega])\) for all \(\pi \in (\alpha,\omega)\). Also, for all  \(l' > -\infty\), there exists a constant \(C>0\) such that
\[
 \|f\|_{W^{k,p}([\pi,\omega])} \leq C \left( \|Af\|_{W^{k-r,p}([\alpha,\omega])} + \|B^j
f\|_{\partial W^{k-r_j,p}([\alpha,\omega])} + \| f \|_{W^{l',p}([\alpha,\pi])}  \right).
\]
In particular, for homogeneous equation, i.e. \(Af = 0, B^j f = 0\), the solution is \(C^\infty\) and an arbitrarily weak estimate in the past gives an arbitrarily strong
estimate in the future.
\end{theorem}

\begin{proof}
Let us explain why the theorem is true in the case of no spatial boundary \(\partial N
=\emptyset\). In this case, there is no distinction between \(l\) and \(l'\). Consider \(A\) as an elliptic operator on \(N\times [\tilde\pi,\omega]\) with \(\tilde\pi = \frac{\alpha +\pi}{2}\) and with no boundary operator, one has the following exact diagram:
\[
\xymatrix{
W^{k,p}([\tilde\pi,\omega]) \ar@{^{(}->}[d] \ar@{->}[r]^<<<<<{{A}} & W^{k-r,p}([\tilde\pi,\omega])   \ar@{^{(}->}[d] \\
W^{l,p}([\tilde\pi,\omega]) \ar@{->}[r]^<<<<<{{A}} & W^{l-r,p}([\tilde\pi,\omega])
}
\]
Therefore the if \(f\in W^{l,p}([\alpha,\omega])\) and \(Af \in W^{k-r,p}([\alpha,\omega])\)then \(f \in W^{k,p}([\tilde\pi,\omega])\subset W^{k,p}([\pi,\omega])\) and 
\begin{align}
\|f\|_{W^{k,p}([\pi,\omega])} \leq \|f\|_{W^{k,p}([\tilde\pi,\omega])} &\leq C \left( \|Af\|_{W^{k-r,p}([\alpha,\omega])} + \| f \|_{W^{l,p}([\alpha,\omega])}  \right) \\ 
			      	   &\leq C \left( \|Af\|_{W^{k-r,p}([\alpha,\omega])} + \| f \|_{W^{l,p}([\alpha,\pi])} + \| f \|_{W^{l,p}([\tilde\pi,\omega])} \right) \label{eq:reg-para}
\end{align}
It remains to check that we can get rid of the \(\| f \|_{W^{l,p}([\tilde\pi,\omega])}\) 
term on the right hand side. Suppose not, then there exists a sequence \(\{f_i\} \subset
W^{l,p}([\alpha,\omega])\) such that \(Af_i \to 0\) in \(W^{k-r,p}([\alpha,\omega])\) and
\(f_i \to 0\) in \(W^{l,p}([\alpha,\pi])\) but \(\| f_i\|_{W^{l,p}([\tilde\pi,\omega])} =
1\). Then by \eqref{eq:reg-para}, \(\{f_i\}\) is a bounded sequence in \(W^{k,p}([\tilde\pi,\omega])\) and, by
Kondrachov's theorem, can be supposed to converge in \(W^{l,p}([\tilde\pi,\omega])\) to a
function \(\tilde f\) which has \(\| \tilde f \|_{W^{l,p}([\tilde\pi,\omega])} =
1\) and \(A\tilde f = 0\) on \([\tilde\pi,\omega]\) because \(A\) commutes with the restriction. Moreover, since \(\|f_i\|_{W^{l,p}([\alpha,\pi])} \to 0\), one has \(\tilde f\in W^{l,p}([\tilde\pi,\omega]/\tilde\pi)\) and the fact that \(\tilde f\ne 0\) contradicts Theorem \ref{thm:para-causality}).
\end{proof}

\begin{remark}
The proof of Theorem \ref{thm:reg-parabolic} in the general case, with spatial boundary
taken into account requires the notion of bigraded Sobolev spaces on half-plan, see
\cite[page 97-100]{hamilton_harmonic_1975}. This is also how the regularity result for
cokernel of elliptic operator, Theorem \ref{thm:reg-coker}, is proved.
\end{remark}


\section{Example: Linear heat equation.}
\label{sec:org0e058a2}

We use the same setup of \(M,N,\alpha,\omega\) as Section
\ref{sec:global-result-para}. Let \(\Delta\) be the (geometer's)
Laplacian
\[ 
-\Delta f:= g^{ij}(x) \left( \frac{\partial^2 f}{\partial x^i \partial x^j} - \Gamma_{ij}^k(x)
\frac{\partial f}{\partial x^k}\right) 
\]
It is easy to check that \(\Delta\) is an elliptic operator with symbol \(\Delta \geq
0\) (there is a factor \(i\) when passing from \(\frac{\partial}{\partial x^i}\) to \(D_{x^i}\)). Hence on \(M = N\times [\alpha,\omega]\) the operator \(\frac{\partial}{\partial t}+ \Delta\) is parabolic.

\subsection{Linear system.}
\label{sec:org5395cc9}
We will look at the linear parabolic system of equations for \(F = (f^1,\dots, f^n): M
\longrightarrow \mathbb{R}^n\):
\begin{equation}
\label{eq:para-system}
\frac{\partial F}{\partial t} + \Delta F + a \nabla F + bF = G
\end{equation}
where in local coordinates \((a \nabla F)^\alpha = a^{\alpha i}_\beta \frac{\partial f^\beta}{\partial x^i}\)
and \((bF)^\alpha = b^\alpha_\beta f^\beta\) and \((\Delta F)^\alpha = \Delta
f^\alpha\) and the coefficients \(a^{\alpha i}_\beta\) and \(b^\alpha_\beta\) are
smooth.

We will say that a function \(F=(f^1,\dots, f^n): M \longrightarrow \mathbb{R}^n\) of class
\(W^{k,p}\) if it is \(W^{k,p}\) component-wise. We also denote abusively by \(W^{k,p}(M)\) the direct sum \(W^{k,p}(M)^{\oplus n}\) where \(F\) belongs to.

\begin{theorem}[Linear heat equation]
\label{thm:lin-heat}
Let \(p > \dim M+1 = \dim N+2\) and \(k\geq 0\), then for all \(G\in
W^{k,p}(N\times[\alpha,\omega]/\alpha)\), there exists a unique \(F\in
W^{k+2,p}(N\times [\alpha,\omega]/\alpha)\) that solves
\eqref{eq:para-system}. Moreover, the operator 
\[ 
F\longmapsto \frac{\partial F}{\partial t} + \Delta F +
a\nabla F + b F
\]
is an isomorphism between Banach spaces \(W^{k+2,p}(N\times[\alpha,\omega]/\alpha) \longrightarrow
W^{k,p}(N\times[\alpha,\omega]/\alpha)\).
\end{theorem}
\begin{proof}
Note that
\begin{align*}
H:\  W^{k+2,p}(N\times[\alpha,\omega]/\alpha) &\longrightarrow W^{k,p}(N\times[\alpha,\omega]/\alpha)\\
     F					      &\longmapsto      \frac{\partial F}{\partial t} + \Delta F
\end{align*}
is a direct sum of parabolic operators in each component, and hence an isomorphism, and 
\begin{align*}
K:\  W^{k+2,p}(N\times[\alpha,\omega]/\alpha) &\longrightarrow W^{k,p}(N\times[\alpha,\omega]/\alpha)\\
     F					      &\longmapsto      a\nabla F + bF
\end{align*}
is a compact operator because it factors through \(W^{k+1}(N\times
[\alpha,\omega]/\alpha)\). Therefore \(H+K\) is a Fredholm map with the same index as \(H\), which is \(0\). It is sufficient to check that the kernel of \(H+K\) is
trivial.

Suppose that \(F = (f^1,\dots, d^n)\in \ker (H+K)\) then \(f^\alpha\in W^{2,p}(N\times
[\alpha,\omega]/\alpha)\), so \(f^\alpha\) and \(\frac{\partial f^\alpha}{\partial
x^i}\) are continuous function on \(N\times[\alpha,\omega]\). Since
\[
 \frac{\partial f^\alpha}{\partial t} + \Delta f^\alpha = -a^{\alpha i}_\beta
\frac{\partial f^\beta}{\partial x^i} - b^\alpha_\beta f^\beta,
\]
by repeated use of Theorem \ref{thm:reg-parabolic} the \(f^\alpha\) are smooth for \(t>\alpha\).

Let \(e:= \frac{1}{2}|F|^2 := \frac{1}{2}\sum_\alpha |f_\alpha|^2\), then \(e\) is
continuous on \(N\times [\alpha,\omega]\), vanishes on \(N\times \{\alpha\}\) and one has
\begin{align*}
 \frac{d e}{dt} &= -\Delta e - |\nabla F|^2 - a^{\alpha i}_\beta f^\alpha \frac{\partial f^\beta}{\partial x^i} - b^\alpha_\beta f^\alpha f^\beta  \\
 	 	&\leq -\Delta e + \frac{1}{2} C |F|^2 = -\Delta e + Ce
\end{align*}
where we used the inequality \(-u^2 - 2uv \leq v^2\) to bound the second and third
terms. We conclude that \(F=0\) since \(e= 0\) by the following Maximum principle. 
\end{proof}

\subsection{Maximum principle and \(L^\infty\)-Comparison theorem.}
\label{sec:org641166d}

With the same proof as for open set in \(\mathbb{R}^n\), one has the maximum principle
for parabolic equation on manifolds. The constant \(C\) in the following Theorem
\ref{thm:max-princ-para} can depend on the point \(x\in M\), but will be most of the time
globally constant, since the manifold \(M\) is compact. The following statement of
Maximum principle will be sufficient for most of our application.

\begin{theorem}[Maximum principle]
\label{thm:max-princ-para}
Let \(f: M \longrightarrow \mathbb{R}\) be a continuous function on \(M = N\times
[\alpha,\omega]\) with \(\restr{f}{\partial_\alpha M} \leq 0\) and \(\restr{f}{
\partial_S M} \leq 0\). Suppose that whenever \(f>0\), \(f\) is smooth satisfies
\[
 \frac{\partial f}{\partial t} \leq -\Delta f + Cf
\]
Then in fact \(f \leq 0\).
\end{theorem}

With the same proof as Theorem \ref{thm:max-princ-para}, one can prove the following \(L^\infty\) Comparison theorem.

\begin{theorem}[\( L^\infty \)-Comparison theorem]
\label{thm:infty-comparison}
Let \(f:\ M=N\times[\alpha,\omega] \longrightarrow \mathbb{R}\) be a continuous function
on \(M\), smooth for time \(t>0\) such that
\begin{equation}
\label{eq:infty-comparison}
 \frac{d f}{dt} = -\Delta f + a\nabla f+ b f \text{ on } N\times (\alpha,\omega]
\end{equation}
where \(a\) is a smooth vector field and \(b\) is a smooth function on \(N\). Then there exists \(B=B(a,b)\) depending only on \(a\) and \(b\) such that
\[
 \|\restr{f}{\omega}\|_{L^\infty} \leq e^{B(\omega-\alpha)}
\|\restr{f}{\alpha}\|_{L^\infty}
\]
\end{theorem}
\begin{proof}
We can suppose \(b \leq -1\) and prove that \(\|f\|_{L^\infty(\partial_\omega M)} \leq
\|f\|_{L^\infty(\partial_\alpha M)}\). Intuitively, this means that since heat spreads
out, the largest density must be attained at time \(t=\alpha\).
In fact, choose \(B = \max_M b + 1\) and define \(\tilde f = f e^{-B(t-\alpha)}\) then
\(\|\restr{\tilde f}{\alpha}\|_{L^\infty} = \| \restr{f}{\alpha} \|_{L^\infty}\) and \(\| \restr{\tilde f}{\omega}\|_{L^\infty} =
e^{-B(\omega-\alpha)}\|\restr{f}{\omega}\|_{l^\infty}\). The function \(\tilde f\)
satisfies the same heat equation \eqref{eq:infty-comparison} as \(f\), with \(b\) replaced by \(b-B \leq -1\).

Now let us prove that under this supposition, \(|f|\) attains it maximum at time \(t=\alpha\). Since we can replace the solution \(f\) of \eqref{eq:infty-comparison} by \(-f\), we can suppose, for sake of contradiction, that \(|f|\) attains it maximum on \(N\times[\alpha,\omega]\) at \((x^*, t^*)\) with \(|f(x^*,t^*)| = f(x^*,t^*) > 0\) and \(t^* >\alpha\). Then one has

\begin{cases}
\nabla f (x^*, t^*)= 0, \\
\frac{d f}{dt}(x^*,t^*) \geq 0, \text{ (this is not true if \( t^*=\alpha \))}\\
\Delta f(x^*, t^*) \geq 0, \\
f(x^*, t^*)>0
\end{cases}

Plugging these in \eqref{eq:infty-comparison}, one has a contradiction.
\end{proof}


\subsection{Backwards heat equation and \(L^1\)-Comparison theorem.}
\label{sec:orgb1e4ade}
We will use backwards heat equation, which is just heat equation with the reversed sense
of time (so with the reversed sign for \(\Delta\) as well), in order to dualise the
estimate of Theorem \ref{thm:infty-comparison} and obtain a \(L^1\) estimate of \(f\) at time \(t=\omega\) in term of its \(L^1\) norm at \(t=\alpha\). In particular,
we prove the following theorem.

\begin{theorem}[\( L^1 \)-comparison theorem]
\label{thm:1-comparison}
Let \(a\) be a smooth, divergence-free vector field on a Riemannian manifold \(N\) and
\(b\) be a smooth function on \(N\).Let \(f:\ N\times[\alpha,\omega] \longrightarrow
\mathbb{R}\) be a continuous function on \(M\) such that
\begin{equation}
\label{eq:1-comparison}
 \frac{d f}{dt} = -\Delta f + a\nabla f+ b f \text{ on } N\times (\alpha,\omega].
\end{equation}
Then there exists \(B=B(a,b)\) depending only on \(a\) and \(b\) such that
\[
 \|\restr{f}{\omega}\|_{L^1} \leq e^{B(\omega-\alpha)}
\|\restr{f}{\alpha}\|_{L^1}
\]
\end{theorem}
\begin{proof}
Since \(L^1\) is the dual space of \(L^\infty\), it is sufficient to prove that for
all \(h\in C^\infty(N)\), one has
\[
\int_{N\times \{\omega\}} fh \leq e^{B(\omega-\alpha)}
\left\|\restr{f}{\alpha}\right\|_{L^1}. \|h\|_{L^\infty}.
\]

Consider the backwards heat equation
\begin{cases}
\frac{d g}{dt} = \Delta g - \tilde a\nabla g - \tilde b g,  & \text{on $ N\times{[\alpha,\omega]}$} \\
\restr{g}{\omega} = h, 
\end{cases}
which is just a heat equation on \(N\times[\alpha,\omega]\) with initial condition at \(\alpha\) if we pose \(\tilde g (t) :=
g(\omega+\alpha - t)\). The solution \(g\) exists and is smooth on \(N\times[\alpha,\omega]\). One has, at any time \(t\)
\begin{align*}
  \int_{N} g \Delta f &= \int_N g\left(-\frac{d f}{dt} + a\nabla f + bf\right)\\
  \int_{N} f\Delta g &= \int_N f\left( \frac{d g}{dt} +\tilde a \nabla g + \tilde b g \right)
\end{align*}
Therefore
\[
 \int_N  f \frac{dg}{dt} + g \frac{df}{dt} = \int_N (a\nabla f)g - (\tilde a \nabla g)f +
(b-\tilde b) fg
\]
Choose \(b=\tilde b\) and \(\tilde a = -a\) then the term \((b-\tilde b)fg\)
vanishes and the two first terms become
\(\int_N \nabla_a (fg) = - \int_N fg \text{ div } a = 0\)
where \(\text{ div } a := \frac{\partial}{\partial x^i} a^i\) is the
divergence. Therefore one has \(\frac{d}{dt}\int_N fg = 0\), meaning that
\[
\int_{N} \restr{f}{\omega}. h = \int_{N\times{\omega}} fg = \int_{N\times{\alpha}} fg \leq
\|\restr{f}{\alpha} \|_{L^1}. \| h \|_{L^\infty} \leq e^{B(\omega-\alpha)} \|\restr{f}{\alpha}\|_{L^1}. \|h\|_{L^\infty}
\]
where we applied Theorem \ref{thm:infty-comparison} to \(g\) (strictly speaking, to \(\tilde
g\)) and  \(B\) only depends on \(\tilde a = -a\) and \(\tilde b = b\).
\end{proof}