\iffalse
\begin{info}
The PDF version of this page can be downloaded by replacing \texttt{html} in the its address by
\texttt{pdf}. 
For example \texttt{/html/sheaf-cohomology.html} should become \texttt{/pdf/sheaf-cohomology.pdf}.
\end{info}
\fi

\begin{definition}
We say that \(P\) is a \textbf{polynomial differential operator of type \((n,k)\)} if \(P\)
is of the form
\[
 P(F) = \sum c_{\alpha_1,\dots, \alpha_\nu} (x, F(x)) D^{\alpha_1} F^{a_1} \dots D^{\alpha_\nu} F^{a_{\nu}}
\]
where the coefficients \(c_{\alpha_1,\dots,\alpha_nu}\) depend smoothly and nonlinearly on \(x\) and \(F\) and \(\alpha_i\in \mathbb{R}^N\) are indices with the weighted norm \(\|\alpha_i\| \leq k\) and \(\sum \|\alpha_i\| \leq n\).
\end{definition}

\begin{exampl}
On \(M\times [\alpha,\omega]\) the tension field \(\tau(F) :=  -\Delta F^\alpha + g^{ij}\Gamma'^\alpha_{\beta\gamma}(F)
F^\beta_i F^\gamma_j\) is a polynomial differential operator of type (2,2). The quadratic
term alone is of type (2,1).
\end{exampl}

\section{A regularity estimate for polynomial differential operator.}
\label{sec:orgde1591a}
Our goal in this part is to prove the following estimate for polynomial differential
operator, in which \(X\) will be \(M\times[\alpha,\omega]\).

\begin{theorem}[Regularity of polynomial differential operator]
\label{thm:reg-poly-diff}
Let \(X\) be a compact Riemannian manifold, \(B\subset \mathbb{R}^N\) is a
large Euclidean ball and \(P\) be a polynomial differential
operator of type \((n,k)\) on \(X\). Suppose that
\begin{equation}
\label{eq:cond:thm:reg-poly-diff}
 r\geq 0,\quad p,q\in (1,\infty),\quad r+k < s, \quad \frac{1}{p}> \frac{r+n}{s} \frac{1}{q}.
\end{equation}
Then for all \(F\in C (X,B) \cap W^{s,q}(X)\), one has \(PF\in W^{r,p}(X)\) and
\[
 \|P F \|_{W^{r,p}} \leq C\left(1 + \|F\|_{W^{s,q}}\right)^{q/p}.
\]
where \(C\) is a constant independent of \(F\).
\end{theorem}

We will prove that the result is \emph{local}, in a sense to be defined. Then we will prove the
local statement using Besov spaces.

\begin{proof}[Proof (reduction of Theorem \ref{thm:reg-poly-diff} to a local statement)]
Let \(\{ \varphi_i: U_i \longrightarrow V_i \}\) be an atlas of \(M\). We denote a point
in \(U_i\) by \(x\) and its coordinates in \(V_i\) by \(\xi\). Let \(\sum \psi_i
= 1\) be a partition of unity subordinated to \(\{U_i\}\) and \(\tilde \psi_i\)
be smooth functions supported in \(U_i\) with \(0\leq \tilde \psi_i \leq 1\) and \(\tilde \psi_i = 1\) in the support of
\(\psi_i\), as in the definition of Sobolev spaces on manifold. We suppose the following local statement is true:
\begin{lemma}[Local statement]
\label{lem:loc-reg-poly-diff}
Let \(P\) be a polynomial differential operator of type \((n,k)\) and coefficients \(c_{\alpha_1,\dots,\alpha_\nu}(x,F)\) are smooth and vanish when \(x\in \mathbb{R}^{\dim X}\) is
outside of a compact. Let \(B\subset \mathbb{R}^N\) be a large Euclidean ball and \(r,p,q,s\) as in \eqref{eq:cond:thm:reg-poly-diff}. Then for all compactly supported  \(F\in C (\mathbb{R}^{\dim X},B)
\cap W^{s,q}(\mathbb{R}^{\dim X})\), one has
\[
 \|P F \|_{W^{r,p}} \leq C\left(1 + \|F\|_{W^{s,q}}\right)^{q/p}
\]
where the constant \(C\) depends only on \(B\) and the support of \(F\), and not on \(F\).
\end{lemma}
One has
\[
\| PF \|_{W^{r,p}} := \sum_i \|\psi_i P F\|_{W^{r,p}}
\]
where viewed in the chart \(U_i\), each \(\psi_i(x) PF(x)\) is \(\sum_\alpha\psi_i(\xi). c_\alpha
(\xi,g_i). D^\alpha g_i\) where \(g_i = f_i\circ \varphi_i^{-1}\) is \(f_i\) viewed in
the chart. Since \(\tilde \psi_i = 1\) in the support of \(\psi_i\), one has
\[
 \psi_i(\xi). c_\alpha (\xi,g_i). D^\alpha g_i = \psi_i(\xi). c_\alpha(\xi,\tilde \psi_i g_i) D^\alpha
(\tilde\psi_i g_i)
\]
hence by the local statement:
\[
\| \psi_i(\xi). c_\alpha (\xi,g_i). D^\alpha g_i \|_{W^{r,p}} \leq C \left( 1 + 
\|\tilde \psi_i g_i\|_{W^{s,q}} \right)^{q/p} \leq C \left( 1 + \|F \|_{W^{s,q}}\right)^{q/p}.
\]
Therefore \(\|PF\|_{W^{r,p}} \leq m C \left( 1 + \|F \|_{W^{s,q}}\right)^{q/p}\) where
\(m\) is the number of charts we used to cover \(M\).
\end{proof}
\begin{remark}
The use of partition of unity in the last proof is to decompose \(PF = \sum \psi_i PF\)
and not \(F = \psi_i F\) since we no longer have linearity of the operator \(P\) in \(F\).
\end{remark}

\section{Review of Besov spaces \(B^{s,p}\).}
\label{sec:org9828763}
In this part, \(X = \mathbb{R}^n\) coordinated by \((x_1,\dots,x_n)\) with weight \((\sigma_1,\dots,\sigma_n)\). We define
\[
 T_j^v f(x_1,\dots,x_n) := f(x_1,\dots, x_j +v,\dots, x_n),\quad \Delta^v_j := T^v_j - {\rm Id}
\]
for \(f\in \mathcal{S}(X)\).

For the notation, we will denote the Besov spaces by \(B^{s,p}\) with \(s\in \mathbb{R}_{>0}\setminus
\mathbb{Z}\) and \(p\in (1,\infty)\) so that they look similar to Sobolev space \(W^{s,p}\). In \href{https://en.wikipedia.org/wiki/Besov\_space}{a more standard notation}, our spaces \(B^{s,p}\) are denoted by \(B^{s}_{p,p}\)

\begin{definition}
We define \(B^{s,p}\) as the completion of \(\mathcal{S}(X)\) under the norm
\[
\|f\|_{B^{s,p}} := \sum_{\|\gamma\| < s} \|D^\gamma f\|_{L^p} +
\sum_{ s- \frac{\sigma}{\sigma_j} < \|\gamma\| < s}\sup_{v} \frac{\|\Delta^v_j D^\gamma
f\|_{L^p}}{|v|^{(s - \|\gamma\|) \sigma_j/\sigma}}
\]
\end{definition}

We cite here some well-known facts
\begin{enumerate}
\item While Sobolev spaces with non-integral regularity are complex interpolation of integral
ones, Besov spaces are their real interpolation.
\item Besov spaces \(B^{s,p}(X)\) are reflexive Banach spaces with their dual spaces being
\(B^{-s,p'}(X)\) where \(\frac{1}{p} + \frac{1}{p'}=1\).
\end{enumerate}

\begin{theorem}
\label{thm:besov-sobolev}
If \(r < s\) then
\[
 W^{s,p}(X) \subset B^{s,p}(X) \subset W^{r,p}(X).
\]
\end{theorem}

\begin{theorem}[Multiplication]
\label{thm:estimate-product}
For \(f,g\in \mathcal{S}(X)\) and \(\begin{cases}
0<\alpha <1, \tilde p \leq p,\tilde q \leq q,\tilde r\leq r \\
\frac{1}{p} + \frac{1}{q} = \frac{1}{r}, \frac{1}{\tilde p} + \frac{1}{q} = \frac{1}{p} +
\frac{1}{\tilde q} = \frac{1}{\tilde r}				      
				       \end{cases}\),  one has
\begin{align}
\|fg\|_{B^{\alpha,\tilde r}} &\leq C \left( \|f\|_{B^{\alpha,\tilde p}}\|g\|_{L^q} + \|f\|_{L^p}\|g\|_{B^{\alpha,\tilde q}} \right) \label{eq:estimate-product-1} \\
\|fg\|_{L^r} &\leq \|f\|_{L^p}\|g\|_{L^q} \label{eq:estimate-product-2}
\end{align}
Therefore by density \eqref{eq:estimate-product-1} is true for all \(f\in L^p\cap
B^{\alpha,\tilde p}, g\in L^q\cap B^{\alpha,\tilde q}\) and \eqref{eq:estimate-product-2}
is true for all \(f\in L^p, g\in L^q\).
\end{theorem}

The reason for which we use the Besov norm is the following estimate:
\begin{theorem}[Composition]
\label{thm:compo-besov}
Let \(\Gamma(x,y)\) be a continuous, nonlinear function of variables \(x\in
\mathbb{R}^n, y\in \mathbb{R}^N\). Suppose that \(\Gamma\) vanishes for all \(x\)
outside of a compact in \(\mathbb{R}^n\) and \(\Gamma\) is \(C\)-Lipschitz in \(y\), and define
\[
\Gamma f:= \left( x\longmapsto \Gamma(x,f(x)) \right).
\]
Then 
\[ 
\|\Gamma f\| \leq C\left( 1+ \|f\|_{B^{\alpha,p}}\right)
\]
\end{theorem}


\section{Proof of the local estimate.}
\label{sec:org63db1f4}
Since \(B^{r+\epsilon,p}(X)\subset W^{r,p}(X)\), by increasing \(r\) a bit, we can
suppose that \(r\not\in \mathbb{Z}\) and replace the \(W^{r,p}\) norm in the statement
by the \(B^{r,p}\) norm, that is to estimate:
\[
 \|PF\|_{B^{r,p}} = \sum_{\|\gamma\| < r} \| D^\gamma(PF)\|_{L^p} + \sum_{
r-\sigma/\sigma_j< \|\gamma\| < r} \frac{\| \Delta^v_j D^\gamma (PF)\|_{L^p}}{|v|^{(r-\|\gamma\|)\sigma_j/\sigma}}
\]
where 
\begin{equation}
\label{eq:term-small}
 D^\gamma (PF) = \sum c_{\beta_1,\dots,\beta_\mu}(x,F) D^{\beta_1} f^{b_1}\dots D^{\beta_\mu}f^{b_\mu}
\end{equation}
with \(\max \|\beta_i\| \leq k +\|\gamma\|\) and \(\sum \|\beta_i\|\leq n + \|\gamma\|\).

Using \(\Delta^v_j (fg) = \Delta^v_j f\ T^v_j g + f \Delta^v_j g\), one can see that \(\Delta^v_j D^\gamma (PF)\) is a sum of terms of 2 types:
\begin{equation}
\label{eq:term-c}
\Delta^v_j c_{\beta_1,\dots,\beta_\mu}\ T^v_j(D^{\beta_1}f^{b_1})\dots T^v_j (D^{\beta_\mu}
f^{b_\mu})
\end{equation}
and
\begin{equation}
\label{eq:term-f}
 c_{\beta_1,\dots,\beta_\mu}\ D^{\beta_1}f^{b_1}\ \dots \ D^{\beta_{i-1}}f^{b_{i-1}}\ \Delta^v_j
(D^{\beta_i} f^{b_i})\ T^v_j(D^{\beta_{i+1}} f^{b_{i+1}})\ \dots\ T^v_j(D^{\beta_\mu} f^{b_\mu})
\end{equation}

Our strategy is to use Theorem \ref{thm:estimate-product} to estimate the terms
\eqref{eq:term-small}, \eqref{eq:term-c} and \eqref{eq:term-f} as follows, where we denote \(\|g\|_p:= \|g\|_{L^p}\)
\begin{equation}
\label{eq:est-term-small}
\left\| c_{\beta_1,\dots,\beta_\mu}(x,F)\ D^{\beta_1} f^{b_1}\dots D^{\beta_\mu}f^{b_\mu} \right\|_{p} \leq \|c_{\beta_1,\dots,\beta_\mu} \|_{\infty}.\|D^{\beta_1} f^{b_1}\|_{p_1}\dots \|D^{\beta_\mu} f^{b_\mu}\|_{p_\mu}
\end{equation}

\begin{equation}
\label{eq:est-term-c}
\left\| \Delta^v_j c_{\beta_1,\dots,\beta_\mu}\ T^v_j(D^{\beta_1}f^{b_1})\dots T^v_j (D^{\beta_\mu}
f^{b_\mu})  \right\|_p \leq \|\Delta^v_j c_{\beta_1,\dots,\beta_\mu}\|_{\tilde p_0}. \| D^{\beta_1} f^{b_1} \|_{p_1}\dots \|D^{\beta_\mu} f^{b_\mu}\|_{p_\mu}
\end{equation}

\begin{multline}
\label{eq:est-term-f}
\left\| c_{\beta_1,\dots,\beta_\mu}\ D^{\beta_1}f^{b_1}\ \dots \ D^{\beta_{i-1}}f^{b_{i-1}}\ \Delta^v_j
(D^{\beta_i} f^{b_i})\ T^v_j(D^{\beta_{i+1}} f^{b_{i+1}})\ \dots\ T^v_j(D^{\beta_\mu} f^{b_\mu}) \right\|_p \leq  \\ \|c_{\beta_1,\dots,\beta_\mu}\|_\infty. \| D^{\beta_1}f^{b_1} \|_{p_1}\dots \| D^{\beta_{i-1}}f^{b_{i-1}}\|_{p_{i-1}}. \|\Delta^v_j
(D^{\beta_i} f^{b_i}) \|_{\tilde p_i}. \| D^{\beta_{i+1}} f^{b_{i+1}} \|_{p_{i+1}}\dots \|D^{\beta_\mu} f^{b_\mu} \|_{p_\mu}
\end{multline}

Then continue by bounding the \(\Delta^v_j\) terms:
\begin{equation}
\label{eq:est-del-c}
\|\Delta^v_j c_{\beta_1,\dots,\beta_\mu}\|_{\tilde p_0} \leq |v|^{\theta \sigma_j/\sigma} C (1+ \|F\|_{B^{\theta,\tilde p_0}}) \leq |v|^{\theta \sigma_j/\sigma} C (1+ \|F\|_{W^{\theta,\tilde p_0}})
\end{equation}
using Theorem \ref{thm:compo-besov}, where \(C\) is the Lipschitz constant of \(c_{\beta_1,\dots,\beta_\mu}(x,F)\) in \(F\), which exists because \(c_{\beta_1,\dots,\beta_\mu}\) is smooth and \(F\) always
remains in a large Euclidean ball \(B\). The next \(\Delta^v_j\) term to bound is,
using Theorem \ref{thm:besov-sobolev}:
\begin{equation}
\label{eq:est-del-f}
\|\Delta^v_j (D^{\beta_i} f^{b_i}) \|_{\tilde p_i}\leq |v|^{\theta \sigma_j/\sigma} \|f^{b_i}\|_{B^{\|\beta_i\| +\theta, \tilde p_i}} \leq |v|^{\theta \sigma_j/\sigma} \|f^{b_i}\|_{W^{\|\beta_i\| +\theta, \tilde p_i}}
\end{equation}

And finally plugging \eqref{eq:est-del-c} and \eqref{eq:est-del-f} in \eqref{eq:est-term-c} and
\eqref{eq:est-term-f}, and noting that \(\|c_{\beta_1,\dots,\beta_\mu} \|_{\infty}\) in
\eqref{eq:est-term-small} is bounded by a constant, it remains to estimate \(\| f^{b_i}
\|_{W^{\|\beta_i\|, p_i}}\), \(\| f^{b_i}
\|_{W^{\|\beta_i\| + \theta, \tilde p_i}}\) and \(\|F\|_{W^{\theta, \tilde p_0}}\) in
term of \(\|F\|_{W^{s,q}}\), for which we will use the following consequence of Interpolation
inequality.

\begin{lemma}
\label{lem:loc-est-reg}
Let \(0\leq r\leq s\) and \(p,q\in (1, \infty)\) such that \(0 <
\frac{1}{p} - \frac{r}{s}\frac{1}{q} < 1-\frac{r}{s}\). Then for all compactly supported \(F\in C(X, B)\cap W^{s,q}\) where \(B\subset \mathbb{R}^N\) is a large Euclidean ball, one has
\[
 \|F\|_{W^{r,p}} \leq C \|F\|^{1-r/s}_\infty \|F\|^{r/s}_{W^{s,q}}\leq C' \|F\|^{r/s}_{W^{s,q}}
\]
where \(C,C'\) depend only on \(B\) and the support of \(F\), but not \(F\).
\end{lemma}

\begin{proof}
Since \(F\) is bounded, \(f^\alpha\in W^{s,q} \cap W^{0,v}\) for all \(v > 1\). By
Interpolation inequality
\[
 \|f^\alpha\|_{W^{r,p}} \leq 2 \|f^\alpha\|^{r/s}_{W^{s,q}} \|f^\alpha\|^{1-r/s}_{W^{0,v}}
\]
then choose \(v\) with \((1 - \frac{r}{s})\frac{1}{v} = \frac{1}{p} -
\frac{r}{s}\frac{1}{q}\).
\end{proof}

To apply Lemma \ref{lem:loc-est-reg}, we have to choose \(p_i,\tilde p_i, \tilde p_0,\theta\) such that 
\begin{cases}
0< \frac{1}{p_i} - \frac{\|\beta_i\|}{s} \frac{1}{q} < 1 - \frac{\|\beta_i\|}{s},   \\
0< \frac{1}{\tilde p_i} - \frac{\|\beta_i +\theta\|}{s}\frac{1}{q} < 1 - \frac{\|\beta_i +\theta\|}{s} \\
0< \frac{1}{\tilde p_0} - \frac{\theta}{s}\frac{1}{q} < 1- \frac{\theta}{s}
\end{cases}
We choose \(\frac{1}{p_i}\) just a bit bigger than \(\frac{\|\beta_i\|}{s}\frac{1}{q}\),
\(\frac{1}{\tilde p_i}\) just a bit bigger than \(\frac{\|\beta_i
+\theta\|}{s}\frac{1}{q}\) and \(\frac{1}{\tilde p_0}\) just a bit bigger than
\(\frac{\theta}{s}\frac{1}{q}\). We will now come back to justify the estimates
\eqref{eq:est-term-small}, \eqref{eq:est-term-c}, \eqref{eq:est-term-f}. Since \(F\) is
bounded in \(B\) and compactly supported in an open set \(V\), we see that \(\|f^\alpha\|_p \leq
C(B,V) \|f^\alpha\|_q\) if \(p\leq q\). Therefore,
\begin{enumerate}
\item For \eqref{eq:est-term-small}, it is sufficient to have
\[
    \frac{1}{p} \geq \frac{1}{p_1}+\dots + \frac{1}{p_\mu} 
   \]
 which is true because the RHS is is a bit bigger than \(\frac{1}{qs}\sum \|\beta_i\|
   \leq \frac{n + \|\gamma\|}{qs} < \frac{n+r}{qs} < \frac{1}{p}\).
\item For \eqref{eq:est-term-c}, it is sufficient to have
\[
    \frac{1}{p} \geq \frac{1}{\tilde p_0} + \frac{1}{p_1}+\dots + \frac{1}{p_\mu} 
   \]
 where the RHS is is a bit bigger than \(\frac{\theta}{s}\frac{1}{q}+ \frac{1}{qs}\sum \|\beta_i\|
   \leq \frac{n + \|\gamma\| + \theta}{qs}\).
\item For \eqref{eq:est-term-f}, it is sufficient to have
\[
    \frac{1}{p} \geq  \frac{1}{p_1}+\dots + \frac{1}{\tilde p_i} + \dots + \frac{1}{p_\mu} 
   \]
where the RHS is is a bit bigger than \(\frac{\theta}{s}\frac{1}{q}+ \frac{1}{qs}\sum \|\beta_i\|
   \leq \frac{n + \|\gamma\| + \theta}{qs}\).
\end{enumerate}

It is sufficient then to take \(\theta = r- \|\gamma\|\). Now the estimates
\eqref{eq:est-term-small}, \eqref{eq:est-term-c}, \eqref{eq:est-term-f} can be continued as
\begin{align}
  RHS \eqref{eq:est-term-small} &\leq \prod_i \|f^{b_i}\|^{\|\beta_i\|/s}_{W^{s,q}} \leq \|F\|_{W^{s,q}}^{\frac{n+\|\gamma\|}{s}} \leq \|F\|_{W^{s,q}}^{q/p}\label{eq:fin-del-small}\\
  RHS \eqref{eq:est-term-c} &\leq |v|^{\theta\sigma_j/\sigma}\left( 1 + \|F\|_{W^{s,q}}^{\theta/s} \right)\prod_i \|f^{b_i}\|^{\|\beta_i\|/s}_{W^{s,q}}\leq |v|^{\theta\sigma_j/\sigma}\left( 1 + \|F\|_{W^{s,q}}^{\theta/s} \right)\|F\|_{W^{s,q}}^{q/p} \label{eq:fin-del-c}\\
  RHS \eqref{eq:est-term-f} &\leq |v|^{\theta\sigma_j/\sigma}\left( 1 + \|f^{b_i}\|_{W^{s,q}}^{\frac{\|\beta_i\| +\theta}{s}} \right)\prod_{u\ne i} \|f^{b_u}\|^{\|\beta_u\|/s}_{W^{s,q}}\leq |v|^{\theta\sigma_j/\sigma}\left( 1 + \|F\|_{W^{s,q}}^{\frac{\|\beta_i\| +\theta}{s}} \right)\|F\|_{W^{s,q}}^{q/p} \label{eq:fin-del-f}
\end{align}
While \eqref{eq:fin-del-small} gives \(\|D^\gamma (PF)\|_p \leq C \|F\|_{W^{s,q}}^{q/p}\),
the last two \eqref{eq:fin-del-c} and \eqref{eq:fin-del-f} give
\[
 \sum_{ s- \frac{\sigma}{\sigma_j} < \|\gamma\| < s}\sup_{v} \frac{\|\Delta^v_j D^\gamma (PF) \|_p}{|v|^{(r-\|\gamma\|)\sigma_j/\sigma}} \leq C \left(
1 + \|F\|^{(n+r)/s}_{W^{q,s}}\right)
\]
We proved the local statement Lemma \ref{lem:loc-reg-poly-diff}.
