\iffalse
\begin{info}
The PDF version of this page can be downloaded by replacing \texttt{html} in the its address by
\texttt{pdf}. 
For example \texttt{/html/sheaf-cohomology.html} should become \texttt{/pdf/sheaf-cohomology.pdf}.
\end{info}
\fi

We will establish in this part a regularity estimate for the quadratic term of nonlinear
heat operator use it to setup a bootstrap scheme that eventually will prove that any
sufficiently regular solution of nonlinear heat equation that is initially \(C^\infty\)
will be always \(C^\infty\).

We will also prove short-time existence using well-known method of Inverse
function theorem for Banach spaces. Since the solution is smooth, we can apply Theorem
\ref{thm:global-eq} to conclude that the it remains in \(M'\subset \mathbb{R}^N\).

\section{Review of Sobolev spaces and Linear equations.}
\label{sec:org3af8194}

The following results are well-known and their statements are written here in the case of our interest (linear heat equation on
manifold). A more careful formulation with complete proofs can be found in the
appendices.

\subsection{Sobolev spaces.}
\label{sec:orge4cbcfa}

Let \(M\) be a Riemannian manifold, the \emph{Sobolev spaces} \(W^{k,p}(M)\) on \(M\) can be defined as the completion of \(C^\infty(M)\) with respect to the Sobolev norms
\[ 
\|\varphi\|_{W^{k,p}} = \sum_{|\alpha|\leq k} \| D^\alpha \varphi \|_{L^p}. 
\]
We will suppose that \(M\) is a compact manifold, then set-theoretically \(W^{k,p}\) does not depend
on the metric of \(M\) and their norm remains in the same equivalent class as the metric
varies. The Sobolev spaces form a family of reflexive Banach spaces that is stable under holomorphic
interpolation:

\begin{theorem}[Interpolation of Sobolev spaces]
\label{thm:interp-sobolev-d}
Let \(p,q\in (1,+\infty)\) and \(k,l\in \mathbb{R}\) and \(M\) be a compact
Riemannian manifold. Then the holomorphic interpolations of 
\[
 A_0:= W^{k,p}(M)\quad \text{and}\quad A_1:=W^{l,q}(M)
\]
are \(A_\theta=W^{s,r}(M)\) where
\[
 \theta l + (1-\theta)k = s,\qquad \theta \frac{1}{q} + (1-\theta) \frac{1}{p} = \frac{1}{r}.
\]
In particular, one has the Interpolation inequality
\[
 \|f\|_{W^{s,r}}\leq 2 \|f \|^\theta_{W^{l,q}} \| f\|^{1-\theta}_{W^{k,p}}.
\]
\end{theorem}

Sobolev embeddings and Kondrachov theorem remain correct on manifold. 

\begin{theorem}[Sobolev embedding]
\label{thm:Sobolev-Rn-d}
Given \(k,l\in \mathbb{Z},\ k>l \geq 0\) and \(p,q\in \mathbb{R},\ p>q\geq 1\). Then
\begin{enumerate}
\item If \(\frac{1}{p}= \frac{1}{q} - \frac{k-l}{n}\) then 
\[
    W^{k,q}(M) \hookrightarrow W^{l,p}(M),
   \]
\item If \(\frac{k-r}{n}> \frac{1}{q}\) then
\[
   W^{k,q}(M) \hookrightarrow C^r(M)
   \]
If \(\frac{k-r-\alpha}{n}\leq \frac{1}{q}\) then 
\[ 
   W^{k,q}(M) \hookrightarrow C^{r,\alpha}(M) 
   \]
\end{enumerate}
where \(C^r(M)\) denotes the space of \(C^r\) functions equipped with the norm \(\| u \|_{C^r} = \max_{l\leq
r}\sup|\nabla^l u|\), and \(C^{r,\alpha}\) is the subspace of \(C^r\) of functions
whose \(r^{\rm th}\)-derivative is \(\alpha\)-Holder, equipped with the norm \(\|
u\|_{C^{r,\alpha}} = \| u \|_{C^r} + \sup_{P\ne Q}\{ \frac{u(P) - u(Q)}{d(P,Q)^\alpha} \}\).
\end{theorem}

\begin{theorem}[Kondrachov]
\label{thm:Kondrachov-Rn-d}
Let \(k\in \mathbb{Z}_{\geq 0}\) and \(p,q\in \mathbb{R}_{>0}\) be such that
\(1\geq \frac{1}{p} > \frac{1}{q} - \frac{k}{n} > 0\) then
\begin{enumerate}
\item The embedding \(W^{k,q}(M) \hookrightarrow  L^p(M)\) is compact,
\item The embedding \(W^{k,q}(M) \hookrightarrow  C^\alpha(M)\) is compact if
\(k-\alpha > \frac{n}{q}\) where \(0\leq \alpha < 1\),
\end{enumerate}
\end{theorem}

It is also natural, for regularity results of parabolic equation, to use weighted Sobolev
spaces because each derivative in time should be counted as twice as that in space. For
example, the space \(W^{2,p}(M\times[\alpha,\omega])\) is the completion of \(C^\infty(M)\) with respect to
the norm
\[
 \|\varphi\|_{W^{2,p}} := \|\varphi\|_{L^p} + \left\| \frac{d\varphi}{dt}\right\|_{L^p} + \sum_{i,j} \left\| \frac{\partial^2\varphi}{\partial x^i \partial
x^j} \right\|_{L^p} + \sum_i \left\| \frac{\partial\varphi}{\partial x^i} \right\|_{L^p}
\]
Similarly, one can define \(W^{2k,p}(M\times[\alpha,\omega])\) using \(L^p\)-norm of
derivatives \(\partial_t^\beta \partial_x^\gamma \varphi\) of \(\varphi\) with \(2\beta + \gamma
\leq 2k\).


We also want to be able to talk about \(W^{k,p}\) when \(k\) is not an integer and not necessarily positive. This
allows us to have a more flexible bootstrap scheme for nonlinear heat equation and to use
Interpolation Theorem \ref{thm:interp-sobolev-d} more efficiently. We claim that these
generalised Sobolev spaces (with weight and with non-integral regularity) can be defined on
manifold and satisfy all the above properties (reflexivity, Interpolation theorem, Sobolev
embedding and Kondrachov theorem) and refer to the appendices for all the details.

\subsection{Trace theorem.}
\label{sec:orgbc9f39d}

It is possible to avoid a discussion on Trace operator if we only want to make sense of the
initial condition of nonlinear heat equation: one can consider only solutions with
regularity greater than \(W^{2,p}(M\times[\alpha,\omega])\) with \(p \geq \dim M + 2\), which can be embedded in \(C(M)\). It is however necessary to investigate regularity of Trace operator to have a complete proof of the bootstrap. We will review
briefly some results.

The following two behaviors of trace are well-known:
\begin{enumerate}
\item If \(-1 + \frac{1}{p} < k < \frac{1}{p}\) then the natural map \(W^{k,p}(M\times[\alpha,\omega]/\alpha) \hookrightarrow W^{k,p}(M\times[\alpha,\omega])\) is an isomorphism, where \(W^{k,p}(M\times[\alpha,\omega]/\alpha)\) denotes the
completion under \(W^{k,p}\)-norm of the space of smooth functions vanishing on a
neighborhood of \(M\times \{\alpha\}\). There is therefore no meaningful notion of trace in this case.
\item If \(k > \frac{1}{p} + l,\ l\geq 0\), then the restriction map
\[
    B:\ C^\infty(M\times[\alpha,\omega]) \longrightarrow C^\infty(M):\ f(x,t) \longmapsto f(x,\alpha)
   \]
extends to a bounded operator \(B:\ W^{k,p}(M\times[\alpha,\omega]) \longrightarrow
   W^{l,p}(M)\), called \emph{Trace operator}.
\end{enumerate}

We will topologise the space \(\partial_\alpha W^{k,p}(M\times[\alpha,\omega])\) of
restrictions to time \(t=\alpha\) of functions in \(W^{k,p}(M\times[\alpha,\omega])\), in case Trace operator is well defined, as cokernel of
\(B\), that is, as a quotient
space of \(W^{k,p}(M\times[\alpha,\omega])\). This makes \(\partial_\alpha
W^{k,p}(M\times[\alpha,\omega])\) a Banach space with stronger norm than any \(W^{l,p}(M)\) for any \(l < k-\frac{1}{p}\).

\subsection{Linear equation on manifolds.}
\label{sec:orgb9f137c}

\subsubsection{Existence and Regularity.}
\label{sec:orgff9fdf8}

It can be easily verified that the linear heat operator \(AF:= \frac{d}{dt}F + \Delta F\)
is a parabolic operator and therefore is also an elliptic operator. All of the following results
holds for operator \(A\).

\begin{theorem}[Regularity for elliptic operator]
\label{thm:elliptic-d}
Let \(M\) be a compact manifold and \(AF:= \frac{d}{dt}F + \Delta F\) be an elliptic
operator of second order. Given \(\frac{1}{p} < l < k <\infty\) and \(F\in
W^{l,p}(M\times[\alpha,\omega])\) and suppose that 
\[
 AF\in W^{k-2,p}(M\times[\alpha,\omega]),\quad \restr{f}{\alpha}\in \partial_\alpha
W^{k,p}(M\times[\alpha,\omega]), \quad  \restr{f}{\omega}\in \partial_\alpha W^{k,p}(M\times[\alpha,\omega]).
\]
Then actually \(F\in W^{k,p}(M\times[\alpha,\omega])\).
\end{theorem}


\begin{theorem}[Causality of parabolic equation]
\label{thm:para-existence-d}
Let \(M\) be a compact manifold and \(AF:= \frac{d}{dt}F + \Delta F\) be an parabolic
operator. Then 
\[
 A:\ W^{k,p}(M\times[\alpha,\omega]/\alpha) \longrightarrow W^{k-2,p}(M\times[\alpha,\omega]/\alpha)
\]
is an isomorphism of Banach spaces.
\end{theorem}


\begin{theorem}[Gårding's Inequality and Regularity for parabolic operator]
\label{thm:para-eq-d }
Let \(M\) be a compact manifold, \(p\in (1,+\infty)\), \(k > l > -\infty\) and \(AF:= \frac{d}{dt}F + \Delta F\) be a parabolic operator. We write \(W^{k,p}([\beta,\gamma])\) shortly for \(W^{k,p}(M\times[\beta,\gamma])\). Suppose that
\[
 F\in W^{l,p}([\alpha,\omega]),\quad AF\in W^{k-2,p}([\alpha,\omega]).
\]
Then \(F\in W^{k,p}([\pi,\omega])\) for all \(\pi\in (\alpha,\omega)\). Also, there
exists a constant \(C>0\) such that 
\[
 \|F\|_{W^{k,p}([\pi,\omega])} \leq C \left( \|AF\|_{W^{k-2,p}([\alpha,\omega])} + \| F \|_{W^{l,p}([\alpha,\pi])}  \right).
\]
In particular for homogeneous equation, the solution is \(C^\infty\) and an arbitrarily weak estimate in the past gives an arbitrarily strong estimate in the future. 
\end{theorem}

\subsubsection{Maximum principle and Comparison theorems.}
\label{sec:orgf25e20e}
Other than regularity results which are generally true for parabolic operators, the linear
heat operator also enjoys the following versions of Maximum principle. See Appendices for
their proofs.

\begin{theorem}[Maximum principle]
\label{thm:max-princ-d}
Let \(M\) be a compact manifold and \(f: M\times[\alpha,\omega] \longrightarrow \mathbb{R}\) be a continuous function
with \(\restr{f}{\alpha}\leq 0\). Suppose that whenever \(f>0\), \(f\) is smooth and
\[
 \frac{\partial f}{\partial t} \leq -\Delta f + Cf.
\]
Then in fact \(f\leq 0\).
\end{theorem}

With the same proof as Theorem \ref{thm:max-princ-d}, one also has:

\begin{theorem}[\( L^\infty \)-Comparison theorem]
\label{thm:infty-comparison-d}
Let \(f: M\times[\alpha,\omega] \longrightarrow \mathbb{R}\) be a continuous function on
\(M\), smooth for all time \(t>0\) such that
\[
  \frac{d f}{dt} = -\Delta f + b f \text{ on } M\times (\alpha,\omega]
\]
where \(b\) is a smooth function on \(M\). Then there exists a constant \(B\) depending
only on \(b\) such that 
\[
 \|\restr{f}{\omega}\|_{L^\infty} \leq e^{B(\omega-\alpha)}
\|\restr{f}{\alpha}\|_{L^\infty}.
\]
\end{theorem}

Using backwards heat equation and Theorem \ref{thm:infty-comparison-d}, one can prove its
version for \(L^1\).

\begin{theorem}[\( L^1 \)-Comparison theorem]
\label{thm:1-comparison-d}
Let \(f: M\times[\alpha,\omega] \longrightarrow \mathbb{R}\) be a continuous function on
\(M\), smooth for all time \(t>0\) such that
\[
  \frac{d f}{dt} = -\Delta f + b f \text{ on } M\times (\alpha,\omega]
\]
where \(b\) is a smooth function on \(M\). Then there exists a constant \(B\) depending
only on \(b\) such that 
\[
 \|\restr{f}{\omega}\|_{L^1} \leq e^{B(\omega-\alpha)}
\|\restr{f}{\alpha}\|_{L^1}.
\]
\end{theorem}


\section{Regularity estimate of the quadratic term.}
\label{sec:org7baff5d}

\begin{theorem}[Regularity of the quadratic term]
\label{thm:reg-quad}
Let \(F:\ M\times [\alpha,\omega] \longrightarrow B\subset \mathbb{R}^N\) be in \(W^{s,q}(M\times[\alpha,\omega])\cap C(M\times [\alpha,\omega])\) and
\[ 
P F := g^{ij}\Gamma'^\alpha_{\beta\gamma}(F) F^\beta_i
F^\gamma_j.
\]
Suppose that
\begin{equation}
\label{eq:cond:thm:reg-poly-diff}
 r\geq 0,\quad p,q\in (1,\infty),\quad r+1 < s, \quad \frac{1}{p}> \frac{r+2}{s} \frac{1}{q}.
\end{equation}
Then one has \(PF\in W^{r,p}(X)\) and
\[
 \|P F \|_{W^{r,p}} \leq C\left(1 + \|F\|_{W^{s,q}}\right)^{q/p}.
\]
where \(C\) is a constant independent of \(F\).
\end{theorem}

\begin{proof}
We will suppose here that \(r,s\) are even integers so that the \(W^{r,p}\)
(respectively \(W^{s,q}\)) norm of \(PF\) (respectively \(F\)) can be written as sum of \(L^p\)
(respectively \(L^q\)) norms of its derivatives. Also, we will use chain rule freely to
differentiate the term \(\Gamma'^\alpha_{\beta\gamma}(F)\) using weak derivatives of \(F\). The general and rigorous proof, which involves fractional Sobolev space to treat non-integral
\(r, s\) and a detour to Besov spaces to justify chain rule, can be found in the appendices.

The derivatives of \(PF\) that appear in its \(W^{r,p}\) norm are of form
\[
 C(x,F)\ \prod_i \partial_t^{b_i} \partial_x^{c_i} F^{\beta_i} 
\]
where \(2\sum b_i + \sum c_i \leq r+2\) and \(\max \{2b_i+ c_i\}\leq r+1\) and \(C(x,F)\) is bounded on \(M\). Using Multiplication theorem for \(L^p\)-spaces, one has
\[
  \left\|C(x,F)\ \prod_i \partial_t^{b_i} \partial_x^{c_i} F^{\beta_i} \right\|_{L^p} \leq
\|C(x,F)\|_{L^\infty} \prod_i \left\|\partial_t^{b_i} \partial_x^{c_i}
F^{\beta_i}\right\|_{L^{p_i}}  \leq \|C(x,F)\|_{L^\infty} \prod_i \left\|F\right\|_{W^{2b_i+c_i,p_i}}
\]
as long as we choose \(p_i\in (1,\infty)\) such that \(\frac{1}{p} \geq \sum
\frac{1}{p_i}\). The
strategy is to choose \(\frac{1}{p_i}\) big enough to have \(W^{s,q}
\hookrightarrow W^{2b_i+c_i,p_i}\) in order to
bound \(\left\|F\right\|_{W^{2b_i+c_i,p_i}}\) by \(\left\|F\right\|_{W^{s,q}}\), then
use the upper bound of \(2b_i +c_i\) to justify that \(\frac{1}{p} > \frac{r+2}{s}
\frac{1}{q} \geq \sum \frac{1}{p_i}\), meaning that such choice of \(p_i\) are valid.

The straightforward way to have a sufficient condition of \(p_i\) such that \(W^{s,q}
\hookrightarrow W^{2b_i+c_i,p_i}\) is to use Sobolev embedding, but the result is sub-optimal
because Sobolev embedding does not take into account the \(L^\infty\)-bounded of \(F\) (its image lies in a compact of \(\mathbb{R}^N\)). A better way is to use
Interpolation inequality, by remarking that \(F\in W^{0,v}\) for all \(v\in (1,+\infty)\) and writing \(W^{2b_i+c_i, p_i}\) as an interpolation space of \(W^{s,q}\) and \(W^{0,v}\). It can be seen, by direct computation, that the sufficient condition for \(W^{s,q}
\hookrightarrow W^{2b_i+c_i,p_i}\) is \(2b_i+c_i < s\) and
\[
 0 < \frac{1}{p_i} - \frac{2b_i+c_i}{s}\frac{1}{q} < 1 - \frac{2b_i + c_i}{s}.
\]
Choose \(\frac{1}{p_i}\) just a bit bigger than \(\frac{2b_i+c_i}{s}\frac{1}{q}\), one
still has
\[
\sum \frac{1}{p_i} \simeq \sum  \frac{2b_i+c_i}{s}\frac{1}{q} \leq \frac{r+2}{s}\frac{1}{q} < \frac{1}{p}.
\]
The conclusion follows.  
\end{proof}



\section{Regularity for nonlinear heat equation.}
\label{sec:org703224a}
Let \(p>\dim M + 2\), using the regularity estimate for the quadratic term, we now can prove:

\begin{theorem}[Bootstrap for nonlinear heat equation]
\label{thm:reg-nonlin-heat}
Let \(F:\ M\times [\alpha,\omega] \longrightarrow B\) such that \(F\in W^{2,p}(M\times
[\alpha,\omega])\) and \(\frac{d F_t}{dt} = \tau(F_t)\), i.e.
\[
 \frac{d F^\alpha}{dt} = -\Delta F^\alpha + g^{ij}\Gamma'^\alpha_{\beta\gamma}(F)
F^\beta_i F^\gamma_j
\]
and \(\restr{F}{M\times\{\alpha\}}\) is smooth. Then \(F\) is smooth on \(M\times [\alpha,\omega]\).
\end{theorem}

\begin{remark}
Note that since \(p > \dim M + 2 = \dim (M\times [\alpha,\omega])+1\), if \(F\in
W^{2,p}(M\times[\alpha,\omega])\) then \(F\) and \(\frac{\partial F}{\partial
x^i}\) are in \(C(M\times[\alpha,\omega])\) by \href{sobolev-riemannian.org}{Sobolev embeddings}. It makes sense then to talk
about:
\begin{enumerate}
\item the restriction and boundary condition at time \(t=\alpha\) (in fact, by \href{interpolation-sobolev.org}{Trace theorem}, \(p>1\) is enough).
\item the pointwise condition \(F: M\times [\alpha,\omega] \longrightarrow
   B\subset V\).
\end{enumerate}
\end{remark}

\begin{proof}
We define the operators \(P F := g^{ij}\Gamma'^\alpha_{\beta\gamma}(F) F^\beta_i
F^\gamma_j\) and \(A F := \frac{dF }{dt} + \Delta F\). We will abusively denote \(W^{k,p}(M\times
[\beta,\gamma])\) by \(W^{k,p}([\beta,\gamma])\). Our bootstrap scheme consists of 3
steps:
\begin{enumerate}
\item Prove that \(F\in W^{2,\tilde p}([\pi,\omega])\) for every \(\pi > \alpha\) and \(\tilde p \in (1,\infty)\). By
compactness of \(M\), it is sufficient to prove this
for a sequence \(\tilde p \to +\infty\).
\item Prove that \(F\) is \(C^\infty\) for all time \(t >\alpha\).
\item Prove that \(F\) is \(C^\infty\) on \(M\times [\alpha,\omega]\).
\end{enumerate}

\emph{Step 1.} By Theorem \ref{thm:reg-quad}, \(AF = PF \in W^{r,q}([\alpha,\omega])\) whenever \(r<1\) and \(\frac{1}{q} > (\frac{r}{2}+1)\frac{1}{p}\). Apply Gårding inequality, for all
\(\pi >\alpha\), \(F\in W^{r+2,q}([\pi,\omega]) \subset W^{2,\tilde p}([\pi,\omega])\)
for \(\frac{1}{\tilde p} = \frac{1}{q} - \frac{r}{\dim M + 1}\). Choose \(\frac{1}{q}\) very close to \((\frac{r}{2}+1)\frac{1}{p}\), one sees that the condition on \(\tilde p\) is \(\frac{1}{\tilde p } > (\frac{r}{2} +1 ) \frac{1}{p} - \frac{r}{p-1}\),
which will be satisfied if \(\frac{1}{\tilde p} > (1-\frac{r}{2})
\frac{1}{p}\), i.e. for all \(\tilde p < \frac{p}{1 -r/2}\). It remains to repeat this
result to finish the first
step. We will say \(F\in W^{2,*}([\pi,\omega])\) for \(F\in W^{2,p}([\pi,\omega])\) for all \(p\in (1,\infty)\).

\emph{Step 2.} By Theorem \ref{thm:reg-quad}, for all \(r<1\), one has \(AF = PF \in
W^{r,*}([\pi,\omega])\), therefore by Gårding inequality, \(F\in W^{r+2, * }([\pi,\omega])\). Iterate this result and one has \(F\in
W^{k,*}([\pi,\omega])\) for all \(k\in [2,\infty)\) and \(\pi >\alpha\). So \(F\)
is smooth for \(t>\alpha\).

\emph{Step 3.} We apply regularity result (Theorem \ref{thm:elliptic-d}) for elliptic operator \(A\) and boundary
operators \(B^0: F \mapsto \restr{F}{M\times\{\alpha\}}\) and \(B^1: F \mapsto
\restr{F}{M\times\{\omega\}}\): For \(q,r\) in Step 1,
one has \(AF = PF \in W^{r,q}([\alpha,\omega])\) and \(B^j F \in \partial W^{r,q}\),
therefore \(F\in W^{r+2,q}([\alpha,\omega])\subset W^{2,\tilde p}([\alpha,\omega])\)
for the same \(\tilde p\) as Step 1. This proves that \(F\in W^{2,*}([\alpha,\omega])\), which also means that one has \(F\in W^{r+2,q}([\alpha,\omega])\) with no additional
condition on \(q\) except \(q\in (1,\infty)\). Iterate and one obtains the regularity
of \(F\) on \([\alpha,\omega]\).
\end{proof}

\begin{remark}
The first 2 steps were to prove the regularity of \(\restr{F}{M\times
\{\omega\}}\), which was then used as a boundary condition in
order to apply regularity result for elliptic operator on manifold with boundary.
\end{remark}




\section{Short-time existence for nonlinear heat equation.}
\label{sec:org6dd8c2b}
We will choose as always \(p > \dim M +2\). As before, \(M\) is a compact Riemannian
manifold and \(B\subset \mathbb{R}^N\) is a large Euclidean ball.

\begin{theorem}[Short-time existence]
\label{thm:short-time}
Let \(F_\alpha: M \longrightarrow  B\) be a smooth map, then there exists \(\epsilon>0\) depending on \(F_\alpha\) and \(F: M\times [\alpha,\alpha + \epsilon] \longrightarrow
B\) such that \(F\in W^{2,p}(M\times [\alpha,\alpha+\epsilon])\) with \(\restr{F}{M\times \{\alpha\}} = F_\alpha\) and
\[
 \frac{d F_t}{dt} = \tau(F_t) \quad \text{on } M\times [\alpha,\alpha +\epsilon]
\]
\end{theorem}

\begin{proof}
We find \(F\) as a sum \(F = F_b + F_\#\) where \(F_b\in C^\infty(M\times
[\alpha,\omega])\) satisfies the initial condition and \(F_\# \in W^{2,p}(M\times
[\alpha,\alpha+\epsilon]/\alpha)\).

The nonlinear heat operator can be written as:
\begin{align*}
  T: W^{2,p}(M\times[\alpha,\omega]/\alpha)^{\oplus N} &\longrightarrow L^p(M\times[\alpha,\omega])^{\oplus N}\\
  	F_\#					       &\longmapsto \tau(F_b + F_\#)
\end{align*} 
where \(\tau(F)^\alpha = - \Delta F^\alpha + g^{ij}\Gamma'^\alpha_{\beta\gamma}(F)
F^\beta_i F^\gamma_j\), which can be rewritten as \(\tau(F) = -\Delta F +
\Gamma(F)(\nabla F)^2\). The derivative of \(T\) at \(F_\#\) in direction \(k\in
W^{2,p}(M\times[\alpha,\omega]/\alpha)^{\oplus N}\) is
\[
 DT(F_\#) k = -\Delta k +D \Gamma(F)\cdot k . (\nabla F)^2 + 2\Gamma(F) \nabla F. \nabla k,
\]
or in local coordinates:
\[
  DT(F_\#)^\alpha = g^{ij}\left( \frac{\partial^2 k^\alpha}{\partial x^i \partial x^j} -
\Gamma^l_{ij} k^\alpha_l \right) + g^{ij} \frac{\partial \Gamma'^\alpha_{\beta\gamma}}{\partial y^\delta} k^\delta F^\beta_i F^\gamma_j + 2g^{ij} \Gamma'^\alpha_{\beta\gamma}(F) F^\beta_i F^\gamma_j 
\]
which is of form \(DT(F_\#) k = -\Delta k - a(x,F) \nabla k -b(x,F) k\) where \(a,b\)
are smooth.

Therefore if we note  
\begin{align*}
  H: W^{2,p}(M\times[\alpha,\omega]/\alpha)^{\oplus N} &\longrightarrow L^p(M\times[\alpha,\omega])^{\oplus N}\\
  	F_\#					       &\longmapsto (\frac{d }{dt} - \tau)(F_b + F_\#)
\end{align*}
then the derivative of \(H\) at \(F_\# = 0\) is
\[
 DH(0)\cdot k = \frac{d k}{d t} + \Delta k + a(x, F_b) \nabla k + b(x, F_b) k
\]
which by Theorem \ref{thm:para-existence-d} is an isomorphism from \(W^{2,p}(M\times[\alpha,\omega]/\alpha)^{\oplus N}\) to
\(W^{0,p}(M\times[\alpha,\omega]/\alpha)^{\oplus N} =  L^p(M\times[\alpha,\omega])^{\oplus
N}\). This shows that \(H\) is a local isomorphism mapping a neighborhood of \(0\) to
a neighborhood of \((\frac{d }{dt}-\tau)F_b\).

Define \(g_\epsilon\in L^p(M\times [\alpha,\omega])^{\oplus N}\) by 
\[
 g_\epsilon:= \begin{cases}
0	      ,  & \text{if $t\in[\alpha,\alpha+\epsilon]$} \\
(\frac{d }{dt}-\tau)F_b	      , & \text{if $t > \alpha+\epsilon$}
	      \end{cases}
\]
which is arbitrarily \(L^p(M\times[\alpha,\omega])\)-close to \((\frac{d }{dt}-\tau)F_b\) for \(0<\epsilon \ll 1\). There exists therefore \(F_\#\in W^{2,p}(M\times
[\alpha,\omega]/\alpha)^{\oplus N}\) such that \(H(F_\#) = g_\epsilon\), meaning that
the function \(F= F_b + F_\#: M \longrightarrow V\) satisfies \(\restr{F}{M\times\{\alpha\}} = F_\alpha\)
and \(\frac{d F}{d t} -\tau(F_t) = 0\) for \(t\in[\alpha,\alpha+\epsilon]\).

By Regularity Theorem \ref{thm:reg-nonlin-heat}, \(F\) is \(C^\infty\) for \(t\in[\alpha,\alpha+\epsilon]\). Theorem \ref{thm:global-eq} assures that the image of \(F\) is in \(M\), hence in \(M'\) for \(t\in [\alpha,\alpha+\epsilon]\).
\end{proof}
