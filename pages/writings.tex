% Created 2018-11-19 Mon 10:40
% Intended LaTeX compiler: pdflatex
\documentclass[11pt]{article}
\usepackage[utf8]{inputenc}
\usepackage[T1]{fontenc}
\usepackage{graphicx}
\usepackage{grffile}
\usepackage{longtable}
\usepackage{wrapfig}
\usepackage{rotating}
\usepackage[normalem]{ulem}
\usepackage{amsmath}
\usepackage{textcomp}
\usepackage{amssymb}
\usepackage{capt-of}
\usepackage{hyperref}
\usepackage{amsthm}
\usepackage{amsmath,amscd,amssymb,mathtools}
\usepackage{tikz-cd}
\usepackage{svg}
\usepackage[all]{xy}
\usepackage{pgfplots}
\newtheorem{remark}{Remark}
\newtheorem{theorem}{Theorem}
\newtheorem{lemma}[theorem]{Lemma}
\newtheorem{corollary}{Corollary}[theorem]
\newtheorem{conjecture}[theorem]{Conjecture}
\newtheorem{proposition}{Proposition}[theorem]
\newtheorem{problem}{Problem}
\newtheorem{exampl}{Example}
\newtheorem{definition}{Definition}
\newtheorem{propdef}[definition]{Proposition-Definition}
\newtheorem{fact}{Fact}
\newtheorem{assertion}{Assertion}
\newcommand{\re}{\mathop{\rm Re}\nolimits}
\newcommand{\im}{\mathop{\rm Im}\nolimits}
\newcommand{\coker}{\mathop{\rm coker}\nolimits}
\newcommand{\supp}{\mathop{\rm supp}\nolimits}
\newcommand{\ord}{\mathop{\rm ord}\nolimits}
\newcommand{\Spec}{\mathop{\rm Spec}\nolimits}
\newcommand{\vol}{\mathop{\rm vol}\nolimits}
\newcommand*{\transp}[2][-3mu]{\ensuremath{\mskip1mu\prescript{\smash{\mathrm t\mkern#1}}{}{\mathstrut#2}}}
\newcommand{\sff}{\mathop{\rm I\*I}\nolimits}
\newcommand{\tr}{\mathop{\rm Tr}\nolimits}
\newcommand{\const}{\mathop{\rm const }\nolimits}
\newcommand{\lcm}{\mathop{\rm lcm}\nolimits}
\newcommand{\gcd}{\mathop{\rm gcd}\nolimits}
\newcommand{\Ric}{\mathop{\rm Ric}\nolimits}
\newcommand{\Riem}{\mathop{\rm Riem}\nolimits}
\newcommand{\Enorm}{\mathop{\mathcal{E}_{\rm norm}}\nolimits}
\newcommand{\Anorm}{\mathop{\mathcal{A}_{\rm norm}}\nolimits}
\newcommand\restr[2]{{% we make the whole thing an ordinary symbol
\left.\kern-\nulldelimiterspace % automatically resize the bar with \right
#1 % the function
\vphantom{\big|} % pretend it's a little taller at normal size
\right|_{#2} % this is the delimiter
}}
\author{Tien}
\date{\today}
\title{}
\hypersetup{
 pdfauthor={Tien},
 pdftitle={},
 pdfkeywords={},
 pdfsubject={},
 pdfcreator={Emacs 26.1 (Org mode 9.0.5)}, 
 pdflang={English}}
\begin{document}

\tableofcontents


\section{First variation of renormalised energy. \textit{<2018-10-26 Fri>}}
\label{sec:org3eec4ba}
\subsection{The results.}
\label{sec:orgfcf699d}

I will start by writing down the result before giving the proofs. Let \((\Sigma,i)\)
be a Riemann surface and \((M,g)\) be an asymptotically hyperbolic Riemannian
four-manifold. We are interested in critical points of renormalised energy \(\Enorm\)
because these maps form a special class among harmonic maps whose energy is critical to all
pertubation (and not just compact pertubation).

Even though we obtained an explicit formula for renormalised energy where a density appears
\[
  \Enorm = -4\pi\chi + \int_\Sigma \left[ 2K(f^*g)\sqrt{{\rm det}_h f^*g} + \tr_h f^*g \right] dA_h,
  \]
it is not easy to compute its variation directly from it. The following
approach gives us a formula that although is not explicit, allows us to see the
following phenomenon.
\begin{theorem}[Critical points of \(\Enorm\) = Harmonic + good germ on the boundary]
\label{thm:crit-Enorm-bndry-germ}
Let \(f: (\Sigma,i) \longrightarrow (M,g)\) be a harmonic map from a Riemann surface \(\Sigma\) to
a AH Riemannian four-manifold \(M\) such that \(q(f)_{\bar g} = 0\) on \(\partial\Sigma\). There is a boundary quantity \(\mathcal{B}_2(f)\) that only concerns the germ of \(f\) on \(\partial\Sigma\) such that:
\begin{center}
\(f\) is critical to \(\Enorm\) if and only if \(\mathcal{B}_2(f) = 0\).
\end{center}
\end{theorem}

We know that the \(\epsilon\)-energy must be writen as
\[
   E_\epsilon (f) = \frac{1}{\epsilon} {\rm length}(\gamma) + \Enorm + O(\epsilon)
  \]
where \(\gamma\) is the boundary curve of \(f\) and the length is taken under the
metric \(\bar g\). Now given a pertubation \(\{f_t\}\) of \(f_0\), we can recover
the variation of \(\Enorm\) by looking at the \(O(1)\) term of \(\frac{dE_\epsilon}{dt}(f_t)\). It turns out that in order to compute the variation of
\(E_\epsilon\), we only need to do the following 2 tasks:
\begin{itemize}
\item Extend the computation made by Eells-Sampson \cite{eells_harmonic_1964} to manifold with boundary.
\item Take care of the variation of the domain of integration \(\Sigma_\epsilon(f_t):=
    f_t^{-1}(r\geq\epsilon)\) which also depends on \(t\) where \(r\) is the boundary
defining function (bdf).
\end{itemize}

\begin{proposition}[Variation of \( E_\epsilon \)]
\label{prop:var-Eeps}
For any pertubation \(\{f_t\}\) of \(f: (\Sigma,\partial\Sigma,i) \longrightarrow (M,\partial M,g)\), one has
\begin{equation}
\label{eq:var-Eeps}
\frac{d}{dt}E_\epsilon(f_t) = \int_{\Sigma_\epsilon(f_0)} \left\langle \tau(f_0), \frac{df_t}{dt} \right\rangle_g dV_h + \frac{1}{\epsilon^2}\int_{\partial\Sigma_\epsilon(f_0)}\left \langle 2 \frac{\partial f_0}{\partial n} + \frac{\nabla^g r(f_0)}{|d(r\circ f_0|_h}  , \frac{df_t}{dt}\right \rangle_{\bar g} ds_h  
\end{equation}
where 
\begin{itemize}
\item \(h\) is a metric in the conformal class \(i\) of \(\Sigma\),
\item \(\frac{\partial}{\partial n}\) is the \(h\)-unit normal vector of \(\partial\Sigma_\epsilon(f_0)\) in \(\Sigma_\epsilon(f_0)\),
\item \(\tau(f_0)\) is the tension field of \(f_0\) with respect to the metric \(h\)
on \(\Sigma\) and \(g\) on \(M\).
\end{itemize}
Denote \(\mathcal{B} := \int_{\partial\Sigma_\epsilon(f_0)}\left
  \langle 2 \frac{\partial f_0}{\partial n} + \frac{\nabla^g r(f_0)}{|d(r\circ f_0|_h}  ,
  \frac{df_t}{dt}\right \rangle_{\bar g} ds_h =: \mathcal{B}_0 + \mathcal{B}_1 \epsilon +
  \mathcal{B}_2 \epsilon^2 + O(\epsilon^3)\) then 
\begin{itemize}
\item \(\mathcal{B}_i\) only depends on the germ of \(f_0\) on \(\partial\Sigma\)
\item \(\mathcal{B}_0 = 2\int_{\partial\Sigma}\left \langle \frac{\partial f_0}{\partial n} +
     \frac{\nabla^{\bar g}r(f_0)}{|\nabla^{\bar g}r(f_0)|_{\bar g}}, \frac{df_t}{dt}\right
     \rangle_{\bar g} ds_h = 0\) where \(n\) is the \(f^* \bar g\)-normal vector of \(\Sigma\) on
the boundary.
\end{itemize}
\end{proposition}

As a straightforward consequence, one has
\begin{proposition}[]
\label{prop:}
Given a map \(f: (\Sigma, \partial\Sigma, i) \longrightarrow (M, \partial M, g)\), then:
\begin{enumerate}
\item \(f_0\) is harmonic and \(\mathcal{B}_2(f_0) = 0\) if and only if \(f_0\) is a
critical point of \(\Enorm\).
\item The first variation of \(\Enorm\) can be writen as:
\begin{align*}
\frac{d \Enorm}{dt}(f_t) &= \int_{\Sigma} \left\langle \tau(f_0),
\frac{df_t}{dt} \right\rangle_g dV_h  \\ &+
\lim_{\epsilon\to 0} \frac{1}{\epsilon^2} \left
[\int_{\partial\Sigma_\epsilon(f_0)}\left \langle 2
\frac{\partial f_0}{\partial n} + \frac{\nabla^g r(f_0)}{|d(r\circ f_0|_h}  ,
\frac{df_t}{dt}\right \rangle_{\bar g} ds_h +  \int_{\Sigma_\epsilon(f_0)} \left
\langle \nabla^{\bar g}_{\dot\gamma} \frac{\dot\gamma}{|\dot\gamma|_h} ,
\frac{df_t}{dt}\right \rangle_{\bar g}\right]      
\end{align*}

where \(\gamma\) is the boundary curve of \(f\), \(\frac{\partial}{\partial n}\) is the \(h\)-unit normal vector of \(\partial\Sigma_\epsilon\) in \(\Sigma_\epsilon\).
\end{enumerate}
\end{proposition}


\subsection{Proofs.}
\label{sec:org4f7be82}
We \ref{Eells-Sampson-bndry} knew that the variation of energy functional can be represented by the tension field, in
case of manifolds with boundary, we obtain an additional term from Stokes formula: 
\begin{align}
   \frac{d}{dt}E_\epsilon(f_t) &= \frac{d}{dt}\int_{\Sigma_\epsilon(f_0)} \tr_h (f_t^* g) +   \frac{d}{dt}\int_{\Sigma_\epsilon(f_t)} \tr_h (f_0^*g)\\
      			       &= \int_{\Sigma_\epsilon(f_0)} \left \langle \tau(f_0), \frac{df_t}{dt} \right \rangle_g dV_h + 2\int_{\partial\Sigma_\epsilon(f_0)} \left \langle \frac{\partial f_0}{\partial n}, \frac{df_t}{dt}\right \rangle_g + \frac{d}{dt} \int_{\Sigma_\epsilon(f_t)} \tr_h( f_0^* g) dV_h
\end{align}
Equation \eqref{eq:var-Eeps} is a straightforward application of the
following lemma for \(\Omega_t = \Sigma_\epsilon(f_t)\), \(F=\tr_h(f_0^* g)\) and \(r_t = r\circ f_t\).

\begin{lemma}[Riemannian Cavalieri]
\label{lem:var-domain}
Let \(\Omega\) be a domain in \(\Sigma\) and  \(\{r_t\}_{t=\overline{0,1}}\) be a familly of functions on \(\Omega\) where \(dr_t\) are non-zero and \(\Omega_t\subset \Omega\) be subdomains of \(\Omega\) defined by \(\Omega_t =
   \{r_t \geq \epsilon\}\). Then for any function \(F\) on \(\Omega\), one has
\[
   \restr{\frac{d}{dt}}{t=0}\int_{\Omega_t}F dV_h = \int_{\partial\Omega_0}
   \frac{r_1}{|\nabla^h r_0|_h}F ds_h
   \]
where \(r_1 = \restr{\frac{dr_t}{dt}}{t=0}\).
\end{lemma}


\begin{proof}
Let us prove the lemma in case \(\Omega_t\) only has one connected component with
non-empty interior, since
the number of such components does not change for \(t\) near 0 (this is because
\(r_0\) has no critical point in \(\Omega\)). Let
us also suppose that \(r_1 \geq 0\) meaning that the domain \(\Omega_t\)
becomes bigger as \(t\) increases from \(0\). This is because one can always
partition the curve \(\gamma = \partial\Omega_0\) into pieces where \(r_1>0, r_1 <0\) or \(r_1=0\)
and the area difference of pieces touching the \(r_1=0\) parts is of \(O(t^2)\). 

The difference \(\Omega_t\setminus \Omega_0\) is the region where \(\epsilon -r_1 t+
   O(t^2)\leq r_0 \leq \epsilon\), therefore \(\Omega_t\setminus \Omega_0\) is of \(O(t^2)\) difference from the region \(\Phi_t = \{ \exp_{\gamma(s)}\frac{\theta r_1\nabla^h
   r_0}{|\nabla^h r_0|^2}:\theta \in[0,t], s\in[0,1] \}\). Therefore
\[
   \restr{\frac{d}{dt}}{t=0}\int_{\Omega_t}F dV_h
   =\restr{\frac{d}{dt}}{t=0} \int_0^t \int_{\gamma} F \frac{r_1}{|\nabla^h r_0|_h} \text{vol }ds_h d\theta = \int_{\gamma} F\frac{r_1}{|\nabla^h r_0|_h} ds_h .
   \]
where \(\frac{r_1}{|\nabla^h r_0|_h}\text{vol}(s,\theta) ds_h d\theta\) is the
pullback of the volume form by exponential map
\((s,\theta)\mapsto \exp_{\gamma(s)} \frac{\theta r_1\nabla^h r_0}{|\nabla^h r_0|_h^2}\), so \(\text{vol}(s,0) = 1\).
\end{proof}




\section{Log term in energy of \(f: \overline{\mathbb{H}^2} \longrightarrow \overline{\mathbb{H}^{n+1}}\).}
\label{sec:org6b8c471}




\section{Commutative diagram revisited}
\label{sec:orgc61064e}

\section{\LaTeX{} in Inkscape: Incompatibility between \texttt{ghostscript} and \texttt{pstoedit} \textit{<2018-03-30 Fri>}}
\label{sec:org9453df4}

\section{\LaTeX{} indentation in org-mode \textit{<2018-02-20 Tue>}}
\label{sec:org25f6eb5}

\section{A (decent) map of mathematics \textit{<2017-10-17 Tue>}}
\label{sec:org4446147}

\section{My 2016-2017 internship \textit{<2017-07-31 Mon>}}
\label{sec:orgd335b8a}
\end{document}