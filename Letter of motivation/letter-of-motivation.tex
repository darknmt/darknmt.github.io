% Created 2018-01-20 Sat 18:34
% Intended LaTeX compiler: pdflatex
\documentclass[11pt]{article}
\usepackage[utf8]{inputenc}
\usepackage[T1]{fontenc}
\usepackage{graphicx}
\usepackage{grffile}
\usepackage{longtable}
\usepackage{wrapfig}
\usepackage{rotating}
\usepackage[normalem]{ulem}
\usepackage{amsmath}
\usepackage{textcomp}
\usepackage{amssymb}
\usepackage{capt-of}
\usepackage{hyperref}
\usepackage{amsthm}
\usepackage{amsmath,amscd,amssymb}
\usepackage{tikz-cd}
\newtheorem{remark}{Remark}
\newtheorem{theorem}{Theorem}
\newtheorem{lemma}[theorem]{Lemma}
\newtheorem{corollary}{Corollary}[theorem]
\newtheorem{conjecture}[theorem]{Conjecture}
\newtheorem{proposition}{Proposition}[theorem]
\newtheorem{problem}{Problem}
\newtheorem{exampl}{Example}
\newtheorem{definition}{Definition}
\newtheorem{propdef}[definition]{Proposition-Definition}
\newcommand{\im}{\mathop{\rm Im}\nolimits}
\newcommand{\supp}{\mathop{\rm supp}\nolimits}
\newcommand{\ord}{\mathop{\rm ord}\nolimits}
\newcommand{\Spec}{\mathop{\rm Spec}\nolimits}
\newcommand{\vol}{\mathop{\rm vol}\nolimits}
\newcommand\restr[2]{{% we make the whole thing an ordinary symbol
\left.\kern-\nulldelimiterspace % automatically resize the bar with \right
#1 % the function
\vphantom{\big|} % pretend it's a little taller at normal size
\right|_{#2} % this is the delimiter
}}
\author{Manh Tien NGUYEN}
\date{\today}
\title{Letter of motivation}
\hypersetup{
 pdfauthor={darknmt},
 pdftitle={Letter of motivation},
 pdfkeywords={},
 pdfsubject={},
 pdfcreator={Emacs 25.3.1 (Org mode 9.0.5)}, 
 pdflang={English}}
\begin{document}

\maketitle

\iffalse
\begin{info}
The PDF version of this page can be downloaded by replacing \texttt{html} in the its address by
\texttt{pdf}. 
For example \texttt{/html/sheaf-cohomology.html} should become \texttt{/pdf/sheaf-cohomology.pdf}.
\end{info}
\fi

The goal of this short letter is to give you a picture of me and my research
interests. Since it is not usual to be attached to a particular mathematical question at my level and
since I tried during last two years to broaden my mathematical scope as much as possible,
I would like to describe my taste and my mathematical identity, in the supposition
that they are getting in form and will not depend much on the problem I approach in my
thesis.

\section{Scientific background}
\label{sec:orgade8ce5}
I started my undergraduate study in a small department in the south of Vietnam which
provides a curriculum narrowly focused in Analysis and PDE, with almost no Algebra or
Geometry covered. I managed however to understand very basic ideas and notions of
mathematics. I later moved to Ecole polytechnique for an engineer program where I enjoyed
a very multidisciplinary program. I took courses primarily on Mathematics, Applied
mathematics, Physics, and Mechanics. I did not take courses in Computer Science as I
prefer to learn them through team projects. Being part of student projects at Ecole
polytechnique was a good experience, hardly did I see elsewhere that Banach-Alaoglu
theorem and weak* convergence of Dirac distribution can be employed to prove a result in
graph theory \footnote{An idea due to J.-P. Serre to prove a theorem of Alon and Boppana.} and
later use that result to create an error-correcting code. \iffalse\footnote{I was supposed to be in the
mathematical side, but then turned to help my friends write the majority of the code.}\fi
Mathematics courses however were good enough to convince me to pursue it as a
career. During my specialized year (Master 1) in Mathematics, I also kept an eye on
Physics courses (e.g. General Relativity, Classical Field Theory) to be familiar with some
basic notions.


\section{Mathematical interests}
\label{sec:orgdd7ef3f}

I got introduced to Differential geometry and Riemannian geometry by excellent lectures in Mathematics and Physics departments, then to Complex geometry during my internship in Marseille and I read a bit
later in that summer about Symplectic geometry in a lecture note of Denis Auroux (\cite{auroux09:lect}). I also
love the joy of problem solving as well as the insight gained after proving by oneself a
mathematical result. From a practical view, I may learn a theorem faster and remember a
technical detail for longer if I put them in use or explain them in simpler, sometimes
unrigorous terms. I often write notes where I try to explain why a statement is true before
proving it and they appear to be useful in many cases. It is also a reason that I would
prefer to attack a concrete problem in my thesis.

After the M1 internship and more advanced courses at Université Paris-Sud, I am however
also convinced that structured and systematized approaches are also important in
mathematics and can help significantly simplify a problem.

Recently, through conversations with S.Boucksom and H.-C. Lu, I heard about and read a
few surveys (\cite{szekelyhidi17:_kahler_einst},\cite{angella17:_kahler_einst}) on canonical metrics and I also find myself interested in the subject. I would be happy
to work on a problem around it.

\section{Brief description of my M1 project.}
\label{sec:orgd060841}
The project, being part of my third year curriculum at Ecole polytechnique, was my first
scientific internship and my first try of serious mathematical research. It was
coordinated by S.Boucksom and supervised by J.Keller at Institut de Mathématiques de
Marseille and the goal was for me to understand the proof of Bogomolov-Beauville
decomposition of complex manifolds with vanishing Chern class and to pick up mathematical
background on the way. There was hence no expect of original results coming from my side
and that was also the case in reality.

The decomposition is a simple, although not formally trivial, translation of a decomposition theorem in Riemannian
geometry to the category of complex manifolds, using Calabi-Yau theorem. In retrospect,
the results that I found to be the most interesting were less or more geometric analysis in nature, including the
decomposition by holonomy (de Rham's theorem) and the phenomena discovered by J.Cheeger and D.Gromoll on
manifolds with non-negative Ricci curvature. It was later that I started to appreciate the
proof of S.-T. Yau for Calabi conjecture. 

There is as I said no original results coming from the project, but since I want to
convince you that beside understanding books and articles or reorganizing lemmas and
theorems into a memoir, I also did mathematics actively, I will refer to an easy
linear algebraic lemma in which I showed that the existence part of de Rham decomposition
theorem implies two things: (1) its uniqueness and (2) the decomposition preserves the Kahler property. More precisely, the lemma shows, using the existence part, that holonomy
representations have a nice property: their irreducible factors are not only unique as
representations (as in Schur lemma) but also as vector subspaces. I do not think the
result is completely obvious, as my reference textbook in Riemannian geometry (\cite{sakai_riemannian_1996}) provides separate proofs for the two phenomena. I also pointed out in the
memoir an inaccuracy in a recent generalized version of Bogomolov-Beauville decomposition (\cite{campana_rationally_2012})
and correct it.

The project won the Research Internship Awards by Mathematics department, along with two others, one in PDE and one in Number Theory. 

\bibliographystyle{alpha}
\bibliography{Stage2017}
\end{document}