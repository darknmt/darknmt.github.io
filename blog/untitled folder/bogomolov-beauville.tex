% Created 2017-06-11 Sun 10:19
% Intended LaTeX compiler: pdflatex
\documentclass[11pt]{article}
\usepackage[utf8]{inputenc}
\usepackage[T1]{fontenc}
\usepackage{graphicx}
\usepackage{grffile}
\usepackage{longtable}
\usepackage{wrapfig}
\usepackage{rotating}
\usepackage[normalem]{ulem}
\usepackage{amsmath}
\usepackage{textcomp}
\usepackage{amssymb}
\usepackage{capt-of}
\usepackage{hyperref}
\usepackage{amsthm}
\usepackage{tikz-cd}
\newtheorem{remark}{Remark}
\newtheorem{theorem}{Theorem}
\newtheorem{lemma}[theorem]{Lemma}
\newtheorem{corollary}{Corollary}[theorem]
\newtheorem{conjecture}[theorem]{Conjecture}
\newtheorem{proposition}{Proposition}[theorem]
\newtheorem{problem}{Problem}
\newtheorem{example}{Example}
\newtheorem{definition}{Definition}
\setcounter{secnumdepth}{3}
\author{darknmt}
\date{\today}
\title{Bogomolov-Beauville classification}
\hypersetup{
 pdfauthor={darknmt},
 pdftitle={Bogomolov-Beauville classification},
 pdfkeywords={},
 pdfsubject={},
 pdfcreator={Emacs 25.2.1 (Org mode 9.0.5)}, 
 pdflang={English}}
\begin{document}

\maketitle
\tableofcontents


\section{From the Riemannian results of de Rham and Berger}
\label{sec:org2ff430d}

We will first prove a (conceptually) straightforward result of \href{de-rham-decomposition.org}{de Rham decomposition} and
\href{Berger-remark-complex}{Berger classification}. The following theorem is taken from Beauville's article
\begin{theorem}[]
\label{thm:beauville-1}
\label{org1c1ad60}
Let \(X\) be a compact Kahler manifold with flat Ricci curvature, then
\begin{enumerate}
\item The universal covering space \(\tilde X\) of \(X\) decomposes isometrically as \[\tilde X =
   E \times\prod_i V_i\times \prod_j X_j\] where \(E = \mathbb{C}^k\), \(V_i\) and \(X_j\) are simply-connected
compact manifolds of dimension \(2m_i\) and \(4r_j\) with irreducible homonomy \(SU(m_i)\) for \(V_i\) and \(Sp(r_j)\) for \(X_j\). One
also has uniqueness in the strong sense as in de Rham decomposition.
\item There exists a finite etale covering space \(X'\) of \(X\) such that \[ X' = T\times \prod_i V_i
   \times \prod_j X_j\] where \(T\) is a complex torus.
\end{enumerate}
\end{theorem}
\begin{proof}
Note that the first point is obtained directly from de Rham decomposition: The one-dimensional
parallel subspaces (of trivial holonomy) are regrouped to \(E\). By \href{Cheeger-Gromoll-splitting.org}{Cheeger-Gromoll splitting},
\(\tilde X = E\times M\) where \(M\) contains no line and is compact (note that we use compactness of
\(X\) here). The irreducible factors in \(M\) are not symmetric spaces as Ricci curvature of
symmetric spaces is non-degenerate. Holonomy of these factors are \(SU(m_i)\) and \(Sp(r_j)\)
according to Berger list since they are Kahler manifolds and Ricci-flat. It remains to prove the
second point.

We will regard each element of \(\pi_1(X)\) by its isometric, free, proper action on \(\tilde X\). As
pointed out the arguments in our discussion of uniqueness of de Rham decomposition, every isometry
of \(\tilde X\) to itself preserves the components \(T_{x_0}E\), \(T_{x_i}V_i\) and \(T_{x_j}X_j\)
of \(T_x \tilde X\), each isometry \(\phi\) of \(\tilde X\) is of form \((\phi_1,\phi_2)\) where \(\phi_1\in
Isom(E)\) and \(\phi_2\in Isom(M)\).

We will use here the fact that if \(M\) is a Kahler manifold, compact and Ricci-flat then \(Isom(M)\)
equiped with compact-open topology is discrete, therefore finite, which will be proved later (Lemma
\ref{org111c804}). We note \(\Gamma := \{\phi = (\phi_1,\phi_2)\in \pi_1(X),\ \phi_2 = Id_M\}\) and
sometime abusively regard \(\Gamma\) as a subgroup of \(Isom(E)\). Note that \(\Gamma\) is a normal
subgroup of \(\pi_1(X)\) with finite index since the quotient is isomorphic to \(Isom(M)\). Therefore
\(\tilde X/\Gamma = E/\Gamma\times M\) is compact as a finite covering of \(X\).

We apply the following theorem of Bieberbach.
\begin{theorem}[Bieberbach]
Let \(E = \mathbb{R}^n\) an Euclidean space and \(\Gamma\) be a subgroup of \(Isom(E)\) satisfies
\begin{enumerate}
\item \(\Gamma\) is discrete under compact-open topology.
\item \(E/\Gamma\) is compact.
\end{enumerate}
Then the subgroup \(\Gamma'\) of translations in \(\Gamma\) is of finite index.
\end{theorem}

Suppose that the two conditions are satisfied and the theorem gives \(\tilde X/\Gamma' =
E/\Gamma'\times M = T\times \prod_i V_i\times \prod_j X_j\) is a finite covering of
\(\tilde X/\Gamma\) as \(\Gamma'\) is a normal subgroup of \(\Gamma\) since

\textbf{Fact.} The subgroup of translations in \(Isom(E)\), where \(E = \mathbb{R^n}\) is an Euclidean space, is
normal.

Therefore \(X' = \tilde X/\Gamma'\) is a finite covering of \(X\) that we want to find.

It remains now to prove that \(\Gamma\) is discrete, which is a consequence of
\begin{enumerate}
\item \(\pi_1(X)\) is discrete, without limit point (obvious).
\item \(Isom(M)\) is finite (see lemma \ref{org111c804})
\end{enumerate}
In fact given any \(\phi = (\phi_1,\phi_2) \in Isom(E)\times Isom(M)\), there exists by (1.) a neighborhood \(\mathcal{U}_1(\phi_1,\phi_2)\times \mathcal{U}_2(\phi_1,\phi_2)\) of \(\phi\) in \(Isom(E)\times
Isom(M)\) such that all points of \(\pi_1(X)\) lying in this region project to \(\phi_1\). By (2.)
we can find a neighborhood \(\mathcal{U}_1\) of \(\phi_1\) in \(Isom(E)\) small enough that \(\mathcal{U}_1(\phi_1)\times Isom(M) \subset \cup_{\phi_2\in Isom(M)}
\mathcal{U}_1(\phi_1,\phi_2)\times \mathcal{U}_2(\phi_1,\phi_2)\). Therefore the projection of
\(\pi_1(X)\) to \(Isom(E)\) is discrete, by consequence \(\Gamma\) is discrete.
\end{proof}

\begin{lemma}[]
\label{lem:Isom-discrete}
\label{org111c804}
Let \(M\) be is a compact, simply-connected, Ricci-flat, Kahler manifold, then the group \(Aut(M)\) of
automorphism of \(M\) equiped with compact-open topology is discrete, therefore \(Isom(M)\) is discrete,
hence finite. 
\end{lemma}
\begin{proof}
The idea is that since \(Aut(M)\) is a Lie group, it suffices to prove that its Lie algebra is of
dimension 0. This is done using these facts. 
\begin{enumerate}
\item The Lie algebra of \(Aut(M)\) can be identified with the vector space of holomorphic vector
fields on \(M\).
\item All holomorphic tensor fields in \(M\) are parallel (Bochner principle). This can be seen by the identity
\(\Delta(\|\tau \|^2) = \|D\tau \|^2\)
\item The only invariant vector of the holonomy representation of \(M\) is \(0\) (obvious).
\end{enumerate}
\end{proof}


\section{To a classification for complex manifold}
\label{sec:org3ea9614}
To obtain a translation of Theorem \ref{org1c1ad60} in a context of complex manifolds (without any
prefered metric a priori), we study the 2 building blocks. To be clear, recall that a complex
manifold \(X\) is called \uline{of Kahler type} if one can equipe \(X\) with an Hermitian structure whose
fundamental form \(\omega\) satisfies \(d\omega= 0\). When we say \(X\) is of Kahler type, we refer to \(X\)
as a complex manifold without fixing a metric on \(X\). We resume here some results, see the
manuscript for their proofs.
\subsection{Special unitary manifold (proper Calabi-Yau manifold)}
\label{sec:org3909a88}
\begin{remark}
Let \(X\) be a compact Kahler manifold with holonomy \(SU(m)\) and complex dimension \(m\) then:
\begin{enumerate}
\item \(X\) is projective.
\item \(H^0(X, \Omega_X^p)=0\) for all \(0 < p < m\), by consequence \$\(\chi\)( \mathcal{O}\(_{\text{X}}\) ) = 1 + (-1)\(^{\text{m}}\) \$.
\end{enumerate}
\end{remark}

\begin{theorem}[]
Given a compact manifold \(X\) of Kahler type and complex dimension \(m\), the followings are equivalent
\begin{enumerate}
\item There exists a compatible metric \(g\) over \(X\) such that \(Hol(X,g) = SU(m)\).
\item \(K_X\) is trivial and \(H^0(X', \Omega_{X'}^p) =0\) for every \(0<p<m\) and \(X'\) a finite covering of \(X\).
\end{enumerate}
\end{theorem}

\subsection{Symplectic manifold}
\label{sec:orgc02533d}
\begin{remark}
Let \(X\) be a compact Kahler manifold with holonomy \(Sp(r)\) and complex dimension \(2r\) then:
\begin{enumerate}
\item There exists a holomorphic 2-form \(\varphi\) non-degenerate at every point.
\item \(H^0(X,\Omega_X^{2l+1}) = 0, H^0(X,\Omega_X^{2l})=\mathbb{C}\varphi^l\) for all \(0\leq l\leq
   r\). By consequence \(\chi(\mathcal{O}_X)=r+1\).
\end{enumerate}
\end{remark}

\begin{theorem}[]
Given a compact manifold \(X\) of Kahler type and complex dimension \(2r\), then:
\begin{itemize}
\item The followings are equivalent:
\begin{enumerate}
\item There exists a compatible metric \(g\) such that \(Hol(X,g) \subset Sp(r)\).
\item There exists a symplectic structure: a 2-form that is closed, holomorphic and non-degenerate
at every point
\end{enumerate}
\item The followings are equivalent, if \(X\) is called symplectic of it satisfies one of them.
\begin{enumerate}
\item There exists a compatible metric \(g\) such that \(Hol(X,g) = Sp(r)\)
\item \(X\) is simply-connected and there exists (uniquely up to a constant) a symplectic structure on \(X\).
\end{enumerate}
\end{itemize}
\end{theorem}

\subsection{Decomposition for complex manifold with vanishing Chern class}
\label{sec:org2d435d3}

Theorem \ref{org1c1ad60} can be translated to a decomposition for complex manifold in the following
way:
\begin{theorem}[Bogomolov- Beauville classification]
Let \(X\) be a compact manifold of Kahler type with vanishing Chern class.
\begin{enumerate}
\item The universal covering \(\tilde X\) of \(X\) is isomorphic to a product \(\mathbb{E}\times \prod_i
   V_i\times\prod_j X_j\) where \(E = \mathbb{C}^k\) and
\begin{enumerate}
\item Each \(V_i\) is a projective simply-connected manifold of complex dimension \(m_i\geq 3\), with trivial
\(K_{V_i}\) and \(H^0(V_i,\Omega_{V_i}^p) = 0\) for \(0 < p < m_i\)
\item Each \(X_j\) is an irreducible compact symplectic manifold of Kahler type.
\end{enumerate}
This decomposition is unique up to an order of \(i\) and \(j\).
\item There exists a finite covering \(X'\) of \(X\) isomorphic to the product \(T\times\prod_i
   V_i\times\prod_j X_j\).
\end{enumerate}
\end{theorem}
The theorem follows directly from Theorem \ref{org1c1ad60}, the only point that needs proof is the
uniqueness, which will be achieved in two steps:
\begin{enumerate}
\item Prove the uniqueness in the case that \(X\) is simply-connected.
\item Prove that every isomorphism \(\phi:\ \mathbb{C}^k\times Y\longrightarrow \mathbb{C}^h\times Z\) is
splitted as \(\phi = (\phi_1,\phi_2)\) where \(\phi_1:\ \mathbb{C}^k\longrightarrow \mathbb{C}^h\) and
\(\phi_2:\ Y\longrightarrow Z\) are isomorphisms (by consequence \(h=k\)).
\end{enumerate}
These two steps will be accomplished in the following two lemmas

\begin{lemma}[]
Let \(Y = \prod_j Y_j\) be a compact, simply-connected manifold of Kahler type with vanishing Chern
class. The Calabi-Yau metrics of \(Y\) are then \(g = \sum_l pr_j^*g_l\) where \(g_l\) are Calabi-Yau metrics
of \(Y_l\).
\end{lemma}
\begin{proof}
Let \(g\) be a Calabi-Yau metric of \(Y\) and \([\omega]\) its class in \(H^{1,1}(Y)\). Since \(Y_j\) are
simply-connected, \([\omega] = \sum_j pr_j^* [\omega_j]\). By \href{calabi-yau.org}{Yau's theorem}, there exist unique Calabi-Yau
metrics \(g_j\) of \(Y_j\) in each class \(\omega_j\). The metric \(g' = \sum_j pr_j^* g_j\) is in the same
class \(\omega\) of \(g\) and is also a Calabi-Yau metric, hence \(g= g' = \sum_j pr_j^*g_j\).
\end{proof}

This lemma affirms that when our manifolds \(Y, Y_j\) are equiped with appropriate Calabi-Yau metrics,
the decomposition map is also a (Riemannian) isometric, we therefore obtain uniqueness of \(V_i, X_j\)
from the uniqueness of Theorem \ref{org1c1ad60}.

\begin{lemma}[]
Let \(Y,Z\) be compact, simply-connected manifold of Kahler type, then any isomorphism \(u:\
\mathbb{C}^k\times Y\longrightarrow \mathbb{C}^h\times Z\) is splitted as \(\phi = (\phi_1,\phi_2)\)
where \(\phi_1:\ \mathbb{C}^k\longrightarrow \mathbb{C}^h\) and \(\phi_2:\ Y\longrightarrow Z\) are
isomorphisms.
\end{lemma}
\begin{proof}
It is clear that the function \(u_1: \mathbb{C}^k\times Y \longrightarrow \mathbb{C}^h \times Z
\longrightarrow \mathbb{C}^h\) is constant in \(Y\), i.e. \(u_1(t,y) = u_1(t)\) as holomorphic
functions on \(Y\) are constant. Therefore  \(u(t,y) = (u_1(t), u_2(t,y))\), as \(u\) is isomorphic, one
has \(h\leq k\) then by the same argument for \(u^{-1}\), one has \(h=k\), \(u_1\) is an isomorphism and
\(u_t(\cdot) := u_2(t,\cdot)\) is an isomorphism from \(Y\) to \(Z\). \(u_0^{-1}\circ u_t\) is then a curve in
\(Aut(Y)\), which is discrete by Lemma \ref{org111c804}. Hence \(u_t= u_0\) independent de \(t\).
\end{proof}
\end{document}