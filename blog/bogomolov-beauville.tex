% Created 2017-08-01 Tue 02:03
% Intended LaTeX compiler: pdflatex
\documentclass[11pt]{article}
\usepackage[utf8]{inputenc}
\usepackage[T1]{fontenc}
\usepackage{graphicx}
\usepackage{grffile}
\usepackage{longtable}
\usepackage{wrapfig}
\usepackage{rotating}
\usepackage[normalem]{ulem}
\usepackage{amsmath}
\usepackage{textcomp}
\usepackage{amssymb}
\usepackage{capt-of}
\usepackage{hyperref}
\usepackage{amsthm}
\usepackage{tikz-cd}
\newtheorem{remark}{Remark}
\newtheorem{theorem}{Theorem}
\newtheorem{lemma}[theorem]{Lemma}
\newtheorem{corollary}{Corollary}[theorem]
\newtheorem{conjecture}[theorem]{Conjecture}
\newtheorem{proposition}{Proposition}[theorem]
\newtheorem{problem}{Problem}
\newtheorem{exampl}{Example}
\newtheorem{definition}{Definition}
\newtheorem{propdef}[definition]{Proposition-Definition}
\author{darknmt}
\date{\today}
\title{Bogomolov-Beauville classification}
\hypersetup{
 pdfauthor={darknmt},
 pdftitle={Bogomolov-Beauville classification},
 pdfkeywords={},
 pdfsubject={},
 pdfcreator={Emacs 25.2.1 (Org mode 9.0.5)}, 
 pdflang={English}}
\begin{document}

\maketitle
\tableofcontents


\section{From the Riemannian results of de Rham and Berger}
\label{sec:org8fd1336}

We will first prove a (conceptually) straightforward result of \href{de-rham-decomposition.org}{de Rham decomposition} and
\href{Berger-remark-complex}{Berger classification}. The following theorem is taken from Beauville's article
\begin{theorem}[Beauville]
\label{thm:beauville-1}
\label{orgdf9aa02}
Let \(X\) be a compact Kähler manifold with flat Ricci curvature, then
\begin{enumerate}
\item The universal covering space \(\tilde X\) of \(X\) decomposes isometrically and holomorphically as
\[\tilde X = E \times\prod_i V_i\times \prod_j X_j\] where \(E = \mathbb{C}^k\), \(V_i\) and \(X_j\)
are simply-connected compact manifolds of real dimension \(2m_i\) and \(4r_j\) with irreducible homonomy
\(SU(m_i)\) for \(V_i\) and \(Sp(r_j)\) for \(X_j\). One also has uniqueness in the strong sense as in de
Rham decomposition.
\item There exists a finite covering space \(X'\) of \(X\) such that \[ X' = T\times \prod_i V_i
   \times \prod_j X_j\] where \(T\) is a complex torus.
\end{enumerate}
\end{theorem}
\begin{proof}
Note that the first point is obtained directly from Cheeger-Gromoll splitting and de Rham
decomposition: The one-dimensional parallel subspaces (of trivial holonomy) are regrouped to \(E\). By
\href{Cheeger-Gromoll-splitting.org}{Cheeger-Gromoll splitting}, \(\tilde X = E\times M\) where \(M\) contains no line and is compact (note
that we use compactness of \(X\) here). The irreducible factors in \(M\) are not symmetric spaces as
Ricci curvature of symmetric spaces is non-degenerate. Holonomy of these factors are \(SU(m_i)\) and
\(Sp(r_j)\) according to Berger list since they are Kähler manifolds and Ricci-flat. It remains to
prove the second point.

We will regard each element of \(\pi_1(X)\) by its isometric, free, proper action on \(\tilde X\). As
pointed out the arguments in our discussion of uniqueness of de Rham decomposition, every isometry
of \(\tilde X\) to itself preserves the components \(T_{x_0}E\), \(T_{x_i}V_i\) and \(T_{x_j}X_j\)
of \(T_x \tilde X\), each isometry \(\phi\) of \(\tilde X\) is of form \((\phi_1,\phi_2)\) where \(\phi_1\in
Isom(E)\) and \(\phi_2\in Isom(M)\).

We will use here the fact that if \(M\) is a Kähler manifold, compact and Ricci-flat then \(Isom(M)\)
equipped with compact-open topology is discrete, therefore finite, which will be proved later (Lemma
\ref{org65d013f}). We note \(\Gamma := \{\phi = (\phi_1,\phi_2)\in \pi_1(X),\ \phi_2 = Id_M\}\) and
sometime abusively regard \(\Gamma\) as a subgroup of \(Isom(E)\). Note that \(\Gamma\) is a normal
subgroup of \(\pi_1(X)\) with finite index since the quotient is isomorphic to a subgroup of
\(Isom(M)\). Therefore \(\tilde X/\Gamma = E/\Gamma\times M\) is compact as a finite cover of \(X\).

We apply the following theorem of Bieberbach.
\begin{theorem}[Bieberbach]
Let \(E = \mathbb{R}^n\) be an Euclidean space and \(\Gamma\) be a subgroup of \(Isom(E)\) that satisfies
\begin{enumerate}
\item \(\Gamma\) is discrete under compact-open topology.
\item \(E/\Gamma\) is compact.
\end{enumerate}
Then the subgroup \(\Gamma'\) of translations in \(\Gamma\) is of finite index.
\end{theorem}

Suppose that the two conditions are satisfied then the theorem gives: \(\tilde X/\Gamma' =
E/\Gamma'\times M = T\times \prod_i V_i\times \prod_j X_j\) is a finite cover of
\(\tilde X/\Gamma\) as \(\Gamma'\) is a normal subgroup of \(\Gamma\):

\textbf{Fact.} The subgroup of translations in \(Isom(E)\), where \(E = \mathbb{R}^n\) is an Euclidean space, is
normal.

Therefore \(X' = \tilde X/\Gamma'\) is a finite cover of \(X\) that we want to find.

It remains to prove that \(\Gamma\) is discrete, which is a consequence of
\begin{enumerate}
\item \(\pi_1(X)\) is discrete, without limit point in \(Isom(E)\times Isom(M)\) (obvious).
\item \(Isom(M)\) is compact.
\end{enumerate}
In fact given any \(\phi = (\phi_1,\phi_2) \in Isom(E)\times Isom(M)\), there exists by (1.) a neighborhood \(\mathcal{U}_1(\phi_1,\phi_2)\times \mathcal{U}_2(\phi_1,\phi_2)\) of \(\phi\) in \(Isom(E)\times
Isom(M)\) such that all points of \(\pi_1(X)\) lying in this region project to \(\phi_1\). By (2.)
we can find a neighborhood \(\mathcal{U}_1\) of \(\phi_1\) in \(Isom(E)\) small enough that \(\mathcal{U}_1(\phi_1)\times Isom(M) \subset \cup_{\phi_2\in Isom(M)}
\mathcal{U}_1(\phi_1,\phi_2)\times \mathcal{U}_2(\phi_1,\phi_2)\). Therefore the projection of
\(\pi_1(X)\) to \(Isom(E)\) is discrete, by consequence \(\Gamma\) is discrete.
\end{proof}

\begin{lemma}[]
\label{lem:Isom-discrete}
\label{org65d013f}
Let \(M\) be is a compact, simply-connected, Ricci-flat, Kähler manifold, then the group \(Aut(M)\) of
automorphism of \(M\) equipped with compact-open topology is discrete, therefore \(Isom(M)\) is discrete,
hence finite. 
\end{lemma}
\begin{proof}
The idea is that since \(Aut(M)\) is a Lie group, it suffices to prove that its Lie algebra is of
dimension 0. This is done using these facts. 
\begin{enumerate}
\item The Lie algebra of \(Aut(M)\) can be identified with the vector space of holomorphic vector
fields on \(M\).
\item \emph{Bochner's principle}: All holomorphic tensor fields on a compact, Ricci-flat Kähler manifold are
parallel.
\item The only invariant vector of the holonomy representation of \(M\) is \(0\) (obvious).
\end{enumerate}
\end{proof}

Bochner principle for holomorphic vector fields comes from the following identity (called \emph{Weitzenbock formula}):
\[
\Delta (\frac{1}{2}\|X\|^2) = \| \Delta X\|^2 + g(X, \nabla \text{div} X) + Ric(X,X)
\]
for every vector field \(X\). If \(X\) is holomorphic then it is harmonic and has \(\text{div} X = 0\). The
fact that \(M\) is Ricci-flat gives \(\Delta (\frac{1}{2}\|X\|^2) = \| \nabla X\|^2\) and the
function \(\| X\|^2\) is subharmonic, therefore constant since \(M\) is compact. We then have \(\nabla X =
0\),i.e. \(X\) is parallel. The method of Bochner also works for tensor fields of any type in
a Ricci-flat Kähler manifold and one also has \(\Delta(\|\tau \|^2) = \|\nabla\tau \|^2\) and that every
holomorphic tensor field is parallel. See P. Petersen, \emph{Riemannian geometry} and A. Besse, \emph{Einstein
Manifolds} for more detail.

\section{Towards a classification for complex manifold}
\label{sec:orgdbfd494}
To obtain a translation of Theorem \ref{orgdf9aa02} in a context of complex manifolds (without any
preferred metric a priori), we study the 2 building blocks: manifolds with holonomy \(SU(m)\) and
\(Sp(r)\). To be clear, recall that a complex manifold \(X\) is called \uline{of Kähler type} if one can
equip \(X\) with an Hermitian structure whose fundamental form \(\omega\) satisfies \(d\omega= 0\). When
we say \(X\) is of Kähler type, we refer to \(X\) as a complex manifold without fixing a metric on
\(X\). 
\subsection{Special unitary manifolds (proper Calabi-Yau manifolds)}
\label{sec:orgf819a06}
\begin{remark}
\label{rem:SU}
\label{org16e8097}
Let \(X\) be a compact Kähler manifold with holonomy \(SU(m)\) and complex dimension \(m\geq 3\) then:
\begin{enumerate}
\item \(H^0(X, \Omega_X^p)=0\) for all \(0 < p < m\), by consequence \(\chi( \mathcal{O}_X ) = 1 + (-1)^m\).
\item \(X\) is \emph{projective}, that is \(X\) can be embedded into \(\mathbb{P}^N\) as zero-locus of some
(finitely) homogeneous polynomials.
\item \(\pi_1(X)\) is finite and if \(m\) is even, \(X\) is simply connected.
\end{enumerate}
\end{remark}
The first point is in fact algebraic in nature: it comes from the fact that the representation of
\(SU(m)\) over \(\bigwedge^pT^*_xM\) is irreducible for all \(p\) et non-trivial for \(0<p<m\), therefore the
action of \(SU(m)\) on \(\bigwedge^pT^*_xM\) for \(0<p<m\) has no invariant element, hence
\(H^0(X,\Omega^p_X)=0\).


The second point follows the following facts:
\begin{enumerate}
\item (Kodaira's theorem) A compact Kähler manifold with \(H^{2,0}=0\) can be embedded in \(\mathbb{P}^N\).
\item (Chow's theorem) A compact complex manifold embedded in \(\mathbb{P}^N\) is algebraic,
i.e. defined by a finite number of homogeneous polynomials.
\end{enumerate}


The third point is a direct consequence of Riemann-Hurwitz formula. In fact, the universal cover
\(\tilde X\) of \(X\) is of holonomy \(SU(m)\). This is due to the following remarks: \(Hol(X)\supset
Hol(X')\supset Hol_0(X') = Hol_0(X)\) and \(Hol_0(X) = Hol(X) = SU(m)\) as \(SU(n)\) is connected.

By Theorem \ref{orgdf9aa02}, \(\tilde X\) is compact by Lemma \ref{org65d013f} a finite covering of
\(X\) as \(\pi_1(X)\) is finite. As \(\chi(\mathcal{O}_X) = \chi(\mathcal{O}_{\tilde X}) = 2\), one has \(X
= \tilde X\), hence \(X\) is simply-connected.



\begin{theorem}[]
\label{thm:SU-alg}
\label{org69e0400}
Given a compact manifold \(X\) of Kähler type and complex dimension \(m\), the following properties are equivalent
\begin{enumerate}
\item There exists a compatible metric \(g\) over \(X\) such that \(Hol(X,g) = SU(m)\).
\item \(K_X\) is trivial and \(H^0(X', \Omega_{X'}^p) =0\) for every \(0 < p < m\) and \(X'\) a finite covering of \(X\).
\end{enumerate}
\end{theorem}

\begin{proof}
(1) implies (2) as a finite covering space \(X'\) of a special unitary manifold \(X\) is still a special
unitary. 

For the implication (2) \(\implies\) (1): by \href{calabi-yau.org}{Yau's theorem} we equip \(X\) with a Ricci-flat metric, by
Theorem \ref{orgdf9aa02}, there exists a finite cover \(X' = T\times \prod_i V_i\times \prod_j X_j\)
where \(T\) is a complex torus, \(Hol(V_i) = SU(m_i), Hol(X_j) = Sp(r_j)\). But
\(H^0(X',\Omega^p_{X'})=0\) for \(0<p<m\), \(X'\) has to be one of the \(V_i\) as
\(H^0(X_j,\Omega_{X_j}^{2})\) and \(H^0(V_i,\Omega_{V_i}^{m_i})\) do not vanish. Therefore \(Hol(X') =
SU(m)\), hence \(Hol(X) = SU(m)\).
\end{proof}

Theorem \ref{org69e0400} allows us to check if a manifold \(X\) is special unitary by looking at the
\(h^{0,p} (0<p<m)\) coefficients of the Hodge diamond of \(X\) and its finite covers. We can see, by
this criteria that the following examples are special unitary manifolds. All of them are algebraically
constructed, since a construction by glueing local charts is difficult (or impossible).

\begin{exampl}[Special unitary manifold]
\begin{enumerate}
\item Elliptic curves over \(\mathbb{C}\) are special unitary, as any statement starting with "for every \(0<p<1\)" is
formally true.
\item A K3 surface (simply-connected surface with trivial canonical bundle) is special
unitary, its Hodge diamond is given below.
\item A quintic threefold (hypersurface of degree 5 in 4-dimensional projective space) is a special
unitary manifold, the Hodge diamond of which is given is given below. In particular, the Fermat
quintic defined by \[\{(z_0:z_1:z_2:z_3:z_4) \in \mathbb{C}\mathbb{P}^{4}:\ \sum z_i^5 =0 \}\]
\item In general, any smooth hypersurface \(X\) of \(\mathbb{C}\mathbb{P}^{m+1}\) of degree \(m+2\) satisfies \(h^{0,p}=0\)
for all \(0<p<m\). If \(X\) is simply-connected then it is a special unitary manifold.
\end{enumerate}
\end{exampl}


\begin{table}[htbp]
\caption{Hodge diamond of a K3 surface.}
\centering
\begin{tabular}{rrrrr}
 &  & 1 &  & \\
 & 0 &  & 0 & \\
1 &  & 20 &  & 1\\
 & 0 &  & 0 & \\
 &  & 1 &  & \\
\end{tabular}
\end{table}

\begin{table}[htbp]
\caption{Hodge diamond of a quintic threefold.}
\centering
\begin{tabular}{rrrrrrr}
 &  &  & 1 &  &  & \\
 &  & 0 &  & 0 &  & \\
 & 0 &  & 1 &  & 0 & \\
1 &  & 101 &  & 101 &  & 1\\
 & 0 &  & 1 &  & 0 & \\
 &  & 0 &  & 0 &  & \\
 &  &  & 1 &  &  & \\
\end{tabular}
\end{table}



\subsection{Irreducible symplectic and hyperkähler manifolds}
\label{sec:orgb162bf1}
\begin{remark}
Let \(X\) be a compact Kähler manifold with holonomy \(Sp(r)\) and complex dimension \(2r\) then:
\begin{enumerate}
\item There exists a holomorphic 2-form \(\varphi\) non-degenerate at every points.
\item \(H^0(X,\Omega_X^{2l+1}) = 0, H^0(X,\Omega_X^{2l})=\mathbb{C}\varphi^l\) for all \(0\leq l\leq
   r\). By consequence \(\chi(\mathcal{O}_X)=r+1\).
\item \(X\) is simply-connected.
\end{enumerate}
\end{remark}

The first point of the remark follows directly from our discussion of \href{Berger-remark-complex.org}{Berger classification}. 

The second point is algebraic in nature: The representation of \(Sp(r)\) on \(\bigwedge^p T^*_xM\) splits
into 
\begin{equation}
\label{eq:decomp-varphi}
\bigwedge^p T^*_xM = P_p \oplus P_{p-2}\varphi(x) \oplus P_{p-4}\varphi^2(x)\oplus \dots
\end{equation}
where \(P_k, 0\leq k\leq r\) are irreducible, non-trivial for \(k>0\) and \(\varphi(x)\in\bigwedge^2
T^*_xM\) uniquely defined up to a constant. Therefore the only invariant elements are
\(c\varphi^{p/2}\) where \(c\) is a scalar.

For the last point, one uses the same arguments as Remark \ref{org16e8097}.


\begin{theorem}[]
\label{thm:Sp-alg}
\label{org84293a6}
Given a compact manifold \(X\) of Kähler type and complex dimension \(2r\), then:
\begin{enumerate}
\item The following properties are equivalent. \(X\) is called \uline{hyperkähler} if it
satisfies one of them.
\begin{enumerate}
\item There exists a compatible metric \(g\) such that \(Hol(X,g) \subset Sp(r)\).
\item There exists a compatible symplectic structure: a 2-form that is closed, holomorphic and non-degenerate
at every point.
\end{enumerate}
\item The following properties are equivalent. \(X\) is called \uline{irreducible symplectic} if it
satisfies one of them.
\begin{enumerate}
\item There exists a compatible metric \(g\) such that \(Hol(X,g) = Sp(r)\)
\item \(X\) is simply-connected and there exists (uniquely up to a constant) a compatible symplectic structure on
\(X\).
\end{enumerate}
\end{enumerate}
By "compatible", we mean "compatible with the complex structure".
\end{theorem}

\begin{proof}
\begin{enumerate}
\item The fact that (a) implies (b) is obvious. For the other way: since \(K_X\) is trivial (existence of
global non-null section) by \href{calabi-yau.org}{Yau's theorem} we equip \(X\) with a Ricci-flat metric, then the
symplectic structure \(\varphi\) of \(X\) is parallel by Bochner's principle. Hence the holonomy is
in \(Sp(r)\).
\item For the implication (a) \(\implies\) (b), it suffices to notice that the invariant elements \(\varphi\)
in the decomposition \eqref{eq:decomp-varphi} is unique. For the direction (b) \(\implies\) (a),
note that \(X\) can be equipped with a Calabi-Yau metric by the (b) \(\implies\) (a) part of (1.), by
Theorem \ref{orgdf9aa02}, \(X = \prod_{j=1}^m X_j\) where \(X_j\) are irreducible compact Kähler
manifolds. The symplectique structure \(\varphi\) on \(X\), restricted on each \(X_j\), gives a
symplectique structure \(\varphi_j\) of \(X_j\). But any form \(\sum_j \lambda_j pr_j^*\varphi_j\) is
another symplectic structure of \(X\), one must have \(m=1\) by uniqueness of \(\varphi\).
\end{enumerate}
\end{proof}

\begin{exampl}[]
\begin{enumerate}
\item One can notice a trivial example: Every special unitary manifold of 2 complex dimensions is
irreducible symplectic because \(SU(2)\) is isomorphic to \(Sp(1)\).
\item Let \(X\) be a smooth cubic hypersurface in \(\mathbb{C}\mathbb{P}^{n+1}\) and \(F(X)= \{ L \in Gr(1,
   \mathbb{C}\mathbb{P}^{n+1}) , L \subset X\} \subset Gr(1, \mathbb{C}\mathbb{P}^{n+1})\) the
manifold formed by lines in \(X\). \(F(X)\) is non-empty when \(n>1\), smooth if \(X\) is smooth and of
dimension \(2n-4\). Beauville and Donagi proved that for \(n=4\), \(F(X)\) is irreducible symplectic,
therefore hyperkähler.
\end{enumerate}
\end{exampl}
\subsection{Decomposition for complex manifold with vanishing Chern class}
\label{sec:org6b838e2}

Theorem \ref{orgdf9aa02} can be translated to a decomposition for complex manifold in the following
way:
\begin{theorem}[Bogomolov-Beauville classification]
\label{thm:beauville-2}
\label{orgb6cc487}
Let \(X\) be a compact manifold of Kähler type of vanishing first Chern class.
\begin{enumerate}
\item The universal covering space \(\tilde X\) of \(X\) is isomorphic to a product \(E\times \prod_i
   V_i\times\prod_j X_j\) where \(E = \mathbb{C}^k\) and
\begin{enumerate}
\item Each \(V_i\) is a projective simply-connected manifold of complex dimension \(m_i\geq 3\), with trivial
\(K_{V_i}\) and \(H^0(V_i,\Omega_{V_i}^p) = 0\) for \(0 < p < m_i\)
\item Each \(X_j\) is an hyperkähler manifold.
\end{enumerate}
This decomposition is unique up to an order of \(i\) and \(j\).
\item There exists a finite cover \(X'\) of \(X\) isomorphic to the product \(T\times\prod_i
   V_i\times\prod_j X_j\).
\end{enumerate}
\end{theorem}
The theorem follows directly from Theorem \ref{orgdf9aa02}, the only point that needs proof is the
uniqueness, which will be achieved in two steps:
\begin{enumerate}
\item Prove the uniqueness in the case that \(X\) is simply-connected.
\item Prove that every isomorphism \(\phi:\ \mathbb{C}^k\times Y\longrightarrow \mathbb{C}^h\times Z\) is
splitted as \(\phi = (\phi_1,\phi_2)\) where \(\phi_1:\ \mathbb{C}^k\longrightarrow \mathbb{C}^h\) and
\(\phi_2:\ Y\longrightarrow Z\) are isomorphisms (by consequence \(h=k\)).
\end{enumerate}
These two steps will be accomplished in the following two lemmas

\begin{lemma}[]
Let \(Y = \prod_j Y_j\) be a finite product of compact, simply-connected manifold of Kähler type with vanishing Chern
class. The Calabi-Yau metrics of \(Y\) are then \(g = \sum_l pr_j^*g_j\) where \(g_j\) are Calabi-Yau metrics
of \(Y_j\).
\end{lemma}
\begin{proof}
Let \(g\) be a Calabi-Yau metric of \(Y\) and \([\omega]\) its class in \(H^{1,1}(Y)\). Since \(Y_j\) are
simply-connected, \([\omega] = \sum_j pr_j^* [\omega_j]\). By \href{calabi-yau.org}{Yau's theorem}, there exist unique Calabi-Yau
metrics \(g_j\) of \(Y_j\) in each class \([\omega_j]\). The metric \(g' = \sum_j pr_j^* g_j\) is in the same
class \(\omega\) of \(g\) and is also a Calabi-Yau metric, hence \(g= g' = \sum_j pr_j^*g_j\).
\end{proof}

This lemma asserts that when our manifolds \(Y, Y_j\) are equipped with appropriate Calabi-Yau metrics,
the decomposition map is also a (Riemannian) isometric, we therefore obtain uniqueness of \(V_i, X_j\)
from uniqueness of Theorem \ref{orgdf9aa02}.

\begin{lemma}[]
Let \(Y,Z\) be compact, simply-connected manifold of Kähler type, then any isomorphism \(u:\
\mathbb{C}^k\times Y\longrightarrow \mathbb{C}^h\times Z\) is splitted as \(\phi = (\phi_1,\phi_2)\)
where \(\phi_1:\ \mathbb{C}^k\longrightarrow \mathbb{C}^h\) and \(\phi_2:\ Y\longrightarrow Z\) are
isomorphisms.
\end{lemma}
\begin{proof}
It is clear that the composed function \(u_1: \mathbb{C}^k\times Y \longrightarrow \mathbb{C}^h \times Z
\longrightarrow \mathbb{C}^h\) is constant in \(Y\), i.e. \(u_1(t,y) = u_1(t)\) as holomorphic
functions on \(Y\) are constant, therefore  \(u(t,y) = (u_1(t), u_2(t,y))\). As \(u\) is isomorphic, one
has \(h\leq k\) then by the same argument for \(u^{-1}\), one has \(h=k\), \(u_1\) is an isomorphism and
\(u_2(t,\cdot)\) is an isomorphism from \(Y\) to \(Z\). \(u_2(0,\cdot)^{-1}\circ u_2(t,\cdot)\) is then a curve in
\(Aut(Y)\), which is discrete by Lemma \ref{org65d013f}. Therefore \(u_2(t,\cdot)= u_2(0,\cdot)\) independent of \(t\).
\end{proof}

`
Emacs 25.2.1 (Org mode 9.0.5)
\end{document}