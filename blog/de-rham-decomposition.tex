% Created 2017-10-28 Sat 18:27
% Intended LaTeX compiler: pdflatex
\documentclass[11pt]{article}
\usepackage[utf8]{inputenc}
\usepackage[T1]{fontenc}
\usepackage{graphicx}
\usepackage{grffile}
\usepackage{longtable}
\usepackage{wrapfig}
\usepackage{rotating}
\usepackage[normalem]{ulem}
\usepackage{amsmath}
\usepackage{textcomp}
\usepackage{amssymb}
\usepackage{capt-of}
\usepackage{hyperref}
\usepackage{amsthm}
\usepackage{amsmath}
\usepackage{tikz-cd}
\newtheorem{remark}{Remark}
\newtheorem{theorem}{Theorem}
\newtheorem{lemma}[theorem]{Lemma}
\newtheorem{corollary}{Corollary}[theorem]
\newtheorem{conjecture}[theorem]{Conjecture}
\newtheorem{proposition}{Proposition}[theorem]
\newtheorem{problem}{Problem}
\newtheorem{exampl}{Example}
\newtheorem{definition}{Definition}
\newtheorem{propdef}[definition]{Proposition-Definition}
\newcommand{\im}{\mathop{\rm Im}\nolimits}
\newcommand{\supp}{\mathop{\rm supp}\nolimits}
\newcommand{\ord}{\mathop{\rm ord}\nolimits}
\newcommand{\Spec}{\mathop{\rm Spec}\nolimits}
\newcommand{\vol}{\mathop{\rm vol}\nolimits}
\newcommand\restr[2]{{% we make the whole thing an ordinary symbol
\left.\kern-\nulldelimiterspace % automatically resize the bar with \right
#1 % the function
\vphantom{\big|} % pretend it's a little taller at normal size
\right|_{#2} % this is the delimiter
}}
\author{darknmt}
\date{\today}
\title{De Rham decomposition}
\hypersetup{
 pdfauthor={darknmt},
 pdftitle={De Rham decomposition},
 pdfkeywords={},
 pdfsubject={},
 pdfcreator={Emacs 25.3.1 (Org mode 9.0.5)}, 
 pdflang={English}}
\begin{document}

\maketitle
\tableofcontents

\iffalse
\begin{info}
The PDF version of this page can be downloaded by replacing \texttt{html} in the its address by
\texttt{pdf}. 
For example \texttt{/html/sheaf-cohomology.html} should become \texttt{/pdf/sheaf-cohomology.pdf}.
\end{info}
\fi

\iffalse
\begin{info}
This post is a part of the \href{../res/Stage2017.pdf}{memoire of my M1 internship} at I2M. The memoire contains,
needless to say, less errors than this page.
\end{info}
\fi

\section{Decomposition theorem of de Rham}
\label{sec:org6744296}

We observe that if a manifold \((M,g)\) is globally a product \((M_1,g_1)\times (M_2,g_2)\) then
\(Hol_{g}(M) = Hol_{g_1}(M_1)\times Hol_{g_2}(M_2)\) and the holonomy representation of \(M\) is
reducible. A result of de Rham says that one can decompose a Riemannian manifold as product of ones
with irreducible holonomy representation.


\begin{theorem}[De Rham decomposition]
\label{orgc11e5b2}
Given \((M,g)\) a simply-connected and complete Riemannian manifold, there exists a unique
decomposition up to isometry and permutation of factors
\[
(M,g) = \prod_{i=1}^n(M_i,g_i)
\]
where \((M_i,g_i)\) are complete, simply connected Riemannian irreducible manifolds. Moreover the
holonomy representation of \(M\) over \(T_xM\) is the product of holonomy representations of \(M_i\) over
\(T_{x_i}M_i\) where \(x = (x_1,\dots, x_n)\)
\end{theorem}

\begin{proof}[Sketch of proof]
The proof of this theorem contains two steps:
\begin{enumerate}
\item Remark that if the holonomy group is reducible then locally \(M\) is a product of Riemannian
manifolds, i.e. for every \(x\in M\) there exists a neighborhood \(U\) containing \(x\) with
\((U,g)=(M_1,g_1)\times (M_2,g_2)\).
\item Obtain the global product structure from local one. This is where completeness is used.
\end{enumerate}


We now discuss the first point with a bit more details. Suppose that \(T_xM = U_x \bigoplus^\perp
V_x\) where \(U_x, V_x\) are stable under action of holonomy group, then by transporting \(U_x, V_x\) to
the tangent space of any point \(y\) (as they are stable by holonomy, the result is independent of the
curve along which the transport is taken), we obtain then two sub bundles \(A\) and \(B\) of \(TM\) over
\(M\) that are stable by parallel transport. Then for every vector field \(u_A\) in \(A\) and \(v\) in \(TM\),
\(\nabla_v u_A \in A\). As the Levi-Civita connection is torsionless, one deduces \([u_A,v_A] =
\nabla_{u_A}v_A - \nabla_{v_A}u_A\) remains in \(A\). By Frobenius theorem, locally at a point \(x\in
M\), there exist manifolds \(M_1, M_2\) whose tangent spaces are \(A\) and \(B\).


\begin{theorem}[Frobenius]
\label{thm:Frobenius}
\label{org907de79}
Given a distribution \(D\) which to each point \(x\) associates a \(k\) -dimensional hyperplane of \(T_xM\)
such that:
\begin{enumerate}
\item \(D\) varies smoothly by \(x\), i.e. for every \(x_0\),there exist \(k\) smooth vector fields locally
defined near \(x_0\) that at each point \(x\) form a base of \(D(x)\).
\item \(D\) is stable by Lie bracket, i.e. for every vector fields \(X,Y\) on \(M\) that take value in \(D\),
\([X,Y]\) takes value in \(D\).
\end{enumerate}
Then at each point \(x\in M\), there exists a maximal \(k\) -dimensional sub-manifold \(N\) of \(M\) containing
\(x\) such that \(D(y)\) is the tangent of \(N\) at \(y\). The maximality means that every sub-manifold of
\(M\) that satisfies this condition is an open sub-manifold of \(N\).
\end{theorem}
For a complete proof that \(M\) is isometric to \(M_1\times M_2\), see Takashi Sakai, \emph{Riemannian
geometry} (Lemma 6.8- Theorem 6.11, chapter III).
\end{proof}



\section{Uniqueness}
\label{sec:org7171d65}
We note that the decomposition is unique in the following sense: 

\begin{proposition}[Uniqueness of de Rham decomposition]
\label{prop:uniqueness}
\label{orgb056b03} If \(M\) is decomposed as \(p_1:\ M\longrightarrow E\times \prod M_i\) and \(p_2:\
M\longrightarrow E'\times\prod M'_j\) where \(M_i, M'j\) are irreducible and \(E,E'\) are maximal
Euclidean components (i.e. none of \(M_i, M'_j\) are isometric to \(\mathbb{R}\)). Then up to a
rearrangement of indice \(j\) the composed map \(f=p_2\circ p_1^{-1}:\ E\times\prod M_i \longrightarrow
E'\times \prod M'_i\) are product of the isometries \(f_E: E\longrightarrow E'\) and \(f_i:M_i
\longrightarrow M'_i\).
\end{proposition}

We first explain the appearance of Euclidean components \(E,E'\). They come from the parallel
transport of trivial representations appeared in the decomposition on each fiber. We call them
Euclidean because they are, up to an isometry, \(\mathbb{R}^k\) with the usual metric. This follows
from the fact that \(\mathbb{R}\) with any Riemannian metric is isometric to \(\mathbb{R}\) with
Euclidean metric.

We first note that the uniqueness stated in Proposition \ref{orgb056b03} comes from the uniqueness of
the decomposition of each tangent fiber, we have the following lemma.

\begin{lemma}[Uniqueness of fiber decomposition]
\label{lem:uniqueness-fiber}
\label{orgb7febfa}
Let \(f: M\longrightarrow M'\) be an isometry that send \(x\in M\) to \(y\in M'\). Let 
\[
T_x M = E \oplus^\perp \bigoplus_i V_i,\quad T_yM' = E' \oplus^\perp \bigoplus_j V'_j
\]
be a decomposition of \(T_xM\) and \(T_yM'\) as direct sum of trivial subspaces \(E,E'\) and irreducible non-trivial
subspaces \(V_i, V_j'\) under holonomy action. Then up to a rearrangement of \(j\), \(f_*\) send \(E\) to
\(E'\) and \(V_i\) to \(V'_i\). 
\end{lemma}

\begin{remark}
One may note that a similar result is \uline{not true} for general representations: one can only prove the
uniqueness of the irreducible factors up to isomorphism and their multiplicity. But the individual irreducible
summands might not map to individual summands (However if one groups all irreducible summands of
the same type, then each group maps to another).
\end{remark}

The supplementary property of holonomy representation put into use here is the following:

\begin{remark}
The holonomy representation \(H\subset SO(V)\) on a fiber \(V=T_xM\) satisfies the \uline{property (H)}: if \(V
= V_1 \oplus^\perp V_2\) where \(V_i\) are stable by \(H\) then \(H = H_1\times H_2\) where \(H_i:= \{ h\in
H: h|_{V_j} = Id, j\ne i\}\). It is obvious that \(H\supset H_1\times H_2\), the other inclusion is a
consequence of de Rham decomposition along \(V_1, V_2\) and the fact that \(Hol(M_1\times M_2) =
Hol(M_1)\times Hol(M_2)\).
\end{remark}

An example of representation that does not satisfies this property (H) is the group \(G = \{\pm
I_2\}\). Take \(V_i = \mathbb{R}e_i\), then \(G_1 = G_2 = \{I_2\}\) therefore \(G_1\times G_2\ne
G\). This also illustrates the fact that not every groups (representations) are holonomy groups
(representations).

We prove the following lemma, which implies Lemma \ref{orgb7febfa}.
\begin{lemma}[Uniqueness of representation decomposition]
\label{lem:unique-representation}
\label{orgf9bc81e}
Let \(G\subset SO(V)\) be an orthonormal representation on a finite dimensional vector space \(V\) with
property (H), given any two orthogonal decompositions
\[
V = E \times \prod V_i = E' \times \prod V'_j
\]
where \(G\) acts trivially on \(E, E'\), \(V_i, V'_j\) are irreducible and of dimension larger than 2,
one has \(E = E'\) and \(V_i = V'_i\) up to a rearrangement of index \(j\).

Moreover, given \(J\in Hom_G(V,V)\cap SO(V)\) then \(J\) sends \(E\) and \(V_i\) to themselves.
\end{lemma}

\begin{proof}
Note that since action of \(G\) is special orthonormal, any one dimensional subspace of \(V\) stable by
\(G\) are trivial under \(G\), that explains why we supposed \(V_i, V'_j\) are of dimension larger
than 2. It suffices to see that every irreducible subspace \(N\) of \(V\) is either contained in \(E\) or equal to
\(V_i\).

Let \(pr_i\) and \(pr_E\) be orthogonal projection of \(V\) to \(V_i\) and \(E\). As \(E\) and \(V_i\) are \(G\) -stable, these
projections are \(G\) -invariant. Let \(N_i = pr_i(N)\), then \(N_i\) is a subspace of \(V_i\) stable
by \(G\), hence either \(0\) or all \(V_i\). If all \(N_i=0\) then clearly \(N\) is perpendicular to \(\bigoplus_i
V_i\), that is \(N\subset E\). If \(pr_E(N)\ne 0\) then \(Hom_G(N, E) \ne 0\) since it contains \(pr_E\). Since
\(N\) is irreducible, \(N\) is \(G\) -isomorphic to a irreducible component of \(N\) by \(pr_E\), therefore
\(N\) is \(G\) -trivial hence \(N\subset E\). Therefore one can suppose that at least one \(N_i = V_i\) and
\(pr_E(N) = 0\), i.e. \(E\perp N\). Note that \(pr_i\) is bijective by Schur lemma.

Let \(G_i = \{g\in G: g|_{V_j} = Id\ \forall j\ne i \}\) then \(Fix(\prod_{j\ne i} G_j)= E\oplus V_i\), in fact if \(v = e +\sum v_i\in Fix(\prod_{j\ne i}
G_j)\) where \(e\in E, v_k\in V_k\), one has \(g_j v_j = v_j \forall g_j \in G_j\), hence \(g v_j = v_j
\forall g\in G\), hence \(v_j = 0\). Now note that as \(pr_i\) commutes with \(G\) and \(N_i\) is fixed by
\(\prod_{j\ne i}G_j\), \(N\) is also fixed by \(\prod_{j\ne i}G_j\). Therefore \(N\subset E \oplus^\perp
V_i\), hence \(N = N_i=V_i\) as \(N\perp E\).

For the last point, note that as \(J\) commutes with all elements of \(G\), \(J\) sends \(Fix(\prod_{j\in
A} G_j)\) to itself. Therefore \(J|_E: E\longrightarrow E\) and \(J|_{E\oplus V_i}: E\oplus
V_i\longrightarrow E\oplus V_i\), hence by orthogonality \(J\) sends \(E\) and \(V_i\) to themselves.
\end{proof}


\section{Application of uniqueness lemma: decomposition for Kähler manifold}
\label{sec:org83afeb1}
Now let apply the de Rham decomposition to a complete Riemannian manifold \(M\) with holonomy \(U(n)
\subset SO(2n)\) (called a \emph{Kähler manifold}). There exists on a fixed fiber \(T_xM\) an automorphism
\(J\) that preserves the Riemannian metric and satisfies \(J^2 = -1\). By transporting \(J\) to every other
fibers of \(TM\) one obtains a \emph{almost complex structure} on \(M\).

Apply Lemma \ref{orgf9bc81e} to \(J\) which obviously commutes with \(G\) and orthonormal, one
see that such structure \(J\) passes to every manifold \(M_i\) and the Euclidean component \(\mathbb{R}^n\) and
remains parallel on these manifolds. We proved that \(M\) is decomposed to \(\mathbb{C}^{n/2}\times
\prod M_i\) where \(M_i\) are Kähler manifold. The decomposition map is both a Riemannian isometry and
a isomorphism between complex manifold (i.e. its preserves complex structure). 
Emacs 25.3.1 (Org mode 9.0.5)
\end{document}