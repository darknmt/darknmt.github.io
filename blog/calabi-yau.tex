% Created 2017-06-14 Wed 12:43
% Intended LaTeX compiler: pdflatex
\documentclass[11pt]{article}
\usepackage[utf8]{inputenc}
\usepackage[T1]{fontenc}
\usepackage{graphicx}
\usepackage{grffile}
\usepackage{longtable}
\usepackage{wrapfig}
\usepackage{rotating}
\usepackage[normalem]{ulem}
\usepackage{amsmath}
\usepackage{textcomp}
\usepackage{amssymb}
\usepackage{capt-of}
\usepackage{hyperref}
\usepackage{amsthm}
\usepackage{tikz-cd}
\newtheorem{remark}{Remark}
\newtheorem{theorem}{Theorem}
\newtheorem{lemma}[theorem]{Lemma}
\newtheorem{corollary}{Corollary}[theorem]
\newtheorem{conjecture}[theorem]{Conjecture}
\newtheorem{proposition}{Proposition}[theorem]
\newtheorem{problem}{Problem}
\newtheorem{example}{Example}
\newtheorem{definition}{Definition}
\author{darknmt}
\date{\today}
\title{Calabi-Yau theorem}
\hypersetup{
 pdfauthor={darknmt},
 pdftitle={Calabi-Yau theorem},
 pdfkeywords={},
 pdfsubject={},
 pdfcreator={Emacs 25.2.1 (Org mode 9.0.5)}, 
 pdflang={English}}
\begin{document}

\maketitle
\tableofcontents


\section{Calabi conjecture}
\label{sec:orga8314a6}
We start with the following fact (which is an exercise in Daniel Huybrechts, \emph{Complex geometry - an introduction})

\begin{proposition}[Ricci curvature and first Chern class]
Let \((X,g)\) be a compact Kähler manifold, then \(iRic(X,g)\) is the curvature of the Chern connection
on the canonical bundle \(K_X\). In other words, \(Ric(X,g)\in -2\pi c_1(K_X)\) where
\(c_1(K_X)\) is the first Chern class of \(K_X\).
\end{proposition}

The Calabi conjecture asked whether there exists for each form \(R\in c_1(K_X)\) a metric \(g'\) with
\(Ric(X,g') = R\). We prefer to work with the fundamental form instead of the metric \(g\) as the former
is antisymmetric and its derivative is hence easy to define.

\begin{definition}
The quadruple \((h, g, \omega, J)\) is said to be \uline{compatible} if \(g\circ J = g\) and \(\omega(a,b) =
g(Ja,b)\) and \(h = g - i\omega\). 
\end{definition}

\begin{remark}
\begin{enumerate}
\item When \(J\) is fixed, one of \(h,g,\omega\) that is invariant by \(J\) determines the two others.
\item For a compatible quadruple, the condition \(\nabla J = 0\) is equivalent to \(d\omega = 0\). The
fundamental form \(\omega\) that satisfies \(d\omega = 0\) is called a \uline{Kähler form}.
\end{enumerate}
\end{remark}


\section{Reduction to local charts, Yau theorem}
\label{sec:orge47df6e}
\paragraph{\(h,g,\omega\) in local coordinates.}
\label{sec:org5d1413c}
We note by \(h_{i\bar j} = h(\partial_{x_i},\partial_{x_j}) =
2g_{\mathbb{C}}(\partial_{z_i},\partial_{z_j})\). By straightforward calculation one has
\begin{align*}
\omega & = -\frac{1}{2} Im h_{i\bar j} (dx^i\wedge dx^j + dy^i\wedge dy^j) + Re h_{i\bar j}dx^i\wedge dy^j\\
& = \frac{i}{2}h_{i\bar j}dz^i\wedge d\bar{z^j}
\end{align*}
and the condition \(d\omega = 0\) is equivalent to
\[
\frac{\partial h_{i\bar j}}{\partial z_k} = \frac{\partial h_{k\bar j}}{\partial z_i}
\]
We also note by \(h^{i\bar j}\) the inverse transposed of \(h_{i\bar j}\), i.e. \(h^{i\bar j}h_{k\bar j}
= \delta_j^k\)
\begin{definition}
Let \(X\) be an almost complex manifold (manifold with an almost complex structure). Then
\(d:\wedge^nT^*X\longrightarrow \wedge^{n+1}T^*X\) sends \(\wedge^{p,q}T^*M\) to
\(\wedge^{p+1,q}T^*M\oplus \wedge^{p,q+1}T^*M\). We denote by \(\partial\) and \(\bar\partial\) the
component of \(d\) in \(\wedge^{p+1,q}T^*M\) and \(\wedge^{p,q+1}T^*M\) respectively. 
\end{definition}
It would be convenient to define \(d^c =i(\bar\partial - \partial)\) then obviously \(dd^c =
2i\partial\bar\partial\). 


\paragraph{The Ricci curvature.}
\label{sec:org8f64d3a}
The Ricci curvature is given by
\[
Ric_{\omega} = -\frac{1}{2}dd^c\log\det(h_{i\bar j})
\]


\paragraph{\(dd^c\) lemma .}
\label{sec:orge83ab23}
We then can state the \(dd^c\) lemma
\begin{lemma}[]
Let \(\alpha\) be a real, (1,1)-form on a compact Kähler manifold \(M\). Then \(\alpha\) is \(d\) -exact if
and only if there exists \(\eta\in C^\infty(M)\) globally defined such that \(\alpha = dd^c\eta\).
\end{lemma}

\paragraph{Yau's theorem.}
\label{sec:org9fd777f}
The \(dd^c\) lemma tells us every form \(R\in c_1(K_X)\) is of form \(Ric_{\omega} + dd^c\eta\). If one
varies the Hermitian product \(h_{i\bar j}\) to \(h_{i\bar j} + \phi_{i\bar j}\) then the new Ricci
curvature is \(dd^c\log\det(h_{i\bar j} + \phi_{i\bar j})\). The Calabi conjecture can be restated as
the existence of \(\phi\) such that \(h_{i\bar j} + \phi_{i\bar j}\) is definite positive and
\begin{equation}
\label{eq:ddc-0}
dd^c\left( \log\det(h_{i\bar j} + \phi_{i\bar j}) - \log\det(h_{i\bar j}) -\eta\right) = 0
\end{equation}

The functions \(f\) that satisfies \(dd^cf = 0\) are called \emph{pluriharmonic}. They also satisfy the
maximum principle. By compactness of \(X\), these functions on \(X\) are exactly constant
functions. Therefore \eqref{eq:ddc-0} is equivalent to
\[
\det(h_{i\bar j} + \phi_{i\bar j}) = e^{c+\eta}\det(h_{i \bar j})
\]
or
\[
(\omega + dd^c\phi)^n = e^{c+\eta}\omega^n
\]
where \(\omega^n\) denotes the repeated wedge product. Note that \((\omega +dd^c\phi)^n - \omega^n\) is exact,
one has \(\int_M (\omega +dd^c\phi)^n = V\), the conjecture of Calabi is therefore a consequence of
the following theorem.

\begin{theorem}[Yau]
Given a function \(f\in C^\infty(M), f>0\) such that \(\int_M f\omega^n = V\). There exists, and unique
up to constant, \(\phi\in C^\infty(M)\) such that \(\omega + dd^c\phi >0\) and
\[
(\omega + dd^c\phi)^n = f\omega^n
\]
\end{theorem}

\section{A sketch of proof}
\label{sec:orgb06fe73}
The uniqueness is straightforward. We will prove the existence of \(\phi\) under the constraint
\(\int_M\phi\omega^n = 0\) (which will be useful to prove that \(\mathcal(N)\) is locally diffeomorphism
later). We will prove that the set \(S\) of \(t\in [0,1]\) such that there exists \(\phi_t\in
C^{k+2,\alpha}(M)\) with \(\int_M \phi_t\omega^n = 0\) that satisfies
\begin{equation}
\label{eq:omega-convex-t}
(\omega + dd^c\phi_t)^n = (tf + 1-t)\omega^n
\end{equation}
is both open and close in \([0,1]\), therefore is the entire interval as \(0\in S\) is non empty.

To see that \(S\) is open, one only has to prove that the function \(\mathcal{N}\) defined by
\[
\phi\mapsto \mathcal{N}(\phi)= \frac{\det(h_{i\bar j} + \phi_{i\bar j})}{\det(h_{i\bar j})}
\]
or in other words \((\omega + dd^c\phi)^n = \mathcal{N}(\phi)\omega^n\), is a local
diffeomorphism. The differential of \(\mathcal{N}\) is given by
\[
D \mathcal{N}(\phi).\eta = \mathcal{N}\Delta\eta
\]
with \(\eta\) varies in \(\{\eta\in C^{k,\alpha}(M):\int_M\eta\omega^n=0\}\). and \(\Delta\) is the
Laplace-Beltrami operator which is known to be bijective between
\[
\left\{\eta\in C^{k+2,\alpha}(M):\ \int_M\eta = 0\right\} \longrightarrow \left\{f\in C^{k,\alpha}(M):\ \int_M f=0\right\}
\]
Therefore \(\mathcal{N}\) is a local diffeomorphism and \(S\) is open.

The proof that \(S\) is closed is more technical and is accomplished in 3 steps:
\begin{enumerate}
\item Using Arzela-Ascoli theorem, it suffices to show that \(\{\phi_t:\ t\in S\}\) is bounded in
\(C^{k+2,\alpha}\). Therefore up to a subsequence, one has the uniform convergence of \(\phi_{t_n}\)
and all its partial derivatives of order \(\leq k+1\). The \(k+2\) -th order follows from \eqref{eq:omega-convex-t}.
\item Using Schauder theory, prove that the above bound follows from a \emph{priori estimate}: There exists
\(\alpha\in (0,1)\) and \(C(X,\|f\|_{1,1}, 1/\inf_M f)>0\) such that every \(\phi\in C^4(X)\)
satisfying \((\omega +dd^c\phi)^n = f\omega^n\) and \(\int_M \phi\omega^n=0\) has \[\|
   \phi\|_{2,\alpha} \leq C\]
\item Establish the priori estimate.
\end{enumerate}








\section{Calabi-Yau manifold}
\label{sec:org260b8b3}
Recall that we defined a Calabi-Yau manifold to be a compact Riemannian manifold of dimension \(2n\)
with holonomy contained in \(SU(n)\). We also remark, using parallel transport, the existence of a compatible
complex structure (\(U(n)\) suffices) and a holomorphic form non-vanishing at every point. We present
here some equivalent definitions of compact Calabi-Yau manifolds.

\begin{theorem}[]
Let \(X\) be a compact manifold of Kähler type and complex dimension \(n\) then:
\begin{enumerate}
\item The followings are equivalent
\begin{enumerate}
\item There exists a Kähler metric such that the global holonomy is in \(SU(n)\).
\item There exists a holomorphic \((n,0)\) form that vanishes nowhere.
\item The canonical bundle \(K_X\) is trivial.
\item The structure group of \(X\) can be reduced to \(SU(n)\).
\end{enumerate}
\item The following are equivalent. If \(X\) is simply-connected, they are equivalent with the 4
statements above.
\begin{enumerate}
\item There exists a Kähler metric such that the local holonomy is in \(SU(n)\).
\item The canonical bundle \(K_X\) is flat.
\item There exists a Kähler metric such that the Ricci curvature vanishes.
\item The first Chern class vanishes.
\end{enumerate}
\end{enumerate}
\end{theorem}

The proof is straightforward (see Manuscript) with the only exception is that one needs Calabi-Yau theorem to
construct Ricci-flat metric.
\end{document}