% Created 2017-08-01 Tue 02:03
% Intended LaTeX compiler: pdflatex
\documentclass[11pt]{article}
\usepackage[utf8]{inputenc}
\usepackage[T1]{fontenc}
\usepackage{graphicx}
\usepackage{grffile}
\usepackage{longtable}
\usepackage{wrapfig}
\usepackage{rotating}
\usepackage[normalem]{ulem}
\usepackage{amsmath}
\usepackage{textcomp}
\usepackage{amssymb}
\usepackage{capt-of}
\usepackage{hyperref}
\usepackage{amsthm}
\usepackage{tikz-cd}
\newtheorem{remark}{Remark}
\newtheorem{theorem}{Theorem}
\newtheorem{lemma}[theorem]{Lemma}
\newtheorem{corollary}{Corollary}[theorem]
\newtheorem{conjecture}[theorem]{Conjecture}
\newtheorem{proposition}{Proposition}[theorem]
\newtheorem{problem}{Problem}
\newtheorem{exampl}{Example}
\newtheorem{definition}{Definition}
\newtheorem{propdef}[definition]{Proposition-Definition}
\author{darknmt}
\date{\today}
\title{Calabi-Yau theorem}
\hypersetup{
 pdfauthor={darknmt},
 pdftitle={Calabi-Yau theorem},
 pdfkeywords={},
 pdfsubject={},
 pdfcreator={Emacs 25.2.1 (Org mode 9.0.5)}, 
 pdflang={English}}
\begin{document}

\maketitle
\tableofcontents


\section{Calabi conjecture}
\label{sec:org0cb0938}

In complex geometry, one usually defines the \emph{Ricci curvature} to be the real (1,1)-form \(\rho\)
with \(\rho(u,v) = Ric(Ju, v) = \text{tr}(w \mapsto R(w,v).Ju)\), as it has the advantage of being an
antisymmetric form.

We will call \(\rho\) the \uline{Ricci form} when it is easy to confuse with the Ricci
curvature tensor in Riemannian geometry. We start with the following fact (which is exercise 4.A.3
in Huybrechts, \emph{Complex geometry: an introduction}).

\begin{remark}
For our convenience when talking about positivity, we would rather use the anticanonical
bundle. Then \(K_{X}^{-1}\) is positive (resp. semi-positive) if and only if \(Ric\) is positive
definite (resp. positive semi-definite) as a symmetric form.
\end{remark}



We start with the following fact (which is exercise 4.A.3 in Daniel Huybrechts, \emph{Complex geometry: an introduction})

\begin{proposition}[Ricci curvature and first Chern class]
Let \((X,g)\) be a compact Kähler manifold. Then \(i\rho(X,g)\) is the curvature of the Chern connection
on the canonical bundle \(K_X\). In other words, \(\rho(X,g)\in -2\pi c_1(K_X)\) where
\(c_1(K_X)\) is the first Chern class of \(K_X\).
\end{proposition}


\begin{definition}
The quadruple \((h, g, \omega, J)\) is said to be \uline{compatible} if \(g\circ J = g\) and \(\omega(a,b) =
g(Ja,b)\) and \(h = g - i\omega\). 
\end{definition}

\begin{remark}
\begin{enumerate}
\item When \(J\) is fixed, one of \(h,g,\omega\) that is invariant by \(J\) determines the two others.
\item For a compatible quadruple, the condition \(\nabla J = 0\) is equivalent to \(d\omega = 0\). The
fundamental form \(\omega\) that satisfies \(d\omega = 0\) is called a \uline{Kähler form}.
\end{enumerate}
\end{remark}


The Calabi conjecture asked whether for each form \(R\in c_1(K_X)\) one can find a metric \(g'\) whose
new fundamental form \(\omega'\) is in the same class of \(\omega\) and \(Ric(X,g') = R\). We prefer to
work with the fundamental form instead of the metric \(g\) as the former is antisymmetric and its
derivative is hence easy to define.

\section{Reduction to local charts, Yau theorem}
\label{sec:orgec41f20}
\paragraph{\(h,g,\omega\) in local coordinates.}
\label{sec:orga12bb61}
We note by \(h_{i\bar j} = h(\partial_{x_i},\partial_{x_j}) =
2g_{\mathbb{C}}(\partial_{z_i},\partial_{z_j})\). By straightforward calculation one has
\begin{align*}
\omega & = -\frac{1}{2} Im h_{i\bar j} (dx^i\wedge dx^j + dy^i\wedge dy^j) + Re h_{i\bar j}dx^i\wedge dy^j\\
& = \frac{i}{2}h_{i\bar j}dz^i\wedge d\bar{z^j}
\end{align*}
and the condition \(d\omega = 0\) is equivalent to
\[
\frac{\partial h_{i\bar j}}{\partial z_k} = \frac{\partial h_{k\bar j}}{\partial z_i}
\]
We also note by \(h^{i\bar j}\) the inverse transposed of \(h_{i\bar j}\), i.e. \(h^{i\bar j}h_{k\bar j}
= \delta_j^k\)
\begin{definition}
Let \(X\) be an almost complex manifold (manifold with an almost complex structure). Then
\(d:\bigwedge^nT^*X\longrightarrow \bigwedge^{n+1}T^*X\) sends \(\bigwedge^{p,q}T^*M\) to
\(\bigwedge^{p+1,q}T^*M\oplus \bigwedge^{p,q+1}T^*M\). We denote by \(\partial\) and \(\bar\partial\) the
component of \(d\) in \(\bigwedge^{p+1,q}T^*M\) and \(\bigwedge^{p,q+1}T^*M\) respectively. 
\end{definition}
It would be convenient to define \(d^c =i(\bar\partial - \partial)\) then obviously \(dd^c =
2i\partial\bar\partial\). 


\paragraph{The Ricci curvature.}
\label{sec:org38fe6b4}
The Ricci curvature form is given in local coordinates by
\[
Ric_{\omega} = -\frac{1}{2}dd^c\log\det(h_{i\bar j})
\]


\paragraph{\(dd^c\) lemma .}
\label{sec:org77ef14c}
We then can state the \(dd^c\) lemma
\begin{lemma}[]
Let \(\alpha\) be a real, (1,1)-form on a compact Kähler manifold \(M\). Then \(\alpha\) is \(d\) -exact if
and only if there exists \(\eta\in C^\infty(M)\) globally defined such that \(\alpha = dd^c\eta\).
\end{lemma}

\paragraph{Yau's theorem.}
\label{sec:org32b43cf}
The \(dd^c\) lemma tells us that every form \(R\in c_1(K_X)\) is of form \(Ric_{\omega} + dd^c\eta\). If one
varies the Hermitian product \(h_{i\bar j}\) to \(h_{i\bar j} + \phi_{i\bar j}\) then the new Ricci
curvature is \(dd^c\log\det(h_{i\bar j} + \phi_{i\bar j})\). The Calabi conjecture can be restated as
the existence of \(\phi\) such that \(h_{i\bar j} + \phi_{i\bar j}\) is definite positive and
\begin{equation}
\label{eq:ddc-0}
dd^c\left( \log\det(h_{i\bar j} + \phi_{i\bar j}) - \log\det(h_{i\bar j}) -\eta\right) = 0
\end{equation}

The functions \(f\) that satisfies \(dd^cf = 0\) are called \emph{pluriharmonic}. They also satisfy the
maximum principle. By compactness of \(X\), these functions on \(X\) are exactly constant
functions. Therefore \eqref{eq:ddc-0} is equivalent to
\[
\det(h_{i\bar j} + \phi_{i\bar j}) = e^{c+\eta}\det(h_{i \bar j})
\]
or by \(dd^c\) lemma:
\[
(\omega + dd^c\phi)^n = e^{c+\eta}\omega^n
\]
where \(\omega^n\) denotes the repeated wedge product. Note that \((\omega +dd^c\phi)^n - \omega^n\) is exact,
one has \(\int_M (\omega +dd^c\phi)^n = V\), the conjecture of Calabi is therefore a consequence of
the following theorem.

\begin{theorem}[Yau]
Given a function \(f\in C^\infty(M), f>0\) such that \(\int_M f\omega^n = V\). There exists, uniquely
up to constant, \(\phi\in C^\infty(M)\) such that \(\omega + dd^c\phi >0\) and
\[
(\omega + dd^c\phi)^n = f\omega^n
\]
\end{theorem}

\section{A sketch of proof}
\label{sec:org351f259}
The uniqueness is straightforward. In fact if \(\phi\) and \(\psi\)
both satisfy \(\omega + dd^c\phi >0\), \(\omega + dd^c \psi >0\) and \((\omega + dd^c\phi)^n = (\omega +
dd^c\psi)^n\) then \(D(\phi - \psi) = 0\) as \[ 0 = \int_M (\phi - \psi)((\omega + dd^c\phi)^n -
(\omega + dd^c\psi)^n) = \int_M d(\phi -\psi)\wedge d^c (\phi -\psi) \wedge T \] where 
\[ 
T =\sum_{j=0}^{n-1}(\omega + dd^c\phi)^j\wedge (\omega + dd^c\psi)^{n-1-j} 
\] 
is a closed (strongly)
positive \((n-1,n-1)\) -form.





We will prove the existence of \(\phi\) under the constraint
\(\int_M\phi\omega^n = 0\) (which will be useful to prove that \(\mathcal(N)\) is locally diffeomorphism
later). We will prove that the set \(S\) of \(t\in [0,1]\) such that there exists \(\phi_t\in
C^{k+2,\alpha}(M)\) with \(\int_M \phi_t\omega^n = 0\) that satisfies
\begin{equation}
\label{eq:omega-convex-t}
(\omega + dd^c\phi_t)^n = (tf + 1-t)\omega^n
\end{equation}
is both open and close in \([0,1]\), therefore is the entire interval as \(0\in S\).

To see that \(S\) is open, one only has to prove that the function \(\mathcal{N}\) defined by
\[
\phi\mapsto \mathcal{N}(\phi)= \frac{\det(h_{i\bar j} + \phi_{i\bar j})}{\det(h_{i\bar j})}
\]
or in other words \((\omega + dd^c\phi)^n = \mathcal{N}(\phi)\omega^n\), is a local
diffeomorphism. The differential of \(\mathcal{N}\) is given by
\[
D \mathcal{N}(\phi).\eta = \mathcal{N}\Delta\eta
\]
with \(\eta\) varies in \(\{\eta\in C^{k,\alpha}(M):\int_M\eta\omega^n=0\}\). and \(\Delta\) is the
Laplace-Beltrami operator which is known to be bijective between
\[
\left\{\eta\in C^{k+2,\alpha}(M):\ \int_M\eta = 0\right\} \longrightarrow \left\{f\in C^{k,\alpha}(M):\ \int_M f=0\right\}
\]
Therefore \(\mathcal{N}\) is a local diffeomorphism and \(S\) is open.

The proof that \(S\) is closed is more technical and is accomplished in 3 steps:
\begin{enumerate}
\item Using Arzela-Ascoli theorem, it suffices to show that \(\{\phi_t:\ t\in S\}\) is bounded in
\(C^{k+2,\alpha}\). Then up to a subsequence, one has the uniform convergence of \(\phi_{t_n}\)
and all its partial derivatives of order \(\leq k+1\). The \(k+2\) -th order follows from \eqref{eq:omega-convex-t}.
\item Using Schauder theory, prove that the above bound follows from \emph{a priori estimate}: \newline There exists
\(\alpha\in (0,1)\) and \(C(X,\|f\|_{1,1}, 1/\inf_M f)>0\) such that every \(\phi\in C^4(X)\)
satisfying \((\omega +dd^c\phi)^n = f\omega^n\), \(\omega + dd^c\phi >0\) and \(\int_M \phi\omega^n=0\)
(we will call such \(\phi\) \emph{admissible})
has \[\|
   \phi\|_{2,\alpha} \leq C.\]
\item Establish the a priori estimate.
\end{enumerate}

To achieve the a priori estimate, one firstly bounds \(\phi\) in \(C^0\), then bound \(\| \Delta\phi\|\)
and finally establishs the \(C^{2,\alpha}\) estimate. We will give here some detail of the first
step. For more detail, see Z. Blocki, \emph{The Calabi-Yau Theorem}.

\begin{proof}[Proof of the $C^0$-estimate.]
Since \(\phi\) is defined up to an additive constant, what we mean by the \(C^0\) -estimate for \(\phi\) is in
fact the estimate of
\[
\text{osc}_M \phi := \max_M \phi - \min_M \phi
\]
by a constant \(C\) that depends only on \(M\) and \(f\). Without losing of generality, one assumes that
\(\int_M \omega^n = 1\) and  \(\max_M \phi = -1\). Therefore \(\| \phi \|_p \leq \| \phi\|_q\) for \(p\leq
q<\infty\).  

One has
\begin{align}
\int_M (-\phi)^p (f-1)\omega^n &= \int_M (-\phi)^p dd^c\phi \wedge \left( \sum_{j=0}^{n-1} (\omega +
dd^c\phi)^j\wedge \omega^{n-1-j}\right) \\
 &= p \int_M (-\phi)^{p-1} d\phi \wedge d^c\phi \wedge \left( \omega^{n-1} + \sum_{j=1}^{n-1}(\omega +
dd^c\phi)^j\wedge \omega^{n-1-j}  \right)\\
&\geq p\int_M (-\phi)^{p-1}d\phi\wedge d^c\phi\wedge \omega^{n-1}\\
&=\frac{4p}{(p+1)^2}\int_M d(-\phi)^{(p+1)/2}\wedge d^c(-\phi)^{(p+1)/2}\wedge \omega^{n-1}\\
&=\frac{c_n p}{(p+1)^2}\| D(-\phi)^{(p+1)/2}\|_2^2
\label{eq:long-edp}
\end{align}
where we used the fact that \(\omega + dd^c\phi >0\) in the inequality, and \(c_n\) is a constant depending only on \(n\).

Now we use the following Sobolev inequality on \(M\) (i.e. use Sobolev inequality in each chart as a
domain of \(\mathbb{R}^m\) then add up the results):
\[
\|v \|_{mq/(m-q)} \leq C(M,q) (\| v\|_q + \|Dv\|_q),\quad \forall v\in W^{1,q}(M), q<m
\]
with \(v = \phi\), \(m=2n\) the real dimension of \(M\) and \(q=2\), then use \eqref{eq:long-edp} to bound
the \(D\phi\) term:
\[
\| (-\phi)^{(p+1)/2}\|_{2n/(n-1)} \leq C_M \left[ \|(-\phi)^{(p+1)/2}\|_2 +
\frac{p+1}{\sqrt{p}}\left(\int_M (-\phi)^p (f-1) \omega^n\right)^{1/2} 
\right]
\]
Replace \(p+1\) by \(p\) and use the fact that \(|\phi|\geq 1\), one has
\[
\|\phi \|_{np/(n-1)} \leq (Cp)^{1/p} \|\phi\|_p,\quad \forall p\geq 2
\]
where \(C\) depends only on \(M\) and \(\|f\|_{\infty}\). 

Repeatedly apply this inequality (this technique is called \emph{Moser's iteration}) one has \(\|\phi
\|_{p_{k+1}} \leq (Cp_k)^{1/p_k} \|\phi\|_{p_k}\) where the sequence \(p_k\) is defined by \(p_0 = 2\)
and \(p_{k+1} = \frac{n}{n-1}p_{k-1} = 2(\frac{n}{n-1})^k\) and \[ \|\phi\|_{\infty} =
\lim_{k\to\infty}\|\phi\|_{p_k} \leq \|\phi\|_2 \prod_{j=0}^\infty (Cp_j)^{1/p_j} \] with
\(\prod_{j=0}^\infty (Cp_j)^{1/p_j} = (n/(n-1))^{n(n-1)/2} (2C)^{n/2}\)

The fact that \(\|\phi\|_2\) is bounded follows directly from the following lemma. 
\end{proof}

\begin{lemma}[$L^p$-boundedness]
For any admissible \(\phi\) with \(\max_M \phi = -1\) one has
\[
\| \phi\|_p\leq C(M,p),\quad \forall 1\leq p\leq\infty
\]
\end{lemma}

\begin{proof}
We will prove the lemma with \(p=1\) first. Let \(g\) be the local potential of the Kähler form
\(\omega\), i.e. a function defined on each chart (not necessarily agrees on zones where
charts are glued together) such that \(\omega = dd^c g = \frac{\sqrt{-1}}{2}g_{i\bar j}dz_i\wedge d\bar
z_j\) where \(g_{i\bar j}\) can also be intepreted as \(\frac{\partial^2}{\partial z_i \partial\bar z_j}
g\). We also suppose that the function \(g\) is negative on every chart. The fact that \(\omega +
dd^c\phi >0\) is rewritten as \((g_{i\bar j} + \phi_{i\bar j}) >0\) in local coordinates.

Note \(u=g + \phi\) the potential of \(\omega + dd^c\phi\) locally defined on each chart, then \(u\)
is negative and plurisubharmonic (psh). For every \(x\in B(y,R)\) one has
\[
u(x) \leq \frac{1}{\text{vol}(B(x,2R)}\int_{B(x,2R)} u \leq \frac{1}{\text{vol}(B(y,2R))}\int_{B(y,R)}u
\]
where the first inequality is due to plurisubharmonicity and the second is due to \(u\leq
0\). Therefore
\[
\| u\|_{L^1(B(y,R))} \leq \text{vol}(B(y,2R)) \inf_{B(y,R)} |u|,
\]
hence
\[
\|\phi\|_{L^1(B(y,R))} \leq \|u\|_{L^1(B(y,R))} \leq \text{vol}(B(y,2R)) (\inf_{B(y,R)} |\phi| + \max_M |g|)
\]
To see that \(\|\phi\|_1\) is bounded, we apply the following Lemma \ref{orgb89ce2a} to the covering of \(M\) by
finitely many ball \(B(y_i,R_i)\), \(c_i = \text{vol}(B(y_i,2R_i))\), \(d_i = c_i \max_M |g|\) and
\(r=1\).

The case \(p>1\) follows analoguously using the following estimate: if \(u\) is negative and psh in \(B(y,2R)\) then
\[
\|u \|_{L^{p}(B(y,R)} \leq C(n,p,R)\|u\|_{L^1(B(y,2R))}
\]
\end{proof}

\begin{lemma}[Combinatoric]
\label{lem:combinatoire}
\label{orgb89ce2a}
Let \(M\) be a connected compact manifold covered by finitely many local charts \(\{V_i\}_{i=1}^{l}\) and
\(r, c_i, d_i>0\). Then for any continuous function \(\phi\) globally
defined on \(M\) such that
\[
\|\phi\|_{L^1(V_i)} \leq c_i \inf_{V_i} |\phi| + d_i,\quad \min_M |\phi| \leq r,
\]
one has \(\|\phi\|_1:= \sum_i \|\phi\|_{L^1(V_i)}\leq C(\{V_i\},\{c_i\}, \{d_i\}, r)\)
\end{lemma}

\begin{proof}
Let \(p\) be a point in \(M\) where \(|\phi|\) attains its minimum. Since \(M\) is connected, for every \(V_i\), there exists a sequence \(V_{i_k}, 0\leq k\leq l\) such that
\[
i_0 = i,\quad V_{i_k}\cap V_{i_{k+1}}\ne \emptyset,\quad p\in V_{i_l}
\]
One has
\begin{align*}
\|\phi\|_{L^1(V_{i_k})} &\leq c_{i_k} \inf_{V_{i_k}} |\phi| + d_{i_k} \leq  c_{i_k} \inf_{V_{i_k}\cap V_{i_{k+1}}} |\phi| + d_{i_k}\\
 &\leq c_{i_k}\frac{1}{\text{vol}(V_{i_k}\cap V_{i_{k+1}})}\|\phi\|_{L^1(V_{i_{k+1}})} + d_{i_k}
\end{align*}
Repeatedly apply this inequality for \(k=0,\dots,l-1\), one has
\begin{align*}
\|\phi\|_{L^1(V_i)} &\leq A(i,\{V_j\},\{c_j\}, \{d_j\}) \|\phi\|_{L^1(V_{i_l})} + B(i,\{V_j\},\{c_j\}, \{d_j\})\\
&\leq A(i,\{V_j\},\{c_j\}, \{d_j\}) (c_{i_l} r+ d_{i_l}) + B(i,\{V_j\},\{c_j\}, \{d_j\})
\end{align*}
Take the sum for all \(i=0,\dots, l\) and the result follows.
\end{proof}








\section{Calabi-Yau manifold}
\label{sec:org58317ae}
Recall that we defined a Calabi-Yau manifold to be a compact Riemannian manifold of dimension \(2n\)
with holonomy contained in \(SU(n)\). We also remark, using parallel transport, the existence of a compatible
complex structure (\(U(n)\) suffices) and a holomorphic form non-vanishing at every point. We present
here some equivalent definitions of compact Calabi-Yau manifolds.

\begin{theorem}[]
Let \(X\) be a compact manifold of Kähler type and complex dimension \(n\) then:
\begin{enumerate}
\item The followings are equivalent
\begin{enumerate}
\item There exists a Kähler metric such that the global holonomy is in \(SU(n)\).
\item There exists a holomorphic \((n,0)\) form that vanishes nowhere.
\item The canonical bundle \(K_X\) is trivial.
\item The structure group of \(X\) can be reduced to \(SU(n)\).
\end{enumerate}
\item The following are equivalent. If \(X\) is simply-connected, they are equivalent with the 4
statements above.
\begin{enumerate}
\item There exists a Kähler metric such that the local holonomy is in \(SU(n)\).
\item The canonical bundle \(K_X\) is flat.
\item There exists a Kähler metric such that the Ricci curvature vanishes.
\item The first Chern class vanishes.
\end{enumerate}
\end{enumerate}
\end{theorem}

The proof is straightforward (see Manuscript) with the only non-trivial part is when one needs Calabi-Yau theorem to
construct Ricci-flat metric.
Emacs 25.2.1 (Org mode 9.0.5)
\end{document}