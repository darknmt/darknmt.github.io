% Created 2017-12-16 Sat 01:08
% Intended LaTeX compiler: pdflatex
\documentclass[11pt]{article}
\usepackage[utf8]{inputenc}
\usepackage[T1]{fontenc}
\usepackage{graphicx}
\usepackage{grffile}
\usepackage{longtable}
\usepackage{wrapfig}
\usepackage{rotating}
\usepackage[normalem]{ulem}
\usepackage{amsmath}
\usepackage{textcomp}
\usepackage{amssymb}
\usepackage{capt-of}
\usepackage{hyperref}
\usepackage{amsthm}
\usepackage{amsmath,amscd}
\usepackage{tikz-cd}
\newtheorem{remark}{Remark}
\newtheorem{theorem}{Theorem}
\newtheorem{lemma}[theorem]{Lemma}
\newtheorem{corollary}{Corollary}[theorem]
\newtheorem{conjecture}[theorem]{Conjecture}
\newtheorem{proposition}{Proposition}[theorem]
\newtheorem{problem}{Problem}
\newtheorem{exampl}{Example}
\newtheorem{definition}{Definition}
\newtheorem{propdef}[definition]{Proposition-Definition}
\newcommand{\im}{\mathop{\rm Im}\nolimits}
\newcommand{\supp}{\mathop{\rm supp}\nolimits}
\newcommand{\ord}{\mathop{\rm ord}\nolimits}
\newcommand{\Spec}{\mathop{\rm Spec}\nolimits}
\newcommand{\vol}{\mathop{\rm vol}\nolimits}
\newcommand\restr[2]{{% we make the whole thing an ordinary symbol
\left.\kern-\nulldelimiterspace % automatically resize the bar with \right
#1 % the function
\vphantom{\big|} % pretend it's a little taller at normal size
\right|_{#2} % this is the delimiter
}}
\author{darknmt}
\date{\today}
\title{Hodge decomposition and Kodaira embedding theorem}
\hypersetup{
 pdfauthor={darknmt},
 pdftitle={Hodge decomposition and Kodaira embedding theorem},
 pdfkeywords={},
 pdfsubject={},
 pdfcreator={Emacs 25.3.1 (Org mode 9.0.5)}, 
 pdflang={English}}
\begin{document}

\maketitle
\tableofcontents

\iffalse
\begin{info}
The PDF version of this page can be downloaded by replacing \texttt{html} in the its address by
\texttt{pdf}. 
For example \texttt{/html/sheaf-cohomology.html} should become \texttt{/pdf/sheaf-cohomology.pdf}.
\end{info}
\fi

This is my review of lectures 15-19 of \href{https://ocw.mit.edu/courses/mathematics/18-966-geometry-of-manifolds-spring-2007/index.htm}{Denis Auroux course} whose goal is to estabish Hodge
theory for compact Kähler varieties and present a proof of Donaldson for the Kodaira
embedding theorem.

\section{Hodge theory}
\label{sec:orgfb951a0}
\subsection{Operators and their dual}
\label{sec:org5e5d0ad}
\subsubsection{Scalar product on \(\Omega^k(M)\)}
\label{sec:orgd657c52}
The scalar product on \(V\) induces one on \(\Omega^k(V)\) by setting \(\langle u_1\wedge \dots \wedge u_k, v_1\wedge\dots \wedge v_k \rangle = \det(\langle
    u_i,v_j \rangle)\).
\begin{exampl}[]
\(\langle \sum \alpha_I dx^I, \beta_J dx^J \rangle =\sum \alpha_I \beta_I\) if \(\{\frac{\partial}{\partial x^i}\}\) form an orthonormal basis.
\end{exampl}
\subsubsection{Hodge star and Hodge dual}
\label{sec:orgc8dfe94}
\begin{definition}
The \textbf{Hodge star} is defined from \(\Omega^k(M) \longrightarrow \Omega^{n-k}(M)\)
such that \(\alpha \wedge *\beta = \langle \alpha, \beta \rangle \text{vol}\) where \(\text{vol}\) is the volume form.
\end{definition}


\begin{remark}
\begin{enumerate}
\item An example: \(*dx^I = dx^{I^C}\) if \(\{\frac{\partial}{\partial x^i}\}\) form an orthonormal
basis and the complement \(I^C\) is chosen so that \(\text{sgn}(I, I^C) = 1\).
\item Note that \(** = (-1)^{k(n-k)}\)
\end{enumerate}
\end{remark}


The \textbf{Hodge dual} of an operator \(P\) will be defined such that \(\langle P \alpha,
\beta \rangle_{L^2} = \langle \alpha, P^* \beta \rangle_{L^2}\) where the \(\langle
\cdot,\cdot\rangle_{L^2}\) is the integral of \(\langle \cdot,\cdot \rangle\) over \(M\). For example,

\begin{definition}
Let \(d\) be the coboundary operator then \(d^* : \Omega^{k}(M) \longrightarrow
\Omega^{k-1}(M)\) is defined by \(d^* = (-1)^{n(k-1)+1}*d*\)
\end{definition}

\begin{definition}
The \textbf{de Rham-Laplace} operator is defined by
\[
\Delta = dd^* + d^* d = (d+d^*)^2
\]
The space of \textbf{harmonic forms} is \(\mathcal{H}^k(M) = \{ \alpha  \in \Omega^k(M) : \Delta
\alpha = 0 \}\).
\end{definition}

\begin{remark}
\begin{enumerate}
\item \(\Delta^* = \Delta\).
\item \(\langle \Delta\alpha, \alpha \rangle  = \|d^*\alpha \|^2 + \|d\alpha\|^2\)
\item A harmonic form is closed and co-closed.
\end{enumerate}
\end{remark}


\subsection{Elliptic theory and Hodge theorem for Riemannian manifolds}
\label{sec:org32ac1ee}
\subsubsection{Symbol of a differential operator}
\label{sec:orgdad7a0c}
\begin{definition}
A mapping \(L: \Gamma(E) \longrightarrow \Gamma(F)\) where \(E,F\) are vector bundles
on a manifold \(M\) is called a \textbf{differential operator} of order \(k\) if in local coordinates,
\[
L(s) = \sum_{|\alpha|\leq k} A_\alpha(x) \frac{\partial^|\alpha| s}{\partial x^\alpha}
\]
where \(A_\alpha(x)\) is a matrix with \(C^\infty\) coefficients.

The \textbf{symbol} of \(L\) is \(\sigma_k(L,\xi) = \sum_\alpha A_{\alpha} \xi_1^{\alpha_1} \dots
\xi_n^{\alpha_n} \in Hom(E_x,F_x)\) where \(\xi = \sum \xi_i dx^i\in T^*M\) in the same coordinate as \(A_\alpha\). 
\end{definition}

\begin{remark}
\begin{enumerate}
\item \(A_\alpha (x)\) depends on the local coordinates and does not transform naturally
when one passes from one coordinates to another. In other words, \(A_\alpha(x)\) is
not in \(Hom(E_x, F_x)\).
\item However, the definition of differential operator does not depend on local coordinates.
\item The symbol transforms naturally (linearly) between coordinates.
\end{enumerate}
\end{remark}

From the third remark, one can define:

\begin{definition}
A differential operator \(L\) is called \textbf{elliptic} if its symbol \(L(x, \xi): E_x
\longrightarrow F_x\) is isomorphic.
\end{definition}


\subsubsection{Elliptic operators}
\label{sec:org3a9009e}
\begin{theorem}[Elliptic operator]
\label{thm:elliptic-operator}
Every elliptic operator \(L: \Gamma(E) \longrightarrow \Gamma(F)\) 
\begin{enumerate}
\item has a pseudoinverse, i.e. there exists \(P:\Gamma(F) \longrightarrow \Gamma(E)\) such
that \(L\circ P - id_{\Gamma(F)}\) and \(P\circ L - id_{\Gamma(E)}\) are smooth operators.
\item is extended to a Fredhom operator \(L_s : W^s(E) \longrightarrow W^{s-k}(F)\),
i.e. \(\ker L = \ker L_s\) and \(coker L_s\) are finite dimensional, \(\im L_s\)
is closed.
\end{enumerate}
Moreover, if \(L : \Gamma(E) \longrightarrow \Gamma(E)\) is elliptic and self-adjoint
then there exists \(H_L, G_L: \Gamma(E) \longrightarrow \Gamma(E)\) such that 
\begin{enumerate}
\item \(\im H_L\subset \ker L\), \(id_{\Gamma(E)} = H_L + L\circ G_L = H_L + G_L\circ L\).
\item \(H_L, G_L\) extend to \(W^s(E) \longrightarrow W^s(E)\).
\item \(\Gamma(E) = \ker L \oplus_{\perp L^2} \im L\circ G_L\).
\end{enumerate}
\end{theorem}

\begin{theorem}[Hodge]
\label{thm:hodge-riemann}
Let \(M\) be a compact, oriented Riemannian manifold, then
\begin{enumerate}
\item \(\Omega^k(M) = \mathcal{H}^k(M) \oplus_{\perp L^2} \im d \oplus_{\perp L^2} \im d^*\).
\item The projection \(\mathcal{H}^k(M) \longrightarrow H^k_{dR}(M, \mathbb{R})\) is
isomorphic. In other words, each class is uniquely represented by a harmonic form.
\end{enumerate}
\end{theorem}

\subsection{Hodge decomposition for Kähler manifolds}
\label{sec:orgbe02f20}
In case of Kähler manifolds, one has the Hodge decomposition of cohomology which comes
from the following two remarks:
\begin{enumerate}
\item The Hodge star \(*: \Omega^{p,q} \longrightarrow \Omega^{n-q, n-p}\). This is due to
the compatible complex structure.
\item The auxilary operator \(L: \alpha \longrightarrow \omega\wedge \alpha\) and its
relation with \(d\). This is due to the compatible symplectic structure.
\end{enumerate}

We resume in the following table the definition, domain and Hodge dual of some operators.


\begin{center}
\begin{tabular}{llll}
Operator & Domain & Definition & Dual\\
\hline
\(L\) & \(\Omega^{p,q} \longrightarrow \Omega^{p+1.q+1}\) & \(\alpha \mapsto \omega\wedge \alpha\) & \(L^* = (-1)^{p+q}*L*\)\\
\(d_c\) & \(\Omega^k \longrightarrow \Omega^{k+1}\) & \(J^{-1} d J\) & \(d_c^* = (-1)^{k+1}Jd^* J\)\\
\(\partial\) & \(\Omega^{p,q} \longrightarrow \Omega^{p+1, q}\) &  & \(\partial^* = -*\bar \partial*\)\\
\(\bar \partial\) & \(\Omega^{p,q} \longrightarrow \Omega^{p,q+1}\) &  & \(\bar \partial^* = -* \partial *\)\\
\(\Box\) & \(\Omega^{p,q} \longrightarrow \Omega^{p,q}\) & \(\partial \partial^* + \partial^* \partial\) & \\
\(\bar \Box\) & \(\Omega^{p,q} \longrightarrow \Omega^{p,q}\) & \(\bar\partial \bar\partial^* + \bar\partial^* \bar\partial\) & \\
\hline
\end{tabular}
\end{center}

In case of Kähler manifold, one has the following relation between these operators.

\begin{lemma}
\lem{lem:}
In a compact Kähler manifold, one has 
\begin{enumerate}
\item \([L,d] = [L^*, d^*] = 0\)
\item \([L, d^*] = d_c\)
\item \([L^*, d] = -d^*_c\)
\item \([L^*,d_c] =d^*\)
\end{enumerate}
Therefore,       
\begin{enumerate}
\item \(\Delta_c = d_c d^*_c + d^*_c d_c = \Delta\)
\item \(\partial^*\) is adjoint to \(\partial\) and \(\bar\partial^*\) to \(\bar\partial\).
\item \(\Delta = 2\Box = 2\bar\Box\)
\end{enumerate}
\end{lemma}

One equip \(\Omega^k\) with the following Hermitian product
\[
\langle \phi, \psi \rangle_{L^2} = \int_M \phi\wedge *\bar \psi 
\]
under which the \(\Omega^{p,q}\) are orthogonal.

One now applies the elliptic theory for \(\bar \Box: \Omega^{p,q} \longrightarrow \Omega^{p,q}\) with \(\mathcal{H}^{p,q}_{\bar \Box} = \ker \Box\) then one sees that
\begin{theorem}[Hodge decomposition]
\label{thm:hodge-decomp}
\begin{enumerate}
\item Each class in the Dolbeault cohomology \(H^{p,q}_{\bar \partial}(M)\) contains
exactly one harmonic form of \(\mathcal{H}^{p,q}_{\bar \Box} = \ker \bar\Box\)
\item \(H^k(M) = \mathcal{H}_\Delta = \bigoplus_{p+q=k} \mathcal{H}^{p,q}_{\bar\Box} = \bigoplus_{p+q=k} H^{p,q}_{\bar\partial}(M)\).
\end{enumerate}
\end{theorem}

\subsection{Hodge symmetries}
\label{sec:orgc6cef03}
Let \(h^{p,q} = \dim_{\mathbb{R}} H^{p,q}_{\bar \partial}(M)\) and \(h^k = \dim
   H^k_{dR}(M, \mathbb{R})\) then one has \(h^k = \sum_{p+q = k} h^{p,q}\). The \(h^{p,q}\) are usually written down as Hodge's diamond

\begin{center}
\begin{tabular}{llll}
\(h^{n,n}\) & \(h^{n, n-1}\) & \(\dots\) & \(h^{n,0}\)\\
\(h^{n-1,n}\) & \(h^{n-1,n-1}\) & \(\dots\) & \(h^{n-1,0}\)\\
\(\dots\) & \(\dots\) & \(\dots\) & \(\dots\)\\
\(h^{0,n}\) & \(h^{0, n-1}\) & \(\dots\) & \(h^{0,0}\)\\
\end{tabular}
\end{center}

with the symmetries
\begin{enumerate}
\item \(h^{p,q} = h^{q,p}\) given by conjugation.
\item \(h^{p,q} =h^{n-q, n-p}\) given by the Hodge star.
\end{enumerate}


\section{Kodaira embedding theorem}
\label{sec:orgef4301b}
Emacs 25.3.1 (Org mode 9.0.5)
\end{document}