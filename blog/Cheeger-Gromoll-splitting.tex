% Created 2017-06-14 Wed 12:51
% Intended LaTeX compiler: pdflatex
\documentclass[11pt]{article}
\usepackage[utf8]{inputenc}
\usepackage[T1]{fontenc}
\usepackage{graphicx}
\usepackage{grffile}
\usepackage{longtable}
\usepackage{wrapfig}
\usepackage{rotating}
\usepackage[normalem]{ulem}
\usepackage{amsmath}
\usepackage{textcomp}
\usepackage{amssymb}
\usepackage{capt-of}
\usepackage{hyperref}
\usepackage{amsthm}
\usepackage{tikz-cd}
\newtheorem{remark}{Remark}
\newtheorem{theorem}{Theorem}
\newtheorem{lemma}[theorem]{Lemma}
\newtheorem{corollary}{Corollary}[theorem]
\newtheorem{conjecture}[theorem]{Conjecture}
\newtheorem{proposition}{Proposition}[theorem]
\newtheorem{problem}{Problem}
\newtheorem{example}{Example}
\newtheorem{definition}{Definition}
\author{darknmt}
\date{\today}
\title{From Busemann function to Cheeger-Gromoll splitting}
\hypersetup{
 pdfauthor={darknmt},
 pdftitle={From Busemann function to Cheeger-Gromoll splitting},
 pdfkeywords={},
 pdfsubject={},
 pdfcreator={Emacs 25.2.1 (Org mode 9.0.5)}, 
 pdflang={English}}
\begin{document}

\maketitle
\tableofcontents

We will prove the following result by Cheeger and Gromoll by a slightly modified approach of
A. Besse.


\begin{theorem}[Cheeger-Gromoll]
\label{orgcf44fbf}
\label{thm:cheeger-gromoll}
Let \(M\) be a complete, connected Riemannian manifold with non negative Ricci curvature. Suppose
that \(M\) contains a line then \(M\) is isometric to \(M'\times \mathbb{R}\) with \(M'\) a complete,
connected Riemannian manifold with non negative Ricci curvature.
\end{theorem}

Note that the notion of \uline{geodesic ray} or \uline{geodesic line} used here is rather strict: A geodesic
line \(\gamma\) is a geodesic parameterized by \(\mathbb{R}\) such that the distance between two point is
exactly the distance \emph{on the geodesic}, for example, geodesic line, if it passes by \(p\in M\) with
velocity \(v\) of norm 1, satisfies is
\[
d(exp_p(tv), exp_p(-sv)) = s+t,\quad \forall s,t>0
\]


\section{Busemann function}
\label{sec:org3ace855}

Let \(\gamma\) be a geodesic ray. We construct the Busemann function \(b\) associated to the ray as
\[
b(x) = \lim_{t\to+\infty}f_t(x) = t - d(x,\gamma(t))
\]
where the limit exists because the sequence \(f_t\) is non-decreasing and bounded by
\(d(x,\gamma(0))\). The convergence is also uniform in every compact. 


In Euclidean space for example, the Busemann function is the orthogonal projection on \(\gamma\). We
will see that in a Riemannian manifold with non negative curvature, the Busemann function will serve
as a projection.


Now with a fixed \(x_0\in M\), the tangent vectors at \(x_0\) of the geodesics connecting \(x_0\) and
\(\gamma(t)\) is in the unit sphere of \(T_xM\), which is compact. Let \(X\) be a limit point of these
tangents vectors, we defined
\[
b_{X,t}(x) = b(x_0) + t - d(x, C_X(t))
\]
where \(C_X(t)\) is the geodesic flow starting at \(x_0\) with velocity \(X\).

\begin{remark}
\label{rem:3-calcul-cheeger-gromoll}
\label{org9362523}
\begin{enumerate}
\item From the construction of \(X\), one has \(b(x_0) + t = b(C_X(t))\), therefore \(b_{X,t} \leq b\) with
equality in \(x_0\). We say that \(b\) is supported by \(b_{X,t}\) at \(x_0\). In general a function \(f\) 
is \uline{supported} by \(g\) at \(x_0\) if \(f(x_0)=g(x_0)\) and \(f\geq g\) in a neighborhood of \(x_0\).
\item \(b_{X,t}\) is smooth and a computation in local coordinate gives \(\Delta b_{X,t} \geq -\frac{\dim
   M - 1}{t}\)
\item \(\|\nabla b_{X,t}\| = 1\)
\end{enumerate}
\end{remark}

The estimation given on the second point of Remark \ref{rem:3-calcul-cheeger-gromoll} is established
using Jacobi fields:

\begin{lemma}[]
The function \(f(x) = d(x,x_0)\) satisfies at a point \(x\) out of the cut-locus of \(x_0\):
\[
\nabla f(x) \leq \frac{n-1}{l}
\]
where \(n=\dim M, l = d(x,x_0) = f(x)\) in Riemannian manifold \(M\) with non-negative Ricci curvature.
\end{lemma}

\begin{proof}
Let \$N(t), \(0\leq t \leq l\) be the velocity of the geodesic \(\gamma\) from \(x_0\) to \(x\), and
\(E_1,\dots, E_{n-1},N\) be a parallel frame along \(\gamma\). Let \(J_i\) be the unique Jacobi
fields along \(\gamma\) with \(J_i(l) = E_i(l)\) and \(J_i(0)=0\) (existence and uniqueness of \(J_i\) is
due to the fact that \(x\) is not in the cut-locus).

Then basic manipulation of Jacobi fields gives (without the fact that curvature is non-negative):
\[
\Delta f(x) = \int_0^l \sum_{i=1}^{n-1}<\nabla_N J_i, \nabla_N J_i> - < R(N,J_i)J_i,N > = \sum_{i=1}^{n-1} I_\gamma(J_i,J_i)
\]
where \(I_\gamma\) is the index form of \(\gamma\). Note that the Jacobi fields \(J_i\) coincide with the
fields \(\frac{t}{l}E(t)\) at \(0\) and \(l\), therefore by the \emph{fundamental inequality} of index form
\[
I_\gamma(J_i,J_i)\leq I_\gamma(\frac{t}{l}E_i,\frac{t}{l}E_i)
\]
hence
\[
\Delta f(x) \leq \int_0^l \sum_{i=1}^{n-1}<\nabla_N \frac{t}{l}E_i, \nabla_N \frac{t}{l}E_i> -
< R(N,\frac{t}{l}E_i)\frac{t}{l}E_i,N >
\]

The curvature term being \(\frac{t^2}{l^2}Ric(N,N)\) hence non-negative, one has
\[
\Delta f(x) \leq \int_0^l \sum_{i=1}^{n-1}<\nabla_N \frac{t}{l}E_i, \nabla_N \frac{t}{l}E_i> = \frac{n-1}{l}
\]
\end{proof}


We also note that it suffices to show that \(b\) is harmonic. In fact, from the smoothness one has
\(\nabla b(x_0) =\nabla b_{X,t}(x_0)\), which means \(\|\nabla b\| = 1\) at every point in \(M\). 
For each point \(y\in M\), there exists a unique \(x\) with \(b(x)=0\) and time \(t\) when the flow of
\(\nabla b\) arrive at \(x\). \(M\) is therefore homeomorphic to \(\bar M\times \mathbb{R}\) by the map
\(F: y\mapsto (x,t)\). We prove that in order that \(F\) is an isometry, it suffices to prove that the gradient field
\(\nabla b\) is parallel. In fact, \(\bar M\) being equiped with the restriction of the metric on \(M\),
the fact that \(F\) is isometric is equivalent to the the fact that the flow \(\Phi^t\) of \(\nabla b\) is
isometric for every time \(t\), which means \(\frac{d}{dt} <\Phi_*^t u, \Phi_*^t u>\) vanishes at
\(t=0\). But
\[
\frac{d}{dt}<\Phi_*^t u,\Phi_*^t u> = 2 <\nabla_{\partial t}\Phi_*^t u, u>|_{t=0} = 2<\nabla_u
\nabla b,u> 
\]
where for the second equality we used Schwarz lemma for \(\Phi(t,x) = \Phi^t(x)\). The vanishing of
\(<\nabla_u \nabla b(x),u>\) for every vector \(u\) is, by bilinearity, equivalent to that of \(\nabla_u
\nabla b\) for every \(u\), meaning that \(\nabla b\) is parallel. 

The fact that \(\nabla b\) is parallel is due to a simple computation:
\[
Ric(N,N) = -N(\Delta b) - \|\nabla N \|^2
\]
where \(\|\nabla N\|^2 = \sum_{i=1}^{n-1}<\nabla_{E_i}N, E_j>^2\). We see that \(N = \nabla b\) is parallel if \(\Delta b =0\).

\begin{remark}
\begin{enumerate}
\item One can show (see A. Besse) that every gradient field \(\nabla b\) of norm 1 at every point is actually harmonic.
\item Using de Rham decomposition, one has directly the splitting of \(M\) since \(N\) is parallel and \(M\) is complete.
\end{enumerate}
\end{remark}

\section{Harmonicity}
\label{sec:org953a134}
The Busemann function associated to a geodesic ray is subharmonic, it is a consequence of the
following lemma.


\begin{lemma}
\label{lem:1}%
In a connected Riemannian manifold, if a continuous function \(f\) is supported at any point \(x\) by
a family \(f_\epsilon\) (depending on \(x\)) with \(\Delta(f_\epsilon)\leq \epsilon\), then \(f\) can not
attains its maximum (unless when \(f\) is constant).
\end{lemma}

\begin{proof}
Given a small geodesic ball \(B\), suppose that we have a function \(h\) on \(B\) with \(\Delta h <0\) in \(B\) and
\(f+h\) attains maximum at \(x\) in the interior of \(B\). Then \(f_\epsilon + h\) also attains maximum at
\(x\), which means \(\Delta f_\epsilon + \Delta h \geq 0\), which is contradictory.

For the construction of the function \(h\), one suppose that \(B\) is small enough such that
\(f|_{\partial B} \leq max_B f=: f(x_0)\) and equality is not attained at every points in \(\partial B\). Then
choose
\[
h = \eta (e^{\alpha \phi} - 1)
\]
with and \(\phi(x) = -1\) if \(x\in \partial B\) and \(f(x) = f(x_0)\), \(\phi(x_0) = 0\),
\(\nabla \phi \ne 0\) and a large  \(\alpha\) such that
\[
\Delta h = \eta (-\alpha^2\| \nabla \phi\| + \alpha \Delta \phi)e^{\alpha \phi}.
\]
is negative.
\end{proof}


Now for subharmonicity of \(b\), given a harmonic function \(h\) that coincides with \(b\) in the boundary \(\partial B\) of a
geodesic ball \(B\), then \(b-h\) is supported by \(b_{X,t} - h\) with \(\Delta (b_{X,t}-h) \to 0\) as \(t\)
tends to \(+\infty\), therefore \(b-h \leq (b-h)|_{\partial B} = 0\) in \(B\). hence \(b\) is subharmonic.

\begin{corollary}
The Busemann function of a geodesic ray in a Riemannian manifold \(M\) with non-negative Ricci
curvature is subharmonic.
\end{corollary}


Now let \(b_+\) be the function previously constructed for the ray \(\gamma|_{[0,+\infty[}\) and \(b_-\)
the Busemann function for the ray \(\tilde\gamma|_{[0,+\infty[}\) where \(\tilde\gamma(t) =
\gamma(-t)\). Note that \(b_+ + b_-\leq 0\) with equality on the line \(\gamma\), but the sum is
subharmonic therefore by maximum principle \(b_+ = -b_-\) and \(b\) is harmonic hence smooth. The
splitting theorem of Cheeger-Gromoll follows.


\section{Application}
\label{sec:org0721a48}
Theorem \ref{orgcf44fbf} gives the following result from J.Cheeger- D.Gromoll, \emph{The splitting
theorem for manifold of nonnegative Ricci curvature} (Theorem 2)
\begin{theorem}[]
\label{thm:decomp-Ricci-non-negative}
\label{org9a6a6a5}
Let \(M\) be a compact Riemannian manifold with non-negative Ricci curvature, then the universal
covering of \(M\) is of form \(\mathcal{M} = \mathbb{R}^n\times \bar M\) where \(\bar M\) does not contain
any lines. Then \(\bar M\) is compact.
\end{theorem}

\begin{proof}
It suffices to prove that if \(\bar M\) is not compact, then it contains a line. In fact, it is easy
to see that such \(\bar M\) must contains a (strict) geodesic ray. In fact it is obvious that with a fixed \(p\in
M\) the function
\[
F: v\mapsto \inf \{t>0: d(p,exp_p(tv)) < t\}
\]
defined on the unit ball \(U_p\) of \(T_p\bar M\) is upper semi-continuous. Therefore if \(F(v)<\infty\)
for all \(v\) unit tangent vector at \(p\) then \(F\) is bounded in \(U_p\) by a constant \(c\). Therefore
\(\bar M\subset exp_p(cU_p)\) which is compact (contradiction). Therefore there exists a minimal ray
at every point \(p\in \bar M\).

The existence of a line in general might not be true, the only extra property of \(\bar M\) that we
will need is that it has a (fundamental) domain \(K\) compact and a family \(\sigma_i\) of isometries
such that \(\bar M = \cup_i \sigma_i K\).

Let us first prove that such domain \(K\) exists. Remark that every isometry of \(\mathcal{M}\) acts
separately on \(\mathcal{M}\), i.e. of form \(\sigma(u) = (\sigma_1(x), \sigma_2(y))\) for \(u=(x,y)\in
\mathcal{M}\) with \(g_1, g_2\) isometries of \(\mathbb{R}^n\) and \(\bar M\). This can be seen by the
uniqueness part of \href{de-rham-decomposition.org}{de Rham decomposition} or simply by noticing that a tangent vector in the \(T_x\mathbb{R}^n\) component is characterized by the fact that its geodesics is a
line. Let \(G_1\) (resp. \(G_2\)) be the group formed by \(\sigma_1\) (resp. \(\sigma_2\)) the \(\mathbb{R}^n\)
(resp. \(\bar M\) of \(\sigma = (\sigma_1,\sigma_2)\in \pi_1(M)\) that acts on \(\mathcal{M}\) we have a
surjection \(M = \mathcal{M}/\pi_1 \longrightarrow \mathbb{R}^n/G_1\times \bar M/G_2\), meaning that
\(M/G_2\) is compact. Such domain \(K\) can be chosen, for example, as a ball in \(\bar M\) large enough that its
image in the quotient \(M/G_2\) contains every equivalent classes.

Now let \(\gamma\) be a minimal ray starting from \(p\in M\), for each \(x\in \gamma\) there exists an
isometry \(\sigma\) of \(\bar M\) such that \(\sigma(x)\in K\). By compactness of \(K\), there exists a
sequence \(t_n \to +\infty\) with \(x_n =\gamma(t_n)\), \(v_n = \dot\gamma(t_n)\) that satisfies \(y_n =
\sigma_n(x_n) \to y\in K\) and even more, \((\sigma_n)_* v_n \to v\in T_y\bar M\) in the tangent bundle \(T\bar
M\). Then the geodesic of \(\bar M\) starting at \(y\) with vector \(v\) is a
line. In fact it suffices to prove that \(d(exp_y(t v), exp_y(-s v)) = s+t\) for \(s,t>0\): for \(n\)
large enough that \(t_n > s\) then
\[
d(exp_{y_n}(tv_n), exp_{y_n}(-sv_n)) = s+t
\]
then let \(n\to +\infty\) and the result follows.
\end{proof}
\end{document}