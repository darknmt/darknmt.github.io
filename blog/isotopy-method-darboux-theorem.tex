% Created 2017-12-16 Sat 01:08
% Intended LaTeX compiler: pdflatex
\documentclass[11pt]{article}
\usepackage[utf8]{inputenc}
\usepackage[T1]{fontenc}
\usepackage{graphicx}
\usepackage{grffile}
\usepackage{longtable}
\usepackage{wrapfig}
\usepackage{rotating}
\usepackage[normalem]{ulem}
\usepackage{amsmath}
\usepackage{textcomp}
\usepackage{amssymb}
\usepackage{capt-of}
\usepackage{hyperref}
\usepackage{amsthm}
\usepackage{amsmath,amscd}
\usepackage{tikz-cd}
\newtheorem{remark}{Remark}
\newtheorem{theorem}{Theorem}
\newtheorem{lemma}[theorem]{Lemma}
\newtheorem{corollary}{Corollary}[theorem]
\newtheorem{conjecture}[theorem]{Conjecture}
\newtheorem{proposition}{Proposition}[theorem]
\newtheorem{problem}{Problem}
\newtheorem{exampl}{Example}
\newtheorem{definition}{Definition}
\newtheorem{propdef}[definition]{Proposition-Definition}
\newcommand{\im}{\mathop{\rm Im}\nolimits}
\newcommand{\supp}{\mathop{\rm supp}\nolimits}
\newcommand{\ord}{\mathop{\rm ord}\nolimits}
\newcommand{\Spec}{\mathop{\rm Spec}\nolimits}
\newcommand{\vol}{\mathop{\rm vol}\nolimits}
\newcommand\restr[2]{{% we make the whole thing an ordinary symbol
\left.\kern-\nulldelimiterspace % automatically resize the bar with \right
#1 % the function
\vphantom{\big|} % pretend it's a little taller at normal size
\right|_{#2} % this is the delimiter
}}
\author{darknmt}
\date{\today}
\title{Moser's Isotopy method and Darboux theorem}
\hypersetup{
 pdfauthor={darknmt},
 pdftitle={Moser's Isotopy method and Darboux theorem},
 pdfkeywords={},
 pdfsubject={},
 pdfcreator={Emacs 25.3.1 (Org mode 9.0.5)}, 
 pdflang={English}}
\begin{document}

\maketitle
\tableofcontents

\iffalse
\begin{info}
The PDF version of this page can be downloaded by replacing \texttt{html} in the its address by
\texttt{pdf}. 
For example \texttt{/html/sheaf-cohomology.html} should become \texttt{/pdf/sheaf-cohomology.pdf}.
\end{info}
\fi

\iffalse
\begin{info}
This post is a part of the \href{../res/Stage2017.pdf}{memoire of my M1 internship} at I2M. The memoire contains,
needless to say, less errors than this page.
\end{info}
\fi

\section{Symplectic geometry does not exist}
\label{sec:org3316df4}
We will prove a symplectic manifold, i.e. a smooth manifold equiped with a closed
everywhere non-degenerate 2-form, \uline{does not have local invariant}. This is a significant
difference between symplectic manifold and riemannian manifold, whose local invariance is
the \uline{curvature}.

To do this, one uses a trick of Moser which in the compact case show that isotopic
symplectic structures \(\omega_0\) and \(\omega_1\) are strongly isotopic, i.e. \(\psi_t^* \omega_t = \omega_0\)
\section{Isotopy method}
\label{sec:orgeb2e2ed}
\subsection{Moser's trick}
\label{sec:org5832dcd}
Let \(M\) be a closed manifold (compact, without boundary) and \(\omega_t\) is a family of
symplectic structures on \(M\) such that
\[
\frac{d}{dt}\omega_t = d\sigma_t
\]
then there exists diffeomorphism \(\psi_t\) of \(M\) such that \(\psi_t^* \omega_t = \omega_0\)


\paragraph{Construction of \(\psi_t\).}
\label{sec:org551d75e}
One constructs \(\psi_t\) by its flow \(\frac{d}{dt}\psi_t = X_t \circ \psi_t\) such that
\[
0 = \frac{d}{dt}\psi_t^*\omega_t = \psi_t^* \left(\frac{d}{dt} \omega_t +
\mathcal{L}_{X_t}\omega_t\right) = \psi_t^* \left( d\sigma_t + X_t \neg d\omega_t + d(X_t
\neg \omega_t) \right)
\]

Since \(\omega_t\) are closed and non-degenerate, it suffits to choose \(X_t\), which exists
uniquely, such that \(X_t \neg \omega = \sigma_t\).

\subsection{Application: Darboux theorem and Moser Stability}
\label{sec:orgb93185e}
Using this trick, we can prove the following theorem of Darboux.

\begin{lemma}
\label{lem1}
Let \(M\) be a closed manifold with symplectic structures \(\omega_0\) and \(\omega_1\) such
that they coincide on a fiber \(T_qM\). Then there exists neighborhoods \(\mathcal{N}_0,
\mathcal{N}_1\) of \(q\) and a diffeomorphism \(\psi: \mathcal{N}_0 \longrightarrow
\mathcal{N}_1\) such that \(\psi^* \omega_1 = \omega_0\). 
\end{lemma}


\begin{proof}
We remark that it is enough to prove that there exists \(\sigma\) locally defined near \(q\)
with \(\omega_1 -\omega_0 = d\sigma\) where \(\sigma = 0\) on \(T_qM\). In fact, let \(\omega_t =
\omega_0 + t(\omega_1 - \omega_0)\) one then has a neighborhood \(\mathcal{N}_0\) of \(q\)
such that \(\omega_t\) are non-degenerate and the field \(X_t\) constructed by Moser's
technique (\(X_t=0\) at \(q\)) has its flow well-defined at time \(t=1\) when starting at \(\mathcal{N}_0\). Then \(\psi_1\) and \(\mathcal{N}_1\) is what we want.

One then uses \uline{another trick} to construct \(\sigma\): Take any Riemannian metric on \(M\) and
let \(\phi_t\) be constructed using the geodesic flow and retricting \(\mathcal{N}_0\) to a
geodesic ball such that \(\phi_0|_{\mathcal{N}_0} = q\), \(\phi_1 = Id\) and \(d\phi_t (q) =
Id_{T_qM}\). Then

\[
\omega_1 - \omega_0 = \int_0^1 \frac{d}{dt}\phi^*_t(\omega_1 - \omega_0) dt = \int_0^1
\phi_t^* d(Y_t \neg(\omega_1 - \omega_0) )= d \left( \int_0^1 dt \phi_t^*(Y_t \neg
(\omega_1 - \omega_0) \right).
\]

It is straight-forward to see that the \(\sigma\) constructed this way works.
\end{proof}

The theorem of Darboux follows easily from Lemma \ref{lem1}.

\begin{theorem}[Darboux]
Every two symplectic form \(\omega_0, \omega_1\) on a closed manifold \(M\) are locally isomorphic.
\end{theorem}
Emacs 25.3.1 (Org mode 9.0.5)
\end{document}