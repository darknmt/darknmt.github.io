% Created 2017-10-28 Sat 18:27
% Intended LaTeX compiler: pdflatex
\documentclass[11pt]{article}
\usepackage[utf8]{inputenc}
\usepackage[T1]{fontenc}
\usepackage{graphicx}
\usepackage{grffile}
\usepackage{longtable}
\usepackage{wrapfig}
\usepackage{rotating}
\usepackage[normalem]{ulem}
\usepackage{amsmath}
\usepackage{textcomp}
\usepackage{amssymb}
\usepackage{capt-of}
\usepackage{hyperref}
\usepackage{amsthm}
\usepackage{amsmath}
\usepackage{tikz-cd}
\newtheorem{remark}{Remark}
\newtheorem{theorem}{Theorem}
\newtheorem{lemma}[theorem]{Lemma}
\newtheorem{corollary}{Corollary}[theorem]
\newtheorem{conjecture}[theorem]{Conjecture}
\newtheorem{proposition}{Proposition}[theorem]
\newtheorem{problem}{Problem}
\newtheorem{exampl}{Example}
\newtheorem{definition}{Definition}
\newtheorem{propdef}[definition]{Proposition-Definition}
\newcommand{\im}{\mathop{\rm Im}\nolimits}
\newcommand{\supp}{\mathop{\rm supp}\nolimits}
\newcommand{\ord}{\mathop{\rm ord}\nolimits}
\newcommand{\Spec}{\mathop{\rm Spec}\nolimits}
\newcommand{\vol}{\mathop{\rm vol}\nolimits}
\newcommand\restr[2]{{% we make the whole thing an ordinary symbol
\left.\kern-\nulldelimiterspace % automatically resize the bar with \right
#1 % the function
\vphantom{\big|} % pretend it's a little taller at normal size
\right|_{#2} % this is the delimiter
}}
\author{darknmt}
\date{\today}
\title{Local results of several complex variables}
\hypersetup{
 pdfauthor={darknmt},
 pdftitle={Local results of several complex variables},
 pdfkeywords={},
 pdfsubject={},
 pdfcreator={Emacs 25.3.1 (Org mode 9.0.5)}, 
 pdflang={English}}
\begin{document}

\maketitle
\tableofcontents

\iffalse
\begin{info}
The PDF version of this page can be downloaded by replacing \texttt{html} in the its address by
\texttt{pdf}. 
For example \texttt{/html/sheaf-cohomology.html} should become \texttt{/pdf/sheaf-cohomology.pdf}.
\end{info}
\fi

\section{Subharmonic and Plurisubharmonic functions}
\label{sec:orgbf45ff7}

Some properties of holomorphic functions that remain in several variables.
\begin{itemize}
\item Cauchy formula
\item Analyticity: series development. Therefore its zeroes never form an open set (except for constant)
\item Maximum modulus
\item Cauchy inequality and Montel's theorem
\end{itemize}


\subsection{Subharmonic functions}
\label{sec:orge070dfc}
We are now in the context of \(\mathbb{R}^n\).

\begin{theorem}[Green kernel]
\label{thm:green-kernel}
Let \(\Omega \Subset \mathbb{R}^n\) be a smoothly bounded domain, then there exists
uniquely a function \(G_\Omega :\bar\Omega\times\bar\Omega \longrightarrow  [-\infty, 0]\), called \uline{the Green kernel} of \(\Omega\), with the following properties:
\begin{enumerate}
\item Regular: \(G_\Omega\) is \(C^\infty\) on \(\bar\Omega\times\bar\Omega\setminus
   \Delta_\Omega\) where \(\Delta_\Omega\) denotes the diagonal,
\item Symetric: \(G_\Omega(x,y) = G_\Omega(y,x)\),
\item Negative: \(G_\Omega(x,y) <0\) on \(\Omega\times\Omega\) and \(G_\Omega(x,y) = 0\) on \(\partial\Omega\times \Omega\),
\item \(\Delta_x G_\Omega(x,y) = \delta_y\) on \(\Omega\) for every \(y\in \Omega\).
\end{enumerate}
\end{theorem}
aa
\begin{exampl}[case \( \Omega = B(0,r) \)]
One can take \(G_{r}=N(x-y) - N(\frac{|y|}{r}(x-\frac{r^2}{|y|^2}y))\) where \(N\) is
the Newton kernel (or \href{https://en.wikipedia.org/wiki/Newtonian\_potential}{Newtonian potential}, the gravitational potential). Explicitly, one
has
\begin{align}
G_r(x,y) &= \frac{1}{4\pi}\log \frac{|x-y|^2}{r^2 -2 \langle x,y \rangle
+\frac{1}{r^2}|x|^2|y|^2} &\text{ if } n=2\\
G_r(x,y) &=\frac{-1}{(m-2)\vol(S^{m-1})}(|x-y|^{2-m} - (r^2 - 2 \langle x,y \rangle  + \frac{1}{r^2}|x|^2|y|^2)^{1-m/2}) &\text{ if } n\geq 3
\end{align}
\end{exampl}

\begin{proposition}[Green-Riesz representation]
For \(u\in C^2(\bar \Omega, \mathbb{R})\) one has
\[
u(x) = \int_{\Omega}G_\Omega(x,y) \Delta u(y) d\lambda(y) + \int_{\partial\Omega} u(y)
\frac{\partial G_\Omega}{\partial \nu_y} d\sigma(y)
\]
In particular, for \(\Omega = B(0,r)\), one has 
\[ 
P_r(x,y) := \frac{\partial G}{\partial
\nu_y} =\frac{1}{\vol(S^{m-1})r}\frac{r^2 - |x|^2}{|x-y|^m}
\]
called the \uline{Poisson kernel}.
\end{proposition}
\begin{proof}
Use the Green-Riesz formula: \(\int_\Omega u\Delta v - v\Delta u = \int_{\partial
\Omega}u \frac{\partial v}{\partial \nu} - v \frac{\partial u}{\partial \nu}\).
\end{proof}

\begin{definition}
Let \(\Omega\subset \mathbb{R}^n\) be an open subset and \(u: \Omega \longrightarrow
[-\infty,\infty)\) a upper semi-continuous function: \[ \limsup_{x \to x_0}u(x) \leq
u(x_0) \] 
One notes by \(\mu_S(u,a,r)\) and \(\mu_B(u,a,r)\) the average of \(u\) in the
sphere and the disk centered in \(a\) of radius \(r\). Then the following properties are equivalent and a function is called
\uline{subharmonic} if they are verified.
\begin{description}
\item[{1)}] \(u(x) \leq P_{a,r}[u](x) \quad \forall a,r, x\in B(a,r)\subset \Omega\),
\item[{2)}] \(u(a) \leq \mu_S(u,a,r)\quad \forall B(a,r)\subset\Omega\),
\item[{2')}] \(u(a) \leq \mu_S(u,a,r)\quad \text{ for } B(a,r_n)\subset\Omega, r_n\to 0\),
\item[{3)}] \(u(a) \leq \mu_B(u,a,r)\quad \forall B(a,r)\subset \Omega\),
\item[{3')}] \(u(a) \leq \mu_B(u,a,r)\quad \text{ for } B(a,r_n)\subset \Omega, r_n\to 0\),
\item[{4)}] If \(u\in C^2\), then \(\Delta u \geq 0\).
\end{description}
The convex cone of subharmonic functions on a domain \(\Omega\) is denoted by \(Sh(\Omega)\).
\end{definition}

\begin{proof}
It is obvious that \((1) \to (2) \to (3) \to (3')\to (2')\). To prove \((2')\to (1)\)
one needs the following 2 facts:

\begin{lemma}[u.s.c function as limit of continuous functions]
Let \(u\) be a u.s.c. function on a compact metric space \(X\), then there exists a
sequence \(u_n\) continuous function on \(X\) that decreases to \(u\) pointwise.
\end{lemma}
\textbf{Proof.} Let \(\tilde u_k(x) = \max \{u(x), -k \}\) to exclude the \(-\infty\) points. Then \(v_k(x) = \sup_{y\in X} \left(u(y) - kd(x,y)\right)\) works.


\begin{lemma}
\((2')\) implies strict maximum principle (see \ref{prop:subhar}).
\end{lemma}
\textbf{Proof.} By restriction to smaller neighborhood, one can suppose that \(u\) attains global
maximum at \(x_0\) in \(\Omega\). Then \(W=\{x\in \Omega:\ u(x) < u(x_0)\}\) is an
open set, and has a point \(y\) in its boundary if \(W\) nonempty. Then \((2')\) is
not satisfied at \(y\) since the measure of open arc is nonzero.


Note that if \(u\) is continue than \((2') \to (1)\): Let \(h = P_{a,r}[u]\)
harmonic then \(u-h\) satisfies \((2')\), therefore the maximum principle, hence \(u-h\leq \restr{(u-h)}{S(a,r)} = 0\).

If \(u\) is u.s.c, take a sequence \(v_k\) continuous that decreases to \(u\) and
let \(h_k = P_{a,r}[v_k]\) then \(h_k\geq v_k\geq u\) and \(h_k\to P_{a,r}[u]\) by
monotone convergence.
\end{proof}







\begin{proposition}[]
\label{prop:subhar}
Let \(u\in Sh(\Omega)\) then
\begin{description}
\item[{(Strict) maximum principle.}] \(u\) cannot attain local maximum unless it is constant in the corresponding connected component,
\item[{Locally integrable.}] \(u\) is \(L^1_{loc}\) on each connected component where \(u\not\equiv -\infty\),
\item[{Pointwise decreasing limit}] The pointwise limit \(u\) of a decreasing sequence \(u_k\) of
subharmonic functions is also subharmonic.
\item[{Regularisation.}] \(\mu_S(u,a,\varepsilon),\mu_B(u,a,\varepsilon),\rho_\varepsilon
                    \ast u\) increase in \(\varepsilon\). Moreover, \(\rho_\varepsilon * u \in
                    Sh(\Omega)\) and decreases to \(u\) pointwise as \(\varepsilon \to
                    0\).
\end{description}
Moreover, for \(u\in \mathcal{D}'(\Omega)\)
\begin{description}
\item[{Positive measure.}] \(u\in Sh(\Omega)\) iff \(\Delta u \geq 0\) is a positive measure.
\end{description}
\end{proposition}

\begin{proof}
\begin{description}
\item[{Locally integrable.}] To see that \(u\in L^1_{loc}(\Omega)\) if \(\Omega\) is
connected and \(u\not\equiv -\infty\), let \(x\) be a point in the boundary of \(W=\{y\in \Omega:\ u\text{ integrable in neighborhood of } y\}\), then apply mean value property in \(a\in W\) such that \(x\in B(a,r)\).
\item[{Pointwise decreasing limit.}] Infimum of a family of u.s.c functions is still u.s.c. The
mean value property comes from monotone convergence.
\item[{Regularisation.}] Check first for \(C^2\) functions, then regularise. One uses the
following Gauss formula: 
\[
     \mu_S(u,a,r) = u(a) + \frac{1}{n}\int_0^r\mu_B(\Delta u, a, t)tdt
     \]
to see that \(\mu_S\) is increasing in \(r\) and
\[
     \mu_B(u,a,r) = m\int_0^1 t^{m-1}\mu_S(u,a,rt)dt
     \]
to see that \(\mu_B\) is increasing. For the convolution, use
\[
     u*\rho_\varepsilon = \vol(S^{n-1})\int_0^1 \mu_S(u,a,\varepsilon t)\rho(t) t^{m-1}dt.
     \]
\item[{Positive measure.}] \(\Delta u * \rho_\varepsilon \geq 0\) as function, therefore the
limit \(\geq 0\) as measure (dominated convergence).
\end{description}
\end{proof}

\begin{proposition}[new harmonic functions from old ones]
let \(u_k \in Sh(\Omega)\) then
\begin{enumerate}
\item If \(\{u_k\}\) decrease to \(u\) then \(u\in Sh(\Omega)\).
\item Let \(\chi\) be a convex function, non-decreasing in each variable then \(\chi(u_1,\dots,u_p) \in Sh(\Omega)\). Therefore, \(\sum u_i\) and \(\max\{u_i\}\)
are subharmonic.
\end{enumerate}
\end{proposition}

\begin{proposition}[Upper regularization]
\label{prop:upper-regularization}
\begin{enumerate}
\item Let \(u\) be a real function on \(\Omega\) then \(u^*(x) = \lim_{\varepsilon\to 0}
   \sup_{x+\varepsilon B} u\), called the \uline{upper envelope} of \(u\) is u.s.c and is in
fact the smallest u.s.c function greater than \(u\).
\item \textbf{Choquet lemma.} Let \(\{u_\alpha\}\) be a family of real function, one defines the
\uline{upper regularization} of \(\{ u_\alpha\}\) by \(u^*\) where \(u=\sup_\alpha u_\alpha\). Then from every such family, on can always find a countable subfamily \(\{v_i\}\)
such that \(u^* = v^*\).
\item If \(\{u_{\alpha}\} \subset Sh(\Omega)\) then \(u^* = u\) a.e. and \(u^*\in Sh(\Omega)\).
\end{enumerate}
\end{proposition}
\begin{proof}
\begin{enumerate}
\item Obvious.
\item Let \(B_i\) be a countable base of the topology and \(x_{i,j}\) be a sequence such
that \(u(x_{ij}) \to \sup_{B_i}u\). Let \(\{ u_{i,j,k} \}\) be a countable subfamily
such that \(u_{ijk}(x_i) \to u(x_i)\) then it is a suitable subfamily.
\item WLOG, suppose that \(\{u_\alpha\} = \{u_i\}\) countable then \(u\) satisfies the
submean value property: \(u(z)\leq \mu_B(u,z,r)\). By the continuity of \(\mu_B(u,z,r)\) one has \(u^*(z)\leq \mu_B(u,z,r)\leq \mu(u^*,z,r)\) therefore \(u^*\in
   Sh(\Omega)\) and \(u^*(z) = \lim_{r\to 0} \mu_B(u^*,z,r) = \lim_{r\to 0}\mu_B(u,z,r)\), from which \(u=u^*\) a.e.
\end{enumerate}
\end{proof}


\section{de Rham currents}
\label{sec:org5130693}

Let \(M\) be a differential \(m\)-dimensional manifold and \(\mathcal{E}^p(M)\) be
the vector space of smooth \(p\)-forms on \(M\) and \(\mathcal{D}^p(M)\) be the
space of those with compact support. Then \(\mathcal{E}^p(M), \mathcal{D}^p(M)\) is a topological vector
space with the pseudonorms \(p_{K,\Omega}^s(\omega) = \max_{K, |\alpha|\leq s}|D^\alpha
  u_I|\) where \(K\Subset \Omega\) an coordinated open set. The \uline{space of de Rham
current} with dimension \(p\) / \uline{degree} \(m - p\) is defined as the dual space of
\(\mathcal{D}^p(M)\), denoted by \(\mathcal{D}'^{m-p}(M)\) or \(D'_{p}(M)\)

\begin{remark}
\begin{enumerate}
\item We are still in \(\mathbb{R}\), but the definition expands to the complex case,
denoted by \(\mathcal{D}'^{m-p, m-q}(M) = \mathcal{D}'_{p,q}(M)\) where \(m\) is
the complex dimension of \(M\).
\item The degree is defined such that the current \(T_\omega: \eta \mapsto \int_M
   \omega\wedge\eta\) is of the same degree as \(\omega\). The dimension is defined so
that the current \(T_{[Z]}: \eta \mapsto \int_Z \eta\) is of the same dimension as \(Z\).
\end{enumerate}
\end{remark}

\begin{definition}
One has the following operation on \(\mathcal{D}'^{m-p}(M)\):
\begin{enumerate}
\item \textbf{Derivative:} \(\langle dT, \omega \rangle  = (-1)^{\deg{T}} \langle T, d\omega \rangle\)
\item \textbf{Wedge product with a form:} \(\langle T\wedge \eta,\omega \rangle  = \langle T, \eta \wedge
   \omega \rangle\)
\item \textbf{Pushforward:} If \(F: X \longrightarrow Y\) proper on \(\supp T\) then \(\langle
   F_*T, \omega \rangle = \langle T, F^*\omega \rangle = \langle T, \chi F_* \omega
   \rangle\) where \(\chi \in C^{\infty}(M)\) identically 1 on \(\supp T\). The
proper condition is such that the pullback of \(\omega\) is compactly support in \(\supp T\)
\item \textbf{Pullback:} Let \(F: X \longrightarrow Y\) submersion then the pushforward of a form
on \(X\) is well-defined by Fubini. One has \(\langle F^* T, \omega \rangle =
   \langle T, F_* \omega \rangle\)
\end{enumerate}
\end{definition}

\begin{remark}
\begin{enumerate}
\item The sign of derivative is chosen so that \(dT_\omega = T_{d\omega}\).
\item Pushforward keeps the dimension, as the arguments are of the same degree.
\item Pullback keeps the codimension, meaning the degree (think \(F^* T_{[Z]} =
   T_{[F^{-1}(Z)]}\)).
\item Locally a current is of form \(T = \sum u_I dx^I\) where \(u_I\) are
distribution. \textbf{Note:} Here distribution are indentified as a current of maximal degree
and \uline{not} zero degree as they naturally are. To be exact, the notation of \(u_I\) is
contravariant and its action is \(\varphi dx^1\wedge\dots\wedge dx^N \mapsto \langle
   u_I, \varphi \rangle dx^1\wedge\dots\wedge dx^n/\text{vol}\) where \(\text{vol}\) is
a canonical volume form.
\end{enumerate}
\end{remark}

The last two remarks explain the sign in the following proposition.

\begin{proposition}[Pushforward and Pullback]
Let \(F: M_1 \longrightarrow M_2\), submersion if needed, then
\begin{enumerate}
\item \(\supp F_* T \subset F(\supp T)\)
\item \(d(F_* T) = F_* dT\) (pushforward of a form is still that form)
\item \(F_*(T\wedge F^* g) = (F_*T)\wedge g\)
\end{enumerate}
and
\begin{enumerate}
\item \(F^*(dT) = (-1)^{m_1-m_2} d(F^* T)\)
\item \(F^*(T\wedge g) = (-1)^{m_1-m_2 -\deg g}(F^* T)\wedge F^* g\)
\end{enumerate}
\end{proposition}






\section{Resolution of \(\bar\partial\), Dolbeault-Grothendieck lemma}
\label{sec:org1c81c81}
The generalized Cauchy formula for several variables is the following (the formula in
\href{https://en.wikipedia.org/wiki/Bochner–Martinelli\_formula}{wikipedia} is \(K^{0,0}_{BM}\))

\begin{theorem}[Bochner–Martinelli-Koppelman formula]
\label{thm:koppelman}
The \uline{Bochner-Martinelli kernel} is the following \((n,n-1)\)-form on \(\mathbb{C}^n\)
\[
k_{BM} = (-1)^{n(n-1)/2}\frac{(n-1)!}{(2\pi i)^n}\sum_{1\leq j\leq n} (-1)^j \frac{\bar
z_j dz_1\wedge \dots\wedge dz_n\wedge d\bar z_1\wedge\dots \wedge \widehat{d\bar z_j} \wedge \dots \wedge d\bar z_n}{|z|^{2n}}
\]
then \(\bar \partial k_{BM} = \delta_0\) on \(\mathbb{C}^n\).

Let \(K_{BM} = \pi^* k_{BM}\) where \(\pi: (z,\zeta)\mapsto z-\zeta\) so that \(\bar \partial K_{BM} = [\Delta]\), then: For any domain \(\Omega\subset \mathbb{C}^n\)
bounded with piecewise \(C^1\) boundary and \(v\) a \((p,q)\)-form of class \(C^1\) on \(\bar \Omega\) then
\[
v(z) = \int_{\partial\Omega}K^{p,q}_{BM}(z,\zeta)\wedge v(\zeta) + \bar \partial
\int_{\Omega} K^{p,q-1}_{BM}(z,\zeta)\wedge v(\zeta) + \int_\Omega
K^{p,q}_{BM}(z,\zeta)\wedge \bar \partial v(\zeta)
\]
where \(K^{p,q}_{BM}\) denotes the component of \(K_{BM}\) type \((p,q)\) in \(z\)
and type \((n-p, n-q-1)\) in \(\zeta\)
\end{theorem}

Another consequence of \ref{thm:koppelman} is the \emph{global} resolution of \(\bar \partial\)
in case of compact support.

\begin{corollary}
If \(v\) is a \((p,q)\)-form with \(q\geq 1\) on \(\mathbb{C}^n\), compactly supported, with regularity of class \(C^s\) such that \(\bar \partial v = 0\) then
there exists an \((p,q-1)\)-form \(u\) on \(\mathbb{C}^n\) with the same regularity
as \(u\) such
that \(\bar \partial u =v\). In fact one can take
\[
u(z) = \int_{\mathbb{C}^n} K_{BM}^{p,q-1}(z,\zeta)\wedge v(\zeta)
\]
In case \((p,q)=(0,1)\) then \(u\) is compactly support. This means that the compact
support \((0,1)\)-Dolbeault cohomology \(H_c^{0,1}(\mathbb{C}^n) = 0\).
\end{corollary}

Since \(K_{BM} = O(|z|^{1-2n})\), one has \(|u(z)| = O(|z|^{1-2n})\) at
infinity. Therefore the compact support of \(u\) in case \((p,q)=(0,1)\) is explained
by Liouville theorem.


The Dolbeault-Grothendieck lemma solves the equation \(\bar \partial u = v\) in a local
scale if the compact support condition is dropped and gives regular result if \(v\) is a \((p,0)\)-form.

\begin{theorem}[Dolbeault-Grothendieck lemma]
\label{thm:dolbeault-grothendieck}
Let \(v \in \mathcal{D}'(p,q)(\Omega)\) such that \(\bar \partial v = 0\).
\begin{enumerate}
\item If \(q=0\) then \(v = \sum v_I dz^I\) where \(v_I\in \mathcal{O}(\Omega)\).
\item If \(q \geq 1\) then there exists \(\omega\subset \Omega\) and \(u\in
   \mathcal{D}'^(p,q-1)(\Omega)\) such that \(\bar \partial u =v\). Moreover, if \(v\in
   \mathcal{E}^{p,q}(\Omega)\) then \(u\in \mathcal{E}^{p,q-1}(\Omega)\)
\end{enumerate}
\end{theorem}

\begin{corollary}[Hypoellipticity in bidegree \((p,0)\)]
\(\bar \partial\) is hypoellipticity in bidegree \((p,0)\), i.e. if \(\bar \partial u
= v\), v of bidegree \((p,1)\) and \(v\) is \(C^\infty\) then \(u\) is also \(C^\infty\) on the entire domain \(\Omega\).
\end{corollary}


\section{Extension theorems, Domain of holomorphy}
\label{sec:org36f7626}

\begin{theorem}[Hartog extension]
\label{thm:hartog-ext}
Let \(\Omega \subset \mathbb{C}^n\) be a domain and \(K \Subset \Omega\) such that \(\Omega\setminus K\) is connected. Then \(\restr{\mathcal{O}(\omega)}{\Omega\setminus K} =
\mathcal{O}(\Omega\setminus K)\) every holomorphic function on \(\Omega\setminus K\) extends to \(\Omega\)
\end{theorem}
\begin{proof}
Let \(f\in \mathcal{O}(\Omega \setminus K\) be the function we want to extend. Let \(\varphi\) be a function with support in a neighborhood of \(K\) and is identically 1 on
\(K\) and \(g = (1-\varphi)f\) which coincides with \(f\) outside of \(\supp
\varphi\). Then \(v = \bar \partial g \in \mathcal{D}^{0,1}\) satisfies \(\bar
\partial v = 0\), therefore there exists \(u\in C_c^\infty(\mathbb{C}^n\) with \(\supp
u \subset \supp \varphi\) such that \(\bar \partial u = v = \bar \partial g\), the
holomorphic function \(g-u\) is well-defined on \(\Omega\) and coincides with \(f\)
(and \(g\)) on \(\Omega\setminus \supp\varphi\), therefore coincides with \(f\) on
\(\Omega\setminus K\).
\end{proof}

Note that although we do not need \(\Omega\) to be small, this theorem counts as a
\uline{local result due to the hypothesis that we are in \(\mathbb{C}^n\)}.


A global result can be obtained using the \href{https://en.wikipedia.org/wiki/Hartogs'\_extension\_theorem}{Hartog figure}, that is the union of an anulus \(\{ (z_1,z'):\ r < |z_1| <
R\}\) and an open set in other dimension \(\{ (z_1,z'):\ z'\in\omega \text{ open}\}\).
and use the interpolation \((z_1,z') \mapsto \int_{C_{R}} \frac{f(\zeta_1,z')}{\zeta_1-z_1}d\zeta_1\) to extend \(f\). The open set in \(z'\)-dimension is to show that the interpolation and \(f\)
coincide on it. With one dimension \(z_1\) to form the anulus an another dimension (says
\(z_2\) to form the open set, one can extend any holomorphic function to a submanifold
of (complex) codimension at least 2.

\begin{theorem}[Riemann extension]
\label{thm:riemann-ext}
Let \(M\) be a complex manifold and \(N\) a sub complex manifold of codimension \(\geq 2\) then any holomorphic function on \(M\setminus N\) extends uniquely to \(M\).
\end{theorem}
Emacs 25.3.1 (Org mode 9.0.5)
\end{document}