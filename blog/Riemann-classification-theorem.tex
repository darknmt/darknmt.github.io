% Created 2017-10-28 Sat 18:27
% Intended LaTeX compiler: pdflatex
\documentclass[11pt]{article}
\usepackage[utf8]{inputenc}
\usepackage[T1]{fontenc}
\usepackage{graphicx}
\usepackage{grffile}
\usepackage{longtable}
\usepackage{wrapfig}
\usepackage{rotating}
\usepackage[normalem]{ulem}
\usepackage{amsmath}
\usepackage{textcomp}
\usepackage{amssymb}
\usepackage{capt-of}
\usepackage{hyperref}
\usepackage{amsthm}
\usepackage{amsmath}
\usepackage{tikz-cd}
\newtheorem{remark}{Remark}
\newtheorem{theorem}{Theorem}
\newtheorem{lemma}[theorem]{Lemma}
\newtheorem{corollary}{Corollary}[theorem]
\newtheorem{conjecture}[theorem]{Conjecture}
\newtheorem{proposition}{Proposition}[theorem]
\newtheorem{problem}{Problem}
\newtheorem{exampl}{Example}
\newtheorem{definition}{Definition}
\newtheorem{propdef}[definition]{Proposition-Definition}
\newcommand{\im}{\mathop{\rm Im}\nolimits}
\newcommand{\supp}{\mathop{\rm supp}\nolimits}
\newcommand{\ord}{\mathop{\rm ord}\nolimits}
\newcommand{\Spec}{\mathop{\rm Spec}\nolimits}
\newcommand{\vol}{\mathop{\rm vol}\nolimits}
\newcommand\restr[2]{{% we make the whole thing an ordinary symbol
\left.\kern-\nulldelimiterspace % automatically resize the bar with \right
#1 % the function
\vphantom{\big|} % pretend it's a little taller at normal size
\right|_{#2} % this is the delimiter
}}
\author{darknmt}
\date{\today}
\title{Classification theorem in Riemannian geometry}
\hypersetup{
 pdfauthor={darknmt},
 pdftitle={Classification theorem in Riemannian geometry},
 pdfkeywords={},
 pdfsubject={},
 pdfcreator={Emacs 25.3.1 (Org mode 9.0.5)}, 
 pdflang={English}}
\begin{document}

\maketitle
\tableofcontents

\iffalse
\begin{info}
The PDF version of this page can be downloaded by replacing \texttt{html} in the its address by
\texttt{pdf}. 
For example \texttt{/html/sheaf-cohomology.html} should become \texttt{/pdf/sheaf-cohomology.pdf}.
\end{info}
\fi

\iffalse
\begin{info}
This post is a part of the \href{../res/Stage2017.pdf}{memoire of my M1 internship} at I2M. The memoire contains,
needless to say, less errors than this page.
\end{info}
\fi




This is part of my reading in the summer of 2017. I will try not to mention the details here as they
will be covered in future posts. The big question is to 
\begin{enumerate}
\item first determine which groups can be a holonomy group (results of E. Cartan then Berger).
\item then study Bogomolov-Beauville result: hyperkahler manifolds and Calabi-Yau manifolds are
building blocks of Kahler manifold with vanishing Chern class.
\end{enumerate}

\section{What is holonomy and why it is important}
\label{sec:org5b05145}

\emph{Holonomy} is the action of closed curves starting from a point \(x\) in a Riemannian manifold \(M\) to
the tangent space \(T_xM\). It is important because

\begin{enumerate}
\item Knowing the holonomy means knowing the parallel tensors, in fact each parallel tensor is uniquely
determined by its value on a fiber \(T_xM\) (for other fibers just transport), and this value has
to be invariant by the action of holonomy.
\item The smaller the holonomy is, the more structure the manifold can have. For example, if somehow
one knows that the holonomy group is in \(SU(n)\) for a manifold of dimension \(2n\), then as \(SU(n)\)
is exactly those that preserve a almost complex structure \(J\), by transporting \(J\), we see that
\(M\) has a almost complex structure.
\item It serves as a geometric invariant (unlike the cohomology, which are topological invariants)
\end{enumerate}


Note that we defined the holonomy group by its representation. In most case, we only consider
Levi-Civita connection and the holonomy representation is then orthogonal.
\section{De Rham decomposition}
\label{sec:orga617747}

De Rham theorem allows us to decompose a Riemannian manifold under certain conditions (complete and
connected) as Riemannian product of complete connected manifold with \emph{irreducible holonomy representation}.

The idea behind this results is to prove that if a fiber \(T_xM\) decompose to subspaces stable by
holonomy, then by parallel transport \(TM\) also decompose. But by direct computation each
component of \(TM\) is involutive and by Frobenius theorem locally tangent to a submanifold of
\(M\). The completeness allows us to join the local pieces.

Now it remains to deal with irreducible holonomies. If the manifold is \emph{locally symmetric} then one
can prove that it is isometric to the homogeneous space \(G/H\) with \(H\) (the holonomy) a closed Lie
subgroup of \(G\). The theory of Lie groups developed by E. Cartan gave a complete list of these spaces.


\section{Berger classification for non-symmetric manifolds}
\label{sec:org271eb6a}

For the non-symmetric manifold, Berger proved that a irreducible holonomy has to be one of the
following
\begin{enumerate}
\item \(SO(n)\)
\item \(U(m)\subset SO(2m)\)
\item \(SU(m)\subset SO(2m)\)
\item \(Sp(m) \subset SO(4m)\)
\item \(SO(m)Sp(1) \subset SO(4m)\)
\item \(G_2\subset SO(7)\)
\item \(\text{Spin}(7)\subset SO(8)\)
\end{enumerate}

where \(n\) is the dimension. Here are some notations, note always that
\[
Sp(m)\subset SU(2m)\subset U(2m)\subset SO(4m)
\]
\begin{enumerate}
\item If \(Hol(g)\subset U(m)\subset SO(2m)\), \(g\) is called a \emph{Kahler metric}.
\item If \(Hol(g)\subset SU(m)\subset SO(2m)\), \(g\) is called a \emph{Calabi-Yau} metric. We will see that a
Calabi-Yau metric is a Kahler metric that is also Ricci-flat.
\item If \(Hol(g)\subset Sp(m)\subset SO(4m)\) then \(g\) is called a \emph{hyperkahler} metric.
\item \(G_2\) and \(\text{Spin}(7)\) are called \emph{exceptional holonomies}
\end{enumerate}

Therefore hyperkahler \(\longrightarrow\) Calabi-Yau \(\longrightarrow\) Kahler

\section{Chern class and Calabi conjecture (Yau theorem)}
\label{sec:org53e8201}
The Chern class \(c_1(M)\) of a Riemannian manifold is the class in \(H^{1,1}(M)\) of the Ricci
curvature. It is easy to see that two different kahler metric give the same Chern class. The
conjecture posed by Calabi asks \textbf{whether we can modify the kahler metric to attain, as Ricci
curvature, every \((1,1)\) form in the original Chern class}. The response, which is positive, was
given by S.T. Yau.

\begin{theorem}[Calabi-Yau]
\label{org0014987}
Let \(M\) be a compact manifold with a Kahler form \(\omega\) and the corresponding Ricci curvature
\(R_\omega\). Let \(R\) be a \((1,1)\) form cohomologous to \(R_\omega\), then there exists a Kahler form
\(\omega'\) cohomologous to \(\omega\) such that \(R'\) is the Ricci curvature of \(\omega'\).
\end{theorem}



\section{Bogomolov-Beauville classification}
\label{sec:org4be41d5}

Bogomolov result, later refined by Beauville showed that the building blocks of Kahler manifolds
with vanishing Chern class are the complex torus, simply-connected Hyperkahler manifolds or
simply-connected Calabi-Yau manifolds.

\begin{theorem}[Bogomolov-Beauville]
\label{orgdda46fb}
Let \(M\) be a compact manifold of Kahler type with \(c_1(M)=0\) then there exists a finite etale
covering space \(\tilde M\) of \(M\) of form
\[
\tilde X = T\times \prod_i V_i \prod_j X_j
\]
where \(V_i\) is a simply-connected, projective manifold with \(H^0(V_i,\Omega_{V_i}^*) = \mathbb{C}
\oplus \mathbb{C}\omega\), \(\omega\in \Omega^{n_i}\) and \(X_j\) irreducible symplectic manifold.
\end{theorem}
Emacs 25.3.1 (Org mode 9.0.5)
\end{document}