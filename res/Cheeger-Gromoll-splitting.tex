% Created 2017-05-23 Tue 00:16
% Intended LaTeX compiler: pdflatex
\documentclass[11pt]{article}
\usepackage[utf8]{inputenc}
\usepackage[T1]{fontenc}
\usepackage{graphicx}
\usepackage{grffile}
\usepackage{longtable}
\usepackage{wrapfig}
\usepackage{rotating}
\usepackage[normalem]{ulem}
\usepackage{amsmath}
\usepackage{textcomp}
\usepackage{amssymb}
\usepackage{capt-of}
\usepackage{hyperref}
\usepackage{amsthm}
\newtheorem{remark}{Remark}
\newtheorem{theorem}{Theorem}
\newtheorem{lemma}[theorem]{Lemma}
\newtheorem{corollary}{Corollary}[theorem]
\newtheorem{conjecture}[theorem]{Conjecture}
\newtheorem{proposition}{Proposition}[theorem]
\newtheorem{problem}{Problem}
\newtheorem{example}{Example}
\newtheorem{definition}{Definition}
\author{darknmt}
\date{\today}
\title{From Busemann function to Cheeger-Gromoll splitting}
\hypersetup{
 pdfauthor={darknmt},
 pdftitle={From Busemann function to Cheeger-Gromoll splitting},
 pdfkeywords={},
 pdfsubject={},
 pdfcreator={Emacs 24.5.1 (Org mode 9.0.5)}, 
 pdflang={English}}
\begin{document}

\maketitle
\tableofcontents

We will prove the following result by Cheeger and Gromoll by a slightly modified approach of
A. Besse.


\begin{theorem}[Cheeger-Gromoll]
\label{org82ebc3a}
Let \(M\) be a complete, connected Riemannian manifold with non negative Ricci curvature. Suppose
that \(M\) contains a line then \(M\) is isometric to \(M'\times \mathbb{R}\) with \(M'\) a complete,
connected Riemannian manifold with non negative Ricci curvature.
\end{theorem}

\section{Busemann function}
\label{sec:org70bd266}

Let \(\gamma\) be a geodesic ray. We construct the Busemann function \(b\) associated to the ray as
\[
b(x) = \lim_{t\to+\infty}f_t(x) = t - d(x,\gamma(t))
\]
where the limit exists because the sequence \(f_t\) is non-decreasing and bounded by
\(d(x,\gamma(0))\). The convergence is also uniform in every compact. 


In Euclidean space for example, the Busemann function is the orthogonal projection on \(\gamma\). We
will see that in a Riemannian manifold with non negative curvature, the Busemann function will serve
as a projection.


Now with a fixed \(x_0\in M\), the tangent vector at \(x_0\) of the geodesics connecting \(x_0\) and
\(\gamma(t)\) is in the unit sphere of \(T_xM\), which is compact. Let \(X\) be a limit point of these
tangents vectors and pose
\[
b_{X,t}(x) = b(x_0) + t - d(x, C_X(t))
\]
where \(C_X(t)\) is the geodesic flow starting at \(x_0\) with velocity \(X\).

\begin{remark}
\begin{enumerate}
\item From the construction of \(X\), one has \(b(x_0) + t = b(C_X(t))\), therefore \(b_{X,t} \leq b\) withe
quality in \(x_0\). We say that \(b\) is supported by \(b_{X,t}\) at \(x_0\). In general a function \(f\)
is \emph{supported} by \(g\) at \(x_0\) if \(f(x_0)=g(x_0)\) and \(f\geq g\) in a neighborhood of \(x_0\).
\item \(b_{X,t}\) is smooth and a computation in local coordinate gives \(\Delta b_{X,t} \leq \frac{\dim
   M - 1}{t}\)
\item \(\|\nabla b_{X,t}\| = 1\)
\end{enumerate}
\end{remark}

We also note that it suffices to show that \(b\) is harmonic. In fact, from the smoothness one has
\(\nabla b(x_0) =\nabla b_{X,t}(x_0)\), which means \(\|\nabla b\| = 1\) at every point in \(M\). 
For each point \(y\in M\), there exists a unique \(x\) with \(b(x)=0\) and time \(t\) when the flow of
\(\nabla b\) arrive at \(x\). \(M\) is therefore homeomorphic to \(\bar M\times \mathbb{R}\) by the above
\(y\mapsto (x,t)\) map. To see that this map is isometric, it remains to prove that \(\nabla b\) is
parallel, which follows from harmonicity of \(b\)
\[
Ric(\nabla b,\nabla b) = -\|\nabla (\nabla b) \| - (\nabla b).(\Delta b)
\]
we see that \(\nabla b\) is parallel if \(\Delta b =0\).

\begin{remark}
One can show (see A. Besse) that every gradient field \(\nabla b\) of norm 1 at every point is
actually harmonic.
\end{remark}

\section{Harmonicity}
\label{sec:org42670c9}
The Busemann function associated to a geodesic ray is subharmonic, it is a consequence of the
following lemma.


\begin{lemma}
\label{lem:1}%
In a connected Riemannian manifold, if a continuous function \(f\) is supported at any point \(x\) by
a family \(f_\epsilon\) (depending on \(x\)) with \(\Delta(f_\epsilon)\leq \epsilon\), then \(f\) can not
attain maximum (unless when \(f\) is constant).
\end{lemma}

\begin{proof}
Given a small geodesic ball \(B\), suppose that we have a function \(h\) on \(B\) with \(\Delta h <0\) in \(B\) and
\(f+h\) attains maximum at \(x\) in the interior of \(B\). Then \(f_\epsilon + h\) also attains maximum at
\(x\), which means \(\Delta f_\epsilon + \Delta h \geq 0\), which is contradictory.

For the construction of the function \(h\), one suppose that \(B\) is small enough such that
\(f|_{\partial B} \leq max_B f=: f(x_0)\) and equality is not attained at every points in \(\partial B\). Then
choose
\[
h = \eta (e^{\alpha \phi} - 1)
\]
with and \(\phi(x) = -1\) if \(x\in \partial B\) and \(f(x) = f(x_0)\), \(\phi(x_0) = 0\),
\(\nabla \phi \ne 0\) and a large  \(\alpha\) such that
\[
\Delta h = \eta (-\alpha^2\| \nabla \phi\| + \alpha \Delta \phi)e^{\alpha \phi}.
\]
is negative.
\end{proof}


Now for subharmonicity of \(b\), given a harmonic function \(h\) that coincides with \(b\) in the boundary \(\partial B\) of a
geodesic ball \(B\), then \(b-h\) is supported by \(b_{X,t} - h\) with \(\Delta (b_{X,t}-h) \to 0\) as \(t\)
tends to \(+\infty\), therefore \(b-h \leq (b-h)|_{\partial B} = 0\) in \(B\). hence \(b\) is subharmonic.

\begin{corollary}
The Busemann function of a geodesic ray in a Riemannian manifold \(M\) with non-negative Ricci
curvature is subharmonic.
\end{corollary}


Now let \(b_+\) be the function previously constructed for the ray \(\gamma|_{[0,+\infty[}\) and \(b_-\)
the Busemann function for the ray \(\tilde\gamma|_{[0,+\infty[}\) where \(\tilde\gamma(t) =
\gamma(-t)\). Note that \(b_+ + b_-\leq 0\) with equality on the line \(\gamma\), but the sum is
subharmonic therefore by maximum principle \(b_+ = -b_-\) and \(b\) is harmonic hence smooth. The
splitting theorem of Cheeger-Gromoll follows.
\end{document}