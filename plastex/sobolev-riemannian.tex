% Created 2018-04-06 Fri 11:07
% Intended LaTeX compiler: pdflatex
\documentclass[11pt]{article}
\usepackage[utf8]{inputenc}
\usepackage[T1]{fontenc}
\usepackage{longtable}
\usepackage{wrapfig}
\usepackage{rotating}
\usepackage{textcomp}
\usepackage{hyperref}
\usepackage{amsthm}
\usepackage{amsmath,amssymb}
\usepackage{darknmt-package}
\author{Tien NGUYEN MANH}
\date{Mar 22, 2018}
\title{A comparison theorem, Sobolev imbeddings and Konrachov theorem for Riemannian manifolds}
\hypersetup{
 pdfauthor={Tien NGUYEN MANH},
 pdftitle={A comparison theorem, Sobolev imbeddings and Konrachov theorem for Riemannian manifolds},
 pdfkeywords={},
 pdfsubject={},
 pdfcreator={Emacs 25.3.1 (Org mode 9.0.5)}, 
 pdflang={English}}
\begin{document}

\maketitle
\tableofcontents

In this part, we will first establish the Sobolev imbeddings theorem and the Kondrachov
theorem for Riemannian manifolds from the Euclidean version of these theorems.

\begin{theorem}[Sobolev Imbedding for \(\mathbb{R}^n\)]
\label{thm:Sobolev-Rn}
Given \(k,l\in \mathbb{Z},\ k>l \geq 0\) and \(p,q\in \mathbb{R},\ p>q\geq 1\). Then
\begin{enumerate}
\item If \(\frac{1}{p}= \frac{1}{q} - \frac{k-l}{n}\) then 
\[
    W^{k,q}(\mathbb{R}^n) \hookrightarrow W^{l,p}(\mathbb{R}^n)
   \] is a continuous imbedding.
\item If \(\frac{k-r}{n}> \frac{1}{q}\) then
\[
   W^{k,q}(\mathbb{R}^n) \hookrightarrow C^r_B(\mathbb{R}^n)
   \]
If \(\frac{k-r-\alpha}{n}\leq \frac{1}{q}\) then 
\[ 
   W^{k,q}(\mathbb{R}^n) \hookrightarrow C^{r,\alpha}(\mathbb{R}^n) 
   \]
\end{enumerate}
where \(C^r_B(\mathbb{R^n})\) denotes the space of \(C^r\) functions with bounded
derivatives up to order \(n\), equipped with the norm \(\| u \|_{C^r_B} = \max_{l\leq
r}\sup|\nabla^l u|\), and \(C^{r,\alpha}\) is the subspace of \(C^r_B\) of functions
whose \(r^{\rm th}\)-derivative is \(\alpha\)-Holder, equipped with the norm \(\|
u\|_{C^r,\alpha} = \| u \|_{C^r_B} + \sup_{P\ne Q}\{ \frac{u(P) - u(Q)}{d(P,Q)^\alpha} \}\).
\end{theorem}

\begin{theorem}[Kondrachov for \( \Omega\subset \mathbb{R}^n \)]
\label{thm:Kondrachov-Rn}
Let \(\Omega\subset \mathbb{R}^n\) be a bounded open subset with regular boundary and
let \(k\in \mathbb{Z}_{\geq 0}\) and \(p,q\in \mathbb{R}_{>0}\) be such that
\(1\geq \frac{1}{p} > \frac{1}{q} - \frac{k}{n} > 0\) then
\begin{enumerate}
\item The imbedding \(W^{k,q}(\Omega) \hookrightarrow  L^p(\Omega)\) is compact.
\item The imbedding \(W^{k,q}(\Omega) \hookrightarrow  C^\alpha(\bar\Omega)\) is compact if
\(k-\alpha > \frac{n}{q}\) where \(0\leq \alpha < 1\).
\item The imbeddings \(W_0^{k,q}(\Omega) \hookrightarrow  L^p(\Omega)\) and \(W_0^{k,q}(\Omega) \hookrightarrow C^\alpha(\bar\Omega)\) are compact, where \(W_0^{k,q}(\Omega)\) denotes the closure of \(C_c^\infty(\Omega)\) in \(W^{k,q}(\Omega)\), i.e. the subspace of functions whose trace vanishes on the boundary
of \(\Omega\).
\end{enumerate}
\end{theorem}

Theorem \ref{thm:Sobolev-Rn} will be generalised for complete manifolds with bounded curvature and
injectivity radius, while Theorem \ref{thm:Kondrachov-Rn} holds for compact Riemannian manifolds.

The generalisation will be done in 2 steps
\begin{enumerate}
\item Compare the volume form of the Riemannian metric \(g\) near a point and
that of the Euclidean metric on the tangent space at that point. Theorem
\ref{thm:vol-comparison} gives an equivalent between the integral under \(g\) and the integral
under Euclidean metric via the exponential map.
\item Reasonably use partition of unity to establish global results from local results (the
Euclidean case). We will need a covering lemma (Calabi's lemma), which essentially reduces to a
combinatorial result (Vitali's covering lemma).
\end{enumerate}

Finally, we will apply imbedding theorems to solve the equation \(-\Delta u = f\) on a
Riemannian manifold when \(f\) is square-integrable.
\section{Quick recall of Jacobi fields, Index inequality}
\label{sec:org0d6bfe8}
\begin{definition}
A \textbf{Jacobi} field is a field \(Y\) defined along a geodesic \(\gamma(t)\) such that
\begin{equation}
\label{eq:jacobi-field}
 \frac{D^2}{dt^2}Y(t) + R(Y(t),\dot\gamma(t))\dot\gamma(t) = 0
\end{equation}
where \(R\) denotes the Riemann curvature tensor.
\end{definition}

\begin{remark}
\label{rem:obvious-jacobi}
\begin{enumerate}
\item Since \eqref{eq:jacobi-field} is linear, a Jacobi field is uniquely defined given \(Y(t_0)\) and \(\dot Y(t_0)\).
\item If \(Y(0) \perp \dot\gamma(0)\) and \(\dot Y(0) \perp \dot\gamma(0)\) then \(\dot Y(t)
   \perp \dot\gamma(t)\) for all \(t\).
\item If \(Y,Z\) are Jacobi fields along \(\gamma\) then 
\[
    \langle Y, \dot Z \rangle - \langle \dot Y, Z \rangle = \const
   \]
In particular, if \(Y,Z\) vanish at a same point \(p_0\) in \(\gamma\) then
\(\langle Y,\dot Z \rangle = \langle \dot Y, Z \rangle\) on \(\gamma\).
\end{enumerate}
\end{remark}


There are two ways to interpret Jacobi fields:
\begin{enumerate}
\item Jacobi fields are derivative of exponential maps
\item Jacobi fields are minimisers of Index form, i.e. the variation of second other of
length.
\end{enumerate}

The first interpretation is the content of the following Proposition.

\begin{proposition}
\label{prop:jacobi-exp}
Let \(Y(t) = D \exp_p(tu). t\xi\) be a vector field defined on a geodesic \(\gamma(t) =
\exp_p tu\). Then \(Y\) satisfies
\begin{equation}
\label{eq:jacobi-exp}
\begin{cases}
Y(0) = 0, \dot Y(0) = \xi,  \\
\ddot Y + R(Y,\dot \gamma)\dot\gamma = 0,
\end{cases}
\end{equation}
hence a Jacobi field.
\end{proposition}

In concrete term, denote by \(\psi\) the exponential function at \(p\in M\) and \(q=\gamma(r) = \exp_p r\dot\gamma(0)\), then
Proposition \ref{prop:jacobi-exp} says that if the Jacobi field \(Y\) vanishes at
\(p =\gamma(0)\), i.e. \(Y(0) = 0\) then \(Y(r)\) at \(\gamma(r)\) is defined as
follow: pull-back \(\dot Y (0)\) by \(\psi\), transport parallelly, w.r.t to the Euclidean
metric of \(T_pM\), \(\psi^*\dot Y(0)\)
from \(0\) to \(X_0 = \psi^{-1} (q)\), then push-forward by \(\psi\), one obtains \(Y(r)\). See Figure \ref{fig:jacobi-exp}.

\label{fig:jacobi-exp}
\begin{figure}[htbp]
\centering
\caption{Jacobi fields and exponential maps.}
\end{figure} 



Since Jacobi fields are derivatives of exponential maps, one can rephrase the phenomenon
of cut-locus by Jacobi fields. Historically, a point \(q\) on a Riemannian manifold
is said to be a \textbf{conjugate} point of \(p\) if there exists, along a geodesic connecting them, a Jacobi
field vanishing on both \(p\) and \(q\). This means that the exponential map with
origin in \(p\) degenerates at a preimage of \(q\). One can also prove that if \(q\)
is in the cut-locus of \(p\) then at least one of the following situation occurs
\begin{enumerate}
\item \(q\) is a conjugate point of \(p\).
\item There exists 2 minimising geodesic from \(p\) to \(q\).
\end{enumerate}



For another interpretation of Jacobi fields, note that given a geodesic \(\gamma\) and a
vector field \(Z\) defined along \(\gamma\), then the first variation of length when
one varies \(\gamma\) by \(Z\) is 0 and the second variation can also be calculated
without difficulty.

\begin{proposition}[Second variation of length]
\label{prop:index-form-variation}
Let \(\gamma:[0,r] \longrightarrow M\) be a geodesic and \(Z\) be a vector field along \(\gamma\) that is
orthogonal to \(\dot\gamma\) at every point. Denote by \(L_\lambda\) length of the
curve \(t\mapsto \exp_{\gamma(t)}\lambda Z\) for \(\lambda \ll 1\), then one has
\begin{equation}
\label{eq:second-var}
 \restr{\frac{d^2}{d\lambda^2}L_\lambda}{\lambda = 0} = I(Z):=   \int_0^r \left( \|Z(t)\|^2 + \langle R(\dot\gamma(t), Z(t))\dot\gamma(t), Z(t) \rangle \right) dt
\end{equation}
\end{proposition}

\begin{definition}
Let \(\gamma:[0,r] \longrightarrow M\) be a geodesic and \(Z\) be a orthogonal vector
field along \(\gamma\). The \textbf{Index form} \(I(Z)\) of \(Z\) is defined by the RHS of \eqref{eq:second-var}.
\end{definition}

\begin{remark}
The curvature term in \eqref{eq:second-var} is \(K(\dot\gamma,Z)\|Z\|^2\) where \(K\) denotes the
sectional curvature of \(M\).
\end{remark}

Jacobi fields can be seen as the unique minimiser of the Index form among vector fields
defined on a geodesic \(\gamma:[0,r] \longrightarrow M\) with the same value at \(\gamma(0)\) and \(\gamma(r)\).

\begin{theorem}[Index inequality]
\label{thm:index-ineq}
Let \(\gamma: [0,r] \longrightarrow M^n\) be a geodesic, \(p=\gamma(0)\) and \(q=\gamma(r)\) such that \(p\) has no conjugate point along \(\gamma\), or
equivalently the exponential map in direction \(\dot\gamma(0)\) does not degenerate.
\begin{itemize}
\item Let \(Z\) be a (piecewise smooth) vector field along \(\gamma\), orthogonal to \(\dot\gamma\) with \(Z(p)=0\).
\item Let \(Y\) be the Jacobi field along \(\gamma\) with \(Y(0) = 0, Y(r) = Z(r)\) and
\(Y\) is orthogonal to \(\dot\gamma\).
\end{itemize}
Then \(I(Y) \leq I(Z)\) and equality occurs if and only if \(Y\equiv Z\).
\end{theorem}

\begin{remark}
\label{rem:jacobi-existence}
Note that such Jacobi field \(Y\) exists and is unique. Firstly, by the second point of Remark
\ref{rem:obvious-jacobi}, one only need \(Y(p) =0\) and \(\dot Y(0) \perp\gamma(0)\). The Jacobi fields satisfying these conditions form a vector space of dimension \(n-1\) (by Cauchy problem, \(\dot Y(0)\) is to be chosen in the orthogonal
space of \(\gamma(0)\)). Since the exponential map does not degenerate on the preimage
of \(\gamma\), each \(\dot Y(0)\) corresponds one-to-one with an \(Y(r)\) by
Proposition \ref{prop:jacobi-exp}. The correspondence is linear, with source and
target spaces of same dimension (\(n-1\)), it follows that each \(Z(r)\perp \gamma(r)\) gives uniquely a Jacobi field \(Y\).

More concretely, let \(\dot V_i(0)\) be a basis of \(\dot\gamma(0)\) in \(T_pM\) and
\(V_i\) be the corresponding Jacobi fields with \(V_i(0)=0\), then
\begin{enumerate}
\item \(\{V_i(t)\}_{i=\overline{1,n-2}}\) is a basis of \(\dot\gamma(t)\) in \(T_{\gamma(t)}M\), where the orthogonal part follows from Remark \ref{rem:obvious-jacobi}
and the linear independence is by the non-degeneration of since \(\exp_p\).
\item If \(Z(t) = \sum f_i(t)V_i(t)\), where \(f_i\) are functions on \([0,r]\), then \(Y(t) = \sum_i f_i(r)V_i(t)\).
\end{enumerate}
\end{remark}

\begin{proof}
As Remark \ref{rem:jacobi-existence}, let \(Z = \sum_i f_i V_i\) and denote \(W = \sum_i
\dot f_i V_i\) then
\[
 I(Z) = \int_0^r\left( \|W\|^2 + 2 \sum_i f_i \langle \dot V_i, W \rangle + \langle \sum_i
f_i\dot V_i,\sum_j f_j\dot V_j \rangle + \langle R(\dot\gamma,\sum f_i V_i)\dot\gamma,\sum
f_j V_j \rangle\right)dt
\]
By definition of Jacobi field, \(R(\dot\gamma, V_i)\dot\gamma = \ddot V_i\), hence the
curvature term is
\begin{equation*}
\begin{split}
\int_0^r \left\langle R(\dot\gamma,\sum f_i V_i)\dot\gamma,\sum f_j V_j \right\rangle &=\sum_{i,j}\int _0^r f_i f_j \langle \ddot V_i, V_j \rangle dt =\sum_{i,j}\int_0^r f_if_j \left(\frac{d }{dt} \langle \dot V_i, V_j \rangle - \langle \dot V_i,\dot V_j \rangle\right)dt\\
			 	       &= -\int_0^r \left\langle \sum_i f_i \dot V_i,\sum_j f_j \dot V_j \right\rangle dt + \langle \dot Y(r), Y(r) \rangle - 2\sum_{i,j}\int_0^r f_i\dot f_j \langle \dot V_i,V_j \rangle dt
\end{split}   
\end{equation*}
where for the second line, we integrated by part and used the fact that
\(\langle \dot V_i, V_j \rangle = \langle V_i, \dot V_j \rangle\) (point 3 of Remark
\ref{rem:obvious-jacobi}). Therefore, one has
\[
 I(Z) = \int_0^r \|W\|^2dt + \langle \dot Y(r), Y(r) \rangle.
\]
In particular \(I(Y) = \langle \dot Y(r), Y(r) \rangle \leq I(Z)\). The equality occurs
if and only if \(W\equiv 0\), i.e. \(Z\equiv Y\).
\end{proof}

\section{Local comparison with space forms}
\label{sec:orgdc0da4c}

Our goal in this section is to prove the following Comparison Theorem. Before going to the
precise statement, let us explain the notation.

\paragraph{Notation.}
\label{sec:org4abfd0a}
Given \(M^n\) a Riemannian manifold and \(B(p,r_0)\) be the geodesic ball centered in
\(p\in M\), of radius \(r_0<\delta_p\) the injectivity radius at \(p\), equipped
with the pullback metric of \(g\) via exponential map \(\exp_p\), which can be expressed in polar geodesic coordinates
as 
\[
(ds)^2 = (dr)^2 + r^2 g_{\theta^i\theta^j}(r,\theta) d\theta^i d\theta^j
\]
where \(\frac{\partial}{\partial\theta^1},\dots,\frac{\partial}{\partial\theta^{n-1}}\)
is an Euclidean orthonormal frame of the sphere
\(r\mathbb{S}^{n-1}\). We note \(|g_\theta| = \det(g_{\theta^i\theta^j})_{ij}\) and
\(g_{\theta\theta}\) be any component \(g_{\theta^i\theta^i}\) for \(i=1,\dots,n-1\).

Abusively, we say that \(\frac{\sin \alpha r}{\alpha} = r\) if \(\alpha = 0\) and \(\sin\alpha r =  \frac{1}{i}\sinh i\alpha r\) and \(\cos \alpha r = \cosh i\alpha r\) if \(\alpha\in i \mathbb{R}\).

\begin{remark}
\label{rem:hairy-ball}
Note that the frame \(\{ \frac{\partial}{\partial\theta^i}\}_i\) may not be global, for
example when \(n\) is odd (Hairy ball theorem). However the quantity \(|g_\theta|\) is
globally defined (except at \(p\)), in fact \(|g_\theta| = r^{-2n+2}|g|\).
\end{remark}

\begin{theorem}[comparison of volume forms]
\label{thm:vol-comparison}
Let \(M^n\) be a Riemannian manifold with
\begin{itemize}
\item sectional curvature \(-a^2 \leq K\leq b^2\)
\item Ricci curvature \({\rm Ric} \geq a' = (n-1)\alpha^2\) where \(\alpha\) can be real or
purely imaginary.
\end{itemize}
Then with the notation of the last paragraph, for all \(r \in (0, r_0)\),
\begin{enumerate}
\item If \(r<\frac{\pi}{b}\) then
\begin{equation}
\label{eq:comp-1}
\begin{split}
 \frac{\partial}{\partial r}\log   \sqrt{g_{\theta\theta}} &\geq \frac{\partial}{\partial r}\log \frac{\sin br}{r}\\  
  g_{\theta\theta} &\geq \left(\frac{\sin br}{br}\right)^2
\end{split}   
\end{equation}
\item One has
\begin{equation}
\label{eq:comp-2}
\begin{split}
 \frac{\partial}{\partial r}\log   \sqrt{g_{\theta\theta}} &\leq \frac{\partial}{\partial r}\log \frac{\sinh ar}{r}\\  
 g_{\theta\theta} &\leq \left(\frac{\sinh ar}{ar}\right)^2
\end{split}   
\end{equation}
\item One has
\begin{equation}
\label{eq:comp-3}
\begin{split}
 \frac{\partial}{\partial r}\log   \sqrt{g_{\theta}} &\leq (n-1) \frac{\partial}{\partial r}\log \frac{\sin \alpha r}{r} \leq -a' \frac{r}{3}\\  
  \sqrt{|g_{\theta}|} &\leq \left(\frac{\sin \alpha r}{\alpha r}\right)^{n-1}
\end{split}   
\end{equation}
\item If \(r<\frac{\pi}{b}\) then
\begin{equation}
\label{eq:comp-4}
\begin{split}
 \frac{\partial}{\partial r}\log   \sqrt{g_{\theta}} &\geq (n-1) \frac{\partial}{\partial r}\log \frac{\sin b r}{r}\\  
 \sqrt{|g_{\theta}|} &\geq \left(\frac{\sin br}{br}\right)^{n-1}       
\end{split}   
\end{equation}
\end{enumerate}
\end{theorem}

\begin{remark}
\label{rem:on-vol-comp}
\begin{enumerate}
\item The moral of the estimates is that if \(r\ll 1\) then the volume form of \(g\),
viewed in the tangent space at \(p\), is equivalent to the Euclidean volume form of \(T_pM\).
\item One can always choose \(\alpha\in i \mathbb{R}\) even when the Ricci curvature is
positive, and RHS of \eqref{eq:comp-3} will be a hyperbolic function and the estimate is
not as sharp as if one choose \(\alpha\in \mathbb{R}\), but it works to prove that
the two volume forms are equivalent when \(r \ll 1\) .
\end{enumerate}
\end{remark}

\begin{remark}
\label{rem:cor-vol-comp}
A few consequences of Theorem \ref{thm:vol-comparison}:
\begin{enumerate}
\item For \(\delta\) small, the metric volume form \(dV\) is \emph{equivalent} to the
Euclidean volume form of tangent space: there exists \(C(\delta)>0\) converging to \(1\) as \(\delta \to 0\) such that \(C(\delta)^{-1}dE \leq  dV \leq C(\delta) dE\).
\item Let \(f\) be a smooth function defined on \(B(p,\delta)\) then the gradient of \(f\) w.r.t the metric \(g\) is closed to the Euclidean gradient of \(f\) viewed in
the chart (namely \(f\circ \exp_p\)):
\begin{equation*}
\begin{split}
\|\nabla f\|_g &= \left|\frac{\partial f}{\partial r}\right|^2 + \sum_\theta \left|\frac{\partial f}{\partial \theta}(r,\theta)\right|^2 g_{\theta\theta}\\ 
\|\nabla (f\circ\exp_p)\|_E &= \left|\frac{\partial f}{\partial r}\right|^2 + \sum_\theta \left|\frac{\partial f}{\partial \theta}(r,\theta)\right|^2
\end{split}   
\end{equation*}
\item Combining the last 2 points, one can see that if \(f\) is supported in a small
geodesic ball \(B(p,\delta)\), then the \(L^p\)-norm of \(\nabla f\) is closed to
the Euclidean \(L^p\) norm of \(\nabla (f\circ \exp_p)\) if \(\delta\) is
sufficiently small.
\end{enumerate}
\end{remark}



The ideal to prove Theorem \ref{thm:vol-comparison} comes from Proposition
\ref{prop:jacobi-exp} and Figure \ref{fig:jacobi-exp}. Given a point \(q\in M\) of distance
\(r< r_0\) from \(p\), then denote by \(Y\) the Jacobi field along the unique
geodesic connecting \(p\) and \(q\) such that \(Y\) vanishes at \(p\) and 
\(Y(r) = \frac{\partial}{\partial \theta}\) at \(q\), then with \(\psi =\exp_p\) as in Figure \ref{fig:jacobi-exp},
\begin{equation}
\label{eq:comp-norm}
\begin{split}
\| Y(r)\|^2  &= \| \psi_{X_0}^* Y(r) \|^2 = \| \psi_0^* \dot Y(0)\|^2_{X_0}\\
     &= r^2 g_{\theta\theta} \|\psi_0^*\dot Y(0)\|_0^2 = r^2 g_{\theta\theta} \|\dot Y(0)\|^2
\end{split}   
\end{equation}
where we used the fact that
\[ g\left(\restr{\frac{\partial}{\partial \theta^i}}{r
\mathbb{S}^{n-1}},\restr{\frac{\partial}{\partial \theta^j}}{r \mathbb{S}^{n-1}}\right) = r^2
g\left(\restr{\frac{\partial}{\partial
\theta^i}}{\mathbb{S}^{n-1}},\restr{\frac{\partial}{\partial \theta^j}}{\mathbb{S}^{n-1}}\right)
= r^2 g_{\theta_i\theta_j} \].

Take logarithm and derive \eqref{eq:comp-norm} w.r.t \(r\), using the fact that
\(\|Y(r)\| = 1\), one obtains
\begin{equation}
\label{eq:comp-norm-ap}
\langle \dot Y(r), Y(r) \rangle = \frac{1}{r} + \frac{\partial}{\partial r}\log g_{\theta\theta}
\end{equation}
It comes to estimate \(\langle \dot Y(r), Y(r) \rangle\), which is in fact the Index
form of \(Y\). The following lemma give an estimate of the Index form in case of bounded
sectional curvature, by comparing the it with the Index form under a metric with constant
sectional curvature. 

\begin{lemma}
\label{lem:compare-const-sec-curv}
Suppose that the sectional curvature \(K \leq b^2\), then for every Jacobi field \(Y\)
defined a long a geodesic \(\gamma: [0,r] \longrightarrow M\) with \(r< \frac{\pi}{2b}\) such that \(Y(0)=0, Y\perp \dot\gamma\). Then
\[
 I(Y) \geq I_b(Y):=\int_0^r \| \dot Y\|^2 - b \|Y\|^2 \geq b\cot br \|Y(r)\|^2
\]
\end{lemma}

\begin{proof}
By the curvature bound, \(I(T) \geq \int_0^r \| \dot Y\|^2 -b^2 \|Y\|^2=:I_b(Y)\). The
quantity \(I_b(Y)\) is exactly the Index form of \(Y\) along \(\gamma\) if the
sectional curvature in constantly \(b\). To be precise, we equip the tubular
neighborhood of \(\gamma\) a metric \(g'\) of constant sectional curvature \(K=b^2\)
such that normal vectors of \(\gamma\) w.r.t the metric \(g\) remain normal under \(g'\). Such \(g'\) is in fact easy to find since: 
\begin{enumerate}
\item The tubular neighborhood is diffeomorphic to
\([0,r]\times \mathbb{B}^{n-1}\) where the diffeomorphism (says \(\iota_1\)) is actually isometry at points
of \(\gamma\), which are mapped to \([0,r]\times \{0\}\);
\item Also, there exists a diffeomorphism \(\iota_2\) mapping \([0,r]\times\mathbb{B}^{n-1}\) to a tubular neighborhood of an arc \(\tilde\gamma\) of length \(r\) on the grand
circle of \(\mathbb{S}^n_{1/b}\) which is isometry on every point of
\([0,r]\times \{0\}\). This is because \(r < \frac{\pi}{2b} < 2\pi \frac{1}{b}\) the length
of the grand circle.
\item One now can identify a tubular neighborhood of \(\gamma\) in \(M\) and that of \(\tilde \gamma\) in \(\mathbb{S}^n_{1/b}\) by \(\iota = \iota_2\circ\iota_1\). Take
\(g'\) to be the pullback of the Eucidean metric on \(\mathbb{S}^n_{1/b}\), which is of sectional curvature \(b^2\).
\end{enumerate}

Now under the metric \(g'\), \(Y\) is no longer a Jacobi field, but it is still
orthogonal to \(\gamma\), denote by \(\tilde Y\) the Jacobi field (under \(g'\)) on
\(\gamma\) that vanishes at \(\gamma(0)\) and has the same value as \(Y\) at \(\gamma(r)\). By Theorem \ref{thm:index-ineq} (Index inequality), one has \(I_b(Y)
\geq I_b(\tilde Y)\). The latter can be computed directly, as the field \(\iota_*\tilde Y\) is given by
\[
 s\mapsto (s,\beta^1\sin bs, \dots, \beta^{n-1}\sin bs), \quad s\in [0,r]
\]
where \((\beta^1,\dots,\beta^{n-1})\) is the coordinates of \({\iota_1}_* Y(r)\) in
\([0,r]\times \mathbb{B}^{n-1}\), hence in this coordinates (also called \emph{Fermi coordinates}), \(\tilde Y(s) = \left(s,\frac{\sin bs}{\sin br}Y(r)\right)\) . Hence
\(I_b(\tilde Y) = b\cot br \|Y(r) \|^2\).
\end{proof}

Now the remaining part of the proof of Theorem \ref{thm:vol-comparison} is straightforward.

\begin{proof}[Proof of Theorem \ref{thm:vol-comparison}]
From \eqref{eq:comp-norm-ap} and Lemma \ref{lem:compare-const-sec-curv}, one has 
\[
 \frac{\partial}{\partial r} \log \sqrt{g_{\theta\theta}} = I(Y) - \frac{1}{r} \geq b\cot br -\frac{1}{r}
\]
This gives the estimates in \eqref{eq:comp-1}.

For \eqref{eq:comp-2}, the sign situation fits Theorem \ref{thm:index-ineq} better, and one
does not need to explicitly evoke the space forms (as Lemma
\ref{lem:compare-const-sec-curv}). It suffices to see that
\begin{equation*}
\begin{split}
 \langle \dot Y(r), Y(r) \rangle &=I(Y) \leq I\left(\frac{\sinh at}{\sinh ar}Y(r)\right)\\
 	      	    &\leq a^2\left(\int_0^r \left(\frac{\cosh at}{\sinh ar}\right)^2 + \int_0^r \left(\frac{\sinh at}{\sinh ar}\right)^2 dt  \right)\|Y(r)\|^2\\
		    &= a\coth ar \|Y(r)\|^2
\end{split}   
\end{equation*}

The estimates in  \eqref{eq:comp-3} comes from the comparison between \(Y\) and the
field  \(t\mapsto \frac{\sin \alpha t}{\sin \alpha r}Y(r)\). Note that the field is
well-defined even when \(\alpha \in \mathbb{R}_{>0}\) (the hyperbolic case (\(\alpha\in
i \mathbb{R}_{>0}\) being obvious). This in fact comes from the following fact:
\begin{theorem}[Myers]
\label{thm:myers}
Let \(M^n\) be a connected, complete manifold with \({\rm Ric}\geq (n-1)\alpha^2 >0\)
then
\begin{enumerate}
\item \(M\) is compact.
\item The diameter of \(M\) is at most \(\pi/\alpha\).
\end{enumerate}
\end{theorem}

Taking sum of inequalities \(I(Y_{i}) \leq I(\frac{\sin \alpha t}{\alpha
r}Y_i(r)\) where \(Y_i\) are Jacobi fields vanishing at \(\gamma(0)\) and
whose values at \(\gamma(r)\) are \(\frac{\partial}{\partial \theta^i}\) respectively,
one has
\begin{equation*}
\begin{split}
\sum_{i=1}^{n-1} \langle \dot Y_i(r), Y_i(r) \rangle &\leq	(n-1)\alpha^2 \int_0^r\left( \frac{\cos \alpha t}{\sin \alpha r}\right)^2 dt -\sum_{i=1}^{n-1}\int_0^r R_{r\theta^ir\theta^i}\left(\frac{\sin \alpha t}{\sin \alpha r}\right)^2 dt\\
		 	 &\leq (n-1)\alpha\cot \alpha r
\end{split}   
\end{equation*}
where for the second line, we used the fact that \(\sum_i R_{r\theta^i r\theta^i}={\rm
Ric}_{rr}\geq (n-1)\alpha^2\). Hence 
\begin{equation*}
\begin{split}
\frac{\partial}{\partial r}\log\sqrt{|g_\theta|}&= \frac{\partial}{\partial
r}\sum_i\log\sqrt{|g_{\theta^i\theta^i}|} = \sum_i \langle \dot
Y_{\theta^i},Y_{\theta^i}\rangle - \frac{n-1}{r}\\
 				 &\leq (n-1)\left(\alpha\cot \alpha r - \frac{1}{r}\right) = (n-1) \frac{\partial}{\partial r}\log \left( \frac{\sin \alpha r}{r}\right)
\end{split}   
\end{equation*}

The proof of \eqref{eq:comp-4} is essentially the same as \eqref{eq:comp-3} where one uses
\eqref{eq:comp-1} for a lower bound of \(I(Y_i) = \langle \dot Y_i(r), Y_i(r) \rangle\).
\end{proof}


As a side note, Lemma \ref{lem:compare-const-sec-curv} can also be used to prove that a small geodeosic ball
is geodesically convex.

\begin{proposition}
\label{prop:geodesic-convex}
Let \(M^n\) be a Riemannian manifold with sectional curvature \(K\leq b^2\) and
injectivity radius \(\delta >0\). Then for every \(r<\min\{\frac{\delta}{2},
\frac{\pi}{4b}\}\), any geodesic ball \(B(p,r)\) is geodesically convex, i.e. any two
points is connected by a geodesic curve inside the ball.
\end{proposition}

\begin{proof}
We first claim that
\begin{lemma}
Given two point \(p,q\) of distance \(d(p,q) = r < \frac{\pi}{2b}\) and \(\Gamma_{p,q}\) the geodesic connecting the them. Let \(\gamma\) be a geodesic staring
from \(q\) with a velocity vector perpendicular to \(\Gamma_{p,q}\), then there exists
a neighborhood of \(q\) inside of which the \(\gamma\) intersects \(\Gamma_{p,q}\)
only at \(q\).
\end{lemma}

First, let us prove that the Lemme implies Proposition \ref{prop:geodesic-convex}. If \(r\) small as in the Proposition and \(q_1, q_2\in B(p,r)\) then
\begin{enumerate}
\item There exists a minimal geodesic \(\Gamma_{q_1,q_2}\) connecting \(q_1, q_2\).
\item By triangle inequality, \(\Gamma_{q_1, q_2}\subset B(p,2r)\): every point \(q\in
   \Gamma_{q_1,q_2}\) has to be \(d(q_1,q_2)/2\)-closed to one \(q_i\), hence \(d(p,q)\leq d(p,q_i) + d(q_i,q) \leq r + \frac{2r}{2} = 2r\).
\end{enumerate}
Let \(T \in \Gamma_{q_1, q_2}\) be the point minimising the distance to \(p\). It
suffices to show that \(T\) is one of the \(q_i\). For the sake of contradiction, if
\(T\) is strictly in the interior of \(\Gamma_{q_1,q_2}\) then
\begin{enumerate}
\item The geodesic \(\Gamma_{p,T}\) connecting \(p\) and \(T\) is orthogonal to \(\Gamma_{q_1,q_2}\) at \(T\). It is not difficult to prove that if the two are not
orthogonal then there exist \(T'\in \Gamma_{q_1,q_2}\) and \(S\in \Gamma_{p,T}\),
both being near to \(T\), such that \(d(p,T) > d(p, S) + d(S,T') \geq d(p,T')\).
\item The ball \(B(p,d(p,T))\cap \Gamma_{q_1,q_2} \supset \Gamma_{q_1,q_2}\).
\end{enumerate}

These contradict the Lemma and prove that \(T\) does not lie in the interior.

It remains to prove the Lemma. Let \(Y\) be the Jacobi field which vanishes at \(p\)
and whose value at \(q\) is \(\dot\gamma\), then by Index inequality (Theorem
\ref{thm:index-ineq}), it suffices to prove that \(I(Y)>0\), because any variation of \(\Gamma_{p,q}\) by orthogonal vector field \(Z\) along \(\gamma\) has \(I(Z) >0\) hence only increases the length,
according to Proposition \ref{prop:index-form-variation}. But by Lemma
\ref{lem:compare-const-sec-curv} gives
\[
 I(Y) \geq I_b(Y) \geq b\cot br \| Y(q) \|^2 >0 \text{ if } r < \frac{\pi}{2b}.
\]
\end{proof}

\section{Some covering lemmas}
\label{sec:org5678b4c}
The goal of this section is to prove a covering lemma for Riemannian manifolds with
injectivity radius \(\delta_0>0\) and bounded curvature (Lemma
\ref{lem:uni-loc-finite-cover}). We start with a covering lemma that not yet requires
curvature bound.

\begin{lemma}[Calabi]
\label{lem:calabi}
Let \(M^n\) be a Riemannian manifold with injectivity radius \(\delta_0 >0\), then for
all \(\delta \in (0,\delta_0)\), there exists \(0 < \gamma < \beta \leq\delta\) and a
partition of \(M = \bigsqcup_{i\in I}\Omega_i\) and \(p_i\in \Omega_i\) such that
\[
 B(p_i,\gamma) \subset \Omega_i \subset B(p_i,\beta)
\]
Moreover, one can choose \(\gamma = \beta/10\) and \(\beta=\delta\).
\end{lemma}

\begin{proof}
Note that it is enough to have 
\begin{equation}
\label{eq:lem:calabi}
\begin{cases}
\bigcup_i B(p_i,\beta) = M,\quad  2\gamma < \beta \\
B(p_i, 2\gamma) \text{ are disjoint}
\end{cases}
\end{equation}
In fact, let \(\Omega'_i = B(p_i,\beta)\setminus \cup_{j\ne i}B(p_j,\gamma)\) then 
\[
\begin{cases}
B(p_j,\gamma) \cap \Omega_i' = \emptyset, B(p_i,\gamma) \subset \Omega_i'\subset B(p_i,\beta) \\
\bigcup_i \Omega_i' = M 
\end{cases}
\]
(for \(\bigcup_i\Omega_i'=M\): If \(x\in M\) satisfies \(x\in B(p_j,\gamma)\subset
B(p_i,\beta)\) then there is no other \(j'\ne j\) such that \(x\in B(p_{j'},\gamma)\), hence \(x\in \Omega_{j}\). Now choose 
\[
 \Omega_1 = \Omega_1', \Omega_2 = \Omega_2'\setminus\Omega_1,\dots,\Omega_n =
\Omega_n'\setminus \cup_{i=1}^{n-1}\Omega_i,\dots
\]

For the existence of \eqref{eq:lem:calabi}, use the following Vitali covering lemma, whose
proof is purely combinatorial in nature.

\begin{lemma}[Vitali covering, Infinite version]
\label{lem:vitali-cover}
Let \(\{B_j: j\in J\}\) be a collection of balls in a metric space such that 
\[
 \sup \{ {\rm rad}(B_j): j\in J \} < +\infty
\]
where \({\rm rad}\) denotes the radius, then there exists a countable subfamily \(J'\subset J\) such that \(\{B_j: j\in J'\}\) are disjoint and 
\[
 \cup_{j\in J}B_j\subset \cup_{j\in J'}5 B_j. 
\]
\end{lemma}

It remains to apply the lemma for the covering \(M= \cup_{x\in M}B(x,2\gamma)\), which
also allows us to choose \(\gamma = \beta/10\) and \(\beta=\delta\).
\end{proof}





\begin{lemma}[Uniformly locally finite covering]
\label{lem:uni-loc-finite-cover}
Let \(M^n\) be a Riemannian manifold with injectivity radius \(\delta_0>0\) and
bounded curvature, then for all \(\delta <\delta_0\) sufficiently small, there exists a \textbf{uniformly locally
finite covering} of \(M\) by balls \(\{B(p_i,\delta)\}_{i\in I}\), i.e. there exists
\(k(\delta)\in \mathbb{Z}_{>0}\) such that for all \(q\in M\), there exists a
neighborhood of \(q\) that intersects at most \(k(\delta)\) balls. Moreover, one can
also require that \(\{B(p_i,\delta/2)\}_{i\in I}\) is still a covering.  
\end{lemma}

\begin{proof}
We will apply Lemma \ref{lem:calabi} with \(\beta = \delta/2\) and
\(\gamma = \beta/10\), then for all \(\delta \ll \delta_0\), the covering
\(\{B(p_i, 2\beta) \}\) satisfies. In fact, for every \(q\in M\), take
\(B(q,\delta)\) as a neighborhood of \(q\) then
\(B(p_i,2\beta)\cap B(q,\gamma)\ne \emptyset\) if and only if
\(p_i\in B(q,2\beta + \gamma)\) Since the balls \(B(p_i,\gamma)\) are disjoint, the
number of \(p_i\) in \(B(q,2\beta +\gamma)\) is bounded by
\[
k = \frac{\max \vol_g (B_{2\beta+2\gamma})}{\min\vol_g (B_\gamma)} \leq C(\delta)\left(\frac{2\beta+2\gamma}{\gamma}\right)^n
\]
where \(\max \vol_g (B_{2\beta+2\gamma})\) and \(\min\vol_g (B_\gamma)\) denote the
maximum and minimum volume of balls of radius \(2\beta + 2\gamma\) and \(\gamma\),
respectively. By Theorem \ref{thm:vol-comparison}, for \(\delta < \epsilon(a',b)\) depending on
the bound \(a'\) and \(b\) of Ricci curvature and sectional curvature, the volume of
these balls are equivalent to that of Euclidean balls of the same radius. The constant of
equivalence was denoted by \(C(\delta)\).
\end{proof}


\section{Sobolev imbeddings for Riemannian manifolds}
\label{sec:org988defc}

The goal of this section is to prove that Sobolev imbeddings are also available for complete Riemannian manifold
with bounded curvature and strictly positive injectivity radius, that is, the following results.

\begin{theorem}[Sobolev imbeddings]
\label{thm:sobolev-imbedding}
Theorem \ref{thm:Sobolev-Rn} holds when one replaces \(\mathbb{R}^{n}\) by a complete Riemannian
manifold of dimension \(n\) with bounded curvature (sectional and Ricci) and injectivity radius \(\delta_0 >0\).
\end{theorem}

The definition of Sobolev spaces as completion of spaces of smooth functions, w.r.t the
Sobolev norms generalises on Riemannian manifolds, namely, we denote by \(W_0^{k,p}(M)\)
the completion of \(C_c^\infty(M)\) w.r.t the norm
\(\|\varphi\|_{W^{k,p}} = \|\varphi\|_{L^p} + \|\nabla \varphi \|_{L^p} + \dots + \|\nabla^k \varphi \|_{L^p}\)
where \(\|\nabla^l\varphi\|_{L^p}\) are computed as follow: the metric \(g\) induces a
fiberwise norm for \(l\)-covariant tensors, integrate that of \(\nabla^l \varphi\), one
obtains \(\|\nabla^l\varphi\|_{L^p}\).

Similarly, the space \(W^{1,p}(M)\) is defined as the completion of \(C^\infty(M)\) w.r.t \(\|\cdot\|_{W^{1,p}}\). 

\begin{remark}
\label{rem:sobolev-def}
\begin{enumerate}
\item Unlike the Euclidean case, one does not define the derivatives term, e.g. \(\nabla_v f\) for \(f\in W^{1,p}(M)\) using integration by part and Riesz representation,
that is, one does not expect a formular such as \(\int_M (\nabla_v f)\varphi dV = -\int_M
   f\nabla_v \varphi dV\) since the "boundary term" \(\int_{M}\nabla_v(f\varphi) dV\) does not vanish, even if \(f\varphi \in C_c^\infty(M)\).
\item The exterior derivative \(df\) can be defined, which is in fact equivalent to de
Rham's notion of current.
\item The term \(\nabla^l f\) for \(f\in W^{k,p}(M)\), when needed, can be defined as a \(L^p\) section of \((TM^*)^{\otimes l}\) giving by the \(L^p\) limit of smooth
sections \(\nabla^l\varphi_i\) for an equivalent class of Cauchy sequence \(\varphi_i\) representing \(f\). The completeness of the space of \(L^p\) sections
of a vector bundle follows from the result in each trivialising chart and the fact that restriction maps
commute with the limit.
\end{enumerate}
\end{remark}


\begin{proposition}[\( W^{1,p} = W_0^{1,p} \)]
\label{prop:Cc-dense-W1p}
If \(M\) is complete then \(C_c^\infty(M)\) is dense in \(W^{1,p}(M)\), equivalently \(W^{1,p}(M) = W^{1,p}_0(M)\).
\end{proposition}
\begin{proof}
It suffices to prove that given a function \(\varphi\in C^\infty(M)\), one can
approximate \(\varphi\) under the norm \(\|\cdot \|_{W^{1,p}}\) by functions in \(C^\infty_c(M)\). Fix \(P\in M\), one uses a cut-off function \(\chi_j\) which is \(1\) on \([0,j]\),
\(0\) on \([j,\infty]\) and linear inside and defines \(\varphi_j (Q) = \varphi(Q)
\chi_j(d(Q,P))\). Note that the distance function is only Lipschitz and not necessarily
smooth (so we did not mind taking a linear cut-off). However, since \(\varphi_j\) is compactly
support and Lipschitz and we can approximate each \(\varphi_j\) by a sequence in \(C_c^\infty(M)\): Let \(K_j\) be the support of \(\varphi_j\) and \(\{\alpha_i\}_i\) be
a finite partition of unity subordinating to an open coordinated cover of \(K\). Since
\(\alpha_i\varphi_j\) is Lipschitz, viewed in a chart, it can be \(W^{1,\infty}\)-approximated by smooth functions, due to the following fact.

\textbf{Fact.} If \(\Omega\subset \mathbb{R}^n\) be a bounded domain with \(\delta \Omega\)
regular, then \({\rm Lip}(\Omega) = W^{1,\infty}(\Omega)\).

The approximation scheme looks like
\(\varphi \approx \varphi_j \approx \sum_i \alpha_{i,K_j}\varphi_j \approx\sum_i \psi_{i,j}\)
where \(\psi_{i,j}\) are smooth and compactly support.
\end{proof}

\begin{remark}
The similar results for higher orders are complicated, for example, one can prove that \(W^{2,p}_0 = W^{2,p}\) under the hypothesis of bounded curvature and strictly positive
injectivity radius. The third order requires extra conditions.
\end{remark}

The second part of the Theorem \ref{thm:sobolev-imbedding} is local in nature, and therefore easier. We will prove
this second part by accepting the first one, which we will come back and prove eventually.

For the imbedding
into \(C^r_B(M)\), it suffices to establish the case \(W^{1,q} \hookrightarrow C^0_B\), the higher order case then follows: If \(\varphi\in W^{k,q}\) then \(\nabla^r\varphi\in W^{k-r,q} \hookrightarrow
   W^{k-r,q} \hookrightarrow W^{1,\tilde q} \hookrightarrow C^0_B\)
where \(\frac{1}{n} \geq \frac{1}{\tilde q}\geq \frac{1}{q}-\frac{k-r-1}{n}\). 

Similarly, for the imbedding into \(C^{r,\alpha(M)}\), it suffices to establish the case
\(W^{1,q} \hookrightarrow C^{0,\alpha}\) for \(\frac{1-\alpha}{n}\geq \frac{1}{q}\).


Since \(W^{1,p}(M) = W^{1,p}_0(M)\), it suffices to prove the following Lemma
\ref{lem:sobolev-case} and Lemma \ref{lem:sobolev-case-holder}.

\begin{lemma}[\( W^{1,q} \hookrightarrow C^0_B \)]
\label{lem:sobolev-case}
Let \(M^n\) be a complete Riemannian manifold with injectivity radius \(\delta_0>0\)
and sectional curvature \(K\leq b^2\), then for all \(\varphi\in C_c^\infty(M)\), one
has
\[
\sup_M |\varphi| \leq C(q) \|\varphi\|_{W^{1,q}},\quad \forall q>n
\]
\end{lemma}
\begin{proof}
Take \(\delta <\min \{\delta_0, \frac{\pi}{2b}\}\) and let \((r,\theta)\) be the
geodesic polar coordinate centered at \(P\in M\), then by Theorem
\ref{thm:vol-comparison}, the ratio of the metric volume form \(dV:= |g|dE\) and the Euclidean
volume form \(dE\) of \(T_P M\) is \(\sqrt{|g_\theta|}\geq \left(\frac{\sin
br}{br}\right)^{n-1}\geq (\frac{2}{\pi})^{n-1}\). 

let \(\chi: \mathbb{R}_{\geq 0} \longrightarrow \mathbb{R}\) be a cut-off function which
is constantly \(1\) near \(0\) and supported in \([0,\delta)\). Then
\[
 \varphi(P) = -\int_0^\delta
\partial_r\left(\varphi(r,\theta)\chi(r)\right)dr,\quad\forall \theta\in \mathbb{S}^{n-1}
\]
Integrate w.r.t \(\theta\in \mathbb{S}^{n-1}\), recall that \(\omega_n\) denotes the
volume of \(\mathbb{S}^{n-1}\):
\begin{equation*}
\begin{split}
 \left|\varphi(p)\right| &\leq (\omega_{n-1})^{-1} \int_B \left| \nabla(\varphi(r,\theta)\chi(r)) \right| r^{1-n}r^{n-1}drd\theta \\
			&\leq (\omega_{n-1})^{-1} \left(\int_B \left| \nabla(\varphi(r,\theta)\chi(r)) \right|^q dE\right)^{1/q}\left(\omega_{n-1}\int_0^\delta r^{(n-1)(1-q)}dr\right)^{1/q'}\\
			&\leq (\frac{\pi}{2})^{n-1} (\omega_{n-1})^{-1/q} \left(\|\nabla\varphi\|_{L^q} + \sup_{[0,\delta]}|\chi'|\|\varphi\|_{L^q} \right) \left(\frac{q-1}{q-n}\delta^{\frac{q-n}{q-1}}\right)^{1/q'}
\end{split}   
\end{equation*}
where   \(q'\) denotes the Hölder conjugate of \(q\) and for we used Hölder inequality
w.r.t \(dE\) for the second inequality and the comparison \(dE \leq
(\frac{\pi}{2})^{n-1}dV\) for the third. The conclusion follows.
\end{proof}

\begin{lemma}[\( W^{1,q} \hookrightarrow C^{0,\alpha} \)]
\label{lem:sobolev-case-holder}
Let \(M^n\) be a complete Riemannian manifold with injectivity radius \(\delta_0>0\)
and bounded curvature, then for all \(\varphi\in C_c^\infty(M)\), one
has
\[
\sup_M |\varphi| + \sup_{P\ne Q} |\varphi(P) -\varphi(Q)|\ d(P,Q)^{-\alpha} \leq
C(\alpha,q) \|\varphi\|_{W^{1,q}},\quad \text{ for all } \frac{1-\alpha}{n}\geq \frac{1}{q}
\]
\end{lemma}
\begin{proof}
By Lemma \ref{lem:sobolev-case}, one can discard the term \(\sup_M |\varphi|\) and only
need to treat the second term of LHS. Let \(\delta \leq \min\{\delta_0, \frac{\pi}{2b}\}\) as in the proof of Lemma \ref{lem:sobolev-case} (\(b^2\) being the upper bound of the
sectional curvature). One only need to consider the case where \(d=d(P,Q) <\delta/2\)
because otherwise \(|\varphi(P)-\varphi(Q)| \leq 2\|\varphi \|_{L^\infty}(\frac{\delta}{2})^{-\alpha}d(P,Q)^\alpha\).

Let \(O\) be the midpoint of \(P,Q\), and denote by \(h:=\varphi\circ\exp_O\)
defined on the Euclidean ball \(B(0,2d)\supset B_O:=B(0,d/2)\). We also denote by \(P,Q\) the
preimages of these points in \(B_O\). See Figure \ref{fig:sobolev-holder}.

\label{fig:sobolev-holder}
\begin{figure}[htbp]
\centering
\caption{Left: the picture viewed in normal polar coordinates at \(O\). Right: the picture viewed in normal polar coordinates at \(Q\).}
\end{figure}

Now place \(B_O\) in polar coordinate centered at \(Q\):
\[
 h(x) - h(Q) = \int_0^r \frac{\partial}{\partial r}h(r,\theta) dr = r\int_0^1
\frac{\partial}{\partial \rho}h(rt,\theta) dt
\]
Integrate on \(B_O\ni x\) w.r.t to the measure \(dE_Q\) given by the normal polar
coordinates at \(Q\):
\begin{equation}
\label{eq:lem:sobolev-case-holder}
\begin{split}
\int_{B_O} |h(x) - \varphi(Q)| dE_Q &\leq \int_{\theta\in \mathbb{S}^{n-1}} \int_{r=0}^{\rho(\theta)} r^{n-1}r \int_0^1|\frac{\partial}{\partial \rho}h(rt,\theta)|dtdrd\theta \\
( u:= rt, \rho(\theta)\leq d)\qquad &\leq \int_{\theta\in \mathbb{S}^{n-1}}\int_{t=0}^1\int_{u=0}^{td}t^{-n-1}u^n \left|\frac{\partial}{\partial \rho}h(u,\theta)\right|dtdud\theta\\
      	  		   &= \int_{t=0}^1 t^{-n-1}\left( \int_{u=0}^{td}\int_{\theta\in \mathbb{S}^{n-1}}\left| \frac{\partial}{\partial \rho}h(u,\theta) \right|u. dE_Q  \right)dt \\
(\text{Holder w.r.t } dE_Q)\qquad &\leq \int_{t=0}^1  t^{-n-1} \left( \int_{u=0}^{td} \int_{\theta\in \mathbb{S}^{n-1}} \left| \frac{\partial}{\partial \rho}h(u,\theta) \right|^q dE_Q \right)^{1/q} \left( \int_0^{td}\omega_{n-1} u^{q'}u^{n-1} du \right)^{1/q'}dt\\
(t\leq 1)\qquad	      &\leq\int_{t=0}^{1} t^{-n-1} \left(\frac{1}{q'+n}(td)^{q'+n}\right)^{1/q'} \left( \int_{u=0}^d\int_{\theta\in \mathbb{S}^{n-1}} |\nabla\varphi|^q dE_Q   \right)^{1/q}dt\\
       &= C_1(q,n)d^{1 + \frac{n}{q'}} \left(\int_{B(Q,d)}|\nabla\varphi|^q dE_Q   \right)^{1/q}
\end{split}   
\end{equation}

Now using the fact that \(\frac{1}{A} dV \leq dE_Q\leq A dV\) since the curvature is bounded, one
has
\[
 \int_{B(O,d/2)}|\varphi(x) -\varphi(Q)|dV \leq C_2(q,n)d^{1 + \frac{n}{q'}} \| \nabla\varphi\|_{L^q}
\]

Taking sum with the same computation for \(P\), one has
\[
\left| \varphi(P)-\varphi(Q)\right|\vol_g(B(O,d/2)) \leq 2C_2(q,n) d^{1 + \frac{n}{q'}}  \| \nabla\varphi\|_{L^q}
\]
since \(\vol_g(B(O,d/2)) \geq A^{-1}\omega_{n-1}d^n\), one has
\[
 \left| \varphi(P)-\varphi(Q)\right| \leq C_3(q,n) \|\nabla\varphi\|_{L^q} d^{1-n/q}
\]
The conclusion follows since \(1-\frac{n}{q}\geq \alpha\).
\end{proof}

For the first part of Theorem \ref{thm:sobolev-imbedding}, it suffices to prove the case
\(k=l+1\), that is, there exists a constant \(C_1, C_2>0\) such that
\(\|u\|_{L^p} \leq C_1\|\nabla u\|_{L^q} + C_2\| u\|_{L^q}\) for
\(u\in W^{1,q}(M)\) and \(\frac{1}{p} = \frac{1}{q}-\frac{1}{n}\).

The proof by \cite{aubin_nonlinear_1998} tries to optimise the constant \(C_1\), in an
attempt to find the best inequality \cite[page 50]{aubin_nonlinear_1998}. We will follow
their arguments, as the extra effort is not much. We will prove that

\begin{proposition}
\label{prop:sobolev-imbedding-1}
Given \(p,q\in \mathbb{R}_{>0}\) such that \(\frac{1}{p}=\frac{1}{q} - \frac{1}{n}>0\),
for any \(\epsilon >0\), there exists \(A_q(\epsilon)\) such that
\[
 \|u\|_p \leq (K(n,q) + \epsilon) \|\nabla u\|_{L^q} + A_q(\epsilon) \| u\|_{L^q}
\]
\end{proposition}
The appearance of the constant \(K(n,q)\), given by 
\[
K(n,q):=\begin{cases}
\frac{q-1}{n(q-1)}\left[\frac{n-q}{n(q-1)}\right]^{1/q}\left[\frac{\Gamma(n+1)}{\Gamma(n/q)\Gamma(n+1-n/q)\omega_{n-1}}\right]^{1/n},  & \text{if $q>1$} \\
\frac{1}{n}\left(\frac{n}{\omega_{n-1}}\right)^{1/n}, & \text{if $q=1$}
\end{cases}
\] 
is due to the following local result.
\begin{theorem}[Aubin]
\label{thm:aubin-loc}
Given \(1\leq q < n\) and \(u\in W^{1,q}(\mathbb{R}^n)\), with \(\frac{1}{p}=\frac{1}{q}-\frac{1}{n}\), one has
\[
 \|u\|_{L^p} \leq K(n,q) \|\nabla u\|_{L^q}.
\]
In fact, \(K(n,q)\) this the norm of the imbedding \(W^{1,q}(\mathbb{R}^n) \hookrightarrow L^p(\mathbb{R}^n)\).
\end{theorem}

We will accept the local result and use the Covering Lemma \ref{lem:uni-loc-finite-cover} to
prove Proposition \ref{prop:sobolev-imbedding-1}, which implies Theorem
\ref{thm:sobolev-imbedding}.

\begin{proof}[Proof of Proposition \ref{prop:sobolev-imbedding-1}]
Note that given any smooth function \(f\) supported in a small geodesic ball \(B(q,\delta)\),
by applying theorem \ref{thm:aubin-loc} to the \(f\), viewed in the chart (that is,
\(f\circ \exp_q\)) and use the fact that \(C(\delta)^{-1} \|\nabla (f\circ\exp_q)\|_{L^q(dE)} \leq \|\nabla f\|_{L^q(dV)} \leq C(\delta)
\|\nabla (f\circ\exp_q)\|_{L^q(dE)}\) (see remark \ref{rem:cor-vol-comp}), one has
\[
\|f\|_{L^p} \leq K_\delta(n,q) \|\nabla f \|_{L^q}
\]
where \(K_\delta(n,q)\) converges to \(K(n,q)\) as \(\delta\to 0\).


It suffice to cover \(M\) by geodesic ball \(B(Q_i,\delta)\) such that there exists a
partition of unity subordinated to \(B(Q_i,\delta)\) such that
\(\|\nabla (h_i^{1/q})\| \leq H=\const\). In fact for \(\varphi \in W^{1,q}(M)\), one has
\begin{equation*}
\begin{split}
\|\varphi\|_p^q &= \left(\int_M |\varphi|^p\right)^{q/p}=\left(\int_M \left(\sum_i |\varphi|^q h_i\right)^{p/q} \right)^{q/p}\\
(\text{since } p\geq q)\qquad		&\leq \sum_i\left(\int_M \left(|\varphi|^q h_i\right)^{p/q} \right)^{q/p} = \sum_i \left\| \varphi h_i^{1/q} \right\|_p^q\\
	       	     			&\leq K^q_\delta(n,q) \sum_i \left\| h_i^{1/q}\nabla\varphi + \varphi\nabla h_i^{1/q}\right\|_q^q
\end{split}   
\end{equation*}
Using the fact that there are at most \(k(\delta)\) balls overlapping at a point and that \((a+b)^q = a^q \left(1+ \frac{b}{a}\right)^q \leq a^q(1 + 2^q \frac{b}{a} +
2^q (\frac{b}{a})^{q})\leq a^q + 2^q ba^{q-1} + 2^{q}b^q\), one has
\begin{equation*}
\begin{split}
\|\varphi\|_{p}^q &\leq K^q_\delta(n,q)\left(\|\nabla\varphi\|_q^q + 2^q k(\delta) H^{q-1}\int_M |\varphi|^{q-1}|\nabla\varphi| + 2^q k(\delta) H^q \| \varphi\|_q^q \right) \\
		  &\leq K^q_\delta(n,q)\left[\|\nabla\varphi\|_q^q + 2^q k(\delta) H^{q-1}\|\nabla\varphi\|_q\|\varphi\|_q^{q-1}  + 2^q k(\delta) H^q \| \varphi\|_q^q \right]\\
\end{split}   
\end{equation*}
It is elementary to see that this implies
\(\|\varphi\|_{p}^q \leq (1+\epsilon)^qK^q(n,q)\left[(1+\epsilon)\|\nabla\varphi\|_q^q +
A(\epsilon) \| \varphi\|_q^q \right]\), from which the conclusion follows.  

For the existence of such \(h_i\), one cover \(M\) by balls \(B(Q_i,\delta)\) using
Lemma \ref{lem:uni-loc-finite-cover}. Denote by \(\varphi_i:B(Q_i,\delta) \longrightarrow
B(0,\delta)\) the inverse of exponential maps and let \(u:  \mathbb{R}^n \longrightarrow
\mathbb{R}\) be the smooth function, choose \(u\) to be a bell curve with maximal value
\(1\) at \(0\), supported in \(B(0,\delta)\) and \(u\leq \frac{1}{2}\) in \(B(0,\delta/2)\) and pose \(u_i = u\circ \varphi_i\). Then 
\[
 \|\nabla u_i\|_{g_M} \leq C_1(g_M,\delta) \|\nabla u\|_E = C_2(g_M,\delta)
\]
Pose \(h_i = \frac{u_i^m}{\sum u_j^m}\) with \(m>q\) then
\begin{equation*}
\begin{split}
\left|\nabla(h_i^{1/q})\right| &= \left|\frac{m}{q}\frac{u_i^{\frac{m}{q} - 1}\nabla u_i}{(\sum u_j^m)^{1/q}} + u_i^{m/q}\left(\frac{-1}{q}\right)\frac{\sum\nabla (u_j^m)}{(\sum u_j^m)^{1+ \frac{1}{q}}}\right|\\
		  &\leq \frac{m}{q.2^{-m/q}}|\nabla u_i| + \frac{1}{q}\sum m \frac{|\nabla u_j|}{(2^{-m})^{1+\frac{1}{q}}}\\
		  &\leq \left( \frac{m}{q}2^{m/q} + \frac{m}{q}2^{m(1+\frac{1}{q})}k(\delta)\right)C_2(g_M,\delta)=\const
\end{split}   
\end{equation*}
where \(k(\delta)\), as in Lemma \ref{lem:uni-loc-finite-cover}, is the upper bound of
number of balls overlapping at the point in question.
\end{proof}






\section{Kondrachov's theorem}
\label{sec:orgb2a0b0a}
The generalised version of Kondrachov's theorem is much easier to prove
\begin{theorem}[Kondrachov]
\label{thm:kondrachov}
Theorem \ref{thm:Kondrachov-Rn} holds when one replaces \(\Omega\) by a compact Riemannian
manifolds of dimension \(n\).
\end{theorem}
\begin{proof}
Cover \(M\) by finitely many small geodesic ball \(B(Q_i,\delta)\) subordinating a partition of unity \(\sum_{i=1}^N \chi_i = 1\), then if a sequence \(\{u_n\}_n \subset W^{k,q}\) is bounded then \(\{\chi_i u_n\}_n\) is also bounded in \(W^{k,q}\). The conclusion follows using Remark \ref{rem:cor-vol-comp} and the
Euclidean version of Kondrachov's theorem.
\end{proof}


\section{Solving \(\Delta u = f\) on a Riemannian manifold.}
\label{sec:org8cc1a23}
With Kondrachov's theorem \ref{thm:kondrachov}, one can uses the familiar "subsequence
extracting" technique to find a minimiser of the quadratic functional
\(\psi \mapsto \frac{1}{2}\int_M \|\nabla\psi\|^2 dV\) in a suitable subspace of
\(W^{1,2}(M)\) (method of Lagrange multiplier), one can prove the following results.

\begin{theorem}[Spectrum of \( \Delta \)]
\label{thm:spec-lap}
Let \(M^n\) be a compact Riemannian manifold then
\begin{enumerate}
\item The eigenvalues of \(\Delta - \nabla^\nu\nabla_\nu\) are \(\geq 0\).
\item The eigenfunctions of \(\delta_0=0\) are constant functions.
\item The eigenvalue \(\lambda_1\) is the minimum value of the functional
\[
    \psi \mapsto \frac{1}{2}\int_M \|\nabla\psi\|^2 dV
   \]
 on the subspace \(\left\{\psi \in W^{1,2}(M): \ \|\psi\|_2 = 1,\int\psi dV=0 \right\}\).
Moreover, first eigenfunctions are smooth.
\end{enumerate}
\end{theorem}


\begin{theorem}
\label{thm:lap-l2}
Given \(M^n\) be a compact Riemannian manifold, consider the Laplace equation on \(M\):
\begin{equation}
\label{eq:lap-M}
\Delta u = f	
\end{equation}
where \(f\in L^2(M)\), then:
\begin{enumerate}
\item There exists \(u\in W^{1,2}(M)\) satisfying \eqref{eq:lap-M} in the weak sense if and only if \(\int_M f
   dV = 0\)
\item \(u\) is unique up to an additive constant.
\item If \(f\in C^{r,\alpha}\) then \(u\in C^{r+2,\alpha}\).
\end{enumerate}
\end{theorem}



\bibliographystyle{alpha}
\bibliography{../res/Stage2018}
Emacs 25.3.1 (Org mode 9.0.5)
\end{document}